% Copyright 2006 by Till Tantau
%
% This file may be distributed and/or modified
%
% 1. under the LaTeX Project Public License and/or
% 2. under the GNU Free Documentation License.
%
% See the file doc/generic/pgf/licenses/LICENSE for more details.


% pgf version is defined in \pgfversion in file
% generic/pgf/utilities/pgfrcs.code.tex 

\def\xcolorversion{2.00}

\usepackage[version=latest]{pgf}

\usepackage{xkeyval,calc,listings,tikz}

% We need lots of libraries...
\usetikzlibrary{%
  arrows,%
  calc,%
  fit,%
  patterns,%
  plotmarks,%
  shapes.geometric,%
  shapes.misc,%
  shapes.symbols,%
  shapes.arrows,%
  shapes.callouts,%
  shapes.multipart,%
  shapes.gates.logic.US,%
  shapes.gates.logic.IEC,%
  er,%
  automata,%
  backgrounds,%
  chains,%
  topaths,%
  trees,%
  petri,%
  mindmap,%
  matrix,%
  calendar,%
  folding,%
  fadings,%
  through,%
  positioning,%
  scopes,%
  decorations.fractals,%
  decorations.shapes,%
  decorations.text,%
  decorations.pathmorphing,%
  decorations.pathreplacing,%
  decorations.footprints,%
  decorations.markings,%
  shadows}

\usepackage[a4paper,left=2.25cm,right=2.25cm,top=2.5cm,bottom=2.5cm,nohead]{geometry}
\usepackage{amsmath,amssymb}
\usepackage{xxcolor}
\usepackage{pifont}
\usepackage{makeidx}
\usepackage[latin1]{inputenc}
\usepackage{amsmath}

% Copyright 2006 by Till Tantau
%
% This file may be distributed and/or modified
%
% 1. under the LaTeX Project Public License and/or
% 2. under the GNU Free Documentation License.
%
% See the file doc/generic/pgf/licenses/LICENSE for more details.

% $Header: /cvsroot/pgf/pgf/doc/generic/pgf/macros/pgfmanual-en-macros.tex,v 1.28 2008/02/20 11:00:42 tantau Exp $


\providecommand\href[2]{\texttt{#1}}


\colorlet{examplefill}{yellow!80!black}
\definecolor{graphicbackground}{rgb}{0.96,0.96,0.8}
\definecolor{codebackground}{rgb}{0.8,0.8,1}

\newenvironment{pgfmanualentry}{\list{}{\leftmargin=2em\itemindent-\leftmargin\def\makelabel##1{\hss##1}}}{\endlist}
\newcommand\pgfmanualentryheadline[1]{\itemsep=0pt\parskip=0pt\item\strut{#1}\par\topsep=0pt}
\newcommand\pgfmanualbody{\parskip3pt}



\newenvironment{pgflayout}[1]{
  \begin{pgfmanualentry}
    \pgfmanualentryheadline{\texttt{\string\pgfpagesuselayout\char`\{\declare{#1}\char`\}}\oarg{options}}
    \index{#1@\protect\texttt{#1} layout}%
    \index{Page layouts!#1@\protect\texttt{#1}}%
    \pgfmanualbody
}
{
  \end{pgfmanualentry}
}


\newenvironment{command}[1]{
  \begin{pgfmanualentry}
    \extractcommand#1\@@
    \pgfmanualbody
}
{
  \end{pgfmanualentry}
}

%% MW: START MATH MACROS
\def\mvar#1{{\rmfamily\textit{#1}}}

\makeatletter

\def\extractmathfunctionname#1{\extractmathfunctionname@#1(,)\tmpa\tmpb}
\def\extractmathfunctionname@#1(#2)#3\tmpb{\def\mathname{#1}}

\def\extractmathoperatorname{\begingroup\def\mvar##1{}\def\ {}\extractmathoperatorname@}
\def\extractmathoperatorname@#1{\xdef\mathname{#1}\endgroup}

\makeatother
	
\def\vskipspecial#1{\vskip#1\vskip0em}

\newenvironment{math-function}[1]{
	\begin{pgfmanualentry}
		\extractmathfunctionname{#1}
		\pgfmanualentryheadline{\texttt{#1}}%
		\index{\mathname @\protect\texttt{\mathname} math function}%
		\index{Math functions!\mathname @\protect\texttt{\mathname}}
		\pgfmanualbody
}
{
	\end{pgfmanualentry}\vskipspecial{-3em}
}

\newenvironment{math-operator}[1]{	
	\begin{pgfmanualentry}
		\extractmathoperatorname{#1}
		\pgfmanualentryheadline{\texttt{#1}}%
		\index{\mathname @\protect\texttt{\mathname} math operator}%
		\index{Math operators!\mathname @\protect\texttt{\mathname}}
    	\pgfmanualbody
}
{%
	\end{pgfmanualentry}\vskipspecial{-3em}
}

\newenvironment{math-constant}[1]{
	\begin{pgfmanualentry}
		\pgfmanualentryheadline{\texttt{#1}}%
		\index{#1@\protect\texttt{#1} math constant}%
		\index{Math constants!#1@\protect\texttt{#1}}
		\pgfmanualbody
}
{
	\end{pgfmanualentry}\vskipspecial{-3em}
}
\def\calcname{\textsc{calc}}
%% MW: END MATH MACROS


\def\extractcommand#1#2\@@{%
  \pgfmanualentryheadline{\declare{\texttt{\string#1}}#2}%
  \removeats{#1}%
  \index{\strippedat @\protect\myprintocmmand{\strippedat}}}


\renewenvironment{environment}[1]{
  \begin{pgfmanualentry}
    \extractenvironement#1\@@
    \pgfmanualbody
}
{
  \end{pgfmanualentry}
}

\def\extractenvironement#1#2\@@{%
  \pgfmanualentryheadline{{\ttfamily\char`\\begin\char`\{\declare{#1}\char`\}}#2}%
  \pgfmanualentryheadline{{\ttfamily\ \ }\meta{environment contents}}%
  \pgfmanualentryheadline{{\ttfamily\char`\\end\char`\{\declare{#1}\char`\}}}%
  \index{#1@\protect\texttt{#1} environment}%
  \index{Environments!#1@\protect\texttt{#1}}}


\newenvironment{plainenvironment}[1]{
  \begin{pgfmanualentry}
    \extractplainenvironement#1\@@
    \pgfmanualbody
}
{
  \end{pgfmanualentry}
}

\def\extractplainenvironement#1#2\@@{%
  \pgfmanualentryheadline{{\ttfamily\declare{\char`\\#1}}#2}%
  \pgfmanualentryheadline{{\ttfamily\ \ }\meta{environment contents}}%
  \pgfmanualentryheadline{{\ttfamily\declare{\char`\\end#1}}}%
  \index{#1@\protect\texttt{#1} environment}%
  \index{Environments!#1@\protect\texttt{#1}}}


\newenvironment{contextenvironment}[1]{
  \begin{pgfmanualentry}
    \extractcontextenvironement#1\@@
    \pgfmanualbody
}
{
  \end{pgfmanualentry}
}

\def\extractcontextenvironement#1#2\@@{%
  \pgfmanualentryheadline{{\ttfamily\declare{\char`\\start#1}}#2}%
  \pgfmanualentryheadline{{\ttfamily\ \ }\meta{environment contents}}%
  \pgfmanualentryheadline{{\ttfamily\declare{\char`\\stop#1}}}%
  \index{#1@\protect\texttt{#1} environment}%
  \index{Environments!#1@\protect\texttt{#1}}}


\newenvironment{shape}[1]{
  \begin{pgfmanualentry}
  	\pgfmanualentryheadline{Shape {\ttfamily\declare{#1}}}%
    \index{#1@\protect\texttt{#1} shape}%
    \index{Shapes!#1@\protect\texttt{#1}}
    \pgfmanualbody
}
{
  \end{pgfmanualentry}
}


\newenvironment{handler}[1]{
  \begin{pgfmanualentry}
    \extracthandler#1\@nil%
    \pgfmanualbody
}
{
  \end{pgfmanualentry}
}

\def\gobble#1{}
\def\extracthandler#1#2\@nil{%
  \pgfmanualentryheadline{Key handler \meta{key}{\ttfamily/\declare{#1}}#2}%
  \index{\gobble#1@\protect\texttt{#1} handler}%
  \index{Key handlers!#1@\protect\texttt{#1}}
}


\makeatletter


\newenvironment{stylekey}[1]{
  \begin{pgfmanualentry}
    \def\extrakeytext{style, }
    \extractkey#1\@nil%
    \pgfmanualbody
}
{
  \end{pgfmanualentry}
}


\newenvironment{key}[1]{
  \begin{pgfmanualentry}
    \def\extrakeytext{}
    %\def\altpath{\emph{\color{gray}or}}%
    \extractkey#1\@nil%
    \pgfmanualbody
}
{
  \end{pgfmanualentry}
}

\def\extractkey#1\@nil{%
  \pgfutil@in@={#1}%
  \ifpgfutil@in@%
    \extractkeyequal#1\@nil
  \else%
    \pgfutil@in@{(initial}{#1}%
    \ifpgfutil@in@%
      \extractequalinitial#1\@nil%
    \else
      \pgfmanualentryheadline{{\ttfamily\declare{#1}}\hfill(\extrakeytext no value)}%
      \def\mykey{#1}%
      \def\mypath{}%
      \def\myname{}%
      \firsttimetrue%
      \decompose#1/\nil%
    \fi
  \fi%
}

\def\extractkeyequal#1=#2\@nil{%
  \pgfutil@in@{(default}{#2}%
  \ifpgfutil@in@%
    \extractdefault{#1}#2\@nil%
  \else%
    \pgfutil@in@{(initial}{#2}%
    \ifpgfutil@in@%
      \extractinitial{#1}#2\@nil%
    \else
      \pgfmanualentryheadline{{\ttfamily\declare{#1}=}#2\hfill(\extrakeytext no default)}%
    \fi%
  \fi%
  \def\mykey{#1}%
  \def\mypath{}%
  \def\myname{}%
  \firsttimetrue%
  \decompose#1/\nil%
}

\def\extractdefault#1#2(default #3)\@nil{%
  \pgfmanualentryheadline{{\ttfamily\declare{#1}\opt{=}}\opt{#2}\hfill (\extrakeytext default {\ttfamily#3})}%
}

\def\extractinitial#1#2(initially #3)\@nil{%
  \pgfmanualentryheadline{{\ttfamily\declare{#1}=}#2\hfill (\extrakeytext no default, initially {\ttfamily#3})}%
}

\def\extractequalinitial#1 (initially #2)\@nil{%
  \pgfmanualentryheadline{{\ttfamily\declare{#1}}\hfill (\extrakeytext initially {\ttfamily#2})}%
  \def\mykey{#1}%
  \def\mypath{}%
  \def\myname{}%
  \firsttimetrue%
  \decompose#1/\nil%
}

\def\keyalias#1{\vspace{-3pt}\item{\small alias {\ttfamily/#1/\myname}}\vspace{-2pt}\par}

\newif\iffirsttime

\makeatother

\def\decompose/#1/#2\nil{%
  \def\test{#2}%
  \ifx\test\empty%
    % aha.
    \index{#1@\protect\texttt{#1} key}%
    \index{\mypath#1@\protect\texttt{#1}}%
    \def\myname{#1}%
  \else%
    \iffirsttime
      \def\mypath{#1@\protect\texttt{/#1/}!}%
      \firsttimefalse
    \else
      \expandafter\def\expandafter\mypath\expandafter{\mypath#1@\protect\texttt{#1/}!}%
    \fi
    \def\firsttime{}
    \decompose/#2\nil%
  \fi%
}


\newenvironment{predefinednode}[1]{
  \begin{pgfmanualentry}
    \pgfmanualentryheadline{Predefined node {\ttfamily\declare{#1}}}%
    \index{#1@\protect\texttt{#1} node}%
    \index{Predefined node!#1@\protect\texttt{#1}}
    \pgfmanualbody
}
{
  \end{pgfmanualentry}
}

\newenvironment{coordinatesystem}[1]{
  \begin{pgfmanualentry}
    \pgfmanualentryheadline{Coordinate system {\ttfamily\declare{#1}}}%
    \index{#1@\protect\texttt{#1} coordinate system}%
    \index{Coordinate systems!#1@\protect\texttt{#1}}
    \pgfmanualbody
}
{
  \end{pgfmanualentry}
}

\newenvironment{snake}[1]{
  \begin{pgfmanualentry}
    \pgfmanualentryheadline{Snake {\ttfamily\declare{#1}}}%
    \index{#1@\protect\texttt{#1} snake}%
    \index{Snakes!#1@\protect\texttt{#1}}
    \pgfmanualbody
}
{
  \end{pgfmanualentry}
}

\newenvironment{decoration}[1]{
  \begin{pgfmanualentry}
    \pgfmanualentryheadline{Decoration {\ttfamily\declare{#1}}}%
    \index{#1@\protect\texttt{#1} decoration}%
    \index{Decorations!#1@\protect\texttt{#1}}
    \pgfmanualbody
}
{
  \end{pgfmanualentry}
}


\def\pgfmanualbar{\char`\|}
\makeatletter
\newenvironment{pathoperation}[3][]{
  \begin{pgfmanualentry}
    \pgfmanualentryheadline{\textcolor{gray}{{\ttfamily\char`\\path}\
        \ \dots}
      \declare{\texttt{#2}}#3\ \textcolor{gray}{\dots\texttt{;}}}%
    \def\pgfmanualtest{#1}%
    \ifx\pgfmanualtest\@empty%
      \index{#2@\protect\texttt{#2} path operation}%
      \index{Path operations!#2@\protect\texttt{#2}}%
    \fi%
    \pgfmanualbody
}
{
  \end{pgfmanualentry}
}
\makeatother

\def\extractcommand#1#2\@@{%
  \pgfmanualentryheadline{\declare{\texttt{\string#1}}#2}%
  \removeats{#1}%
  \index{\strippedat @\protect\myprintocmmand{\strippedat}}}

\def\doublebs{\texttt{\char`\\\char`\\}}


\newenvironment{package}[1]{
  \begin{pgfmanualentry}
    \pgfmanualentryheadline{{\ttfamily\char`\\usepackage\char`\{\declare{#1}\char`\}\space\space \char`\%\space\space  \LaTeX}}
    \index{#1@\protect\texttt{#1} package}%
    \index{Packages and files!#1@\protect\texttt{#1}}%
    \pgfmanualentryheadline{{\ttfamily\char`\\input \declare{#1}.tex\space\space\space \char`\%\space\space  plain \TeX}}
    \pgfmanualentryheadline{{\ttfamily\char`\\usemodule[\declare{#1}]\space\space \char`\%\space\space  Con\TeX t}}
    \pgfmanualbody
}
{
  \end{pgfmanualentry}
}


\newenvironment{pgfmodule}[1]{
  \begin{pgfmanualentry}
    \pgfmanualentryheadline{{\ttfamily\char`\\usepgfmodule\char`\{\declare{#1}\char`\}\space\space\space
        \char`\%\space\space  \LaTeX\space and plain \TeX\space and pure pgf}}
    \index{#1@\protect\texttt{#1} module}%
    \index{Modules!#1@\protect\texttt{#1}}%
    \pgfmanualentryheadline{{\ttfamily\char`\\usepgfmodule[\declare{#1}]\space\space \char`\%\space\space  Con\TeX t\space and pure pgf}}
    \pgfmanualbody
}
{
  \end{pgfmanualentry}
}

\newenvironment{pgflibrary}[1]{
  \begin{pgfmanualentry}
    \pgfmanualentryheadline{{\ttfamily\char`\\usepgflibrary\char`\{\declare{#1}\char`\}\space\space\space
        \char`\%\space\space  \LaTeX\space and plain \TeX\space and pure pgf}}
    \index{#1@\protect\texttt{#1} library}%
    \index{Libraries!#1@\protect\texttt{#1}}%
    \pgfmanualentryheadline{{\ttfamily\char`\\usepgflibrary[\declare{#1}]\space\space \char`\%\space\space  Con\TeX t\space and pure pgf}}
    \pgfmanualentryheadline{{\ttfamily\char`\\usetikzlibrary\char`\{\declare{#1}\char`\}\space\space
        \char`\%\space\space  \LaTeX\space and plain \TeX\space when using \tikzname}}
    \pgfmanualentryheadline{{\ttfamily\char`\\usetikzlibrary[\declare{#1}]\space
        \char`\%\space\space  Con\TeX t\space when using \tikzname}}
    \pgfmanualbody
}
{
  \end{pgfmanualentry}
}

\newenvironment{tikzlibrary}[1]{
  \begin{pgfmanualentry}
    \pgfmanualentryheadline{{\ttfamily\char`\\usetikzlibrary\char`\{\declare{#1}\char`\}\space\space \char`\%\space\space  \LaTeX\space and plain \TeX}}
    \index{#1@\protect\texttt{#1} library}%
    \index{Libraries!#1@\protect\texttt{#1}}%
    \pgfmanualentryheadline{{\ttfamily\char`\\usetikzlibrary[\declare{#1}]\space \char`\%\space\space Con\TeX t}}
    \pgfmanualbody
}
{
  \end{pgfmanualentry}
}



\newenvironment{filedescription}[1]{
  \begin{pgfmanualentry}
    \pgfmanualentryheadline{File {\ttfamily\declare{#1}}}%
    \index{#1@\protect\texttt{#1} file}%
    \index{Packages and files!#1@\protect\texttt{#1}}%
    \pgfmanualbody
}
{
  \end{pgfmanualentry}
}


\newenvironment{packageoption}[1]{
  \begin{pgfmanualentry}
    \pgfmanualentryheadline{{\ttfamily\char`\\usepackage[\declare{#1}]\char`\{pgf\char`\}}}
    \index{#1@\protect\texttt{#1} package option}%
    \index{Package options for \textsc{pgf}!#1@\protect\texttt{#1}}%
    \pgfmanualbody
}
{
  \end{pgfmanualentry}
}



\newcommand\opt[1]{{\color{black!50!green}#1}}
\newcommand\ooarg[1]{{\ttfamily[}\meta{#1}{\ttfamily]}}

\def\opt{\afterassignment\pgfmanualopt\let\next=}
\def\pgfmanualopt{\ifx\next\bgroup\bgroup\color{black!50!green}\else{\color{black!50!green}\next}\fi}



\def\beamer{\textsc{beamer}}
\def\pdf{\textsc{pdf}}
\def\pgfname{\textsc{pgf}}
\def\tikzname{Ti\emph{k}Z}
\def\pstricks{\textsc{pstricks}}
\def\prosper{\textsc{prosper}}
\def\seminar{\textsc{seminar}}
\def\texpower{\textsc{texpower}}
\def\foils{\textsc{foils}}

{
  \makeatletter
  \global\let\myempty=\@empty
  \global\let\mygobble=\@gobble
  \catcode`\@=12
  \gdef\getridofats#1@#2\relax{%
    \def\getridtest{#2}%
    \ifx\getridtest\myempty%
      \expandafter\def\expandafter\strippedat\expandafter{\strippedat#1}
    \else%
      \expandafter\def\expandafter\strippedat\expandafter{\strippedat#1\protect\printanat}
      \getridofats#2\relax%
    \fi%
  }

  \gdef\removeats#1{%
    \let\strippedat\myempty%
    \edef\strippedtext{\stripcommand#1}%
    \expandafter\getridofats\strippedtext @\relax%
  }
  
  \gdef\stripcommand#1{\expandafter\mygobble\string#1}
}

\def\printanat{\char`\@}

\def\declare{\afterassignment\pgfmanualdeclare\let\next=}
\def\pgfmanualdeclare{\ifx\next\bgroup\bgroup\color{red!75!black}\else{\color{red!75!black}\next}\fi}


\let\textoken=\command
\let\endtextoken=\endcommand

\def\myprintocmmand#1{\texttt{\char`\\#1}}

\def\example{\par\smallskip\noindent\textit{Example: }}
\def\themeauthor{\par\smallskip\noindent\textit{Theme author: }}


\def\indexoption#1{%
  \index{#1@\protect\texttt{#1} option}%
  \index{Graphic options and styles!#1@\protect\texttt{#1}}%
}

\def\itemcalendaroption#1{\item \declare{\texttt{#1}}%
  \index{#1@\protect\texttt{#1} date test}%
  \index{Date tests!#1@\protect\texttt{#1}}%
}



\def\class#1{\list{}{\leftmargin=2em\itemindent-\leftmargin\def\makelabel##1{\hss##1}}%
\extractclass#1@\par\topsep=0pt}
\def\endclass{\endlist}
\def\extractclass#1#2@{%
\item{{{\ttfamily\char`\\documentclass}#2{\ttfamily\char`\{\declare{#1}\char`\}}}}%
  \index{#1@\protect\texttt{#1} class}%
  \index{Classes!#1@\protect\texttt{#1}}}

\def\partname{Part}

\makeatletter
\def\index@prologue{\section*{Index}\addcontentsline{toc}{section}{Index}
  This index only contains automatically generated entries. A good
  index should also contain carefully selected keywords. This index is
  not a good index.
  \bigskip
}
\c@IndexColumns=2
  \def\theindex{\@restonecoltrue
    \columnseprule \z@  \columnsep 29\p@
    \twocolumn[\index@prologue]%
       \parindent -30pt
       \columnsep 15pt
       \parskip 0pt plus 1pt
       \leftskip 30pt
       \rightskip 0pt plus 2cm
       \small
       \def\@idxitem{\par}%
    \let\item\@idxitem \ignorespaces}
  \def\endtheindex{\onecolumn}
\def\noindexing{\let\index=\@gobble}



\newcommand\symarrow[1]{
  \index{#1@\protect\texttt{#1} arrow tip}%
  \index{Arrow tips!#1@\protect\texttt{#1}}
  \texttt{#1}& yields thick  
  \begin{tikzpicture}[arrows={#1-#1},thick,baseline]
    \useasboundingbox (0pt,-0.5ex) rectangle (1cm,2ex);
    \draw (0pt,.5ex) -- (1cm,.5ex);
  \end{tikzpicture} and thin
  \begin{tikzpicture}[arrows={#1-#1},thin,baseline]
    \useasboundingbox (0pt,-0.5ex) rectangle (1cm,2ex);
    \draw (0pt,.5ex) -- (1cm,.5ex);
  \end{tikzpicture}
}

\newcommand\sarrow[2]{
  \index{#1@\protect\texttt{#1} arrow tip}%
  \index{Arrow tips!#1@\protect\texttt{#1}}
  \index{#2@\protect\texttt{#2} arrow tip}%
  \index{Arrow tips!#2@\protect\texttt{#2}}
  \texttt{#1-#2}& yields thick  
  \begin{tikzpicture}[arrows={#1-#2},thick,baseline]
    \useasboundingbox (0pt,-0.5ex) rectangle (1cm,2ex);
    \draw (0pt,.5ex) -- (1cm,.5ex);
  \end{tikzpicture} and thin
  \begin{tikzpicture}[arrows={#1-#2},thin,baseline]
    \useasboundingbox (0pt,-0.5ex) rectangle (1cm,2ex);
    \draw (0pt,.5ex) -- (1cm,.5ex);
  \end{tikzpicture}
}

\newcommand\carrow[1]{
  \index{#1@\protect\texttt{#1} arrow tip}%
  \index{Arrow tips!#1@\protect\texttt{#1}}
  \texttt{#1}& yields for line width 1ex
  \begin{tikzpicture}[arrows={#1-#1},line width=1ex,baseline]
    \useasboundingbox (0pt,-0.5ex) rectangle (1.5cm,2ex);
    \draw (0pt,.5ex) -- (1.5cm,.5ex);
  \end{tikzpicture}
}
\def\myvbar{\char`\|}
\newcommand\plotmarkentry[1]{%
  \index{#1@\protect\texttt{#1} plot mark}%
  \index{Plot marks!#1@\protect\texttt{#1}}
  \texttt{\char`\\pgfuseplotmark\char`\{\declare{#1}\char`\}} &
  \tikz\draw[color=black!25] plot[mark=#1,mark options={fill=examplefill,draw=black}] coordinates{(0,0) (.5,0.2) (1,0) (1.5,0.2)};\\
}
\newcommand\plotmarkentrytikz[1]{%
  \index{#1@\protect\texttt{#1} plot mark}%
  \index{Plot marks!#1@\protect\texttt{#1}}
  \texttt{mark=\declare{#1}} & \tikz\draw[color=black!25]
  plot[mark=#1,mark options={fill=examplefill,draw=black}] 
    coordinates {(0,0) (.5,0.2) (1,0) (1.5,0.2)};\\
}



\ifx\scantokens\@undefined
  \PackageError{pgfmanual-macros}{You need to use extended latex
    (elatex) or (pdfelatex) to process this document}{}
\fi

\begingroup
\catcode`|=0
\catcode`[= 1
\catcode`]=2
\catcode`\{=12
\catcode `\}=12
\catcode`\\=12 |gdef|find@example#1\end{codeexample}[|endofcodeexample[#1]]
|endgroup

\begingroup
\catcode`\^=7
\catcode`\^^M=13
\catcode`\ =13%
\gdef\returntospace{\catcode`\ =13\def {\space}\catcode`\^^M=13\def^^M{}}%
\endgroup

\begingroup
\catcode`\%=13
\catcode`\^^M=13
\gdef\commenthandler{\catcode`\%=13\def%{\@gobble@till@return}}
\gdef\@gobble@till@return#1^^M{}
\gdef\@gobble@till@return@ignore#1^^M{\ignorespaces}
\gdef\typesetcomment{\catcode`\%=13\def%{\@typeset@till@return}}
\gdef\@typeset@till@return#1^^M{{\def%{\char`\%}\textsl{\char`\%#1}}\par}
\endgroup

\define@key{codeexample}{width}{\setlength\codeexamplewidth{#1}}
\define@key{codeexample}{graphic}{\colorlet{graphicbackground}{#1}}
\define@key{codeexample}{code}{\colorlet{codebackground}{#1}}
\define@key{codeexample}{execute code}{\csname code@execute#1\endcsname}
\define@key{codeexample}{code only}[]{\code@executefalse}
\define@key{codeexample}{pre}{\def\code@pre{#1}}
\define@key{codeexample}{post}{\def\code@post{#1}}
\define@key{codeexample}{vbox}[]{\def\code@pre{\vbox\bgroup\setlength{\hsize}{\linewidth-6pt}}\def\code@post{\egroup}}
\define@key{codeexample}{ignorespaces}[]{\let\@gobble@till@return=\@gobble@till@return@ignore}
\define@key{codeexample}{leave comments}[]{\def\code@catcode@hook{\catcode`\%=12}\let\commenthandler=\relax\let\typesetcomment=\relax}

\def\code@pre{}
\def\code@post{}
\def\code@catcode@hook{}

\newdimen\codeexamplewidth
\newif\ifcode@execute
\newbox\codeexamplebox
\def\codeexample[#1]{%
  \begingroup%
  \code@executetrue
  \setlength\codeexamplewidth{4cm+7pt}
  \setkeys{codeexample}{#1}%
  \parindent0pt
  \begingroup%
  \par%
  \medskip%
  \let\do\@makeother%
  \dospecials%
  \obeylines%
  \@vobeyspaces%
  \catcode`\%=13%
  \catcode`\^^M=13%
  \code@catcode@hook%
  \relax%
  \find@example}
\def\endofcodeexample#1{%
  \endgroup%
  \ifcode@execute%
    \setbox\codeexamplebox=\hbox{%
      {%
        {%
          \returntospace%
          \commenthandler%
          \xdef\code@temp{#1}% removes returns and comments
        }%
        \colorbox{graphicbackground}{\color{black}\ignorespaces%
          \code@pre\expandafter\scantokens\expandafter{\code@temp\ignorespaces}\code@post\ignorespaces}%
      }%
    }%
    \ifdim\wd\codeexamplebox>\codeexamplewidth%
      \def\code@start{\par}%
      \def\code@flushstart{}\def\code@flushend{}%
      \def\code@mid{\parskip2pt\par\noindent}%
      \def\code@width{\linewidth-6pt}%
      \def\code@end{}%
    \else%
      \def\code@start{%
        \linewidth=\textwidth%
        \parshape \@ne 0pt \linewidth
        \leavevmode%
        \hbox\bgroup}%
      \def\code@flushstart{\hfill}%
      \def\code@flushend{\hbox{}}%
      \def\code@mid{\hskip6pt}%
      \def\code@width{\linewidth-12pt-\codeexamplewidth}%
      \def\code@end{\egroup}%
    \fi%
    \code@start%
    \noindent%
    \begin{minipage}[t]{\codeexamplewidth}\raggedright
      \hrule width0pt%
      \footnotesize\vskip-1em%
      \code@flushstart\box\codeexamplebox\code@flushend%
      \vskip-1ex
      \leavevmode%
    \end{minipage}%
  \else%
    \def\code@mid{\par}
    \def\code@width{\linewidth-6pt}
    \def\code@end{}
  \fi%
  \code@mid%  
  \colorbox{codebackground}{%
    \begin{minipage}[t]{\code@width}%
      {%
        \let\do\@makeother
        \dospecials
        \frenchspacing\@vobeyspaces
        \normalfont\ttfamily\footnotesize
        \typesetcomment%
        \@tempswafalse
        \def\par{%
          \if@tempswa
          \leavevmode \null \@@par\penalty\interlinepenalty
          \else
          \@tempswatrue
          \ifhmode\@@par\penalty\interlinepenalty\fi
          \fi}%
        \obeylines
        \everypar \expandafter{\the\everypar \unpenalty}%
        #1}
    \end{minipage}}%
  \code@end%
  \par%
  \medskip
  \end{codeexample}
}

\def\endcodeexample{\endgroup}


\makeatother


%%% Local Variables: 
%%% mode: latex
%%% TeX-master: "beameruserguide"
%%% End: 


\makeindex

\makeatletter
\renewcommand*\l@subsection{\@dottedtocline{2}{1.5em}{2.8em}}
\renewcommand*\l@subsubsection{\@dottedtocline{3}{4.3em}{3.2em}}
\makeatother

%\includeonly{}

% Global styles:
\tikzset{
  every plot/.style={prefix=plots/pgf-},
  shape example/.style={
    color=black!30,
    draw,
    fill=yellow!30,
    line width=.5cm,
    inner xsep=2.5cm,
    inner ysep=0.5cm}
}

\index{Options for graphics|see{Graphic options and styles}}
\index{Styles for graphics|see{Graphic options and styles}}
\index{Options for packages|see{Package options}}
\index{Handlers for keys|see{Key handlers}}
\index{File|see{Packages and files}}
\index{Layout|see{Page layout}}
\index{Node|see{Predefined node}}

\begin{document}

%
��!�}	� �
����DEV��INO��SYN��SV~�pgfmanual-en-base-decorations.tex�J��J���J�j���P�_.j�
%\end{document}



% The titlepage

{
  \parindent0pt
  \null
  \colorlet{mintgreen}{green!50!black!50}
  
  \thispagestyle{empty}
  \vskip3cm
  \vfill
  \hfil
  \begin{tikzpicture}[overlay]
    \coordinate (front) at (0,0);
    \coordinate (horizon) at (0,.31\paperheight);
    \coordinate (bottom) at (0,-.6\paperheight);
    \coordinate (sky) at (0,.57\paperheight);
    \coordinate (left) at (-.51\paperwidth,0);
    \coordinate (right) at (.51\paperwidth,0);
    
    \shade [bottom color=blue!30!black!10,top color=blue!30!black!50]
      ([yshift=-5mm]horizon -|  left) rectangle (sky -| right);
    \shade [bottom color=black!70!green!25,top color=black!70!green!10]
      (front -| left) -- (horizon -| left)
      decorate [decoration=random steps] { -- (horizon -| right) }
      -- (front -| right) -- cycle;
    \shade [top color=black!70!green!25,bottom color=black!25]
      ([yshift=-5mm-1pt]front -| left) rectangle ([yshift=1pt]front -| right);
    \fill [black!25] (bottom -| left) rectangle ([yshift=-5mm]front -| right);

    \def\nodeshadowed[#1]#2;{\node[scale=2,above,#1]{#2};\node[scale=2,above,#1,yscale=-1,scope fading=south,opacity=0.4]{#2};}

    \nodeshadowed [at={(-5,5  )},yslant=0.05] {\Huge Ti\textcolor{orange}{\emph{k}}Z};
    \nodeshadowed [at={( 0,5.3)}] {\huge \textcolor{mintgreen}{\&}};
    \nodeshadowed [at={( 5,5  )},yslant=-0.05] {\Huge \textsc{PGF}};
    \nodeshadowed [at={( 0,2  )}] {Manual for Version \pgftypesetversion};

    \foreach \i in {0.5,0.6,...,2}
      \fill [white,decoration=Koch snowflake,opacity=.9]
            [shift=(horizon),shift={(rand*11,rnd*7)},scale=\i]
            [double copy shadow={opacity=0.2,shadow xshift=0pt,shadow
              yshift=3*\i pt,fill=white,draw=none}]
        decorate { 
          decorate { 
            decorate {
              (0,0) -- ++(60:1) -- ++(-60:1) -- cycle
            }
          }
        };

  \node (left text) [text width=.5\paperwidth-2cm,below right,at={(-.5\paperwidth+1cm,-1.5cm)}]
  {
    \fontfamily{pcr}
    \def\textbraceleft{\char`\{}
    \def\textbraceright{\char`\}}
    \def\textbackslash{\char`\\}
    \begin{lstlisting}[basicstyle=\scriptsize\color{black},
                       keywordstyle=\bfseries\color{white},
                       identifierstyle=\bfseries\color{black},
                       keywords={tikzpicture,shade,fill,draw,path,node},
                       literate={-}{{-}}1]
\begin{tikzpicture}
  \coordinate (front) at (0,0);
  \coordinate (horizon) at (0,.31\paperheight);
  \coordinate (bottom) at (0,-.6\paperheight);
  \coordinate (sky) at (0,.57\paperheight);
  \coordinate (left) at (-.51\paperwidth,0);
  \coordinate (right) at (.51\paperwidth,0);
    
  \shade [bottom color=white,
          top color=blue!30!black!50]
              ([yshift=-5mm]horizon -|  left)
    rectangle (sky -| right);
  
  \shade [bottom color=black!70!green!25,
          top color=black!70!green!10]
    (front -| left) -- (horizon -| left)
    decorate [decoration=random steps] {
      -- (horizon -| right)  }
    -- (front -| right) -- cycle;
    
  \shade [top color=black!70!green!25,
         bottom color=black!25]
              ([yshift=-5mm-1pt]front -| left)
    rectangle ([yshift=1pt]front -| right);

  \fill [black!25]
              (bottom -| left)
    rectangle ([yshift=-5mm]front -| right);
 
  \def\nodeshadowed[#1]#2;{
    \node[scale=2,above,#1]{#2};
    \node[scale=2,above,#1,yscale=-1,
          scope fading=south,opacity=0.4]{#2};
  }
\end{lstlisting}
};

  \node (right text) [text width=.5\paperwidth-2cm,below right,at={(1cm,-1.5cm)}]
  {
    \fontfamily{pcr}
    \def\textbraceleft{\char`\{}
    \def\textbraceright{\char`\}}
    \def\textbackslash{\char`\\}
    \begin{lstlisting}[basicstyle=\scriptsize\color{black},
                       keywordstyle=\bfseries\color{white},
                       identifierstyle=\bfseries\color{black},
                       keywords={tikzpicture,shade,fill,draw,path,node},
                       literate={-}{{-}}1]
  \nodeshadowed [at={(-5,8  )},yslant=0.05]
    {\Huge Ti\textcolor{orange}{\emph{k}}Z};
  \nodeshadowed [at={( 0,8.3)}]
    {\huge \textcolor{green!50!black!50}{\&}};
  \nodeshadowed [at={( 5,8  )},yslant=-0.05]
    {\Huge \textsc{PGF}};
  \nodeshadowed [at={( 0,5  )}]
    {Manual for Version \pgftypesetversion};

  \foreach \i in {0.5,0.6,...,2}
    \fill
      [white,opacity=\i/2,
       decoration=Koch snowflake,
       shift=(horizon),shift={(rand*11,rnd*7)},
       scale=\i,double copy shadow={
         opacity=0.2,shadow xshift=0pt,
         shadow yshift=3*\i pt,
         fill=white,draw=none}]
      decorate {
        decorate { 
          decorate { 
            (0,0)- ++(60:1) -- ++(-60:1) -- cycle
          } } };

   \node (left text) ...
   \node (right text) ...

   \fill [decorate,
          decoration={footprints,foot of=gnome},
          opacity=.5,brown] (left text.south)
     to [out=-45,in=135]    (right text.north);
   \fill [decorate,
     decoration={footprints,foot of=felis silvestris,
       foot length=5pt,stride length=15pt,foot angle=0},
     opacity=.5,green!50!black] (left text.south)
    to [out=20,in=180] (right text.north west);
\end{tikzpicture}
  \end{lstlisting}
  };

  \fill [decorate,decoration=footprints,
         decoration={footprints,foot of=gnome},
         opacity=.5,brown] (left text.south)
    to [out=-45,in=135]    (right text.north);
  \fill [decorate,decoration={footprints,foot length=5pt,foot of=felis
    silvestris,stride  length=15pt,foot angle=0},
     opacity=.5,green!50!black] (left text.south)
    to [out=20,in=180] (right text.north west);
\end{tikzpicture}
\vfill
\vbox{}
\clearpage
}

{
  \vbox{}
  \vskip0pt plus 1fill
  F�r meinen Vater, damit er noch viele sch�ne \TeX-Graphiken
  erschaffen kann.
  \vskip1em
  \hfill\emph{Till}
  \vskip0pt plus 3fill

  \parindent=0pt
  Copyright 2007 by Till Tantau

  \medskip  
  Permission is granted to copy, distribute and/or modify \emph{the documentation}
  under the terms of the \textsc{gnu} Free Documentation License, Version 1.2
  or any later version published by the Free Software Foundation;
  with no Invariant Sections, no Front-Cover Texts, and no Back-Cover Texts.
  A copy of the license is included in the section entitled \textsc{gnu}
  Free Documentation License.

  \medskip  
  Permission is granted to copy, distribute and/or modify \emph{the
    code of the package} under the terms of the \textsc{gnu} Public License, Version 2
  or any later version published by the Free Software Foundation.
  A copy of the license is included in the section entitled \textsc{gnu}
  Public License.

  \medskip  
  Permission is also granted to distribute and/or modify \emph{both
    the documentation and the code} under the conditions of the LaTeX
  Project Public License, either version 1.3 of this license or (at
  your option) any later version. A copy of the license is included in
  the section entitled \LaTeX\ Project Public License. 

  \vbox{}
  \clearpage
}


\title{\bfseries The \tikzname\ and {\Large PGF} Packages\\
  \large Manual for version \pgfversion\\[1mm]
\large\href{http://sourceforge.net/projects/pgf}{\texttt{http://sourceforge.net/projects/pgf}}}
\author{Till Tantau\footnote{Editor of this documentation. Parts of
    this documentation have been written by other authors as indicated
    in these parts or chapters and in Section~\ref{section-authors}.}\\
  \normalsize Institut f\"ur Theoretische Informatik\\[-1mm]
  \normalsize Universit\"at zu L\"ubeck}

\maketitle

\tableofcontents

\clearpage



���}	� �
����DEV��INO��SYN��SV~�pgfmanual-en-introduction.tex�J��J���J�j��P�_.j�


\part{Tutorials and Guidelines}

{\Large \emph{by Till Tantau}}

\bigskip
\noindent
To help you get started with \tikzname, instead of a long installation
and configuration section, this manual starts with tutorials. They
explain all the basic and some of the more advanced features of the
system, without going into all the details. This part also contains
some guidelines on how you should proceed when creating graphics using
\tikzname. 

\vskip3cm

\begin{codeexample}[graphic=white,width=0pt]
\tikz \draw[thick,rounded corners=8pt]
  (0,0) -- (0,2) -- (1,3.25) -- (2,2) -- (2,0) -- (0,2) -- (2,2) -- (0,0) -- (2,0);
\end{codeexample}

% Copyright 2006 by Till Tantau
%
% This file may be distributed and/or modified
%
% 1. under the LaTeX Project Public License and/or
% 2. under the GNU Free Documentation License.
%
% See the file doc/generic/pgf/licenses/LICENSE for more details.

\section{Tutorial: A Picture for Karl's Students}

This tutorial is intended for new users of \pgfname\ and \tikzname. It
does not give an exhaustive account of all the features of \tikzname\ or
\pgfname, just of those that you are likely to use right away.

Karl is a math and chemistry high-school teacher. He used to create
the graphics in his worksheets and exams using \LaTeX's |{picture}|
environment. While the results were acceptable, creating the graphics
often turned out to be a lengthy process. Also, there tended to be
problems with lines having slightly wrong angles and circles also
seemed to be hard to get right. Naturally, his students could not care
less whether the lines had the exact right angles and they find
Karl's exams too difficult no matter how nicely they were drawn. But
Karl was never entirely satisfied with the result.

Karl's son, who was even less satisfied with the results (he did not
have to take the exams, after all),  told Karl that he might wish
to try out a new package for creating graphics. A bit confusingly,
this package seems to have two names: First, Karl had to download and 
install a package called \pgfname. Then it turns out that inside this
package there is another package called \tikzname, which is supposed to
stand for ``\tikzname\ ist \emph{kein}  Zeichenprogramm.'' Karl finds this
all a bit strange and \tikzname\ seems to indicate that the package
does not do what he needs. However, having used \textsc{gnu}
software for quite some time and ``\textsc{gnu} not being Unix,''
there seems to be hope yet. His son assures him that \tikzname's name is
intended to warn people that \tikzname\ is not a program that you can
use to draw graphics with your mouse or tablet. Rather, it is more
like a ``graphics language.''


\subsection{Problem Statement}

Karl wants to put a graphic on the next worksheet for his
students. He is currently teaching his students about sine and
cosine. What he would like to have is something that looks like this
(ideally):

\noindent
\begin{tikzpicture}
  [scale=3,line cap=round,
   % Styles
   axes/.style=,
   important line/.style={very thick},
   information text/.style={rounded corners,fill=red!10,inner sep=1ex}]

  % Local definitions
  \def\costhirty{0.8660256}

  % Colors
  \colorlet{anglecolor}{green!50!black}
  \colorlet{sincolor}{red}
  \colorlet{tancolor}{orange!80!black}
  \colorlet{coscolor}{blue}

  % The graphic
  \draw[help lines,step=0.5cm] (-1.4,-1.4) grid (1.4,1.4);
  
  \draw (0,0) circle (1cm);

  \begin{scope}[axes]
    \draw[->] (-1.5,0) -- (1.5,0) node[right] {$x$};
    \draw[->] (0,-1.5) -- (0,1.5) node[above] {$y$};

    \foreach \x/\xtext in {-1, -.5/-\frac{1}{2}, 1}
      \draw[xshift=\x cm] (0pt,1pt) -- (0pt,-1pt) node[below,fill=white] {$\xtext$};
  
    \foreach \y/\ytext in {-1, -.5/-\frac{1}{2}, .5/\frac{1}{2}, 1}
      \draw[yshift=\y cm] (1pt,0pt) -- (-1pt,0pt) node[left,fill=white] {$\ytext$};
  \end{scope}
    
  \filldraw[fill=green!20,draw=anglecolor] (0,0) -- (3mm,0pt) arc(0:30:3mm);
  \draw (15:2mm) node[anglecolor] {$\alpha$};
    
  \draw[important line,sincolor]
    (30:1cm) -- node[left=1pt,fill=white] {$\sin \alpha$} +(0,-.5);
  
  \draw[important line,coscolor]
    (0,0) -- node[below=2pt,fill=white] {$\cos \alpha$} (\costhirty,0);
  
  \draw[important line,tancolor] (1,0) --
    node [right=1pt,fill=white]
    {
      $\displaystyle \tan \alpha \color{black}=
      \frac{{\color{sincolor}\sin \alpha}}{\color{coscolor}\cos \alpha}$
    } (intersection of 0,0--30:1cm and 1,0--1,1) coordinate (t);

  \draw (0,0) -- (t);
  
  \draw[xshift=1.85cm] node [right,text width=6cm,information text]
    {
      The {\color{anglecolor} angle $\alpha$} is $30^\circ$ in the
      example ($\pi/6$ in radians). The {\color{sincolor}sine of
        $\alpha$}, which is the height of the red line, is
      \[
      {\color{sincolor} \sin \alpha} = 1/2.
      \]
      By the Theorem of Pythagoras we have ${\color{coscolor}\cos^2 \alpha} +
      {\color{sincolor}\sin^2\alpha} =1$. Thus the length of the blue
      line, which is the {\color{coscolor}cosine of $\alpha$}, must be
      \[
      {\color{coscolor}\cos\alpha} = \sqrt{1 - 1/4} = \textstyle
      \frac{1}{2} \sqrt 3. 
      \]%
      This shows that {\color{tancolor}$\tan \alpha$}, which is the
      height of the orange line, is  
      \[
      {\color{tancolor}\tan\alpha} = \frac{{\color{sincolor}\sin
          \alpha}}{\color{coscolor}\cos \alpha} = 1/\sqrt 3.
      \]%
    };
\end{tikzpicture}


\subsection{Setting up the Environment}

In \tikzname, to draw a picture, at the start of the picture
you need to tell \TeX\ or \LaTeX\ that you want to start a picture. In
\LaTeX\ this is done using the environment |{tikzpicture}|, in plain
\TeX\ you just use |\tikzpicture| to start the picture and
|\endtikzpicture| to end it.

\subsubsection{Setting up the Environment in \LaTeX}

Karl, being a \LaTeX\ user, thus sets up his file as follows:

\begin{codeexample}[code only]
\documentclass{article} % say
\usepackage{tikz}
\begin{document}
We are working on
\begin{tikzpicture}
  \draw (-1.5,0) -- (1.5,0);
  \draw (0,-1.5) -- (0,1.5);
\end{tikzpicture}.
\end{document}
\end{codeexample}

When executed, that is, run via |pdflatex| or via |latex| followed by
|dvips|, the resulting will contain something that looks like this:

\begin{codeexample}[width=7cm]
We are working on
\begin{tikzpicture}
  \draw (-1.5,0) -- (1.5,0);
  \draw (0,-1.5) -- (0,1.5);
\end{tikzpicture}.
\end{codeexample}

Admittedly, not quite the whole picture, yet, but we
do have the axes established. Well, not quite, but we have the lines
that make up the axes drawn. Karl suddenly has a sinking feeling
that the picture is still some way off. 

Let's have a more detailed look at the code. First, the package
|tikz| is loaded. This package is a so-called ``frontend'' to the
basic \pgfname\ system. The basic layer, which is also described in this
manual, is somewhat more, well, basic and thus harder to use. The
frontend makes things easier by providing a simpler syntax.

Inside the environment there are two |\draw| commands. They mean:
``The path, which is specified following the command up to the
semicolon, should be drawn.'' The first path is specified
as |(-1.5,0) -- (0,1.5)|, which means ``a straight line from the point
at position $(-1.5,0)$ to the point at position $(0,1.5)$.'' Here, the
positions are specified within a special coordinate system in which,
initially, one unit is 1cm.

Karl is quite pleased to note that the environment automatically
reserves enough space to encompass the picture.


\subsubsection{Setting up the Environment in Plain \TeX}

Karl's wife Gerda, who also happens to be a math teacher, is not a
\LaTeX\ user, but uses plain \TeX\ since she prefers to do things
``the old way.'' She can also use \tikzname. Instead of
|\usepackage{tikz}| she has to write |\input tikz.tex| and instead of
|\begin{tikzpicture}| she writes |\tikzpicture| and  instead of
  |\end{tikzpicture}| she writes |\endtikzpicture|. 

Thus, she would use:
\begin{codeexample}[code only]
%% Plain TeX file
\input tikz.tex
\baselineskip=12pt
\hsize=6.3truein
\vsize=8.7truein
We are working on
\tikzpicture
  \draw (-1.5,0) -- (1.5,0);
  \draw (0,-1.5) -- (0,1.5);
\endtikzpicture.
\bye
\end{codeexample}

Gerda can typeset this file using either |pdftex| or |tex| together
with |dvips|. \tikzname\ will automatically discern which driver she is
using. If she wishes to use |dvipdfm| together with |tex|, she 
either needs to modify the file |pgf.cfg| or can write
|\def\pgfsysdriver{pgfsys-dvipdfm.def}| somewhere \emph{before} she
inputs |tikz.tex| or |pgf.tex|.



\subsubsection{Setting up the Environment in Con\TeX t}

Karl's uncle Hans uses Con\TeX t. Like Gerda, Hans can also use
\tikzname. Instead of |\usepackage{tikz}| he says
|\usemodule[tikz]|. Instead of |\begin{tikzpicture}| he writes
  |\starttikzpicture| and  instead of |\end{tikzpicture}| he writes
|\stoptikzpicture|.  

His version of the example looks like this:
\begin{codeexample}[code only]
%% ConTeXt file
\usemodule[tikz]

\starttext
  We are working on
  \starttikzpicture
    \draw (-1.5,0) -- (1.5,0);
    \draw (0,-1.5) -- (0,1.5);
  \stoptikzpicture.
\stoptext  
\end{codeexample}

Hans will now typeset this file in the usual way using |texexec|. 



\subsection{Straight Path Construction}

The basic building block of all pictures in \tikzname\ is the path. 
A \emph{path} is a series of straight lines and curves that are
connected (that is not the whole picture, but let us ignore the
complications for the moment). You start a path by specifying the
coordinates of the start position as a point in round brackets, as in
|(0,0)|. This is followed by a series of ``path extension
operations.'' The simplest is |--|, which we used already. It must be
followed by another coordinate and it extends the path in a straight
line to this new position. For example, if we were to turn the two
paths of the axes into one path, the following would result:

\begin{codeexample}[]
\tikz \draw (-1.5,0) -- (1.5,0) -- (0,-1.5) -- (0,1.5);
\end{codeexample}

Karl is a bit confused by the fact that there is no |{tikzpicture}|
environment, here. Instead, the little command |\tikz| is used. This
command either takes one argument (starting with an opening brace as in
|\tikz{\draw (0,0) -- (1.5,0)}|, which yields \tikz{\draw (0,0)
 --(1.5,0);}) or collects everything up to the next semicolon and
puts it inside a |{tikzpicture}| environment. As a rule of thumb, all
\tikzname\ graphic drawing commands must occur as an argument of |\tikz|
or inside a |{tikzpicture}| environment. Fortunately, the command
|\draw| will only be defined inside this environment, so there is
little chance that you will accidentally do something wrong here. 



\subsection{Curved Path Construction}

The next thing Karl wants to do is to draw the circle. For this,
straight lines obviously will not do. Instead, we need some way to
draw curves. For this, \tikzname\ provides a special syntax. One or two
``control points'' are needed. The math behind them is not quite
trivial, but here is the basic idea: Suppose you are at point $x$ and
the first control point is $y$. Then the curve will start ``going in
the direction of~$y$ at~$x$,'' that is, the tangent of the curve at $x$
will point toward~$y$. Next, suppose the curve should end at $z$ and
the second support point is $w$. Then the curve will, indeed, end at
$z$ and the tangent of the curve at point $z$ will go through $w$.

Here is an example (the control points have been added for clarity):
\begin{codeexample}[]
\begin{tikzpicture}
  \filldraw [gray] (0,0) circle (2pt)
                   (1,1) circle (2pt)
                   (2,1) circle (2pt)
                   (2,0) circle (2pt);
  \draw (0,0) .. controls (1,1) and (2,1) .. (2,0);
\end{tikzpicture}
\end{codeexample}

The general syntax for extending a path in a ``curved'' way is
|.. controls| \meta{first control point} |and| \meta{second control
  point} |..| \meta{end point}. You can leave out the |and|
\meta{second control point}, which causes the first one to be used 
twice.

So, Karl can now add the first half circle to the picture:

\begin{codeexample}[]
\begin{tikzpicture}
  \draw (-1.5,0) -- (1.5,0);
  \draw (0,-1.5) -- (0,1.5);
  \draw (-1,0) .. controls (-1,0.555) and (-0.555,1) .. (0,1)
               .. controls (0.555,1) and (1,0.555) .. (1,0);
\end{tikzpicture}
\end{codeexample}

Karl is happy with the result, but finds specifying circles in this
way to be extremely awkward. Fortunately, there is a much simpler way.


\subsection{Circle Path Construction}

In order to draw a circle, the path construction operation |circle| can
be used. This operation is followed by a radius in round brackets as in
the following example: (Note that the previous position is used as the
\emph{center} of the circle.)

\begin{codeexample}[]
\tikz \draw (0,0) circle (10pt);
\end{codeexample}

You can also append an ellipse to the path using the |ellipse|
operation. Instead of a single radius you can specify two of them, one
for the $x$-direction and one for the $y$-direction, separated by
|and|: 

\begin{codeexample}[]
\tikz \draw (0,0) ellipse (20pt and 10pt);
\end{codeexample}

To draw an ellipse whose axes are not horizontal and vertical, but
point in an arbitrary direction (a ``turned ellipse'' like \tikz
\draw[rotate=30] (0,0) ellipse (6pt and 3pt);) you can use
transformations, which are explained later. The code for the little
ellipse is |\tikz \draw[rotate=30] (0,0) ellipse (6pt and 3pt);|, by
the way. 

So, returning to Karl's problem, he can write
|\draw (0,0) circle (1cm);| to draw the circle:

\begin{codeexample}[]
\begin{tikzpicture}
  \draw (-1.5,0) -- (1.5,0);
  \draw (0,-1.5) -- (0,1.5);
  \draw (0,0) circle (1cm);
\end{tikzpicture}
\end{codeexample}


At this point, Karl is a bit alarmed that the circle is so small when
he wants the final picture to be much bigger. He is pleased to learn
that \tikzname\ has powerful transformation options and scaling
everything by a factor of three is very easy. But let us leave the
size as it is for the moment to save some space. 




\subsection{Rectangle Path Construction}

The next things we would like to have is the grid in the background.
There are several ways to produce it. For example, one might draw lots of
rectangles. Since rectangles are so common, there is a special syntax
for them: To add a rectangle to the current path, use the |rectangle|
path construction operation. This operation should be followed by another
coordinate and will append a rectangle to the path such that the
previous coordinate and the next coordinates are corners of the
rectangle. So, let us add two rectangles to the picture:

\begin{codeexample}[]
\begin{tikzpicture}
  \draw (-1.5,0) -- (1.5,0);
  \draw (0,-1.5) -- (0,1.5);
  \draw (0,0) circle (1cm);
  \draw (0,0) rectangle (0.5,0.5);
  \draw (-0.5,-0.5) rectangle (-1,-1);
\end{tikzpicture}
\end{codeexample}

While this may be nice in other situations, this is not really leading
anywhere with Karl's problem: First, we would need an awful lot of
these rectangles and then there is the border that is not ``closed.''

So, Karl is about to resort to simply drawing four vertical and four
horizontal lines using the nice |\draw| command, when he learns that
there is a |grid| path construction operation.



\subsection{Grid Path Construction}

The |grid| path operation adds a grid to the current path. It will add
lines making up a grid that fills the rectangle whose one corner is
the current point and whose other corner is the point following the
|grid| operation. For example, the code
|\tikz \draw[step=2pt] (0,0) grid (10pt,10pt);| produces \tikz
\draw[step=2pt] (0,0) grid (10pt,10pt);. Note how the optional
argument for |\draw| can be used to specify a grid width (there are
also |xstep| and |ystep| to define the steppings independently). As
Karl will learn soon, there are \emph{lots} of things that can be
influenced using such options.

For Karl, the following code could be used:

\begin{codeexample}[]
\begin{tikzpicture}
  \draw (-1.5,0) -- (1.5,0);
  \draw (0,-1.5) -- (0,1.5);
  \draw (0,0) circle (1cm);
  \draw[step=.5cm] (-1.4,-1.4) grid (1.4,1.4);
\end{tikzpicture}
\end{codeexample}

Having another look at the desired picture, Karl notices that it would
be nice for the grid to be more subdued. (His son told him that grids
tend to be distracting if they are not subdued.) To subdue the grid,
Karl adds two more options to the |\draw| command that draws the
grid. First, he uses the color |gray| for the grid lines. Second, he
reduces the line width to |very thin|. Finally, he swaps the ordering
of the commands so that the grid is drawn first and everything else on
top. 

\begin{codeexample}[]
\begin{tikzpicture}
  \draw[step=.5cm,gray,very thin] (-1.4,-1.4) grid (1.4,1.4);
  \draw (-1.5,0) -- (1.5,0);
  \draw (0,-1.5) -- (0,1.5);
  \draw (0,0) circle (1cm);
\end{tikzpicture}
\end{codeexample}


\subsection{Adding a Touch of  Style}

Instead of the options |gray,very thin| Karl could also have
said |help lines|. \emph{Styles} are predefined sets of options
that can be used to organize how a graphic is drawn. By saying
|help lines| you say ``use the style that I (or someone else)
has set for drawing help lines.'' If Karl decides, at some later
point, that grids should be drawn, say, using the color |blue!50|
instead of |gray|, he could provide the following option somewhere:
\begin{codeexample}[code only]
help lines/.style={color=blue!50,very thin}
\end{codeexample}
The effect of this ``style setter'' is that in the current
scope or environement the |help lines| option has the same effect as
|color=blue!50,very thin|.

Using styles makes your graphics code more flexible. You can
change the way things look easily in a consistent manner. 
Normally, styles are defined at the beginning of a picture. However,
you may sometimes wish to define a style globally, so that all
pictures of your document can use this style. Then you can easily
change the way all graphics look by changing this one style. In this
situation you can use the |\tikzset| command at the beginning of the
document as in  
\begin{codeexample}[code only]
\tikzset{help lines/.style=very thin}
\end{codeexample}

To build a hierarchy of styles you can have one style use
another. So in order to define a style |Karl's grid| that is based on
the |grid| style Karl could say
\begin{codeexample}[code only]
\tikzset{Karl's grid/.style={help lines,color=blue!50}}
...
\draw[Karl's grid] (0,0) grid (5,5);
\end{codeexample}

Styles are made even more powerful by parametrization. This means
that, like other options, styles can also be used with a
parameter. For instance, Karl could parametrize his grid so that, by
default, it is blue, but he could also use another color. 

\begin{codeexample}[code only]
\begin{tikzpicture}
  [Karl's grid/.style  ={help lines,color=#1!50},
   Karl's grid/.default=blue]

  \draw[Karl's grid]     (0,0) grid (1.5,2);
  \draw[Karl's grid=red] (2,0) grid (3.5,2);
\end{tikzpicture}
\end{codeexample}


\subsection{Drawing Options}

Karl wonders what other options there are that influence how a path is
drawn. He saw already that the |color=|\meta{color} option can be used
to set the line's color. The option |draw=|\meta{color} does nearly
the same, only it sets the color for the lines only and a different
color can be used for filling (Karl will need this when he fills the
arc for the angle).

He saw that the style |very thin| yields very thin lines. Karl is not
really surprised by this and neither is he surprised to learn that |thin|
yields thin lines,  |thick| yields thick lines, |very thick| yields
very thick lines, |ultra thick| yields really, really thick lines and
|ultra thin| yields lines that are so thin that low-resolution printers
and displays will have trouble showing them. He wonders what gives
lines of ``normal'' thickness. It turns out that |thin| is the correct
choice. This seems strange to Karl, but his son explains him that
\LaTeX\ has two commands called |\thinlines| and |\thicklines| and
that |\thinlines| gives the line width of ``normal'' lines, more
precisely, of the thickness that, say, the stem of a letter like ``T''
or ``i'' has. Nevertheless, Karl would like to know whether there is
anything ``in the middle'' between |thin| and |thick|. There is:
|semithick|.

Another useful thing one can do with lines is to dash or dot them. For
this, the two styles |dashed| and |dotted| can be used, yielding
\tikz \draw[dashed] (0,0) -- (10pt,0pt); and \tikz \draw[dotted] (0,0)
-- (10pt,0pt);. Both options also exist in a loose and a dense
version, called |loosely dashed|, |densely dashed|, |loosely dotted|,
and |densely dotted|. If he really, really  needs to, Karl can also
define much more complex dashing patterns with the |dash pattern|
option, but his son insists that dashing is to be used with utmost
care and mostly distracts. Karl's son claims that complicated dashing
patterns are evil. Karl's students do not care about dashing patterns. 



\subsection{Arc Path Construction}

Our next obstacle is to draw the arc for the angle. For this, the
|arc| path construction operation is useful, which draws part of a
circle or ellipse. This |arc| operation must be followed by a triple in 
rounded brackets, where the components of the triple are separated by
colons. The first two components are angles, the last one is a
radius. An example would be |(10:80:10pt)|, which means ``an arc from
10 degrees to 80 degrees on a circle of radius 10pt.'' Karl obviously
needs an arc from $0^\circ$ to $30^\circ$. The radius should be
something relatively small, perhaps around one third of the circle's
radius. This gives: |(0:30:3mm)|.

When one uses the arc path construction operation, the specified arc will
be added with its starting point at the current position. So, we first
have to ``get there.'' 

\begin{codeexample}[]
\begin{tikzpicture}
  \draw[step=.5cm,gray,very thin] (-1.4,-1.4) grid (1.4,1.4);
  \draw (-1.5,0) -- (1.5,0);
  \draw (0,-1.5) -- (0,1.5);
  \draw (0,0) circle (1cm);
  \draw (3mm,0mm) arc (0:30:3mm);
\end{tikzpicture}
\end{codeexample}

Karl thinks this is really a bit small and he cannot continue unless
he learns how to do scaling. For this, he can add the |[scale=3]|
option. He could add this option to each |\draw| command, but that
would be awkward. Instead, he adds it to the whole environment, which
causes this option to apply to everything within.

\begin{codeexample}[]
\begin{tikzpicture}[scale=3]
  \draw[step=.5cm,gray,very thin] (-1.4,-1.4) grid (1.4,1.4);
  \draw (-1.5,0) -- (1.5,0);
  \draw (0,-1.5) -- (0,1.5);
  \draw (0,0) circle (1cm);
  \draw (3mm,0mm) arc (0:30:3mm);
\end{tikzpicture}
\end{codeexample}

As for circles, you can specify ``two'' radii in order to get an
elliptical arc.

\begin{codeexample}[]
  \tikz \draw (0,0) arc (0:315:1.75cm and 1cm);
\end{codeexample}


\subsection{Clipping a Path}

In order to save space in this manual, it would be nice to clip Karl's
graphics a bit so that we can focus on the ``interesting''
parts. Clipping is pretty easy in \tikzname. You can use the |\clip|
command clip all subsequent drawing. It works like |\draw|, only it
does not draw anything, but uses the given path to clip everything
subsequently. 

\begin{codeexample}[]
\begin{tikzpicture}[scale=3]
  \clip (-0.1,-0.2) rectangle (1.1,0.75);
  \draw[step=.5cm,gray,very thin] (-1.4,-1.4) grid (1.4,1.4);
  \draw (-1.5,0) -- (1.5,0);
  \draw (0,-1.5) -- (0,1.5);
  \draw (0,0) circle (1cm);
  \draw (3mm,0mm) arc (0:30:3mm);
\end{tikzpicture}
\end{codeexample}

You can also do both at the same time: Draw \emph{and} clip a
path. For this, use the |\draw| command and add the |clip|
option. (This is not the whole picture: You can also use the |\clip|
command and add the |draw| option. Well, that is also not the whole
picture: In reality, |\draw| is just a shorthand for |\path[draw]|
and |\clip| is a shorthand for |\path[clip]| and you could also say
|\path[draw,clip]|.) Here is an example: 

\begin{codeexample}[]
\begin{tikzpicture}[scale=3]
  \clip[draw] (0.5,0.5) circle (.6cm);
  \draw[step=.5cm,gray,very thin] (-1.4,-1.4) grid (1.4,1.4);
  \draw (-1.5,0) -- (1.5,0);
  \draw (0,-1.5) -- (0,1.5);
  \draw (0,0) circle (1cm);
  \draw (3mm,0mm) arc (0:30:3mm);
\end{tikzpicture}
\end{codeexample}


\subsection{Parabola and Sine Path Construction}

Although Karl does not need them for his picture, he is pleased to
learn that there are |parabola| and |sin| and |cos| path operations for
adding parabolas and sine and cosine curves to the current path. For the
|parabola| operation, the current point will lie on the parabola as
well as the point given after the parabola operation. Consider
the following example:

\begin{codeexample}[]
\tikz \draw (0,0) rectangle (1,1)  (0,0) parabola (1,1);
\end{codeexample}

It is also possible to place the bend somewhere else:

\begin{codeexample}[]
\tikz \draw[x=1pt,y=1pt] (0,0) parabola bend (4,16) (6,12);
\end{codeexample}

The operations |sin| and |cos| add a sine or cosine curve in the interval
$[0,\pi/2]$ such that the previous current point is at the start of
the curve and the curve ends at the given end point. Here are two
examples:
\begin{codeexample}[]
A sine \tikz \draw[x=1ex,y=1ex] (0,0) sin (1.57,1); curve.
\end{codeexample}

\begin{codeexample}[]
\tikz \draw[x=1.57ex,y=1ex] (0,0) sin (1,1) cos (2,0) sin (3,-1) cos (4,0)
                            (0,1) cos (1,0) sin (2,-1) cos (3,0) sin (4,1);
\end{codeexample}



\subsection{Filling and Drawing}

Returning to the picture, Karl now wants the angle to be ``filled''
with a very light green. For this he uses |\fill| instead of
|\draw|. Here is what Karl does:

\begin{codeexample}[]
\begin{tikzpicture}[scale=3]
  \clip (-0.1,-0.2) rectangle (1.1,0.75);
  \draw[step=.5cm,gray,very thin] (-1.4,-1.4) grid (1.4,1.4);
  \draw (-1.5,0) -- (1.5,0);
  \draw (0,-1.5) -- (0,1.5);
  \draw (0,0) circle (1cm);
  \fill[green!20!white] (0,0) -- (3mm,0mm) arc (0:30:3mm) -- (0,0);
\end{tikzpicture}
\end{codeexample}

The color |green!20!white| means 20\% green and 80\% white mixed
together. Such color expression are possible since \pgfname\ uses Uwe
Kern's |xcolor| package, see the documentation of that package for
details on color expressions.

What would have happened, if Karl had not ``closed'' the path using
|--(0,0)| at the end? In this case, the path is closed automatically,
so this could have been omitted. Indeed, it would even have been
better to write the following, instead:
\begin{codeexample}[code only]
  \fill[green!20!white] (0,0) -- (3mm,0mm) arc (0:30:3mm) -- cycle;
\end{codeexample}
The |--cycle| causes the current path to be closed (actually the
current part of the current path) by smoothly joining the first and
last point. To appreciate the difference, consider the following
example:

\begin{codeexample}[]
\begin{tikzpicture}[line width=5pt]
  \draw (0,0) -- (1,0) -- (1,1) -- (0,0);
  \draw (2,0) -- (3,0) -- (3,1) -- cycle;
  \useasboundingbox (0,1.5); % make bounding box higher
\end{tikzpicture}
\end{codeexample}

You can also fill and draw a path at the same time using the
|\filldraw| command. This will first draw the path, then fill it. This
may not seem too useful, but you can specify different colors to be
used for filling and for stroking. These are specified as optional
arguments like this:

\begin{codeexample}[]
\begin{tikzpicture}[scale=3]
  \clip (-0.1,-0.2) rectangle (1.1,0.75);
  \draw[step=.5cm,gray,very thin] (-1.4,-1.4) grid (1.4,1.4);
  \draw (-1.5,0) -- (1.5,0);
  \draw (0,-1.5) -- (0,1.5);
  \draw (0,0) circle (1cm);
  \filldraw[fill=green!20!white, draw=green!50!black]
    (0,0) -- (3mm,0mm) arc (0:30:3mm) -- cycle;
\end{tikzpicture}
\end{codeexample}



\subsection{Shading}

Karl briefly considers the possibility of making the angle ``more
fancy'' by \emph{shading} it. Instead of filling the with a uniform
color, a smooth transition between different colors is used. For this,
|\shade| and |\shadedraw|, for shading and drawing at the same time,
can be used: 

\begin{codeexample}[]
  \tikz \shade (0,0) rectangle (2,1)  (3,0.5) circle (.5cm);
\end{codeexample}
The default shading is a smooth transition from gray to white. To
specify different colors, you can use options:

\begin{codeexample}[]
\begin{tikzpicture}[rounded corners,ultra thick]
  \shade[top color=yellow,bottom color=black] (0,0) rectangle +(2,1);
  \shade[left color=yellow,right color=black] (3,0) rectangle +(2,1);
  \shadedraw[inner color=yellow,outer color=black,draw=yellow] (6,0) rectangle +(2,1);
  \shade[ball color=green] (9,.5) circle (.5cm);
\end{tikzpicture}
\end{codeexample}

For Karl, the following might be appropriate:

\begin{codeexample}[]
\begin{tikzpicture}[scale=3]
  \clip (-0.1,-0.2) rectangle (1.1,0.75);
  \draw[step=.5cm,gray,very thin] (-1.4,-1.4) grid (1.4,1.4);
  \draw (-1.5,0) -- (1.5,0);
  \draw (0,-1.5) -- (0,1.5);
  \draw (0,0) circle (1cm);
  \shadedraw[left color=gray,right color=green, draw=green!50!black]
    (0,0) -- (3mm,0mm) arc (0:30:3mm) -- cycle;
\end{tikzpicture}
\end{codeexample}

However, he wisely decides that shadings usually only distract without
adding anything to the picture.


\subsection{Specifying Coordinates}

Karl now wants to add the sine and cosine lines. He knows already that
he can use the |color=| option to set the lines's colors. So, what is
the best way to specify the coordinates?

There are different ways of specifying coordinates. The easiest way is
to say something like |(10pt,2cm)|. This means 10pt in $x$-direction
and 2cm in $y$-directions. Alternatively, you can also leave out the
units as in |(1,2)|, which means ``one times the current $x$-vector
plus twice the current $y$-vector.'' These vectors default to 1cm in
the $x$-direction and 1cm in the $y$-direction, respectively.

In order to specify points in polar coordinates, use the notation
|(30:1cm)|, which means 1cm in direction 30 degree. This is obviously
quite useful to ``get to the point $(\cos 30^\circ,\sin 30^\circ)$ on
the circle.'' 

You can add a single |+| sign in front of a coordinate or two of
them as in |+(1cm,0cm)| or |++(0cm,2cm)|. Such coordinates are interpreted
differently: The first form means ``1cm upwards from the previous
specified position'' and the second means ``2cm to the right of the
previous specified position, making this the new specified position.''
For example, we can draw the sine line as follows:

\begin{codeexample}[]
\begin{tikzpicture}[scale=3]
  \clip (-0.1,-0.2) rectangle (1.1,0.75);
  \draw[step=.5cm,gray,very thin] (-1.4,-1.4) grid (1.4,1.4);
  \draw (-1.5,0) -- (1.5,0);
  \draw (0,-1.5) -- (0,1.5);
  \draw (0,0) circle (1cm);
  \filldraw[fill=green!20,draw=green!50!black]
    (0,0) -- (3mm,0mm) arc (0:30:3mm) -- cycle;
  \draw[red,very thick] (30:1cm) -- +(0,-0.5);
\end{tikzpicture}
\end{codeexample}

Karl used the fact $\sin 30^\circ = 1/2$. However, he very much
doubts that his students know this, so it would be nice to have a way
of specifying ``the point straight down from |(30:1cm)| that lies on
the $x$-axis.'' This is, indeed, possible using a special syntax: Karl
can write \verb!(30:1cm |- 0,0)!. In general, the meaning of
|(|\meta{p}\verb! |- !\meta{q}|)| is ``the intersection of a vertical
line through $p$ and a horizontal line through $q$.''

Next, let us draw the cosine line. One way would be to say
\verb!(30:1cm |- 0,0) -- (0,0)!. Another way is the following: we
``continue'' from where the sine ends: 

\begin{codeexample}[]
\begin{tikzpicture}[scale=3]
  \clip (-0.1,-0.2) rectangle (1.1,0.75);
  \draw[step=.5cm,gray,very thin] (-1.4,-1.4) grid (1.4,1.4);
  \draw (-1.5,0) -- (1.5,0);
  \draw (0,-1.5) -- (0,1.5);
  \draw (0,0) circle (1cm);
  \filldraw[fill=green!20,draw=green!50!black] (0,0) -- (3mm,0mm) arc
  (0:30:3mm) -- cycle;
  \draw[red,very thick]  (30:1cm) -- +(0,-0.5);
  \draw[blue,very thick] (30:1cm) ++(0,-0.5) -- (0,0);
\end{tikzpicture}
\end{codeexample}

Note the there is no |--| between |(30:1cm)| and |+(0,-0.5)|. In
detail, this path is interpreted as follows: ``First, the |(30:1cm)|
tells me to move by pen to $(\cos 30^\circ,1/2)$. Next, there comes
another coordinate specification, so I move my pen there without drawing
anything. This new point is half a unit down from the last position,
thus it is at $(\cos 30^\circ,0)$. Finally, I move the pen to the
origin, but this time drawing something (because of the |--|).''

To appreciate the difference between |+| and |++| consider the
following example:

\begin{codeexample}[]
\begin{tikzpicture}
  \def\rectanglepath{-- ++(1cm,0cm)  -- ++(0cm,1cm)  -- ++(-1cm,0cm) -- cycle}
  \draw (0,0) \rectanglepath;
  \draw (1.5,0) \rectanglepath;
\end{tikzpicture}
\end{codeexample}

By comparison, when using a single |+|, the coordinates are different:

\begin{codeexample}[]
\begin{tikzpicture}
  \def\rectanglepath{-- +(1cm,0cm)  -- +(1cm,1cm)  -- +(0cm,1cm) -- cycle}
  \draw (0,0) \rectanglepath;
  \draw (1.5,0) \rectanglepath;
\end{tikzpicture}
\end{codeexample}


Naturally, all of this could have been written more clearly and more
economically like this (either with a single of a double |+|): 
\begin{codeexample}[]
\tikz \draw (0,0) rectangle +(1,1)  (1.5,0) rectangle +(1,1);
\end{codeexample}



Karl is left with the line for $\tan \alpha$, which seems difficult to
specify using transformations and polar coordinates. For this he needs
another way of specifying coordinates: Karl can specify intersections
of lines as coordinates. The line for $\tan \alpha$ starts at $(1,0)$
and goes upward to a point that is at the intersection of a line going
``up'' and a line going from the origin through |(30:1cm)|. The syntax
for this point is the following:

\begin{codeexample}[code only]
\draw[very thick,orange] (1,0) -- (intersection of 1,0--1,1 and 0,0--30:1cm);
\end{codeexample}

In the following, two final examples of how to use relative
positioning are presented. Note that the transformation options,
which are explained later, are often more useful for shifting than
relative positioning. 

\begin{codeexample}[]
\begin{tikzpicture}[scale=0.5]
  \draw (0,0) -- (90:1cm) arc (90:360:1cm) arc (0:30:1cm) -- cycle;
  \draw (60:5pt) -- +(30:1cm) arc (30:90:1cm) -- cycle;

  \draw (3,0)  +(0:1cm) -- +(72:1cm) -- +(144:1cm) -- +(216:1cm) --
               +(288:1cm) -- cycle;
\end{tikzpicture}
\end{codeexample}



\subsection{Adding Arrow Tips}

Karl now wants to add the little arrow tips at the end of the axes. He has
noticed that in many plots, even in scientific journals, these arrow tips
seem to missing, presumably because the generating programs cannot
produce them. Karl thinks arrow tips belong at the end of axes. His
son agrees. His students do not care about arrow tips.

It turns out that adding arrow tips is pretty easy: Karl adds the option
|->| to the drawing commands for the axes:

\begin{codeexample}[]
\begin{tikzpicture}[scale=3]
  \clip (-0.1,-0.2) rectangle (1.1,1.51);
  \draw[step=.5cm,gray,very thin] (-1.4,-1.4) grid (1.4,1.4);
  \draw[->] (-1.5,0) -- (1.5,0);
  \draw[->] (0,-1.5) -- (0,1.5);
  \draw (0,0) circle (1cm);
  \filldraw[fill=green!20,draw=green!50!black] (0,0) -- (3mm,0mm) arc
  (0:30:3mm) -- cycle;
  \draw[red,very thick]    (30:1cm) -- +(0,-0.5);
  \draw[blue,very thick]   (30:1cm) ++(0,-0.5) -- (0,0);
  \draw[orange,very thick] (1,0) -- (intersection of 1,0--1,1 and 0,0--30:1cm);
\end{tikzpicture}
\end{codeexample}

If Karl had used the option |<-| instead of |->|, arrow tips would
have been put at the beginning of the path. The option |<->| puts
arrow tips at both ends of the path.

There are certain restrictions to the kind of paths to which arrow tips
can be added. As a rule of thumb, you can add arrow tips only to a
single open ``line.'' For example, you should not try to add tips to,
say, a rectangle or a circle. (You can try, but no guarantees as to what
will happen now or in future versions.) However, you can add arrow
tips to curved paths and to paths that have several segments, as in
the following examples:

\begin{codeexample}[]
\begin{tikzpicture}
  \draw [<->] (0,0) arc (180:30:10pt);
  \draw [<->] (1,0) -- (1.5cm,10pt) -- (2cm,0pt) -- (2.5cm,10pt);
\end{tikzpicture}
\end{codeexample}

Karl has a more detailed look at the arrow that \tikzname\ puts at the
end. It looks like this when he zooms it: \tikz { \useasboundingbox
  (0pt,-.5ex) rectangle (10pt,.5ex); \draw[->,line width=1pt] (0pt,0pt) --
  (10pt,0pt); }. The shape seems vaguely familiar and, indeed, this is
exactly the end of \TeX's standard arrow used in something like
$f\colon A \to B$.


Karl likes the arrow, especially since it is not ``as thick'' as the
arrows offered by many other packages. However, he expects that,
sometimes, he might need to use some other kinds of arrow.
To do so, Karl can say |>=|\meta{right arrow tip kind}, where
\meta{right arrow tip kind} is a special arrow tip specification. For
example, if Karl says |>=stealth|, then he tells \tikzname\
that he would like  ``stealth-fighter-like'' arrow tips: 

\begin{codeexample}[]
\begin{tikzpicture}[>=stealth]
  \draw [->] (0,0) arc (180:30:10pt);
  \draw [<<-,very thick] (1,0) -- (1.5cm,10pt) -- (2cm,0pt) -- (2.5cm,10pt);
\end{tikzpicture}
\end{codeexample}%>>

Karl wonders whether such a military name for the arrow type is really
necessary. He is not really mollified when his son tells him that
Microsoft's PowerPoint uses the same name. He decides to have his
students discuss this at some point.

In addition to |stealth| there are several other predefined arrow tip
kinds Karl can choose from, see
Section~\ref{section-library-arrows}. Furthermore, he can define
arrows types himself, if he needs new ones. 




\subsection{Scoping}

Karl saw already that there are numerous graphic options that affect how
paths are rendered. Often, he would like to apply certain options to
a whole set of graphic commands. For example, Karl might wish to draw
three paths using a |thick| pen, but would like everything else to
be drawn ``normally.''

If Karl wishes to set a certain graphic option for the whole picture,
he can simply pass this option to the |\tikz| command or to the
|{tikzpicture}| environment (Gerda would pass the options to
|\tikzpicture| and Hans passes them to |\starttikzpicture|). However,
if Karl wants to apply graphic options to a local group, he put these
commands inside a |{scope}| environment (Gerda uses |\scope| and
|\endscope|, Hans uses |\startscope| and |\stopscope|). This
environment takes graphic options as an optional argument and these
options apply to everything inside the scope, but not to anything outside.

Here is an example:

\begin{codeexample}[]
\begin{tikzpicture}[ultra thick]
  \draw (0,0) -- (0,1);
  \begin{scope}[thin]
    \draw (1,0) -- (1,1);
    \draw (2,0) -- (2,1);
  \end{scope}
  \draw (3,0) -- (3,1);  
\end{tikzpicture}
\end{codeexample}

Scoping has another interesting effect: Any changes to the clipping
area are local to the scope. Thus, if you say |\clip| somewhere inside
a scope, the effect of the |\clip| command ends at the end of the
scope. This is useful since there is no other way of ``enlarging'' the
clipping area.

Karl has also already seen that giving options to commands like
|\draw| apply only to that command. In turns out that the situation is
slightly more complex. First, options to a command like |\draw| are
not really options to the command, but they are ``path options'' and
can be given anywhere on the path. So, instead of
|\draw[thin] (0,0) -- (1,0);| one can also write
|\draw (0,0) [thin] -- (1,0);| or |\draw (0,0) -- (1,0) [thin];|; all
of these have the same effect. This might seem strange since in the
last case, it would appear that the |thin| should take effect only
``after'' the line from $(0,0)$ to $(1,0)$ has been draw. However,
most graphic options only apply to the whole path. Indeed, if you say
both |thin| and |thick| on the same path, the last option given will
``win.''

When reading the above, Karl notices that only ``most'' graphic
options apply to the whole path. Indeed, all transformation options do
\emph{not} apply to the whole path, but only to ``everything following
them on the path.'' We will have a more detailed look at this in a
moment. Nevertheless, all options given during a path construction
apply only to this path. 



\subsection{Transformations}

When you specify a  coordinate like |(1cm,1cm)|, where is that
coordinate placed on the page? To determine the position, \tikzname,
\TeX, and \textsc{pdf} or PostScript all apply certain transformations
to the given coordinate in order to determine the finally position on
the page. 

\tikzname\ provides numerous options that allow you to transform
coordinates in \pgfname's private coordinate system. For example, the
|xshift| option allows you to shift all subsequent points by a certain
amount:

\begin{codeexample}[]
\tikz \draw (0,0) -- (0,0.5) [xshift=2pt] (0,0) -- (0,0.5);
\end{codeexample}

It is important to note that you can change transformation ``in the
middle of a path,'' a feature that is not supported by \pdf\
or PostScript. The reason is that \pgfname\ keeps track of its own
transformation matrix.

Here is a more complicated example:
\begin{codeexample}[]
\begin{tikzpicture}[even odd rule,rounded corners=2pt,x=10pt,y=10pt]
  \filldraw[fill=examplefill] (0,0)   rectangle (1,1)
    [xshift=5pt,yshift=5pt]   (0,0)   rectangle (1,1)
                [rotate=30]   (-1,-1) rectangle (2,2);
\end{tikzpicture}
\end{codeexample}

The most useful transformations are |xshift| and |yshift| for
shifting, |shift| for shifting to a given point as in |shift={(1,0)}|
or |shift={+(0,0)}| (the braces are necessary so that \TeX\ does not
mistake the comma for separating options), |rotate| for rotating by a
certain angle (there is also a |rotate around| for rotating around a
given point), |scale| for scaling by a certain factor, |xscale| and
|yscale| for scaling only in the $x$- or $y$-direction (|xscale=-1| is
a flip), and |xslant| and |yslant| for slanting. If these
transformation and those that I have not mentioned are not
sufficient,  the |cm| option allows you to apply an arbitrary
transformation matrix. Karl's students, by the way, do not know what a
transformation matrix is. 



\subsection{Repeating Things: For-Loops}

Karl's next aim is to add little ticks on the axes at positions $-1$,
$-1/2$, $1/2$, and $1$. For this, it would be nice to use some kind of
``loop,'' especially since he wishes to do the same thing at each of
these positions. There are different packages for doing this. \LaTeX\
has its own internal command for this, |pstricks| comes along with the
powerful |\mulitdo| command. All of these can be used together with
\pgfname\ and \tikzname, so if you are familiar with them, feel free to
use them. \pgfname\ introduces yet another command, called |\foreach|,
which I introduced since I could never remember the syntax of the other
packages. |\foreach| is defined in the package |pgffor| and can be used
independently of \pgfname. \tikzname\ includes it automatically.

In its basic form, the |\foreach| command is easy to use:
\begin{codeexample}[]
\foreach \x in {1,2,3} {$x =\x$, }
\end{codeexample}

The general syntax is |\foreach| \meta{variable}| in {|\meta{list of
    values}|} |\meta{commands}. Inside the \meta{commands}, the
\meta{variable} will be assigned to the different values. If the
\meta{commands} do not start with a brace, everything up to the
next semicolon is used as \meta{commands}.

For Karl and the ticks on the axes, he could use the following code:

\begin{codeexample}[]
\begin{tikzpicture}[scale=3]
  \clip (-0.1,-0.2) rectangle (1.1,1.51);
  \draw[step=.5cm,gray,very thin] (-1.4,-1.4) grid (1.4,1.4);
  \filldraw[fill=green!20,draw=green!50!black] (0,0) -- (3mm,0mm) arc
  (0:30:3mm) -- cycle;
  \draw[->] (-1.5,0) -- (1.5,0);
  \draw[->] (0,-1.5) -- (0,1.5);
  \draw (0,0) circle (1cm);

  \foreach \x in {-1cm,-0.5cm,1cm}
    \draw (\x,-1pt) -- (\x,1pt);
  \foreach \y in {-1cm,-0.5cm,0.5cm,1cm}
    \draw (-1pt,\y) -- (1pt,\y);
\end{tikzpicture}
\end{codeexample}

As a matter of fact, there are many different ways of creating the
ticks. For example, Karl could have put the |\draw ...;| inside curly
braces. He could also have used, say,
\begin{codeexample}[code only]
\foreach \x in {-1,-0.5,1}
  \draw[xshift=\x cm] (0pt,-1pt) -- (0pt,1pt);
\end{codeexample}

Karl is curious what would happen in a more complicated situation
where there are, say, 20 ticks. It seems bothersome to explicitly
mention all these numbers in the set for |\foreach|. Indeed, it is
possible to use |...| inside the |\foreach| statement to iterate over 
a large number of values (which must, however, be dimensionless
real numbers) as in the following example: 

\begin{codeexample}[]
\tikz \foreach \x in {1,...,10}
        \draw (\x,0) circle (0.4cm);
\end{codeexample}

If you provide \emph{two} numbers before the |...|, the |\foreach|
statement will use their difference for the stepping:

\begin{codeexample}[]
\tikz \foreach \x in {-1,-0.5,...,1}
       \draw (\x cm,-1pt) -- (\x cm,1pt);
\end{codeexample}

We can also nest loops to create interesting effects:

\begin{codeexample}[]
\begin{tikzpicture}
  \foreach \x in {1,2,...,5,7,8,...,12}
    \foreach \y in {1,...,5}
    {
      \draw (\x,\y) +(-.5,-.5) rectangle ++(.5,.5);
      \draw (\x,\y) node{\x,\y};
    }
\end{tikzpicture}
\end{codeexample}

The |\foreach| statement can do even trickier stuff, but the above
gives the idea.




\subsection{Adding Text}

Karl is, by now, quite satisfied with the picture. However, the most
important parts, namely the labels, are still missing! 

\tikzname\ offers an easy-to-use and powerful system for adding text and,
more generally, complex shapes to a picture at specific positions. The
basic idea is the following: When \tikzname\ is constructing a path and
encounters the keyword |node| in the middle of a path, it
reads a \emph{node specification}. The keyword |node| is typically
followed by some options and then some text between curly braces. This
text is put inside a normal \TeX\ box (if the node specification
directly follows a coordinate, which is usually the case, \tikzname\ is
able to perform some magic so that it is even possible to use verbatim
text inside the boxes) and then placed at the current position, that
is, at the last specified position (possibly shifted a bit, according
to the given options). However, all nodes are drawn only after the
path has been completely drawn/filled/shaded/clipped/whatever.  

\begin{codeexample}[]
\begin{tikzpicture}
  \draw (0,0) rectangle (2,2);
  \draw (0.5,0.5) node [fill=examplefill]
                       {Text at \verb!node 1!}
     -- (1.5,1.5) node {Text at \verb!node 2!};
\end{tikzpicture}
\end{codeexample}

Obviously, Karl would not only like to place nodes \emph{on} the last
specified position, but also to the left or the 
right of these positions. For this, every node object that you
put in your picture is equipped with several \emph{anchors}. For
example, the |north| anchor is in the middle at the upper end of the shape,
the |south| anchor is at the bottom and the |north east| anchor is in
the upper right corner. When you given the option |anchor=north|, the
text will be placed such that this northern anchor will lie on the
current position and the text is, thus, below the current
position. Karl uses this to draw the ticks as follows:

\begin{codeexample}[]
\begin{tikzpicture}[scale=3]
  \clip (-0.6,-0.2) rectangle (0.6,1.51);
  \draw[step=.5cm,help lines] (-1.4,-1.4) grid (1.4,1.4);
  \filldraw[fill=green!20,draw=green!50!black]
    (0,0) -- (3mm,0mm) arc (0:30:3mm) -- cycle;
  \draw[->] (-1.5,0) -- (1.5,0);   \draw[->] (0,-1.5) -- (0,1.5);
  \draw (0,0) circle (1cm);

  \foreach \x in {-1,-0.5,1}
    \draw (\x cm,1pt) -- (\x cm,-1pt) node[anchor=north] {$\x$};
  \foreach \y in {-1,-0.5,0.5,1}
    \draw (1pt,\y cm) -- (-1pt,\y cm) node[anchor=east] {$\y$};
\end{tikzpicture}
\end{codeexample}

This is quite nice, already. Using these anchors, Karl can now add
most of the other text elements. However, Karl thinks that, though
``correct,'' it is quite counter-intuitive that in order to place something
\emph{below} a given point, he has to use the \emph{north} anchor. For
this reason, there is an option called |below|, which does the
same as |anchor=north|. Similarly, |above right| does the same as
|anchor=south east|. In addition, |below| takes an optional
dimension argument. If given, the shape will additionally be shifted
downwards by the given amount. So, |below=1pt| can be used to put
a text label below some point and, additionally shift it  1pt
downwards. 

Karl is not quite satisfied with the ticks. He would like to have
$1/2$ or $\frac{1}{2}$ shown instead of $0.5$, partly to show off the
nice capabilities of \TeX\ and \tikzname, partly because for positions
like $1/3$ or $\pi$ it is certainly very much preferable to have the
``mathematical'' tick there instead of just the ``numeric'' tick.
His students, on the other hand, prefer $0.5$ over $1/2$
since they are not too fond of fractions in general.

Karl now faces a problem: For the |\foreach| statement, the position
|\x| should still be given as |0.5| since \tikzname\ will not know where
|\frac{1}{2}| is supposed to be. On the other hand, the typeset text
should really be  |\frac{1}{2}|. To solve this problem, |\foreach|
offers a special syntax: Instead of having one variable |\x|, Karl can
specify two (or even more) variables separated by a slash as in
|\x / \xtext|. Then, the elements in the set over which |\foreach|
iterates must also be of the form \meta{first}|/|\meta{second}. In
each iteration, |\x| will be set to \meta{first} and |\xtext| will be
set to \meta{second}. If no \meta{second} is given, the \meta{first}
will be used again. So, here is the new code for the ticks: 

\begin{codeexample}[]
\begin{tikzpicture}[scale=3]
  \clip (-0.6,-0.2) rectangle (0.6,1.51);
  \draw[step=.5cm,help lines] (-1.4,-1.4) grid (1.4,1.4);
  \filldraw[fill=green!20,draw=green!50!black]
    (0,0) -- (3mm,0mm) arc (0:30:3mm) -- cycle;
  \draw[->] (-1.5,0) -- (1.5,0); \draw[->] (0,-1.5) -- (0,1.5);
  \draw (0,0) circle (1cm);

  \foreach \x/\xtext in {-1, -0.5/-\frac{1}{2}, 1}
    \draw (\x cm,1pt) -- (\x cm,-1pt) node[anchor=north] {$\xtext$};
  \foreach \y/\ytext in {-1, -0.5/-\frac{1}{2}, 0.5/\frac{1}{2}, 1}
    \draw (1pt,\y cm) -- (-1pt,\y cm) node[anchor=east] {$\ytext$};
\end{tikzpicture}
\end{codeexample}

Karl is quite pleased with the result, but his son points out that
this is still not perfectly satisfactory: The grid and the circle
interfere with the numbers and decrease their legibility. Karl is not
very concerned by this (his students do not even notice), but his son
insists that there is an easy solution: Karl can add the
|[fill=white]| option to fill out the background of the text shape
with a white color. 

The next thing Karl wants to do is to add the labels like $\sin
\alpha$. For this, he would like to place a label ``in the middle of
line.'' To do so, instead of specifying the label
|node {$\sin\alpha$}|  directly after one of the endpoints of the line
(which would place 
the label at that endpoint), Karl can give the label directly after
the |--|, before the coordinate. By default, this places the label in
the middle of the line, but the |pos=| options can be used to modify
this. Also, options like |near start| and |near end| can be used to
modify this position:


\begin{codeexample}[]
\begin{tikzpicture}[scale=3]
  \clip (-2,-0.2) rectangle (2,0.8);
  \draw[step=.5cm,gray,very thin] (-1.4,-1.4) grid (1.4,1.4);
  \filldraw[fill=green!20,draw=green!50!black] (0,0) -- (3mm,0mm) arc
  (0:30:3mm) -- cycle;
  \draw[->] (-1.5,0) -- (1.5,0) coordinate (x axis);
  \draw[->] (0,-1.5) -- (0,1.5) coordinate (y axis);
  \draw (0,0) circle (1cm);
    
  \draw[very thick,red]
    (30:1cm) -- node[left=1pt,fill=white] {$\sin \alpha$} (30:1cm |- x axis);
  \draw[very thick,blue]
    (30:1cm |- x axis) -- node[below=2pt,fill=white] {$\cos \alpha$} (0,0);
  \draw[very thick,orange] (1,0) -- node [right=1pt,fill=white]
    {$\displaystyle \tan \alpha \color{black}=
      \frac{{\color{red}\sin \alpha}}{\color{blue}\cos \alpha}$}
    (intersection of 0,0--30:1cm and 1,0--1,1) coordinate (t);

  \draw (0,0) -- (t);
  
  \foreach \x/\xtext in {-1, -0.5/-\frac{1}{2}, 1}
    \draw (\x cm,1pt) -- (\x cm,-1pt) node[anchor=north,fill=white] {$\xtext$};
  \foreach \y/\ytext in {-1, -0.5/-\frac{1}{2}, 0.5/\frac{1}{2}, 1}
    \draw (1pt,\y cm) -- (-1pt,\y cm) node[anchor=east,fill=white] {$\ytext$};
\end{tikzpicture}
\end{codeexample}

You can also position labels on curves and, by adding the |sloped|
option, have them rotated such that they match the line's slope. Here
is an example:

\begin{codeexample}[]
\begin{tikzpicture}
  \draw (0,0) .. controls (6,1) and (9,1) ..
    node[near start,sloped,above] {near start}
    node {midway}
    node[very near end,sloped,below] {very near end} (12,0);
\end{tikzpicture}
\end{codeexample}

It remains to draw the explanatory text at the right of the
picture. The main difficulty here lies in limiting the width of the
text ``label,'' which is quite long, so that line breaking is
used. Fortunately, Karl can use the option |text width=6cm| to get the
desired effect. So, here is the full code:

\begin{codeexample}[code only]
\begin{tikzpicture}
  [scale=3,line cap=round
  % Styles
  axes/.style=,
  important line/.style={very thick},
  information text/.style={rounded corners,fill=red!10,inner sep=1ex}]

  % Local definitions
  \def\costhirty{0.8660256}

  % Colors
  \colorlet{anglecolor}{green!50!black}
  \colorlet{sincolor}{red}
  \colorlet{tancolor}{orange!80!black}
  \colorlet{coscolor}{blue}

  % The graphic
  \draw[help lines,step=0.5cm] (-1.4,-1.4) grid (1.4,1.4);
  
  \draw (0,0) circle (1cm);

  \begin{scope}[axes]
    \draw[->] (-1.5,0) -- (1.5,0) node[right] {$x$} coordinate(x axis);
    \draw[->] (0,-1.5) -- (0,1.5) node[above] {$y$} coordinate(y axis);

    \foreach \x/\xtext in {-1, -.5/-\frac{1}{2}, 1}
      \draw[xshift=\x cm] (0pt,1pt) -- (0pt,-1pt) node[below,fill=white] {$\xtext$};
  
    \foreach \y/\ytext in {-1, -.5/-\frac{1}{2}, .5/\frac{1}{2}, 1}
      \draw[yshift=\y cm] (1pt,0pt) -- (-1pt,0pt) node[left,fill=white] {$\ytext$};
  \end{scope}
    
  \filldraw[fill=green!20,draw=anglecolor] (0,0) -- (3mm,0pt) arc(0:30:3mm);
  \draw (15:2mm) node[anglecolor] {$\alpha$};
    
  \draw[important line,sincolor]
    (30:1cm) -- node[left=1pt,fill=white] {$\sin \alpha$} (30:1cm |- x axis);
  
  \draw[important line,coscolor]
    (30:1cm |- x axis) -- node[below=2pt,fill=white] {$\cos \alpha$} (0,0);
  
  \draw[important line,tancolor] (1,0) -- node[right=1pt,fill=white] {
    $\displaystyle \tan \alpha \color{black}=
    \frac{{\color{sincolor}\sin \alpha}}{\color{coscolor}\cos \alpha}$}
    (intersection of 0,0--30:1cm and 1,0--1,1) coordinate (t);

  \draw (0,0) -- (t);
  
  \draw[xshift=1.85cm]
    node[right,text width=6cm,information text]
    {
      The {\color{anglecolor} angle $\alpha$} is $30^\circ$ in the
      example ($\pi/6$ in radians). The {\color{sincolor}sine of
        $\alpha$}, which is the height of the red line, is
      \[
      {\color{sincolor} \sin \alpha} = 1/2.
      \]
      By the Theorem of Pythagoras ...
    };
\end{tikzpicture}
\end{codeexample}



���}	� �
����DEV��INO��SYN��SV~�pgfmanual-en-tutorial-nodes.tex�J��J׀�J�j��rP�_.j�
% Copyright 2006 by Till Tantau
%
% This file may be distributed and/or modified
%
% 1. under the LaTeX Project Public License and/or
% 2. under the GNU Free Documentation License.
%
% See the file doc/generic/pgf/licenses/LICENSE for more details.


\section{Tutorial: Euclid's Amber Version of the \emph{Elements}}

In this third tutorial we have a look at how \tikzname\ can be used to
draw geometric constructions.

Euclid is currently quite busy writing his new book series, whose
working title is ``Elements'' (Euclid is not quite sure whether this
title will convey the message of the series to future generations
correctly, but he intends to change the title before it goes to the
publisher). Up to know, he wrote down his text and graphics on
papyrus, but his publisher suddenly insists that he must submit in
electronic form. Euclid tries to argue with the publisher that 
electronics will only be discovered thousands of years later, but the
publisher informs him that the use of amber is no longer cutting edge
technology and Euclid will just have to keep up with modern tools.

Slightly disgruntled, Euclid starts converting his papyrus
entitled ``Book I, Proposition I'' to an amber version.  

\subsection{Book I, Proposition I}

The drawing on his papyrus looks like this:\footnote{The text is taken
from the wonderful interactive version of Euclid's Elements by David
E. Joyce, to be found on his website at Clark University.}

\bigskip
\noindent
\begin{tikzpicture}[thick,help lines/.style={thin,draw=black!50}]
  \def\A{\textcolor{input}{$A$}}
  \def\B{\textcolor{input}{$B$}}
  \def\C{\textcolor{output}{$C$}}
  \def\D{$D$}
  \def\E{$E$}
  
  \colorlet{input}{blue!80!black}
  \colorlet{output}{red!70!black}
  \colorlet{triangle}{orange}
  
  \coordinate [label=left:\A]
    (A) at ($ (0,0) + .1*(rand,rand) $);
  \coordinate [label=right:\B]
    (B) at ($ (1.25,0.25) + .1*(rand,rand) $);

  \draw [input] (A) -- (B);
  
  \node [help lines,draw,label=left:\D] (D) at (A) [circle through=(B)] {};
  \node [help lines,draw,label=right:\E] (E) at (B) [circle through=(A)] {};
  
  \coordinate [label=above:\C]
    (C) at (intersection 2 of D and E);

  \draw [output] (A) -- (C);
  \draw [output] (B) -- (C);

  \foreach \point in {A,B,C}
    \fill [black,opacity=.5] (\point) circle (2pt);

  \begin{pgfonlayer}{background}
    \fill[triangle!80] (A) -- (C) -- (B) -- cycle;
  \end{pgfonlayer}
  
  \node [below right,text width=10cm,text justified] at (4,3)
  {
    \small
    \textbf{Proposition I}\par
    \emph{To construct an \textcolor{triangle}{equilateral triangle}
      on a given \textcolor{input}{finite straight line}.}
    \par
    \vskip1em
    Let \A\B\ be the given \textcolor{input}{finite straight line}. It
    is required to construct an \textcolor{triangle}{equilateral
      triangle} on the \textcolor{input}{straight line}~\A\B. 

    Describe the circle \B\C\D\ with center~\A\ and radius \A\B. Again
    describe the circle \A\C\E\ with center~\B\ and radius \B\A. Join the
    \textcolor{output}{straight lines} \C\A\ and \C\B\ from the
    point~\C\ at which the circles cut one another to the points~\A\ and~\B.

    Now, since the point~\A\ is the center of the circle \C\D\B,
    therefore \A\C\ equals \A\B. Again, since the point \B\ is the
    center of the circle \C\A\E, therefore \B\C\ equals \B\A. But
    \A\C\ was proved equal to \A\B, therefore each of the straight
    lines \A\C\ and \B\C\ equals \A\B. And 
    things which equal the same thing also equal one another,
    therefore \A\C\ also equals \B\C. Therefore the three straight
    lines \A\C, \A\B, and \B\C\ equal one another. 
    Therefore the \textcolor{triangle}{triangle} \A\B\C\ is
    equilateral, and it has been  constructed on the given finite
    \textcolor{input}{straight line}~\A\B.  
  };
\end{tikzpicture}
\bigskip

Let us have a look at how Euclid can turn this into \tikzname\ code.

\subsubsection{Setting up the Environment}

As in the previous tutorials, Euclid needs to load \tikzname, together
with some libraries. These libraries are |calc|, |through|, and
|backgrounds|. Depending on which format he uses, Euclid would use one
of the following in the preamble: 

\begin{codeexample}[code only]
% For LaTeX:
\usepackage{tikz}
\usetikzlibrary{calc,through,backgrounds}
\end{codeexample}

\begin{codeexample}[code only]
% For plain TeX:
\input tikz.tex
\usetikzlibrary{calc,through,backgrounds}
\end{codeexample}

\begin{codeexample}[code only]
% For ConTeXt:
\usemodule[tikz]
\usetikzlibrary[calc,through,backgrounds]
\end{codeexample}


\subsubsection{The Line \emph{AB}}

The first part of the picture that Euclid wishes to draw is the line
$AB$. That is easy enough, something like |\draw (0,0) -- (2,1);|
might do. However, Euclid does not wish to reference the two points
$A$ and $B$ as $(0,0)$ and $(2,1)$ subsequently. Rather, he wishes to
just write |A| and |B|. Indeed, the whole point of his book is that
the points $A$ and $B$ can be arbitrary and all other points (like
$C$) are constructed in terms of their positions. It would not do
if Euclid were to write down the coordinates of $C$ explicitly.

So, Euclid starts with defining two coordinates using the
|\coordinate| command:
\begin{codeexample}[]
\begin{tikzpicture}
  \coordinate (A) at (0,0);
  \coordinate (B) at (1.25,0.25);

  \draw[blue] (A) -- (B);
\end{tikzpicture}
\end{codeexample}

That was easy enough. What is missing at this point are the labels for
the coordinates. Euclid does not want them \emph{on} the points, but
next to them. He decides to use the |label| option:
\begin{codeexample}[]
\begin{tikzpicture}
  \coordinate [label=left:\textcolor{blue}{$A$}]  (A) at (0,0);
  \coordinate [label=right:\textcolor{blue}{$B$}] (B) at (1.25,0.25);

  \draw[blue] (A) -- (B);
\end{tikzpicture}
\end{codeexample}

At this point, Euclid decides that it would be even nicer if the
points $A$ and $B$ were in some sense ``random.'' Then, neither Euclid
nor the reader can make the mistake of taking ``anything for granted''
concerning these position of these points. Euclid is pleased to learn
that there is a |rand| function in \tikzname\ that does exactly what
he needs: It produces a number between $-1$ and $1$. Since \tikzname\
can do a bit of math, Euclid can change the coordinates of the points
as follows:
\begin{codeexample}[code only]
\coordinate [...] (A) at (0+0.1*rand,0+0.1*rand);
\coordinate [...] (B) at (1.25+0.1*rand,0.25+0.1*rand);
\end{codeexample}

This works fine. However, Euclid is not quite satisfied since he would
prefer that the ``main coordinates'' $(0,0)$ and $(1.25,0.25)$ are
``kept separate'' from the perturbation
$0.1(\mathit{rand},\mathit{rand})$. This means, he would like to
specify that coordinate $A$ as ``The point that is at $(0,0)$ plus one
tenth of the vector  $(\mathit{rand},\mathit{rand})$.''

It turns out that the |calc| library allows him to do exactly this
kind of computation. When this library is loaded, you can use special
coordinates that start with |($| and end with |$)| rather than just
|(| and~|)|. Inside these special coordinates you can give a linear
combination of coordinates. (Note that the dollar signs are only
intended to signal that a ``computation'' is going on; no mathematical
typesetting is done.)

The new code for the coordinates is the following:

\begin{codeexample}[code only]
\coordinate [...] (A) at ($ (0,0) + .1*(rand,rand) $);
\coordinate [...] (B) at ($ (1.25,0.25) + .1*(rand,rand) $);
\end{codeexample}

Note that if a coordinate in such a computation has a factor (like
|.1|) you must place a |*| directly before the opening parenthesis of
the coordinate. You can nest such computations.



\subsubsection{The Circle Around \emph{A}}

The first tricky construction is the circle around~$A$. We will see
later how to do this in a very simple manner, but first let us do it
the ``hard'' way.

The idea is the following: We draw a circle around the point $A$ whose
radius is given by the length of the line $AB$. The difficulty lies in
computing the length of this line.

Two ideas ``nearly'' solve this problem: First, we can write
|($ (A) - (B) $)| for the vector that is the difference between $A$
and~$B$. All we need is the length of this vector. Second, given two
numbers $x$ and $y$, one can write |veclen(|$x$|,|$y$|)| inside a
mathematical expression. This gives the value $\sqrt{x^2+y^2}$, which
is exactly the desired length.

The only remaining problem is to access the $x$- and $y$-coordinate of
the vector~$AB$. For this, we need a new concept: the \emph{let
  operation}. A let operation can be given anywhere on a path where a
normal path operation like a line-to or a move-to is expected. The
effect of a let operation is to evaluate some coordinates and to
assign the results to special macros. These macros make it easy to
access the $x$- and $y$-coordinates of the coordinates.

Euclid would write the following:
\begin{codeexample}[]
\begin{tikzpicture}
  \coordinate [label=left:$A$]  (A) at (0,0);
  \coordinate [label=right:$B$] (B) at (1.25,0.25);
  \draw (A) -- (B);

  \draw (A) let
              \p1 = ($ (B) - (A) $)
            in
              circle ({veclen(\x1,\y1)});
\end{tikzpicture}
\end{codeexample}

Each assignment in a let operation starts with |\p|, usually followed
by a \meta{digit}. Then comes an equal sign and a coordinate. The
coordinate is evaluated and the result is stored internally. From
then on you can use the following expressions: 
\begin{enumerate}
\item |\x|\meta{digit} yields the $x$-coordinate of the resulting point.
\item |\y|\meta{digit} yields the $y$-coordinate of the resulting
  point.
\item |\p|\meta{digit} yields the same as |\x|\meta{digit}|,\y|\meta{digit}.
\end{enumerate}
You can have multiple assignments in a let operation, just separate
them with commas. In later assignments you can already use the results
of earlier assignments.

Note that |\p1| is not a coordinate in the usual sense. Rather, it
just expands to a string like |10pt,20pt|. So, you cannot write, for
instance, |(\p1.center)| since this would just expand to
|(10pt,20pt.center)|, which makes no sense.

Next, we want to draw both circles at the same time. Each time the
radius is |veclen(\x1,\y1)|. It seems natural to compute this radius
only once. For this, we can also use a let operation: Instead of
writing |\p1 = ...|, we write |\n2 = ...|. Here, ``n'' stands for
``number'' (while ``p'' stands for ``point''). The assignment of a
number should be followed by a number in curly braces.
\begin{codeexample}[]
\begin{tikzpicture}
  \coordinate [label=left:$A$]  (A) at (0,0);
  \coordinate [label=right:$B$] (B) at (1.25,0.25);
  \draw (A) -- (B);

  \draw let \p1 = ($ (B) - (A) $),
            \n2 = {veclen(\x1,\y1)}
        in
          (A) circle (\n2)
          (B) circle (\n2);
\end{tikzpicture}
\end{codeexample}
In the above example, you may wonder, what |\n1| would yield? The
answer is that it would be undefined -- the |\p|, |\x|, and |\y|
macros refer to the same logical point, while the |\n| macro has ``its
own namespace.'' We could even have replaced |\n2| in the example by
|\n1| and it would still work. Indeed, the digits following these
macros are just normal \TeX\ parameters. We could also use a longer
name, but then we have to use curly braces:
\begin{codeexample}[]
\begin{tikzpicture}
  \coordinate [label=left:$A$]  (A) at (0,0);
  \coordinate [label=right:$B$] (B) at (1.25,0.25);
  \draw (A) -- (B);

  \draw let \p1        = ($ (B) - (A) $),
            \n{radius} = {veclen(\x1,\y1)}
        in
          (A) circle (\n{radius})
          (B) circle (\n{radius});
\end{tikzpicture}
\end{codeexample}

At the beginning of this section it was promised that there is an
easier way to create the desired circle. The trick is to use the
|through| library. As the name suggests, it contains code for creating
shapes that go through a given point.

The option that we are looking for is |circle through|. This option is
given to a \emph{node} and has the following effects: First, it causes
the node's inner and outer separations to be set to zero. Then it sets
the shape of the node to |circle|. Finally, it sets the radius of the
node such that it goes through the parameter given to
|circle through|. This radius is computed in essentially the same way
as above.

\begin{codeexample}[]
\begin{tikzpicture}
  \coordinate [label=left:$A$]  (A) at (0,0);
  \coordinate [label=right:$B$] (B) at (1.25,0.25);
  \draw (A) -- (B);

  \node [draw,circle through=(B),label=left:$D$] at (A) {};
\end{tikzpicture}
\end{codeexample}


\subsubsection{The Intersection of the Circles}

Euclid can now draw the line and the circles. The final problem is to
compute the intersection of the two circles. This computation is a bit
involved if you want to do it ``by hand.'' Fortunately, the so-called
intersection coordinate system allows us to specify points as the
intersection of two objects (in order for the following code to work,
the |calc| library must be loaded; it defines the necessary code for
computing the intersection of circles):
\begin{codeexample}[]
\begin{tikzpicture}
  \coordinate [label=left:$A$]  (A) at (0,0);
  \coordinate [label=right:$B$] (B) at (1.25,0.25);
  \draw (A) -- (B);

  \node (D) [draw,circle through=(B),label=left:$D$]  at (A) {};
  \node (E) [draw,circle through=(A),label=right:$E$] at (B) {};

  \coordinate [label=above:$C$] (C) at (intersection 2 of D and E);

  \draw [red] (A) -- (C);
  \draw [red] (B) -- (C);
\end{tikzpicture}
\end{codeexample}

We could also have written |intersection 1 of| or just
|intersection of| to get access to the other intersection of the
circles.

Although Euclid does not need it for the current picture, it is just a
small step to computing the bisection of the line $AB$:

\begin{codeexample}[]
\begin{tikzpicture}
  \coordinate [label=left:$A$]  (A) at (0,0);
  \coordinate [label=right:$B$] (B) at (1.25,0.25);
  \draw (A) -- (B);

  \node (D) [draw,circle through=(B),label=left:$D$]  at (A) {};
  \node (E) [draw,circle through=(A),label=right:$E$] at (B) {};

  \coordinate [label=above:$C$]  (C)  at (intersection 2 of D and E);
  \coordinate [label=below:$C'$] (C') at (intersection 1 of D and E);

  \draw [red] (C) -- (C');
  \node [fill=red,inner sep=1pt,label=-45:$F$] (F) at (intersection of C--C' and A--B) {};
\end{tikzpicture}
\end{codeexample}



\subsubsection{The Complete Code}

Back to Euclid's code. He introduces a few macros to make life
simpler, like a |\A| macro for typesetting a blue $A$. He also uses the
|background| layer for drawing the triangle behind everything at the
end. 

\begin{codeexample}[]
\begin{tikzpicture}[thick,help lines/.style={thin,draw=black!50}]
  \def\A{\textcolor{input}{$A$}}     \def\B{\textcolor{input}{$B$}}
  \def\C{\textcolor{output}{$C$}}    \def\D{$D$}
  \def\E{$E$}
  
  \colorlet{input}{blue!80!black}    \colorlet{output}{red!70!black}
  \colorlet{triangle}{orange}
  
  \coordinate [label=left:\A]  (A) at ($ (0,0) + .1*(rand,rand) $);
  \coordinate [label=right:\B] (B) at ($ (1.25,0.25) + .1*(rand,rand) $);

  \draw [input] (A) -- (B);
  
  \node [help lines,draw,label=left:\D]   (D) at (A) [circle through=(B)] {};
  \node [help lines,draw,label=right:\E]  (E) at (B) [circle through=(A)] {};
  
  \coordinate [label=above:\C] (C) at (intersection 2 of D and E);

  \draw [output] (A) -- (C) -- (B);

  \foreach \point in {A,B,C}
    \fill [black,opacity=.5] (\point) circle (2pt);

  \begin{pgfonlayer}{background}
    \fill[triangle!80] (A) -- (C) -- (B) -- cycle;
  \end{pgfonlayer}
  
  \node [below right, text width=10cm,text justified] at (4,3) {
    \small\textbf{Proposition I}\par
    \emph{To construct an \textcolor{triangle}{equilateral triangle}
      on a given \textcolor{input}{finite straight line}.}
    \par\vskip1em
    Let \A\B\ be the given \textcolor{input}{finite straight line}.  \dots
  };
\end{tikzpicture}
\end{codeexample}


\subsection{Book I, Proposition II}

The second proposition in the Elements is the following:

\bigskip\noindent
\begin{tikzpicture}[thick,help lines/.style={thin,draw=black!50}]
  \def\A{\textcolor{orange}{$A$}}   \def\B{\textcolor{input}{$B$}}
  \def\C{\textcolor{input}{$C$}}    \def\D{$D$}
  \def\E{$E$}                       \def\F{$F$}
  \def\G{$G$}                       \def\H{$H$}
  \def\K{$K$}                       \def\L{\textcolor{output}{$L$}}
  
  \colorlet{input}{blue!80!black}    \colorlet{output}{red!70!black}
  
  \coordinate [label=left:\A]  (A) at ($ (0,0) + .1*(rand,rand) $);
  \coordinate [label=right:\B] (B) at ($ (1,0.2) + .1*(rand,rand) $);
  \coordinate [label=above:\C] (C) at ($ (1,2) + .1*(rand,rand) $);

  \draw [input] (B) -- (C);
  \draw [help lines] (A) -- (B);

  \coordinate [label=above:\D] (D) at ($ (A)!.5!(B) ! {sin(60)*2} ! 90:(B) $);

  \draw [help lines] (D) -- ($ (D)!3.75!(A) $) coordinate [label=-135:\E] (E);
  \draw [help lines] (D) -- ($ (D)!3.75!(B) $) coordinate [label=-45:\F] (F);

  \node (H) at (B) [help lines,circle through=(C),draw,label=135:\H] {};

  \coordinate [label=right:\G] (G) at (intersection of B--F and H);

  \node (K) at (D) [help lines,circle through=(G),draw,label=135:\K] {};

  \coordinate [label=below:\L] (L) at (intersection of A--E and K);

  \draw [output] (A) -- (L);

  \foreach \point in {A,B,C,D,G,L}
    \fill [black,opacity=.5] (\point) circle (2pt);
  
  \node [below right, text width=9cm,text justified] at (4,4) {
    \small\textbf{Proposition II}\par
    \emph{To place a \textcolor{output}{straight line} equal to a
      given \textcolor{input}{straight line} with 
      one end at a \textcolor{orange}{given point}.} 
    \par\vskip1em
    Let \A\ be the given point, and \B\C\ the given
    \textcolor{input}{straight line}. 
    It is required to place a \textcolor{output}{straight line} equal
    to the given \textcolor{input}{straight line} \B\C\ with one end
    at the point~\A.  

    Join the straight line \A\B\ from the point \A\ to the point \B, and
    construct the equilateral triangle \D\A\B\ on it.
    
    Produce the straight lines \A\E\ and \B\F\ in a straight line with
    \D\A\ and \D\B. Describe the circle \C\G\H\ with center \B\ and
    radius \B\C, and  again, describe the circle \G\K\L\ with center
    \D\ and radius \D\G. 	

    Since the point \B\ is the center of the circle \C\G\H, therefore
    \B\C\ equals \B\G. Again, since the point \D\ is the center of the
    circle \G\K\L, therefore \D\L\ equals \D\G. And in these \D\A\
    equals \D\B, therefore the remainder \A\L\ equals the remainder
    \B\G. But \B\C\ was also proved  equal to \B\G, therefore each of
    the straight lines \A\L\ and \B\C\ equals \B\G. And things which
    equal the same thing also equal one another, therefore \A\L\ also
    equals \B\C. 
    
    Therefore the \textcolor{output}{straight line} \A\L\ equal to the
    given \textcolor{input}{straight line} \B\C\  has been placed with
    one end at the \textcolor{orange}{given point}~\A.  
  };
\end{tikzpicture}




\subsubsection{Using Partway Calculations for the Construction of \emph{D}}

Euclid's construction starts with ``referencing'' Proposition~I for
the construction of the point~$D$. Now, while we could simply repeat the
construction, it seems a bit bothersome that one has to draw all these
circles and do all these complicated constructions.

For this reason, \tikzname\ supports some simplifications. First,
there is a simple syntax for computing a point that is ``partway'' on
a line from $p$ to~$q$: You place these two points in a coordinate
calculation -- remember, they start with |($| and end with |$)| -- and
then combine them using |!|\meta{part}|!|. A \meta{part} of |0| refers
to the \emph{first} coordinate, a \meta{part} of |1| refers to the
second coordinate, and a value in between refers to a point on the
line from $p$ to~$q$. Thus, the syntax is similar to the |xcolor|
syntax for mixing colors.

Here is the computation of the point in the middle of the line $AB$:
\begin{codeexample}[]
\begin{tikzpicture}
  \coordinate [label=left:$A$]  (A) at (0,0);
  \coordinate [label=right:$B$] (B) at (1.25,0.25);
  \draw (A) -- (B);
  \node [fill=red,inner sep=1pt,label=below:$X$] (X) at ($ (A)!.5!(B) $) {};
\end{tikzpicture}
\end{codeexample}

The computation of the point $D$ in Euclid's second proposition is a
bit more complicated. It can be expressed as follows: Consider the
line from $X$ to $B$. Suppose we 
rotate this line around $X$ for 90$^\circ$ and then stretch it by a
factor of $\sin(60^\circ)/2$. This yields the desired point~$D$. We
can do the stretching using the partway modifier above, for the
rotation we need a new modifier: the rotation modifier. The idea is
that the second coordinate in a partway computation can be prefixed by
an angle. Then the partway point is computed normally (as if no angle
were given), but the resulting point is rotated by this angle around
the first point.  

\begin{codeexample}[]
\begin{tikzpicture}
  \coordinate [label=left:$A$]  (A) at (0,0);
  \coordinate [label=right:$B$] (B) at (1.25,0.25);
  \draw (A) -- (B);
  \node [fill=red,inner sep=1pt,label=below:$X$] (X) at ($ (A)!.5!(B) $) {};
  \node [fill=red,inner sep=1pt,label=above:$D$] (D) at
    ($ (X) ! {sin(60)*2} ! 90:(B) $) {};
  \draw (A) -- (D) -- (B);
\end{tikzpicture}
\end{codeexample}

Finally, it is not necessary to explicitly name the point $X$. Rather,
again like in the |xcolor| package, it is possible to chain partway
modifiers:

\begin{codeexample}[]
\begin{tikzpicture}
  \coordinate [label=left:$A$]  (A) at (0,0);
  \coordinate [label=right:$B$] (B) at (1.25,0.25);
  \draw (A) -- (B);
  \node [fill=red,inner sep=1pt,label=above:$D$] (D) at
    ($ (A) ! .5 ! (B) ! {sin(60)*2} ! 90:(B) $) {};
  \draw (A) -- (D) -- (B);
\end{tikzpicture}
\end{codeexample}


\subsubsection{Intersecting a Line and a Circle}

The next step in the construction is to draw a circle around $B$
through $C$, which is easy enough to do using the |circle through|
option. Extending the lines $DA$ and $DB$ can be done using partway
calculations, but this time with a part value outside the range
$[0,1]$: 

\begin{codeexample}[]
\begin{tikzpicture}
  \coordinate [label=left:$A$]  (A) at (0,0);
  \coordinate [label=right:$B$] (B) at (0.75,0.25);
  \coordinate [label=above:$C$] (C) at (1,1.5);
  \draw (A) -- (B) -- (C);
  \coordinate [label=above:$D$] (D) at
    ($ (A) ! .5 ! (B) ! {sin(60)*2} ! 90:(B) $) {};
  \node (H) [label=135:$H$,draw,circle through=(C)] at (B) {};
  \draw (D) -- ($ (D) ! 3.5 ! (B) $) coordinate [label=below:$F$] (F);
  \draw (D) -- ($ (D) ! 2.5 ! (A) $) coordinate [label=below:$E$] (E);
\end{tikzpicture}
\end{codeexample}

We now face the problem of finding the point $G$, which is the
intersection of the line $BF$ and the circle $H$. One way is to use
yet another variant of the partway computation: Normally, a partway
computation has the form \meta{p}|!|\meta{factor}|!|\meta{q},
resulting in the point $(1-\meta{factor})\meta{p} +
\meta{factor}\meta{q}$. Alternatively, instead of \meta{factor} you
can also use a \meta{dimension} between the points. In this case, you
get the point that is \meta{dimension} removed from \meta{p} on the
straight line to \meta{q}.

We know that the point $G$ is on the way from $B$ to $F$. The distance
is given by the radius of the circle~$H$. Here is the code form
computing $H$:
\begin{codeexample}[pre={
\begin{tikzpicture}
  \coordinate [label=left:$A$]  (A) at (0,0);
  \coordinate [label=right:$B$] (B) at (0.75,0.25);
  \coordinate [label=above:$C$] (C) at (1,1.5);
  \draw (A) -- (B) -- (C);
  \coordinate [label=above:$D$] (D) at
    ($ (A) ! .5 ! (B) ! {sin(60)*2} ! 90:(B) $) {};
  \node (H) [label=135:$H$,draw,circle through=(C)] at (B) {};
  \draw (D) -- ($ (D) ! 3.5 ! (B) $) coordinate [label=below:$F$] (F);
  \draw (D) -- ($ (D) ! 2.5 ! (A) $) coordinate [label=below:$E$] (E);
},post={\end{tikzpicture}}]
  \path let \p1 = ($ (B) - (C) $) in
    coordinate [label=left:$G$] (G) at ($ (B) ! veclen(\x1,\y1) ! (F) $);
  \fill[red,opacity=.5] (G) circle (2pt);
\end{codeexample}

However, there is a simpler way: As for circles, we can also intersect
a line and a circle using the |intersection| coordinate system:

\begin{codeexample}[pre={
\begin{tikzpicture}
  \coordinate [label=left:$A$]  (A) at (0,0);
  \coordinate [label=right:$B$] (B) at (0.75,0.25);
  \coordinate [label=above:$C$] (C) at (1,1.5);
  \draw (A) -- (B) -- (C);
  \coordinate [label=above:$D$] (D) at
    ($ (A) ! .5 ! (B) ! {sin(60)*2} ! 90:(B) $) {};
  \node (H) [label=135:$H$,draw,circle through=(C)] at (B) {};
  \draw (D) -- ($ (D) ! 3.5 ! (B) $) coordinate [label=below:$F$] (F);
  \draw (D) -- ($ (D) ! 2.5 ! (A) $) coordinate [label=below:$E$] (E);
},post={\end{tikzpicture}}]
  \coordinate [label=left:$G$] (G) at (intersection of B--F and H);
  \fill[red,opacity=.5] (G) circle (2pt);
\end{codeexample}

\subsubsection{The Complete Code}

\begin{codeexample}[]
\begin{tikzpicture}[thick,help lines/.style={thin,draw=black!50}]
  \def\A{\textcolor{orange}{$A$}}   \def\B{\textcolor{input}{$B$}}
  \def\C{\textcolor{input}{$C$}}    \def\D{$D$}
  \def\E{$E$}                       \def\F{$F$}
  \def\G{$G$}                       \def\H{$H$}
  \def\K{$K$}                       \def\L{\textcolor{output}{$L$}}
  
  \colorlet{input}{blue!80!black}    \colorlet{output}{red!70!black}
  
  \coordinate [label=left:\A]  (A) at ($ (0,0) + .1*(rand,rand) $);
  \coordinate [label=right:\B] (B) at ($ (1,0.2) + .1*(rand,rand) $);
  \coordinate [label=above:\C] (C) at ($ (1,2) + .1*(rand,rand) $);

  \draw [input] (B) -- (C);
  \draw [help lines] (A) -- (B);

  \coordinate [label=above:\D] (D) at ($ (A)!.5!(B) ! {sin(60)*2} ! 90:(B) $);

  \draw [help lines] (D) -- ($ (D)!3.75!(A) $) coordinate [label=-135:\E] (E);
  \draw [help lines] (D) -- ($ (D)!3.75!(B) $) coordinate [label=-45:\F] (F);

  \node (H) at (B) [help lines,circle through=(C),draw,label=135:\H] {};

  \coordinate [label=right:\G] (G) at (intersection of B--F and H);

  \node (K) at (D) [help lines,circle through=(G),draw,label=135:\K] {};

  \coordinate [label=below:\L] (L) at (intersection of A--E and K);

  \draw [output] (A) -- (L);

  \foreach \point in {A,B,C,D,G,L}
    \fill [black,opacity=.5] (\point) circle (2pt);

  % \node ...
\end{tikzpicture}
\end{codeexample}

�� �}	� �
����DEV��INO��SYN��SV~�pgfmanual-en-tutorial-chains.tex�J��JՀ�J�j��nP�_.j�

���}	� �
����DEV��INO��SYN��SV~�pgfmanual-en-guidelines.tex�J��J���J�j��P�_.j�



\part{Installation and Configuration}

{\Large \emph{by Till Tantau}}


\bigskip
\noindent
This part explains how the system is installed. Typically, someone has
already done so for your system, so this part can be skipped; but if
this is not the case and you are the poor fellow who has to do the
installation, read the present part. 


\vskip1cm

\begin{codeexample}[graphic=white]
\begin{tikzpicture}[->,>=stealth',shorten >=1pt,auto,node distance=2.8cm,on grid,semithick,
                    every state/.style={fill=red,draw=none,circular drop shadow,text=white}]

  \node[initial,state] (A)                    {$q_a$};
  \node[state]         (B) [above right=of A] {$q_b$};
  \node[state]         (D) [below right=of A] {$q_d$};
  \node[state]         (C) [below right=of B] {$q_c$};
  \node[state]         (E) [below=of D]       {$q_e$};

  \path (A) edge              node {0,1,L} (B)
            edge              node {1,1,R} (C)
        (B) edge [loop above] node {1,1,L} (B)  
            edge              node {0,1,L} (C)
        (C) edge              node {0,1,L} (D)
            edge [bend left]  node {1,0,R} (E)    
        (D) edge [loop below] node {1,1,R} (D)
            edge              node {0,1,R} (A)
        (E) edge [bend left]  node {1,0,R} (A);

   \node [right=1cm,text width=8cm] at (C)
   {
     The current candidate for the busy beaver for five states. It is
     presumed that this Turing machine writes a maximum number of
     $1$'s before halting among all Turing machines with five states
     and the tape alphabet $\{0, 1\}$. Proving this conjecture is an
     open research problem.
   };
\end{tikzpicture}
\end{codeexample}

% Copyright 2006 by Till Tantau
%
% This file may be distributed and/or modified
%
% 1. under the LaTeX Project Public License and/or
% 2. under the GNU Free Documentation License.
%
% See the file doc/generic/pgf/licenses/LICENSE for more details.

\section{Installation}

There are different ways of installing \pgfname, depending
on your system and needs, and you may need to install other
packages as well as, see below. Before installing, you may wish to
review the licenses under which the package is
distributed, see Section~\ref{section-license}. 

Typically, the package will already be installed on your
system. Naturally, in this case you do not need to worry about the
installation process at all and you can skip the rest of this
section. 


\subsection{Package and Driver Versions}

This documentation is part of version \pgfversion\ of the \pgfname\
package. In order to run \pgfname, you need a reasonably recent 
\TeX\ installation. When using \LaTeX, you need the following packages
installed (newer versions should also work):
\begin{itemize}
\item
  |xcolor| version \xcolorversion.
\end{itemize}
With plain \TeX, |xcolor| is not needed, but you obviously do not
get its (full) functionality. 

Currently, \pgfname\ supports the following backend drivers:
\begin{itemize}
\item
  |pdftex| version 0.14 or higher. Earlier versions do not work.
\item
  |dvips| version 5.94a or higher. Earlier versions may also work.

  For inter-picture connections, you need process pictures using
  |pdftex| version 1.40 or higher running in DVI mode.
\item
  |dvipdfm| version 0.13.2c or higher. Earlier versions may also work.

  For inter-picture connections, you need process pictures using
  |pdftex| version 1.40 or higher running in DVI mode.
\item
  |tex4ht| version 2003-05-05 or higher. Earlier versions may also work.
\item
  |vtex| version 8.46a or higher. Earlier versions may also work.
\item
  |textures| version 2.1 or higher. Earlier versions may also work.
\item
  |xetex| version 0.996 or higher. Earlier versions may also work.
\end{itemize}

Currently, \pgfname\ supports the following formats:
\begin{itemize}
\item
  |latex| with complete functionality.
\item
  |plain| with complete functionality, except for graphics inclusion,
  which works only for pdf\TeX.
\item
  |context| with complete functionality, except for graphics inclusion,
  which works only for pdf\TeX.
\end{itemize}

For more details, see Section~\ref{section-formats}.



\subsection{Installing Prebundled Packages}

I do not create or manage prebundled packages of \pgfname, but,
fortunately, nice other people do. I cannot give detailed instructions
on how to install these packages, since I do not manage them, but I
\emph{can} tell you were to find them. If you have a problem with
installing, you might wish to have a look at the Debian page or the
Mik\TeX\ page first.


\subsubsection{Debian}

The command ``|aptitude install pgf|'' should do the trick. Sit back
and relax. In detail, the following packages are installed:  
\begin{verbatim}
http://packages.debian.org/pgf
http://packages.debian.org/latex-xcolor
\end{verbatim}


\subsubsection{MiKTeX}

For MiK\TeX, use the update wizard to install the (latest versions of
the) packages called |pgf| and |xcolor|. 




\subsection{Installation in a texmf Tree}

For a permanent installation, you place the files of the
the \textsc{pgf} package in an appropriate |texmf| tree. 

When you ask \TeX\ to use a certain class or package, it usually looks
for the necessary files in so-called |texmf| trees. These trees
are simply huge directories that contain these files. By default,
\TeX\ looks for files in three different |texmf| trees:
\begin{itemize}
\item
  The root |texmf| tree, which is usually located at
  |/usr/share/texmf/| or |c:\texmf\| or somewhere similar.
\item
  The local  |texmf| tree, which is usually located at
  |/usr/local/share/texmf/| or |c:\localtexmf\| or somewhere similar.
\item
  Your personal  |texmf| tree, which is usually located in your home
  directory at |~/texmf/| or |~/Library/texmf/|.   
\end{itemize}

You should install the packages either in the local tree or in
your personal tree, depending on whether you have write access to the
local tree. Installation in the root tree can cause problems, since an
update of the whole \TeX\ installation will replace this whole tree.


\subsubsection{Installation that Keeps Everything Together}

Once you have located the right texmf tree, you must decide whether
you want to install \pgfname\ in such a way that ``all its files are
kept in one place'' or whether you want to be
``\textsc{tds}-compliant,'' where \textsc{tds} means ``\TeX\ directory
structure.''

If you want to keep ``everything in one place,'' inside the |texmf|
tree that you have chosen create a sub-sub-directory called
|texmf/tex/generic/pgf| or
|texmf/tex/generic/pgf-|\texttt{\pgfversion}, if you prefer. Then
place all files of the |pgf| package in this directory. Finally,
rebuild \TeX's filename database. This is done by running the command
|texhash| or |mktexlsr| (they are the same). In Mik\TeX, there is a
menu option to do this. 


\subsubsection{Installation that is TDS-Compliant}

While the above installation process is the most ``natural'' one and
although I would like to recommend it since it makes updating and
managing the \pgfname\ package easy, it is not
\textsc{tds}-compliant. If you want to be \textsc{tds}-compliant,
proceed as follows: (If you do not know what \textsc{tds}-compliant
means, you probably do not want to be \textsc{tds}-compliant.)

The |.tar| file of the |pgf| package contains the following files and
directories at its root: |README|, |doc|,  |generic|, |plain|, and
|latex|. You should ``merge'' each of the four directories with the
following directories |texmf/doc|, |texmf/tex/generic|,
|texmf/tex/plain|, and |texmf/tex/latex|. For example, in the |.tar|
file the |doc| directory contains just the directory |pgf|, and this
directory has to be moved to |texmf/doc/pgf|. The root |README| file
can be ignored since it is reproduced in |doc/pgf/README|.

You may also consider keeping everything in one place and using
symbolic links to point from the \textsc{tds}-compliant directories to
the central installation.

\vskip1em
For a more detailed explanation of the standard installation process
of packages, you might wish to consult
\href{http://www.ctan.org/installationadvice/}{|http://www.ctan.org/installationadvice/|}.
However, note that the \pgfname\ package does not come with a
|.ins| file (simply skip that part).


\subsection{Updating the Installation}

To update your installation from a previous version, all you need to
do is to replace everything in the directory |texmf/tex/generic/pgf|
with the files of the new version (or in all the directories where
|pgf| was installed, if you chose a \textsc{tds}-compliant
installation). The easiest way to do this is to first delete the old
version and then proceed as described above. Sometimes, there are
changes in the syntax of certain command from version to version. If
things no longer work that used to work, you may wish to have a look
at the release notes and at the change log. 



���}	� �
����DEV��INO��SYN��SV~�pgfmanual-en-license.tex�J��Jǀ�J�j��:P�_.j�
% Copyright 2006 by Till Tantau
%
% This file may be distributed and/or modified
%
% 1. under the LaTeX Project Public License and/or
% 2. under the GNU Free Documentation License.
%
% See the file doc/generic/pgf/licenses/LICENSE for more details.


\section{Input and Output Formats}
\label{section-formats}


\TeX\ was designed to be a flexible system. This is true both for the
\emph{input} for \TeX\ as well as for the \emph{output}. The present
section explains which input formats there are and how they are
supported by \pgfname. It also explains which different output formats
can be produced.



\subsection{Supported Input Formats}

\TeX\ does not prescribe exactly how your input should be
formatted. While it is \emph{customary} that, say, an opening brace
starts a scope in \TeX, this is by no means necessary. Likewise, it is
\emph{customary} that environments start with |\begin|, but \TeX\
could not really care less about the exact command name.

Even though \TeX\ can be reconfigured, users can not. For this reason,
certain \emph{input formats} specify a set of commands and conventions
how input for \TeX\ should be formatted. There are currently three
``major'' formats: Donald Knuth's original |plain| \TeX\ format,
Leslie Lamport's popular \LaTeX\ format, and Hans Hangen's Con\TeX t
format.


\subsubsection{Using the  \LaTeX\ Format}

Using \pgfname\ and \tikzname\ with the \LaTeX\ format is easy: You
say |\usepackage{pgf}| or |\usepackage{tikz}|. Usually, that is all
you need to do, all configuration will be done automatically and
(hopefully) correctly.

The style files used for the \LaTeX\ format reside in the subdirectory
|latex/pgf/| of the \pgfname-system. Mainly, what these files do is to
include files in the directory |generic/pgf|. For example, here is the
content of the file |latex/pgf/frontends/tikz.sty|:

\begin{codeexample}[code only]
% Copyright 2006 by Till Tantau
%
% This file may be distributed and/or modified
%
% 1. under the LaTeX Project Public License and/or
% 2. under the GNU Public License.
%
% See the file doc/generic/pgf/licenses/LICENSE for more details.


\RequirePackage{pgf,pgffor}


��
�}	� �
����DEV��INO��SYN��SV~�tikz.code.tex�K��K���K�lf��P�_.lg

\endinput
\end{codeexample}

The files in the |generic/pgf| directory do the actual work.



\subsubsection{Using the Plain \TeX\ Format}

When using the plain \TeX\ format, you say |% Copyright 2006 by Till Tantau
%
% This file may be distributed and/or modified
%
% 1. under the LaTeX Project Public License and/or
% 2. under the GNU Public License.
%
% See the file doc/generic/pgf/licenses/LICENSE for more details.


\edef\pgfatcode{\the\catcode`\@}
\catcode`\@=11


\input pgfrcs.tex
\ProvidesPackageRCS $Header: /cvsroot/pgf/pgf/plain/pgf/basiclayer/pgf.tex,v 1.10 2008/01/15 17:17:22 tantau Exp $

\input pgfcore.tex

\usepgfmodule{shapes,plot}

%\input pgfbasesnakes.tex
%\input pgfbasedecorations.tex
%\input pgfbasematrix.tex

\catcode`\@=\pgfatcode

\endinput
| or
|
���}	� �
����DEV��INO��SYN��SV~�tikz.tex�L��L���L�nG�_P�_.nH|. Then, instead of  |\begin{pgfpicture}| and
  |\end{pgfpicture}| you use  |\pgfpicture| and |\endpgfpicture|. 

Unlike for the \LaTeX\ format, \pgfname\ is not as good at discerning
the appropriate configuration for the plain \TeX\ format. In
particular, it can only automatically determine the correct output
format if you use |pdftex| or |tex| plus |dvips|. For all other output
formats you need to set the macro |\pgfsysdriver| to the correct
value. See the description of using output formats later on. 

\pgfname\ was originally written for use with \LaTeX\ and this shows
in a number of places. Nevertheless, the plain \TeX\ support is
reasonably good.

Like the \LaTeX\ style files, the plain \TeX\ files like |tikz.tex|
also just include the correct |tikz.code.tex| file.



\subsubsection{Using the Con\TeX t Format}

When using the Con\TeX t format, you say |\usemodule[pgf]| or
|\usemodule[tikz]|. As for the plain \TeX\ format you also have
to replace the start- and end-of-environment tags as follows: Instead
of  |\begin{pgfpicture}| and |\end{pgfpicture}| you use
|\startpgfpicture| and |\stoppgfpicture|; similarly, instead of
|\begin{tikzpicture}| and |\end{tikzpicture}| you use must now use
|\starttikzpicture| and |\stoptikzpicture|; and so on for other
environments. 

The Con\TeX t support is very similar to the plain \TeX\ support, so
the same restrictions apply: You may have to set the output
format directly and graphics inclusion may be a problem.

In addition to |pgf| and |tikz| there also exist modules like
|pgfcore| or |pgfmodulematrix|. To
use them, you may need to include the module |pgfmod| first (the
modules |pgf| and |tikz| both include |pgfmod| for you, so typically
you can skip this). This special module is necessary since Con\TeX t
satanically restricts the length of module names to 6 characters
and \pgfname's long names are mapped
to cryptic 6-letter-names for you by the module |pgfmod|.





\subsection{Supported Output Formats}
\label{section-drivers}

An output format is a format in which \TeX\ outputs the text it has
typeset. Producing the output is (conceptually) a two-stage process:
\begin{enumerate}
\item
  \TeX\ typesets your text and graphics. The result of this
  typesetting is mainly a long list of letter--coordinate pairs, plus 
  (possibly) some ``special'' commands. This long list of pairs
  is written to something called a |.dvi|-file.
\item
  Some other program reads this |.dvi|-file and translates the
  letter--coordinate pairs into, say, PostScript commands for placing
  the given letter at the given coordinate.
\end{enumerate}

The classical example of this process is the combination of |latex|
and |dvips|. The |latex| program (which is just the |tex| program
called with the \LaTeX-macros preinstalled) produces a |.dvi|-file as
its output. The |dvips| program takes this output and produces a
|.ps|-file (a PostScript) file. Possibly, this file is further
converted using, say, |ps2pdf|, whose name is supposed to mean
``PostScript to PDF.'' Another example of programs using this
process is the combination of |tex| and |dvipdfm|. The |dvipdfm|
program takes a |.dvi|-file as 
input and translates the letter--coordinate pairs therein into
\pdf-commands, resulting in a |.pdf| file directly. Finally, the
|tex4ht| is also a program that takes a |.dvi|-file and produces an
output, this time it is a |.html| file. The programs |pdftex| and
|pdflatex| are special: They directly produce a |.pdf|-file without
the intermediate |.dvi|-stage. However, from the programmer's point of
view they behave exactly as if there where an intermediate stage.

Normally, \TeX\ only produces letter--coordinate pairs as its
``output.'' This obviously makes is difficult to draw, say, a
curve. For this, ``special'' commands can be used. Unfortunately,
these special commands are not the same for the different programs
that process the |.dvi|-file. Indeed, every program that takes a
|.dvi|-file as input has a totally different syntax for the special
commands.

One of the main jobs of \pgfname\ is to ``abstract way'' the
difference in the syntax of the different programs. However, this
means that support for each program has to be ``programmed,'' which is
a time-consuming and complicated process. 


\subsubsection{Selecting the Backend Driver}

When \TeX\ typesets your document, it does not know which program
you are going to use to transform the |.dvi|-file. If your |.dvi|-file
does not contain any special commands, this would be fine; but these
days almost all |.dvi|-files contain lots of special commands. It is
thus necessary to tell \TeX\ which program you are going to use later
on.

Unfortunately, there is no ``standard'' way of telling this to
\TeX. For the \LaTeX\ format a sophisticated mechanism exists inside
the |graphics| package and \pgfname\ plugs into this mechanism. For
other formats and when this plugging does not work as expected, it is
necessary to tell \pgfname\ directly which program you are going to
use. This is done by redefining the macro |\pgfsysdriver| to an
appropriate value \emph{before} you load |pgf|. If you are going to
use the |dvips| program, you set this macro to the value
|pgfsys-dvips.def|; if you use |pdftex| or |pdflatex|, you set it to
|pgfsys-pdftex.def|; and so on. In the following, details of the
support of the different programs are discussed.


\subsubsection{Producing PDF Output}

\pgfname\ supports three programs that produce \pdf\ output (\pdf\ means
``portable document format'' and was invented by the Adobe company):
|dvipdfm|, |pdftex|, and |vtex|. The |pdflatex| program is the same as the
|pdftex| program: it uses a different input format, but the output is
exactly the same.

\begin{filedescription}{pgfsys-pdftex.def}
  This is the driver file for use with pdf\TeX, that is, with the
  |pdftex| or |pdflatex| command. It includes
  |pgfsys-common-pdf.def|.

  This driver has the ``complete'' functionality. This means,
  everything \pgfname\ ``can do at all'' is implemented in this
  driver. 
\end{filedescription}

\begin{filedescription}{pgfsys-dvipdfm.def}
  This is a driver file for use with (|la|)|tex| followed by |dvipdfm|. It
  includes |pgfsys-common-pdf.def|.

  This driver supports most of \pgfname's features, but there are some
  restrictions:
  \begin{enumerate}
  \item
    In \LaTeX\ mode it uses |graphicx| for the graphics
    inclusion and does not support masking.
  \item
    In plain \TeX\ mode it does not support image inclusion.
  \item
    For remembering of pictures (inter-picture connections) you need
    to use a recent version of |pdftex| running in DVI-mode.
  \item
    Patterns are not (cannot) be supported.
  \item
    Functional shadings are not (cannot) be supported.
  \end{enumerate}
\end{filedescription}

\begin{filedescription}{pgfsys-xetex.def}
  This is a driver file for use with |xe|(|la|)|tex| followed by
  |xdvipdfmx|. This driver supports the same operations as the dvipdfm
  driver,  except that remembering of pictures (inter-picture
  connections)   always works.
\end{filedescription}

\begin{filedescription}{pgfsys-vtex.def}
  This is the driver file for use with the commercial \textsc{vtex}
  program. Even though it produces  \textsc{pdf} output, it
  includes |pgfsys-common-postscript.def|. Note that the
  \textsc{vtex} program can produce \emph{both} Postscript and
  \textsc{pdf} output, depending on the command line
  parameters. However, whether you produce Postscript or
  \textsc{pdf} output does not change anything with respect to the
  driver. 

  This driver supports most of \pgfname's features, except for
  the following restrictions:
  \begin{enumerate}
  \item
    In \LaTeX\ mode it uses |graphicx| for the graphics
    inclusion and does not support masking.
  \item
    In plain \TeX\ mode it does not support image inclusion.
  \item
    Shading is fully implemented, but yields the same quality as the
    implementation for |dvips|.
  \item
    Opacity is not supported.
  \item
    Remembering of pictures (inter-picture connections) is not
    supported. 
  \end{enumerate}
\end{filedescription}

It is also possible to produce a |.pdf|-file by first producing a
PostScript file (see below) and then using a PostScript-to-\pdf\
conversion program like |ps2pdf| or the Acrobat Distiller.


\subsubsection{Producing PostScript Output}

\begin{filedescription}{pgfsys-dvips.def}
  This is a driver file for use with (|la|)|tex| followed by
  |dvips|. It includes |pgfsys-common-postscript.def|.

  This driver also supports most of \pgfname's features, except for
  the following restrictions:
  \begin{enumerate}
  \item
    In \LaTeX\ mode it uses |graphicx| for the graphics
    inclusion and does not support masking.
  \item
    In plain \TeX\ mode it does not support image inclusion.
  \item
    Shading is fully implemented, but the results will not be 
    as good as with a driver producing |.pdf| as output. 
  \item
    Opacity works only in conjunction with newer versions of
    GhostScript. 
  \item
    For remembering of pictures (inter-picture connections) you need
    to use a recent version of |pdftex| running in DVI-mode.
  \end{enumerate}
\end{filedescription}

\begin{filedescription}{pgfsys-textures.def}
  This is a driver file for use with the \textsc{textures} program. It
  includes |pgfsys-common-postscript.def|. 

  This driver has exactly the same restrictions as the driver for
  |dvips|. 
\end{filedescription}

You can also use the |vtex| program together with |pgfsys-vtex.def| to
produce Postscript output.



\subsubsection{Producing HTML / SVG Output}

The |tex4ht| program converts |.dvi|-files to |.html|-files. While the
\textsc{html}-format cannot be used to draw graphics, the
\textsc{svg}-format can. Using the following driver, you can ask
\pgfname\ to produce an \textsc{svg}-picture for each \pgfname\
graphic in your text.

\begin{filedescription}{pgfsys-tex4ht.def}
  This is a driver file for use with the |tex4ht| program. It includes
  |pgfsys-common-svg.def|.

  When using this driver you should be aware of the following
  restrictions: 
  \begin{enumerate}
  \item
    In \LaTeX\ mode it uses |graphicx| for the graphics
    inclusion.    
  \item
    In plain \TeX\ mode it does not support image inclusion.
  \item
    Remembering of pictures (inter-picture connections) is not
    supported. 
  \item
    Text inside |pgfpicture|s is not supported very well. The reason
    is that the \textsc{svg} specification currently does not support
    text very well and it is also not possible to correctly ``escape
    back'' to \textsc{html}. All these problems will hopefully
    disappear in the future, but currently only two kinds of text work
    reasonably well: First, plain text without math mode, special
    characters or anything else special. Second, \emph{very} simple
    mathematical text that contains subscripts or superscripts. Even
    then, variables are not correctly set in italics and, in general,
    text simple does not look very nice.
  \item
    If you use text that contains anything special, even something as
    simple as |$\alpha$|, this may corrupt the graphic since |text4ht|
    does not always produce valid \textsc{xml} code. So, once more,
    \emph{stick to very simple node text inside graphics.} Sorry.
  \item
    Unlike for other output formats, the bounding box of a picture
    ``really crops'' the picture.
  \item
    Matrices do not work.
  \item
    Functional shadings are not supported.
  \end{enumerate}

  The driver basically works as follows: When a |{pgfpicture}| is
  started, appropriate |\special| commands are used to directed the
  output of |tex4ht| to a new file called |\jobname-xxx.svg|, where
  |xxx| is a number that is increased for each graphic. Then, till the
  end of the picture, each (system layer) graphic command creates a
  special that inserts appropriate \textsc{svg} literal text into the
  output file. The exact details are a bit complicated since the
  imaging model and the processing model of PostScript/\pdf\ and
  \textsc{svg} are not quite the same; but they are ``close enough''
  for \pgfname's purposes.
\end{filedescription}


\subsubsection{Producing Perfectly Portable DVI Output}

\begin{filedescription}{pgfsys-dvi.def}
  This is a driver file that can be used with any output driver,
  except for |tex4ht|.

  The driver will produce perfectly portable |.dvi| files by composing
  all pictures entirely of black rectangles, the basic and only graphic
  shape supported by the \TeX\ core. Even straight, but slanted lines
  are tricky to get right in this model (they need to be composed of
  lots of little squares).

  Naturally, \emph{very little} is possible with this driver. In fact,
  so little is possible that it is easier to list what is possible:
  \begin{itemize}
  \item Text boxes can be placed in the normal way.
  \item Lines and curves can be drawn (stroked). If they are not
    horizontal or vertical, they are composed of hundred of small
    rectangles.
  \item Lines of different width are supported.
  \item Transformations are supported.
  \end{itemize}
  Note that, say, even filling is not supported! (Let alone color or
  anything fancy.)

  This driver has only one real application: It might be useful when
  you only need horizontal or vertical lines in a picture. Then, the
  results are quite satisfactory.
\end{filedescription}



\part{Ti\emph{k}Z ist \emph{kein} Zeichenprogramm}
\label{part-tikz}

{\Large \emph{by Till Tantau}}


\bigskip
\noindent
\vskip3cm
\begin{codeexample}[graphic=white]
\begin{tikzpicture}
  \draw[fill=yellow] (0,0) -- (60:.75cm) arc (60:180:.75cm);
  \draw(120:0.4cm) node {$\alpha$};

  \draw[fill=green!30] (0,0) -- (right:.75cm) arc (0:60:.75cm);
  \draw(30:0.5cm) node {$\beta$};

  \begin{scope}[shift={(60:2cm)}]
    \draw[fill=green!30] (0,0) -- (180:.75cm) arc (180:240:.75cm);
    \draw (30:-0.5cm) node {$\gamma$};

    \draw[fill=yellow] (0,0) -- (240:.75cm) arc (240:360:.75cm);
    \draw (-60:0.4cm) node {$\delta$};
  \end{scope}

  \begin{scope}[thick]
    \draw  (60:-1cm) node[fill=white] {$E$} -- (60:3cm) node[fill=white] {$F$};
    \draw[red]                   (-2,0) node[left] {$A$} -- (3,0) node[right]{$B$};
    \draw[blue,shift={(60:2cm)}] (-3,0) node[left] {$C$} -- (2,0) node[right]{$D$};
  
    \draw[shift={(60:1cm)},xshift=4cm]
    node [right,text width=6cm,rounded corners,fill=red!20,inner sep=1ex]
    {
      When we assume that $\color{red}AB$ and $\color{blue}CD$ are
      parallel, i.\,e., ${\color{red}AB} \mathbin{\|} \color{blue}CD$,
      then $\alpha = \delta$ and $\beta = \gamma$.
    };
  \end{scope}
\end{tikzpicture}
\end{codeexample}



% Copyright 2006 by Till Tantau
%
% This file may be distributed and/or modified
%
% 1. under the LaTeX Project Public License and/or
% 2. under the GNU Free Documentation License.
%
% See the file doc/generic/pgf/licenses/LICENSE for more details.

\section{Design Principles}

This section describes the design principles behind the \tikzname\
frontend, where \tikzname\ means ``\tikzname\ ist \emph{kein}
Zeichenprogramm.'' To use \tikzname, as a \LaTeX\ user say
|\usepackage{tikz}| somewhere in the preamble, as a plain \TeX\ user
say |\input tikz.tex|. \tikzname's job is to make your life easier by
providing an easy-to-learn and easy-to-use syntax for describing
graphics. 

The commands and syntax of \tikzname\ were influenced by several
sources. The basic command names and the notion of  path operations is
taken from \textsc{metafont}, the option mechanism comes from
\textsc{pstricks}, the notion of styles is reminiscent of
\textsc{svg}. To make it all work together, some compromises were
necessary. I also added some ideas of my own, like coordinate
transformations.  

The following basic design principles underlie \tikzname:
\begin{enumerate}
\item Special syntax for specifying points.
\item Special syntax for path specifications.
\item Actions on paths.
\item Key-value syntax for graphic parameters.
\item Special syntax for nodes.
\item Special syntax for trees.
\item Grouping of graphic parameters.
\item Coordinate transformation system.
\end{enumerate}



\subsection{Special Syntax For Specifying Points}

\tikzname\ provides a special syntax for specifying points and
coordinates. In the simplest case, you provide two \TeX\ dimensions,
separated by commas, in round brackets as in |(1cm,2pt)|.

You can also specify a point in polar coordinates by using a colon
instead of a comma as in |(30:1cm)|, which means ``1cm in a 30
degrees direction.'' 

If you do not provide a unit, as in |(2,1)|, you specify a point in
\pgfname's $xy$-coordinate system. By default, the unit $x$-vector
goes 1cm to the right and the unit $y$-vector goes 1cm upward.

By specifying three numbers as in |(1,1,1)| you specify a point in
\pgfname's $xyz$-coordinate system.

It is also possible to use an anchor of a previously defined shape
as in |(first node.south)|.

You can add two plus signs before a coordinate as in
|++(1cm,0pt)|. This means ``1cm to the right of the last point
used.'' This allows you to easily specify relative movements. For
example, |(1,0) ++(1,0) ++(0,1)| specifies the three coordinates
|(1,0)|, then |(2,0)|, and |(2,1)|.

Finally, instead of two plus signs, you can also add a single
one. This also specifies a point in a relative manner, but it does
not ``change'' the current point used in subsequent relative
commands. For example, |(1,0) +(1,0) +(0,1)| specifies the three
coordinates |(1,0)|, then |(2,0)|, and |(1,1)|.

\subsection{Special Syntax For Path Specifications}

When creating a picture using \tikzname, your main job is the
specification of \emph{paths}. A path is a series of straight or curved
lines, which need not be connected. \tikzname\ makes it easy to
specify paths, partly using the syntax of \textsc{metapost}. For
example, to specify a triangular path you use
\begin{codeexample}[code only]
(5pt,0pt) -- (0pt,0pt) -- (0pt,5pt) -- cycle
\end{codeexample}
and you get \tikz \draw (5pt,0pt) -- (0pt,0pt) -- (0pt,5pt) -- cycle;
when you draw this path.

\subsection{Actions on Paths}

A path is just a series of straight and curved lines, but it is not
yet specified what should happen with it. One can \emph{draw} a
path, \emph{fill} a path, \emph{shade} it, \emph{clip} it, or do any
combination of these. Drawing (also known as \emph{stroking}) can be
thought of as taking a pen of a certain thickness and moving it
along the path, thereby drawing on the canvas. Filling means that
the interior of the path is filled with a uniform color. Obviously,
filling makes sense only for \emph{closed} paths and a path is
automatically closed prior to filling, if necessary.

Given a path as in |\path (0,0) rectangle (2ex,1ex);|, you can draw
it by adding the |draw| option as in
|\path[draw] (0,0) rectangle (2ex,1ex);|, which yields \tikz \path[draw]
(0,0) rectangle (2ex,1ex);. The |\draw| command is just an abbreviation for
|\path[draw]|. To fill a path, use the |fill| option or the |\fill|
command, which is an abbreviation for |\path[fill]|. The
|\filldraw| command is an abbreviation for
|\path[fill,draw]|. Shading is caused by the |shade| option (there
are |\shade| and |\shadedraw| abbreviations) and clipping by the
|clip| option. There is is also a |\clip| command, which does the
same as |\path[clip]|, but not commands like |\drawclip|. Use, say,
|\draw[clip]| or |\path[draw,clip]| instead.

All of these commands can only be used inside |{tikzpicture}|
environments. 

\tikzname\ allows you to use different colors for filling and
stroking.

\subsection{Key-Value Syntax for Graphic Parameters}

Whenever \tikzname\ draws or fills a path, a large number of graphic
parameters influenced the rendering. Examples include the colors
used, the dashing pattern, the clipping area, the line width, and
many others. In \tikzname, all these options are specified as lists
of so called key-value pairs, as in |color=red|, that are
passed as optional parameters to the path drawing and filling
commands. This usage is similar to \textsc{pstricks}. For
example, the following will draw a thick, red triangle;
\begin{codeexample}[]
\tikz \draw[line width=2pt,color=red] (1,0) -- (0,0) -- (1,0) -- cycle; 
\end{codeexample}

\subsection{Special Syntax for Specifying Nodes}
\tikzname\ introduces a special syntax for adding text or, more
generally, nodes to a graphic. When you specify a path, add nodes as
in the following example:
\begin{codeexample}[]
\tikz \draw (1,1) node {text} -- (2,2);
\end{codeexample}
Nodes are inserted at the current position of
the path, but only \emph{after} the path has been rendered. When
special options are given, as in
|\draw (1,1) node[circle,draw] {text};|, the text is not just put 
at the current position. Rather, it is surrounded by a circle and
this circle is ``drawn.'' 

You can add a name to a node for later reference either by using the
option   |name=|\meta{node name} or by stating the node name in
parentheses outside the text as in |node[circle](name){text}|.

Predefined shapes include |rectangle|, |circle|, and |ellipse|, but
it is possible (though a bit challenging) to define new shapes.

\subsection{Special Syntax for Specifying Trees}

In addition to the ``node syntax,'' \tikzname\ also introduces a
special syntax for drawing trees. The syntax is intergrated with the
special node syntax and only few new commands need to be remebered.
In essence, a |node| can be followed by any number of children, each
introduced by the keyword |child|. The children are nodes themselves,
each of which may have children in turn.

\begin{codeexample}[]
\begin{tikzpicture}
  \node {root}
    child {node {left}}
    child {node {right}
      child {node {child}}
      child {node {child}}
    };
\end{tikzpicture}
\end{codeexample}

Since trees are made up from nodes, it is possible to use options to
modify the way trees are drawn. Here are two examples of the above tree,
redrawn with different options:

\begin{codeexample}[]
\begin{tikzpicture}
  [edge from parent fork down,
   every node/.style={fill=red!30,rounded corners},
   edge from parent/.style={red,-o,thick,draw}]
  \node {root}
      child {node {left}}
      child {node {right}
        child {node {child}}
        child {node {child}}
      };
\end{tikzpicture}
\end{codeexample}

\begin{codeexample}[]
\begin{tikzpicture}
  [parent anchor=east,child anchor=west,grow=east,
   every node/.style={ball color=red,circle,text=white}
   edge from parent/.style={draw,dashed,thick,red}]
  \node {root}
      child {node {left}}
      child {node {right}
        child {node {child}}
        child {node {child}}
      };
\end{tikzpicture}
\end{codeexample}

\subsection{Grouping of Graphic Parameters}

Graphic parameters should often apply to several path drawing or
filling commands. For example, we may wish to draw numerous lines all
with the same line width of 1pt. For this, we put these commands
in a |{scope}| environment that takes the desired graphic options
as an optional parameter. Naturally, the specified graphic
parameters apply only to the drawing and filling commands inside the
environment. Furthermore, nested |{scope}| environments or
individual drawing commands can override the graphic parameters of
outer |{scope}| environments. In the following example, three red
lines, two green lines, and one blue line are drawn:

\begin{codeexample}[]
\begin{tikzpicture}
  \begin{scope}[color=red]
    \draw (0mm,10mm) -- (10mm,10mm);
    \draw (0mm, 8mm) -- (10mm, 8mm);
    \draw (0mm, 6mm) -- (10mm, 6mm);
  \end{scope}
  \begin{scope}[color=green]
    \draw             (0mm, 4mm) -- (10mm, 4mm);
    \draw             (0mm, 2mm) -- (10mm, 2mm);
    \draw[color=blue] (0mm, 0mm) -- (10mm, 0mm);
  \end{scope}
\end{tikzpicture}
\end{codeexample}

The |{tikzpicture}| environment itself also behaves like a
|{scope}| environment, that is, you can specify graphic parameters
using an optional argument. These optional apply to all commands in
the picture.


\subsection{Coordinate Transformation System}

\tikzname\ supports both \pgfname's \emph{coordinate} transformation
system to perform transformations as well as \emph{canvas}
transformations, a more low-level transformation system. (For 
details on the difference between coordinate transformations and
canvas transformations see Section~\ref{section-design-transformations}.) 

The syntax is setup in such a way that is harder to use canvas
transformations than coordinate transformations. There are two reasons
for this: First, the canvas transformation must be used with great
care and often results in ``bad'' graphics with changing line width
and text in wrong sizes. Second, \pgfname\ looses track of where nodes
and shapes are positioned when canvas transformations are used.
So, in almost all circumstances, you should use coordinate
transformations rather than canvas transformations.


���}	� �
����DEV��INO��SYN��SV~�pgfmanual-en-tikz-scopes.tex�J��JӀ�J�j��dP�_.j�

��!�}	� �
����DEV��INO��SYN��SV~�pgfmanual-en-tikz-coordinates.tex�J��Jπ�J�j��XP�_.j�
% Copyright 2006 by Till Tantau
%
% This file may be distributed and/or modified
%
% 1. under the LaTeX Project Public License and/or
% 2. under the GNU Free Documentation License.
%
% See the file doc/generic/pgf/licenses/LICENSE for more details.

\section{Syntax for Path Specifications}

A \emph{path} is a series of straight and curved line segments. It is
specified following a |\path| command and the specification must
follow a special syntax, which is described in the subsections of the
present section.


\begin{command}{\path\meta{specification}|;|}
  This command is available only inside a |{tikzpicture}| environment.

  The \meta{specification} is a long stream of \emph{path
  operations}. Most of these path operations tell \tikzname\ how the path
  is build. For example, when you write |--(0,0)|, you use a
  \emph{line-to operation} and it means ``continue the path from
  wherever you are to the origin.''

  At any point where \tikzname\ expects a path operation, you can also
  give some graphic options, which is a list of options in brackets,
  such as |[rounded corners]|. These options can have different
  effects:
  \begin{enumerate}
  \item
    Some options take ``immediate'' effect and apply to all subsequent
    path operations on the path. For example, the |rounded corners|
    option will round all following corners, but not the corners
    ``before'' and if the |sharp corners| is given later on the path
    (in a new set of brackets), the rounding effect will end.

\begin{codeexample}[]
\tikz \draw (0,0) -- (1,1)
           [rounded corners] -- (2,0) -- (3,1)
           [sharp corners] -- (3,0) -- (2,1);
\end{codeexample}
    Another example are the transformation options, which also apply
    only to subsequent coordinates.
  \item
    The options that have immediate effect can be ``scoped'' by
    putting part of a path in curly braces. For example, the above
    example could also be written as follows:

\begin{codeexample}[]
\tikz \draw (0,0) -- (1,1)
           {[rounded corners] -- (2,0) -- (3,1)}
           -- (3,0) -- (2,1);
\end{codeexample}
  \item
    Some options only apply to the path as a whole. For example, the
    |color=| option for determining the color used for, say, drawing
    the path always applies to all parts of the path. If several
    different colors are given for different parts of the path, only
    the last one (on the outermost scope) ``wins'':
 
\begin{codeexample}[]
\tikz \draw (0,0) -- (1,1)
           [color=red] -- (2,0) -- (3,1)
           [color=blue] -- (3,0) -- (2,1);
\end{codeexample}

    Most options are of this type. In the above example, we would have
    had to ``split up'' the path into several |\path| commands:
\begin{codeexample}[]
\tikz{\draw (0,0) -- (1,1);
      \draw [color=red] (2,0) -- (3,1);
      \draw [color=blue] (3,0) -- (2,1);}
\end{codeexample}
  \end{enumerate}

  By default, the |\path| command does ``nothing'' with the
  path, it just ``throws it away.'' Thus, if you write
  |\path(0,0)--(1,1);|, nothing is drawn 
  in your picture. The only effect is that the area occupied by the
  picture is (possibly) enlarged so that the path fits inside the
  area. To actually ``do'' something with the path, an option like
  |draw| or |fill| must be given somewhere on the path. Commands like
  |\draw| do this implicitly.
  
  Finally, it is also possible to give \emph{node specifications} on a
  path. Such specifications can come at different locations, but they
  are always allowed when a normal path operation could follow. A node
  specification starts with |node|. Basically, the effect is to
  typeset the node's text as normal \TeX\ text and to place
  it at the ``current location'' on the path. The details are explained
  in Section~\ref{section-nodes}.

  Note, however, that the nodes are \emph{not} part of the path in any
  way. Rather, after everything has been done with the path what is
  specified by the path options (like filling and drawing the path due
  to a |fill| and a |draw| option somewhere in the
  \meta{specification}), the nodes are added in a post-processing
  step.   
  
  The following style influences scopes:
  \begin{stylekey}{/tikz/every path (initially \normalfont empty)}
    This style is installed at the beginning of every path. This can
    be useful for (temporarily) adding, say, the |draw| option to
    everything in a scope.
\begin{codeexample}[]
\begin{tikzpicture}
  [fill=examplefill,          % only sets the color
   every path/.style={draw}]  % all paths are drawn
  \fill  (0,0) rectangle +(1,1);
  \shade (2,0) rectangle +(1,1);
\end{tikzpicture}
\end{codeexample}
  \end{stylekey}

\end{command}




\subsection{The Move-To Operation}

The perhaps simplest operation is the move-to operation, which is
specified by just giving a coordinate where a path operation is
expected.

\begin{pathoperation}[noindex]{}{\meta{coordinate}}
  \index{empty@\protect\meta{empty} path operation}%
  \index{Path operations!empty@\protect\texttt{\meta{empty}}}%
  The move-to operation normally starts a path at a certain
  point. This does not cause a line segment to be created, but it  
  specifies the starting point of the next segment. If a path is
  already under construction, that is, if several segments have
  already been created, a move-to operation will start a new part of the
  path that is not connected to any of the previous segments.

\begin{codeexample}[]
\begin{tikzpicture}
  \draw (0,0) --(2,0) (0,1) --(2,1);
\end{tikzpicture}
\end{codeexample}

  In the specification |(0,0) --(2,0) (0,1) --(2,1)| two move-to
  operations are specified: |(0,0)| and |(0,1)|. The other two
  operations, namely |--(2,0)| and |--(2,1)| are line-to operations,
  described next.
\end{pathoperation}


\subsection{The Line-To Operation}


\subsubsection{Straight Lines}

\begin{pathoperation}{--}{\meta{coordinate}}
  The line-to operation extends the current path from the current
  point in a straight line to the given coordinate. The ``current
  point'' is the endpoint of the previous drawing operation or the point
  specified by a prior move-to operation.

  You use two minus signs followed by a coordinate in round
  brackets. You can add spaces before and after the~|--|.

  When a line-to operation is used and some path segment has just been
  constructed, for example by another line-to operation, the two line
  segments become joined. This means that if they are drawn, the point
  where they meet is ``joined'' smoothly. To appreciate the difference,
  consider the following two examples: In the left example, the path
  consists of two path segments that are not joined, but that happen to
  share a point, while in the right example a smooth join is shown.

\begin{codeexample}[]
\begin{tikzpicture}[line width=10pt]
  \draw (0,0) --(1,1)  (1,1) --(2,0);
  \draw (3,0) -- (4,1) -- (5,0);
  \useasboundingbox (0,1.5); % make bounding box higher
\end{tikzpicture}
\end{codeexample}

\end{pathoperation}


\subsubsection{Horizontal and Vertical Lines}

Sometimes you want to connect two points via straight lines that are
only horizontal and vertical. For this, you can use two path
construction operations.

{\catcode`\|=12
\begin{pathoperation}[noindex]{-|}{\meta{coordinate}}
  \index{--1@\protect\texttt{-\protect\pgfmanualbar} path operation}%
  \index{Path operations!--1@\protect\texttt{-\protect\pgfmanualbar}}%
  This operation means ``first horizontal, then vertical.''

  \begin{codeexample}[]
\begin{tikzpicture}
  \draw (0,0) node(a) [draw] {A}  (1,1) node(b) [draw] {B};
  \draw (a.north) |- (b.west);
  \draw[color=red] (a.east) -| (2,1.5) -| (b.north);
\end{tikzpicture}
\end{codeexample}
\end{pathoperation}
\begin{pathoperation}[noindex]{|-}{\meta{coordinate}}
  \index{--2@\protect\texttt{\protect\pgfmanualbar-} path operation}%
  \index{Path operations!--2@\protect\texttt{\protect\pgfmanualbar-}}%
  This operations means  ``first vertical, then horizontal.''
\end{pathoperation}
}


\subsection{The Curve-To Operation}

The curve-to operation allows you to extend a path using a B�zier
curve.

\begin{pathoperation}{..}{\declare{|controls|}\meta{c}\opt{|and|\meta{d}}\declare{|..|\meta{y}}}
  This operation extends the current path from the current
  point, let us call it $x$, via a curve to a the current point~$y$.
  The curve is a cubic B�zier curve. For such a curve, 
  apart from $y$, you also specify two control points $c$ and $d$. The
  idea is that the curve starts at $x$, ``heading'' in the direction
  of~$c$. Mathematically spoken, the tangent of the curve at $x$ goes
  through $c$. Similarly, the curve ends at $y$, ``coming from'' the
  other control point,~$d$. The larger the distance between $x$ and~$c$
  and between $d$ and~$y$, the larger the curve will be.

  If the ``|and|\meta{d}'' part is not given, $d$ is assumed to be
  equal to $c$.

\begin{codeexample}[]
\begin{tikzpicture}
  \draw[line width=10pt] (0,0) .. controls (1,1) .. (4,0)
                               .. controls (5,0) and (5,1) .. (4,1);
  \draw[color=gray] (0,0) -- (1,1) -- (4,0) -- (5,0) -- (5,1) -- (4,1);
\end{tikzpicture}
\end{codeexample}

  As with the line-to operation, it makes a difference whether two curves
  are joined because they resulted from consecutive curve-to or line-to
  operations, or whether they just happen to have the same ending:

\begin{codeexample}[]
\begin{tikzpicture}[line width=10pt]
  \draw (0,0) -- (1,1) (1,1) .. controls (1,0) and (2,0) .. (2,0);
  \draw (3,0) -- (4,1) .. controls (4,0) and (5,0) .. (5,0);
  \useasboundingbox (0,1.5); % make bounding box higher
\end{tikzpicture}
\end{codeexample}
\end{pathoperation}


\subsection{The Cycle Operation}

\begin{pathoperation}{--cycle}{}
  This operation adds a straight line from the current
  point to the last point specified by a move-to operation. Note that
  this need not be the beginning of the path. Furthermore, a smooth join
  is created between the first segment created after the last move-to
  operation and the straight line appended by the cycle operation.

  Consider the following example. In the left example, two triangles are
  created using three straight lines, but they are not joined at the
  ends. In the second example cycle operations are used.

\begin{codeexample}[]
\begin{tikzpicture}[line width=10pt]
  \draw (0,0) -- (1,1) -- (1,0) -- (0,0) (2,0) -- (3,1) -- (3,0) -- (2,0);
  \draw (5,0) -- (6,1) -- (6,0) -- cycle (7,0) -- (8,1) -- (8,0) -- cycle;
  \useasboundingbox (0,1.5); % make bounding box higher
\end{tikzpicture}
\end{codeexample}
\end{pathoperation}



\subsection{The Rectangle Operation}

A rectangle can obviously be created using four straight lines and a
cycle operation. However, since rectangles are needed so often, a
special syntax is available for them.

\begin{pathoperation}{rectangle}{\meta{corner}}
  When this operation is used, one corner will be the current point,
  another corner is given by \meta{corner}, which becomes the new
  current point.

\begin{codeexample}[]
\begin{tikzpicture}
  \draw (0,0) rectangle (1,1);
  \draw (.5,1) rectangle (2,0.5) (3,0) rectangle (3.5,1.5) -- (2,0);
\end{tikzpicture}
\end{codeexample}
\end{pathoperation}


\subsection{Rounding Corners}

All of the path construction operations mentioned up to now are
influenced by the following option:
\begin{key}{/tikz/rounded corners=\meta{inset} (default 4pt)}
  When this option is in force, all corners (places where a line is
  continued either via line-to or a curve-to operation) are replaced by
  little arcs so that the corner becomes smooth. 

\begin{codeexample}[]
\tikz \draw [rounded corners] (0,0) -- (1,1)
           -- (2,0) .. controls (3,1) .. (4,0);
\end{codeexample}

  The \meta{inset} describes how big the corner is. Note that the
  \meta{inset} is \emph{not} scaled along if you use a scaling option
  like |scale=2|. 

\begin{codeexample}[]
\begin{tikzpicture}
  \draw[color=gray,very thin] (10pt,15pt) circle (10pt);
  \draw[rounded corners=10pt] (0,0) -- (0pt,25pt) -- (40pt,25pt);
\end{tikzpicture}
\end{codeexample}

  You can switch the rounded corners on and off ``in the middle of
  path'' and different corners in the same path can have different
  corner radii:

\begin{codeexample}[]
\begin{tikzpicture}
  \draw (0,0) [rounded corners=10pt] -- (1,1) -- (2,1)
                     [sharp corners] -- (2,0)
               [rounded corners=5pt] -- cycle;
\end{tikzpicture}
\end{codeexample}

Here is a rectangle with rounded corners:
\begin{codeexample}[]
\tikz \draw[rounded corners=1ex] (0,0) rectangle (20pt,2ex);
\end{codeexample}

  You should be aware, that there are several pitfalls when using this
  option. First, the rounded corner will only be an arc (part of a
  circle) if the angle is $90^\circ$. In other cases, the rounded
  corner will still be round, but ``not as nice.''

  Second, if there are very short line segments in a path, the
  ``rounding'' may cause inadverted effects. In such case it may be
  necessary to temporarily switch off the rounding using
  |sharp corners|. 
\end{key}

\begin{key}{/tikz/sharp corners}
  This options switches off any rounding on subsequent corners of the
  path.   
\end{key}



\subsection{The Circle and Ellipse Operations}

A circle can be approximated well using four B�zier curves. However,
it is difficult to do so correctly. For this reason, a special syntax
is available for adding such an approximation of a circle to the
current path.

\begin{pathoperation}{circle}{|(|\meta{radius}|)|}
  The center of the circle is given by the current point. The new
  current point of the path will remain to be the center of the
  circle.  
\end{pathoperation}

\begin{pathoperation}{ellipse}{|(|\meta{half width}| and |\meta{half height}|)|}
  Note that you can add spaces after |ellipse|, but you have to place
  spaces around |and|.

\begin{codeexample}[]
\begin{tikzpicture}
  \draw (1,0) circle (.5cm);
  \draw (3,0) ellipse (1cm and .5cm) -- ++(3,0) circle (.5cm)
    -- ++(2,-.5) circle (.25cm);
\end{tikzpicture}
\end{codeexample}
\end{pathoperation}


\subsection{The Arc Operation}

The \emph{arc operation} allows you to add an arc to the current
path.
\begin{pathoperation}{arc}{|(|\meta{start angle}|:|\meta{end
    angle}|:|\meta{radius}\opt{| and |\meta{half height}}|)|}
  The arc operation adds a part of a circle of the given radius
  between the given angles. The arc will start at the current point
  and will end at the end of the arc.

  \begin{codeexample}[]
\begin{tikzpicture}
  \draw (0,0) arc (180:90:1cm) -- (2,.5) arc (90:0:1cm);
  \draw (4,0) -- +(30:1cm) arc (30:60:1cm) -- cycle;
  \draw (8,0) arc (0:270:1cm and .5cm) -- cycle;
\end{tikzpicture}
\end{codeexample}

\begin{codeexample}[]
\begin{tikzpicture}
  \draw (-1,0) -- +(3.5,0);
  \draw (1,0) ++(210:2cm) -- +(30:4cm);
  \draw (1,0) +(0:1cm) arc (0:30:1cm);      
  \draw (1,0) +(180:1cm) arc (180:210:1cm);
  \path (1,0) ++(15:.75cm) node{$\alpha$};
  \path (1,0) ++(15:-.75cm) node{$\beta$};
\end{tikzpicture}
\end{codeexample}
\end{pathoperation}


\subsection{The Grid Operation}

You can add a grid to the current path using the |grid| path
operation. 

\begin{pathoperation}{grid}{\opt{\oarg{options}}\meta{corner}}
  This operations adds a grid filling a rectangle whose two corners
  are given by \meta{corner} and by the previous coordinate. Thus, the
  typical way in which a grid is drawn is |\draw (1,1) grid (3,3);|,
  which yields a grid filling the rectangle whose corners are at
  $(1,1)$ and $(3,3)$. All coordinate transformations apply to the
  grid.

\begin{codeexample}[]
\tikz[rotate=30] \draw[step=1mm] (0,0) grid (2,2);
\end{codeexample}

  The \meta{options}, which are local to the |grid| operation, can be
  used to influence the appearance of the grid. The stepping of the
  grid is governed by the following options: 

\begin{key}{/tikz/step=\meta{number or dimension or coordinate}
    (initially 1cm)}
  Sets the stepping in both the
  $x$ and $y$-direction. If a dimension is provided, this is used
  directly. If a number is provided, this number is interpreted in the
  $xy$-coordinate system. For example, if you provide the number |2|,
  then the $x$-step is twice the $x$-vector and the $y$-step is twice
  the $y$-vector set by the |x=| and |y=| options. Finally, if you
  provide a coordinate, then the $x$-part of this coordinate will be
  used as the $x$-step and the $y$-part will be used as the
  $y$-coordinate.

\begin{codeexample}[]
\begin{tikzpicture}[x=.5cm]
  \draw[thick] (0,0) grid [step=1]     (3,2);
  \draw[red]   (0,0) grid [step=.75cm] (3,2);
\end{tikzpicture}
\begin{tikzpicture}
  \draw        (0,0) circle (1);
  \draw[blue]  (0,0) grid [step=(45:1)] (3,2);
\end{tikzpicture}
\end{codeexample}  

  A complication arises when the $x$- and/or $y$-vector do not point
  along the axes. Because of this, the actual rule for computing the
  $x$-step and the $y$-step is the following: As the $x$- and
  $y$-steps we use the $x$- and $y$-components or the following two
  vectors: The first vector is either $(\meta{x-grid-step-number},0)$
  or $(\meta{x-grid-step-dimension},0\mathrm{pt})$, the second vector
  is  $(0,\meta{y-grid-step-number})$ or
  $(0\mathrm{pt},\meta{x-grid-step-dimension})$. 
\end{key}  

\begin{key}{/tikz/xstep=\meta{dimension or number} (initially 1cm)}
  Sets the stepping in the $x$-direction. 
\begin{codeexample}[]
\tikz \draw (0,0) grid [xstep=.5,ystep=.75] (3,2);
\end{codeexample}
\end{key}

\begin{key}{/tikz/ystep=\meta{dimension or number} (initially 1cm)}
  Sets the stepping in the $y$-direction. 
\end{key}

  It is important to note that the grid is always ``phased'' such that
  it contains the point $(0,0)$ if that point happens to be inside the
  rectangle. Thus, the grid does \emph{not} always have an intersection
  at the corner points; this occurs only if the corner points are
  multiples of the stepping. Note that due to rounding errors, the
  ``last'' lines of a grid may be omitted. In this case, you have to
  add an epsilon to the corner points.

  The following style is useful for drawing grids:
\begin{stylekey}{/tikz/help lines (initially {line width=0.2pt,gray})}
  This style makes lines ``subdued'' by using thin gray lines for
  them. However, this style is not installed automatically and you
  have to say for example:
\begin{codeexample}[]
\tikz \draw[help lines] (0,0) grid (3,3);
\end{codeexample}
\end{stylekey}

\end{pathoperation}



\subsection{The Parabola Operation}

The |parabola| path operation continues the current path with a
parabola. A parabola is a (shifted and scaled) curve defined by the
equation $f(x) = x^2$ and looks like this: \tikz \draw (-1ex,1.5ex)
parabola[parabola height=-1.5ex] +(2ex,0ex);.

\begin{pathoperation}{parabola}{\opt{\oarg{options}|bend|\meta{bend
        coordinate}}\meta{coordinate}}
  This operation adds a parabola through the current point and the
  given \meta{coordinate}. If the |bend| is given, it specifies where
  the bend should go; the \meta{options} can also be used to specify
  where the bend is. By default, the bend is at the old current point. 

\begin{codeexample}[]
\begin{tikzpicture}
  \draw               (0,0) rectangle                (1,1.5)
                      (0,0) parabola                 (1,1.5);
  \draw[xshift=1.5cm] (0,0) rectangle                (1,1.5)
                      (0,0) parabola[bend at end]    (1,1.5);
  \draw[xshift=3cm]   (0,0) rectangle                (1,1.5)
                      (0,0) parabola bend (.75,1.75) (1,1.5);
\end{tikzpicture}
\end{codeexample}

  The following options influence parabolas:
\begin{key}{/tikz/bend=\meta{coordinate}}
  Has the same effect as saying |bend|\meta{coordinate} outside the
  \meta{options}. The option specifies that the bend of the parabola
  should be at the given \meta{coordinate}. You have to take care
  yourself that the bend position is a ``valid'' position; which means
  that if there is no parabola of the form $f(x) = a x^2 + b x + c$
  that goes through the old current point, the given bend, and the new
  current point, the result will not be a parabola.

  There is one special property of the \meta{coordinate}: When a
  relative coordinate is given like |+(0,0)|, the position relative
  to which this coordinate is ``flexible.'' More precisely, this
  position lies somewhere on a line from the old current point to the
  new current point. The exact position depends on the next
  option.
\end{key}

\begin{key}{/tikz/bend pos=\meta{fraction}}
  Specifies where the ``previous'' point is relative to which the bend
  is calculated. The previous point will be at the \meta{fraction}th
  part of the line from the old current point to the new current
  point.

  The idea is the following: If you say |bend pos=0| and
  |bend +(0,0)|, the bend will be at the old current point. If you say
  |bend pos=1| and |bend +(0,0)|, the bend will be at the new current
  point. If you say |bend pos=0.5| and |bend +(0,2cm)| the bend will
  be 2cm above the middle of the line between the start and end
  point. This is most useful in situations such as the following:
\begin{codeexample}[]
\begin{tikzpicture}
  \draw[help lines] (0,0) grid (3,2);
  \draw (-1,0) parabola[bend pos=0.5] bend +(0,2) +(3,0);
\end{tikzpicture}
\end{codeexample}

  In the above example, the |bend +(0,2)| essentially means ``a
  parabola that is 2cm high'' and |+(3,0)| means ``and 3cm wide.''
  Since this situation arises often, there is a special shortcut
  option:
  \begin{key}{/tikz/parabola height=\meta{dimension}}
    This option has the same effect as
    |[bend pos=0.5,bend={+(0pt,|\meta{dimension}|)}]|. 
\begin{codeexample}[]
\begin{tikzpicture}
  \draw[help lines] (0,0) grid (3,2);
  \draw (-1,0) parabola[parabola height=2cm] +(3,0);
\end{tikzpicture}
\end{codeexample}
  \end{key}
\end{key}

The following styles are useful shortcuts:
\begin{stylekey}{/tikz/bend at start}
  This places the bend at the start of a
  parabola. It is a shortcut for the following options:
  |bend pos=0,bend={+(0,0)}|.
\end{stylekey}

\begin{stylekey}{/tikz/bend at end}
  This places the bend at the end of a parabola.
\end{stylekey}

\end{pathoperation}


\subsection{The Sine and Cosine Operation}

The |sin| and |cos| operations are similar to the |parabola|
operation. They, too, can be used to draw (parts of) a sine or cosine
curve.

\begin{pathoperation}{sin}{\meta{coordinate}}
  The effect of |sin| is to draw a scaled and shifted version of a sine
  curve in the interval $[0,\pi/2]$. The scaling and shifting is done in
  such a way that the start of the sine curve in the interval is at the
  old current point and that the end of the curve in the interval is at
  \meta{coordinate}. Here is an example that should clarify this:

\begin{codeexample}[]
\tikz \draw (0,0) rectangle (1,1)     (0,0) sin (1,1)
            (2,0) rectangle +(1.57,1) (2,0) sin +(1.57,1);
\end{codeexample}
\end{pathoperation}

\begin{pathoperation}{cos}{\meta{coordinate}}
  This operation works similarly, only a cosine in the interval
  $[0,\pi/2]$ is drawn. By correctly alternating |sin| and |cos|
  operations, you can create a complete sine or cosine curve:

\begin{codeexample}[]
\begin{tikzpicture}[xscale=1.57]
  \draw (0,0) sin (1,1) cos (2,0) sin (3,-1) cos (4,0) sin (5,1);
  \draw[color=red] (0,1.5) cos (1,0) sin (2,-1.5) cos (3,0) sin (4,1.5) cos (5,0);
\end{tikzpicture}
\end{codeexample}
\end{pathoperation}

Note that there is no way to (conveniently) draw an interval on a sine
or cosine curve whose end points are not multiples of $\pi/2$.



\subsection{The Plot Operation}

The |plot| operation can be used to append a line or curve to the path
that goes through a large number of coordinates. These coordinates are
either given in a simple list of coordinates, read from some file, or
they are computed on the fly.

Since the syntax and the behaviour of this command are a bit complex,
they are described in the separated Section~\ref{section-tikz-plots}.

\subsection{The To Path Operation}

The |to| operation is used to add a user-defined path
from the previous coordinate to the following coordinate. When you
write |(a) to (b)|, a straight line is added from |a|
to |b|, exactly as if you had written |(a) -- (b)|. However, if you
write |(a) to [out=135,in=45] (b)| a curve is added to the path,
which leaves at an angle of 135$^\circ$ at |a| and arrives at an angle
of 45$^\circ$ at |b|. This is because the options |in| and |out|
trigger a special path to be used instead of the straight line. 

\begin{pathoperation}{to}{\opt{|[|\meta{options}|]|}
    \opt{\meta{nodes}} |(|\meta{coordinate}|)|}

  This path operation inserts the path current set via the |to path|
  option at the current position. The \meta{options} can be used to
  modify (perhaps implicitly) the |to path| and to setup how the path
  will be rendered. 

  Before the |to path| is inserted, a number of macros are setup that
  can ``help'' the |to path|. These are |\tikztostart|,
  |\tikztotarget|, and |\tikztonodes|; they are explained in the
  following. 
  
  \medskip
  \textbf{Start and Target Coordinates.}\ \ 
  The |to| operation is always followed by a \meta{coordinate}, called
  the target coordinate. The macro |\tikztotarget| is set to this
  coordinate (without the parentheses). There is also a \emph{start
    coordinate}, which is the coordinate preceding the |to|
  operation. This coordinate can be accessed via the macro
  |\tikztostart|. In the following example, for the first |to|, the
  macro |\tikztostart| is |0pt,0pt| and the |\tikztotarget| is
  |0,2|. For the second |to|, the macro |\tikztostart| is |10pt,10pt|
  and |\tikztotarget| is |a|.
  
\begin{codeexample}[]
\begin{tikzpicture}
  \draw[help lines] (0,0) grid (3,2);
  
  \draw      (0,0)       to (0,2);
  \node      (a)         at (2,2) {a};
  \draw[red] (10pt,10pt) to (a);
\end{tikzpicture}
\end{codeexample}


  \medskip
  \textbf{Nodes on tos.}\ \
  It is possible to add nodes to the paths constructed by a |to|
  operation. To do so, you specify the nodes between the |to|
  keyword and the coordinate (if there are options to the |to|
  operation, these come first). The effect of |(a) to node {x} (b)| 
  (typically) is the same as if you had written
  |(a) -- node {x} (b)|, namely that the node is placed on the
  to. This can be used to add labels to tos: 
  
\begin{codeexample}[]
\begin{tikzpicture}
  \draw (0,0) to node [sloped,above] {x} (3,2);

  \draw (0,0) to[out=90,in=180] node [sloped,above] {x} (3,2);
\end{tikzpicture}
\end{codeexample}

  \medskip
  \textbf{Styles for nodes.}\ \
  In addition to the \meta{options} given after the |to| operation,
  the following style is also set at the beginning of the to path:
  \begin{stylekey}{/tikz/every to (initially \normalfont empty)}
    This style is installed at the beginning of every to. By
    default, it is set to |draw|. 
\begin{codeexample}[]
\tikz[every to/.style={draw,dashed}]
  \path (0,0) to (3,2);
\end{codeexample}
  \end{stylekey}

  \medskip
  \textbf{Options.}\ \ 
  The \meta{options} given with the |to| allow you to influence the
  appearance of the |to path|. Mostly, these options are used to
  change the |to path|. This can be used to change the path from a 
  straight line to, say, a curve.

  The path used is set using the following option:
  \begin{key}{/tikz/to path=\meta{path}}
    Whenever an |to| operation is used, the \meta{path} is
    inserted. More precisely, the following path is added:

    \begin{quote}
      |[every to,|\meta{options}|] |\meta{path}
    \end{quote}
  
    The \meta{options} are the options given to the |to| operation,
    the \meta{path} is the path set by this option |to path|.

    Inside the \meta{path}, different macros are used to reference the
    from- and to-coordinates. In detail, these are:
    \begin{itemize}
    \item \declare{|\tikztostart|} will expand to the from-coordinate
      (without the parentheses).
    \item \declare{|\tikztotarget|} will expand to the to-coordinate.
    \item \declare{|\tikztonodes|} will expand to the nodes between
      the |to| operation and the coordinate. Furthermore, these
      nodes will have the |pos| option set implicitly.      
    \end{itemize}

    Let us have a look at a simple example. The standard straight line
    for an to is achieved by the following \meta{path}:
    \begin{quote}
      |-- (\tikztotarget) \tikztonodes|
    \end{quote}

    Indeed, this is the default setting for the path. When we write
    |(a) to (b)|, the \meta{path} will expand to |(a) -- (b)|, when
    we write
    \begin{quote}
      |(a) to[red] node {x} (b)|
    \end{quote}
    the \meta{path} will expand to
    \begin{quote}
      |(a) -- (b) node[pos] {x}|
    \end{quote}

    It is not possible to specify the path
    \begin{quote}
      |-- \tikztonodes (\tikztotarget)|
    \end{quote}
    since \tikzname\ does not allow one to have a macro after |--|
    that expands to a node.

    Now let us have a look at how we can modify the \meta{path}
    sensibly. The simplest way is to use a curve.
    
\begin{codeexample}[]
\begin{tikzpicture}[to path={
    .. controls +(1,0) and +(1,0) .. (\tikztotarget) \tikztonodes}]

  \node (a) at (0,0) {a};
  \node (b) at (2,1) {b};
  \node (c) at (1,2) {c};
                  
  \draw (a) to node {x} (b)
        (a) to          (c);
\end{tikzpicture}
\end{codeexample}

    Here is another example:

\begin{codeexample}[]
\tikzset{
  my loop/.style={->,to path={
    .. controls +(80:1) and +(100:1) .. (\tikztotarget) \tikztonodes}},
  my state/.style={circle,draw}}
                           
\begin{tikzpicture}[shorten >=2pt]
  \node [my state] (a) at (210:1) {$q_a$};
  \node [my state] (b) at (330:1) {$q_b$};
                  
  \draw (a) to           node[below]       {1} (b)
            to [my loop] node[above right] {0} (b);
\end{tikzpicture}
\end{codeexample}

    \begin{key}{/tikz/execute at begin to=\meta{code}}
      The \meta{code} is executed prior to the to. This can be used to
      draw one or more additional paths or to do additional
      computations.
    \end{key}

    \begin{key}{/tikz/executed at end to=\meta{code}}
      Works like the previous option, only this code is executed after
      the to path has been added.
    \end{key}


    \begin{stylekey}{/tikz/every to (initially \normalfont empty)}
      This style is installed at the beginning of every to.
    \end{stylekey}
  \end{key}
\end{pathoperation}

There are a number of predefined |to path|s, see
Section~\ref{library-to-paths} for a reference.


\subsection{The Let Operation}

The \emph{let operation} is the first of a number of path operations
that do not actually extend that path, but have different, mostly
local, effects.

\begin{pathoperation}{let}{\meta{assignment}
    \opt{|,|\meta{assignment}}%
    \opt{|,|\meta{assignment}\dots}\declare{| in |}}
  When this path operation is encountered, the \meta{assignment}s are 
  evaluated, one by one. This will store coordinate and number in
  special \emph{registers} (which are local to \tikzname, they have
  nothing to do with \TeX\ registers). Subsequently, one can access the
  contents of these registers using the macros |\p|, |\x|, |\y|, and
  |\n|.

  The first kind of permissible \meta{assignment}s have the following
  form:
  \begin{quote}
    |\n|\meta{number register}|={|\meta{formula}|}|
  \end{quote}
  When an assignment has this form, the \meta{formula} is evaluated
  using the |\pgfmathparse| operation. The result stored in the
  \meta{number register}. If the \meta{formula} involves a dimension
  anywhere (as in |2*3cm/2|), then the \meta{number register} stores
  the resulting dimension with a trailing |pt|.  A \meta{number register} can
  be named arbitrarily and is a normal \TeX\  parameter to the |\n|
  macro. Possible names are |{left corner}|, but also just a single
  digit like~|5|.  

  Let us call the path that follows a let operation its
  \emph{body}. Inside the body, the |\n| macro can be used to access
  the register.
  \begin{command}{\n\marg{number register}}
    When this macro is used on the left-hand side of an |=|-sign in a
    let operation, it has no effect and is just there for
    readability. When the macro is used on the right-hand side of an
    |=|-sign or in the body of the let operation, then it expands to
    the value stored in the \meta{number register}. This will either
    be a dimensionless number like |2.0| or a dimension like |5.6pt|.

    For instance, if we say |let \n1={1pt+2pt}, \n2={1+2} in ...|, then
    inside the |...| part the macro |\n1| will expand to |3pt| and
    |\n2| expands to |3|.
  \end{command}

  The second kind of \meta{assignements} have the following form:
  \begin{quote}
    |\p|\meta{point register}|={|\meta{formula}|}|
  \end{quote}
  Point position registers store a single point, consisting
  of an $x$-part and a $y$-part measured in \TeX\ points (|pt|). In
  particular, point registers do not stored nodes or node names.
  Here is an example:
\begin{codeexample}[]
\begin{tikzpicture}
  \draw [help lines] (0,0) grid (3,2);

  \draw let \p{foo} = (1,1), \p2 = (2,0) in
          (0,0) -- (\p2) -- (\p{foo});
\end{tikzpicture}
\end{codeexample}

  \begin{command}{\p\marg{point register}}
    When this macro is used on the left-hand side of an |=|-sign in a
    let operation, it has no effect and is just there for
    readability. When the macro is used on the right-hand side of an
    |=|-sign or in the body of the let operation, then it expands to
    the $x$-part (measured in \TeX\ points) of the coordinate stored
    in the \meta{register}, followed, by a comma, followed by the
    $y$-part. 

    For instance, if we say |let \p1=(1pt,1pt+2pt) in ...|, then
    inside the |...| part the macro |\p1| will expand to exactly the
    seven characters ``1pt,3pt''. This means that you when you write
    |(\p1)|, this expands to |(1pt,3pt)|, which is presumably exactly
    what you intended.
  \end{command}
  \begin{command}{\x\marg{point register}}
    This macro expand just to the $x$-part of the point
    register. If we say as above, as we did above,
    |let \p1=(1pt,1pt+2pt) in ...|, then
    inside the |...| part the macro |\x1| expands to |1pt|.
  \end{command}
  \begin{command}{\y\marg{point register}}
    Works like |\x|, only for the $y$-part.
  \end{command}
  Note that the above macros are available only inside a let
  operation. 

  Here is an example where let clauses are used to assemble a coordinate
  from the $x$-coordinate of a first point and the $y$-coordinate of a
  second point. Naturally, using the \verb!|-! notation, this could be
  written much more compactly.
\begin{codeexample}[]
\begin{tikzpicture}
  \draw [help lines] (0,0) grid (3,2);

  \draw    (1,0) coordinate (first point)
        -- (3,2) coordinate (second point);

  \fill[red] let \p1 = (first point),
                 \p2 = (second point) in
               (\x1,\y2) circle (2pt);
\end{tikzpicture}
\end{codeexample}

  Note that the effect of a let operation is local to the body of the
  let operation. If you wish to access a computed coordinate outside
  the body, you must use a |coordinate| path operation:
\begin{codeexample}[]
\begin{tikzpicture}
  \draw [help lines] (0,0) grid (3,2);

  \path % let's define some points:
    let
      \p1        = (1,0),
      \p2        = (3,2),
      \p{center} = ($ (\p1) !.5! (\p2) $) % center
    in
      coordinate (p1) at (\p1)
      coordinate (p2) at (\p2)
      coordinate (center) at (\p{center});
      
  \draw (p1) -- (p2);
  \fill[red] (center) circle (2pt);
\end{tikzpicture}
\end{codeexample}

  For a more useful application of the let operation, let use draw a
  circle that touches a given line:
\begin{codeexample}[]
\begin{tikzpicture}
  \draw [help lines] (0,0) grid (3,3);

  \coordinate (a) at (rnd,rnd);
  \coordinate (b) at (3-rnd,3-rnd);
  \draw (a) -- (b);

  \node (c) at (1,2) {x};

  \draw let \p1 = ($ (a)!(c)!(b) - (c) $),
            \n1 = {veclen(\x1,\y1)}
        in (c) circle (\n1);
\end{tikzpicture}
\end{codeexample}
\end{pathoperation}


\subsection{The Scoping Operation}

When \tikzname\ encounters and opening or a closing brace (|{| or~|}|) at
some point where a path operation should come, it will open or close a
scope. All options that can be applied ``locally'' will be scoped
inside the scope. For example, if you apply a transformation like
|[xshift=1cm]| inside the scoped area, the shifting only applies to
the scope. On the other hand, an option like |color=red| does not have
any effect inside a scope since it can only be applied to the path as
a whole.

Concerning the effect of scopes on relative coordinates,
please see Section~\ref{section-scopes-relative}.


\subsection{The Node and Edge Operations}

There are two more operations that can be found in paths:
|node| and |edge|. The first is used to add a so-called node to a
path. This operation is special in the following sense: It does not 
change the current path in any way. In other words, this operation 
is not really a path operation, but has an effect that is
``external''  to the path. The |edge| operation has similar effect in
that it adds something \emph{after} the main path has been
drawn. However, it works like the |to| operation, that is, it adds a
|to| path to the picture after the main path has been drawn.

Since these operations are quite complex, they are described in the
separate Section~\ref{section-nodes}.  



\subsection{The PGF-Extra Operation}

In some cases you may need to ``do some calculations or some other
stuff'' while a path is constructed. For this, you would like to
suspend the construction of the path and suspend \tikzname's parsing
of the path, you would then like to have some \TeX\ code executed, and
would then like to resume the parsing of the path. This effect can be
achieved using the following path operation |\pgfextra|. Note that
this operation should only be used by real experts and should only be
used deep inside clever macros, not on normal paths.

\begin{command}{\pgfextra\marg{code}}
  This command may only be used inside a \tikzname\ path. There it is
  used like a normal path operation. The construction of the path is
  temporarily suspended and the \meta{code} is executed. Then, the
  path construction is resumed.

\begin{codeexample}[]
\newdimen\mydim
\begin{tikzpicture}
  \mydim=1cm
  \draw (0pt,\mydim) \pgfextra{\mydim=2cm} -- (0pt,\mydim);
\end{tikzpicture}
\end{codeexample}
\end{command}

\begin{command}{\pgfextra \meta{code} \texttt{\char`\\endpgfextra}}
  This is an alternative syntax for the |\pgfextra| command. If the
  code following |\pgfextra| does not start with a brace, the
  \meta{code} is executed until |\endpgfextra| is encountered. What
  actually happens is that |\pgfextra| that is not followed by a brace
  completely shuts down the \tikzname\ parse and |\endpgfextra| is a
  normal macro that restarts the parser.

\begin{codeexample}[]
\newdimen\mydim
\begin{tikzpicture}
  \mydim=1cm
  \draw (0pt,\mydim)
    \pgfextra \mydim=2cm \endpgfextra -- (0pt,\mydim);
\end{tikzpicture}
\end{codeexample}
\end{command}

% Copyright 2006 by Till Tantau
%
% This file may be distributed and/or modified
%
% 1. under the LaTeX Project Public License and/or
% 2. under the GNU Free Documentation License.
%
% See the file doc/generic/pgf/licenses/LICENSE for more details.

\section{Actions on Paths}

\subsection{Overview}

Once a path has been constructed, different things can be done with
it. It can be drawn (or stroked) with a ``pen,'' it can be filled with
a color or shading, it can be used for clipping subsequent drawing, it
can be used to specify the extend of the picture---or  any
combination of these actions at the same time.

To decide what is to be done with a path, two methods can be
used. First, you can use a special-purpose command like |\draw| to
indicate that the path should be drawn. However, commands like |\draw|
and |\fill| are just abbreviations for special cases of the more
general method: Here, the |\path| command is used to specify the
path. Then, options encountered on the path indicate what should be
done with the path.

For example, |\path (0,0) circle (1cm);| means ``This is a path
consisting of a circle around the origin. Do not do anything with it
(throw it away).'' However, if the option |draw| is encountered
anywhere on the path, the circle will be drawn. ``Anywhere'' is any
point on the path where an option can be given, which is everywhere
where a path command like |circle (1cm)| or |rectangle (1,1)| or even
just |(0,0)| would also be allowed. Thus, the following commands all
draw the same circle:
\begin{codeexample}[code only]
\path [draw] (0,0) circle (1cm);
\path (0,0) [draw] circle (1cm);
\path (0,0) circle (1cm) [draw];
\end{codeexample}
Finally, |\draw (0,0) circle (1cm);| also draws a path, because
|\draw| is an abbreviation for |\path [draw]| and thus the command
expands to the first line of the above example.

Similarly, |\fill| is an abbreviation for |\path[fill]| and
|\filldraw| is an abbreviation for the command
|\path[fill,draw]|. Since options accumulate, the following commands
all have the same effect: 
\begin{codeexample}[code only]
\path [draw,fill]   (0,0) circle (1cm);
\path [draw] [fill] (0,0) circle (1cm);
\path [fill] (0,0) circle (1cm) [draw];
\draw [fill] (0,0) circle (1cm);
\fill (0,0) [draw] circle (1cm);
\filldraw (0,0) circle (1cm);
\end{codeexample}

In the following subsection the different actions are explained that
can be performed on a path. The following commands are abbreviations for
certain sets of actions, but for many useful combinations there are no
abbreviations:

\begin{command}{\draw}
  Inside |{tikzpicture}| this is an abbreviation for |\path[draw]|.
\end{command}

\begin{command}{\fill}
  Inside |{tikzpicture}| this is an abbreviation for |\path[fill]|.
\end{command}

\begin{command}{\filldraw}
  Inside |{tikzpicture}| this is an abbreviation for |\path[fill,draw]|.
\end{command}

\begin{command}{\pattern}
  Inside |{tikzpicture}| this is an abbreviation for |\path[pattern]|.
\end{command}

\begin{command}{\shade}
  Inside |{tikzpicture}| this is an abbreviation for |\path[shade]|.
\end{command}

\begin{command}{\shadedraw}
  Inside |{tikzpicture}| this is an abbreviation for |\path[shade,draw]|.
\end{command}

\begin{command}{\clip}
  Inside |{tikzpicture}| this is an abbreviation for |\path[clip]|.
\end{command}

\begin{command}{\useasboundingbox}
  Inside |{tikzpicture}| this is an abbreviation for |\path[use as bounding box]|.
\end{command}



\subsection{Specifying a Color}

The most unspecific option for setting colors is the following:

\begin{key}{/tikz/color=\meta{color name}}
  \indexoption{color option}%
  This option sets the color that is used for fill, drawing, and text
  inside the current scope. Any special settings for filling colors or
  drawing colors are immediately ``overruled'' by this option.

  The \meta{color name} is the name of a previously defined color. For
  \LaTeX\ users, this is just a normal ``\LaTeX-color'' and the
  |xcolor| extensions are allows. Here is an example:

\begin{codeexample}[]
\tikz \fill[color=red!20] (0,0) circle (1ex);
\end{codeexample}

  It is possible to ``leave out'' the |color=| part and you can also
  write:
\begin{codeexample}[]
\tikz \fill[red!20] (0,0) circle (1ex);
\end{codeexample}
  What happens is that every option that \tikzname\ does not know, like
  |red!20|, gets a ``second chance'' as a color name.

  For plain \TeX\ users, it is not so easy to specify colors since
  plain \TeX\ has no ``standardized'' color naming
  mechanism. Because of this, \pgfname\ emulates the |xcolor| package,
  though the emulation is \emph{extremely basic} (more precisely, what
  I could hack together in two hours or so). The emulation allows you
  to do the following:
  \begin{itemize}
  \item Specify a new color using |\definecolor|. Only the two color
    models |gray| and |rgb| are supported.
    \example |\definecolor{orange}{rgb}{1,0.5,0}|
  \item Use |\colorlet| to define a new color based on an old
    one. Here, the |!| mechanism is supported, though only ``once''
    (use multiple |\colorlet| for more fancy colors).
    \example |\colorlet{lightgray}{black!25}|
  \item Use |\color|\marg{color name} to set the color in the current
    \TeX\ group. |\aftergroup|-hackery is used to restore the color
    after the group.
  \end{itemize}
\end{key}

As pointed out above, the |color=| option applies to ``everything''
(except to shadings), which is not always what you want. Because of
this, there are several more specialized color options. For example,
the |draw=| option sets the color used for drawing, but does not
modify the color used for filling. These color options are documented
where the path action they influence is described.


\subsection{Drawing a Path}

You can draw a path using the following option:
\begin{key}{/tikz/draw=\meta{color} (default \normalfont is scope's color setting)}
  Causes the path to be drawn. ``Drawing'' (also known as
  ``stroking'') can be thought of as picking up a pen and moving it
  along the path, thereby leaving ``ink'' on the canvas.

  There are numerous parameters that influence how a line is drawn,
  like the thickness or the dash pattern. These options are explained
  below.

  If the optional \meta{color} argument is given, drawing is done
  using the given \meta{color}. This color can be different from the
  current filling color, which allows you to draw and fill a path with
  different colors. If no \meta{color} argument is given, the last
  usage of the |color=| option is used.

  If the special color name |none| is given, this option causes
  drawing to be ``switched off.'' This is useful if a style has
  previously switched on drawing and you locally wish to undo this
  effect. 

  Although this option is normally used on paths to indicate that the
  path should be drawn, it also makes sense to use the option with a
  |{scope}| or |{tikzpicture}| environment. However, this will
  \emph{not} cause all path to drawn. Instead, this just sets the
  \meta{color} to be used for drawing paths inside the environment.

\begin{codeexample}[]
\begin{tikzpicture}
  \path[draw=red] (0,0) -- (1,1) -- (2,1) circle (10pt);
\end{tikzpicture}
\end{codeexample}
\end{key}


The following subsections list the different options that influence
how a path is drawn. All of these options only have an effect if the
|draw| options is given (directly or indirectly).

\subsubsection{Graphic Parameters: Line Width, Line Cap, and Line Join}

\label{section-cap-joins}

\begin{key}{/tikz/line width=\meta{dimension} (initially 0.4pt)}
  Specifies the line width. Note the space.

\begin{codeexample}[]
  \tikz \draw[line width=5pt] (0,0) -- (1cm,1.5ex);
\end{codeexample}
\end{key}

There are a number of predefined styles that provide more ``natural''
ways of setting the line width. You can also redefine these
styles. 

\begin{stylekey}{/tikz/ultra thin}
  Sets the line width to 0.1pt.
\begin{codeexample}[]
  \tikz \draw[ultra thin] (0,0) -- (1cm,1.5ex);
\end{codeexample}
\end{stylekey}

\begin{stylekey}{/tikz/very thin}
  Sets the line width to 0.2pt.
\begin{codeexample}[]
  \tikz \draw[very thin] (0,0) -- (1cm,1.5ex);
\end{codeexample}
\end{stylekey}

\begin{stylekey}{/tikz/thin}
  Sets the line width to 0.4pt.
\begin{codeexample}[]
  \tikz \draw[thin] (0,0) -- (1cm,1.5ex);
\end{codeexample}
\end{stylekey}

\begin{stylekey}{/tikz/semithick}
  Sets the line width to 0.6pt.
\begin{codeexample}[]
  \tikz \draw[semithick] (0,0) -- (1cm,1.5ex);
\end{codeexample}
\end{stylekey}

\begin{stylekey}{/tikz/thick}
  Sets the line width to 0.8pt.
\begin{codeexample}[]
  \tikz \draw[thick] (0,0) -- (1cm,1.5ex);
\end{codeexample}
\end{stylekey}

\begin{stylekey}{/tikz/very thick}
  Sets the line width to 1.2pt.
\begin{codeexample}[]
  \tikz \draw[very thick] (0,0) -- (1cm,1.5ex);
\end{codeexample}
\end{stylekey}

\begin{stylekey}{/tikz/ultra thick}
  Sets the line width to 1.6pt.
\begin{codeexample}[]
  \tikz \draw[ultra thick] (0,0) -- (1cm,1.5ex);
\end{codeexample}
\end{stylekey}


\begin{key}{/tikz/line cap=\meta{type} (initially butt)}
  Specifies how lines ``end.'' Permissible \meta{type} are |round|,
  |rect|, and |butt|. They have the following effects:

\begin{codeexample}[]
\begin{tikzpicture}
  \begin{scope}[line width=10pt]
    \draw[line cap=rect]  (0,0 ) -- (1,0);
    \draw[line cap=butt]  (0,.5) -- (1,.5);
    \draw[line cap=round] (0,1 ) -- (1,1);
  \end{scope}
  \draw[white,line width=1pt]
    (0,0 ) -- (1,0) (0,.5) -- (1,.5) (0,1 ) -- (1,1);
\end{tikzpicture}
\end{codeexample}
\end{key}

\begin{key}{/tikz/line join=\meta{type} (initially miter)}
  Specifies how lines ``join.'' Permissible \meta{type} are |round|,
  |bevel|, and |miter|. They have the following effects:

\begin{codeexample}[]
\begin{tikzpicture}[line width=10pt]
  \draw[line join=round] (0,0) -- ++(.5,1) -- ++(.5,-1);
  \draw[line join=bevel] (1.25,0) -- ++(.5,1) -- ++(.5,-1); 
  \draw[line join=miter] (2.5,0) -- ++(.5,1) -- ++(.5,-1); 
  \useasboundingbox (0,1.5); % make bounding box bigger
\end{tikzpicture}
\end{codeexample}

  \begin{key}{/tikz/miter limit=\meta{factor} (initially 10)}
    When you use the miter join and there is a very sharp corner (a
    small angle), the miter join may protrude very far over the actual
    joining point. In this case, if it were to protrude by 
    more than \meta{factor} times the line width, the miter join is
    replaced by a bevel join. 

\begin{codeexample}[]
\begin{tikzpicture}[line width=5pt]
  \draw                 (0,0) -- ++(5,.5) -- ++(-5,.5);
  \draw[miter limit=25] (6,0) -- ++(5,.5) -- ++(-5,.5);
  \useasboundingbox (14,0); % make bounding box bigger
\end{tikzpicture}
\end{codeexample}
  \end{key}
\end{key}

\subsubsection{Graphic Parameters: Dash Pattern}

\begin{key}{/tikz/dash pattern=\meta{dash pattern}}
  Sets the dashing pattern. The syntax is the same as in
  \textsc{metafont}. For example following pattern
  |on 2pt off 3pt on 4pt off 4pt| means ``draw
  2pt, then leave out 3pt, then draw 4pt once more, then leave out 4pt
  again, repeat''. 

\begin{codeexample}[]
\begin{tikzpicture}[dash pattern=on 2pt off 3pt on 4pt off 4pt]
  \draw (0pt,0pt) -- (3.5cm,0pt);
\end{tikzpicture}
\end{codeexample}
\end{key}

\begin{key}{/tikz/dash phase=\meta{dash phase} (initially 0pt)}
  Shifts the start of the dash pattern by \meta{phase}.

\begin{codeexample}[]
\begin{tikzpicture}[dash pattern=on 20pt off 10pt]
  \draw[dash phase=0pt] (0pt,3pt) -- (3.5cm,3pt);
  \draw[dash phase=10pt] (0pt,0pt) -- (3.5cm,0pt);
\end{tikzpicture}
\end{codeexample}
\end{key}

As for the line thickness, some predefined styles allow you to set the
dashing conveniently.

\begin{stylekey}{/tikz/solid}
  Shorthand for setting a solid line as ``dash pattern.'' This is the default.

\begin{codeexample}[]
\tikz \draw[solid] (0pt,0pt) -- (50pt,0pt);
\end{codeexample}
\end{stylekey}

\begin{stylekey}{/tikz/dotted}
  Shorthand for setting a dotted dash pattern.

\begin{codeexample}[]
\tikz \draw[dotted] (0pt,0pt) -- (50pt,0pt);
\end{codeexample}
\end{stylekey}

\begin{stylekey}{/tikz/densely dotted}
  Shorthand for setting a densely dotted dash pattern.

\begin{codeexample}[]
\tikz \draw[densely dotted] (0pt,0pt) -- (50pt,0pt);
\end{codeexample}
\end{stylekey}

\begin{stylekey}{/tikz/loosely dotted}
  Shorthand for setting a loosely dotted dash pattern.

\begin{codeexample}[]
\tikz \draw[loosely dotted] (0pt,0pt) -- (50pt,0pt);
\end{codeexample}
\end{stylekey}

\begin{stylekey}{/tikz/dashed}
  Shorthand for setting a dashed dash pattern.

\begin{codeexample}[]
\tikz \draw[dashed] (0pt,0pt) -- (50pt,0pt);
\end{codeexample}
\end{stylekey}

\begin{stylekey}{/tikz/densely dashed}
  Shorthand for setting a densely dashed dash pattern.

\begin{codeexample}[]
\tikz \draw[densely dashed] (0pt,0pt) -- (50pt,0pt);
\end{codeexample}
\end{stylekey}

\begin{stylekey}{/tikz/loosely dashed}
  Shorthand for setting a loosely dashed dash pattern.

\begin{codeexample}[]
\tikz \draw[loosely dashed] (0pt,0pt) -- (50pt,0pt);
\end{codeexample}
\end{stylekey}


\subsubsection{Graphic Parameters: Draw Opacity}

When a line is drawn, it will normally ``obscure'' everything behind
it as if you has used perfectly opaque ink. It is also possible to ask
\tikzname\ to use an ink that is a little bit (or a big bit)
transparent using the |draw opacity| option. This is explained in
Section~\ref{section-tikz-transparency} on transparency in more detail.



\subsubsection{Graphic Parameters: Arrow Tips}

When you draw a line, you can add arrow tips at the ends. It is
only possible to add one arrow tip at the start and one at the end. If
the path consists of several segments, only the last segment gets
arrow tips. The behavior for paths that are closed is not specified
and may change in the future.

\begin{key}{/tikz/arrows=\meta{start arrow kind}|-|\meta{end arrow kind}}
  This option sets the start and end arrow tips (an empty value as in |->|
  indicates that no arrow tip should be drawn at the start).%
  \indexoption{arrows}

  \emph{Note: Since the arrow option is so often used, you can leave
    out the text |arrows=|.} What happens is that every option that
  contains a |-| is interpreted as an arrow specification.

\begin{codeexample}[]
\begin{tikzpicture}
  \draw[->]        (0,0)   -- (1,0);
  \draw[o-stealth] (0,0.3) -- (1,0.3);
\end{tikzpicture}
\end{codeexample}

  The permissible values are all predefined arrow tips, though
  you can also define new arrow tip kinds as explained in
  Section~\ref{section-arrows}. This is often necessary to obtain
  ``double'' arrow tips and arrow tips that have a fixed size. Since
  |pgflibraryarrows| is loaded by default, all arrow tips described in
  Section~\ref{section-library-arrows} are available.

  One arrow tip kind is special: |>| (and all arrow tip kinds containing the
  arrow tip kind such as |<<| or \verb!>|!). This arrow tip type is not  
  fixed. Rather, you can redefine it using the |>=| option, see
  below. 

  \example You can also combine arrow tip types as in
\begin{codeexample}[]
\begin{tikzpicture}[thick]
  \draw[to reversed-to]   (0,0) .. controls +(.5,0) and +(-.5,-.5) .. +(1.5,1);
  \draw[[-latex reversed] (1,0) .. controls +(.5,0) and +(-.5,-.5) .. +(1.5,1);
  \draw[latex-)]          (2,0) .. controls +(.5,0) and +(-.5,-.5) .. +(1.5,1);
  \useasboundingbox (-.1,-.1) rectangle (3.1,1.1); % make bounding box bigger
\end{tikzpicture}
\end{codeexample}
\end{key}

\begin{key}{/tikz/>=\meta{end arrow kind}}
  This option can be used to redefine the ``standard'' arrow tip |>|. The
  idea is that different people have different ideas what arrow tip kind
  should normally be used. I prefer the arrow tip of \TeX's |\to| command
  (which is used in things like $f\colon A \to B$). Other people will
  prefer \LaTeX's standard arrow tip, which looks like this: \tikz
  \draw[-latex] (0,0) -- (10pt,1ex);. Since the arrow tip kind |>| is
  certainly the most ``natural'' one to use, it is kept free of any
  predefined meaning. Instead, you can change it by saying |>=to| to
  set the ``standard'' arrow tip kind to \TeX's arrow tip, whereas |>=latex|
  will set it to \LaTeX's arrow tip and |>=stealth| will use a
  \textsc{pstricks}-like arrow tip.

  Apart from redefining the arrow tip kind |>| (and |<| for the start),
  this option also redefines the following arrow tip kinds: |>| and |<| as
  the swapped version of \meta{end arrow kind}, |<<| and |>>| as
  doubled versions, |>>| and |<<| as swapped doubled versions, %>>
  and \verb!|<! and \verb!>|! as arrow tips ending with a vertical bar.

\begin{codeexample}[]
\begin{tikzpicture}[scale=2]
  \begin{scope}[>=latex]
    \draw[->]    (0pt,6ex) -- (1cm,6ex);
    \draw[>->>]  (0pt,5ex) -- (1cm,5ex); 
    \draw[|<->|] (0pt,4ex) -- (1cm,4ex);
  \end{scope}
  \begin{scope}[>=diamond]
    \draw[->]    (0pt,2ex) -- (1cm,2ex);
    \draw[>->>]  (0pt,1ex) -- (1cm,1ex);
    \draw[|<->|] (0pt,0ex) -- (1cm,0ex);
  \end{scope} 
\end{tikzpicture}
\end{codeexample} 
% <<
\end{key}

\begin{key}{/tikz/shorten >=\meta{dimension} (initially 0pt)}
  This option will shorten the end of lines by the given
  \meta{dimension}. If you specify an arrow tip, lines are already
  shortened a bit such that the arrow tip touches the specified endpoint
  and does not ``protrude over'' this point. Here is an example:

\begin{codeexample}[]
\begin{tikzpicture}[line width=20pt]
  \useasboundingbox (0,-1.5) rectangle (3.5,1.5);
  \draw[red]        (0,0) -- (3,0);
  \draw[gray,->]    (0,0) -- (3,0);
\end{tikzpicture}
\end{codeexample}

  The |shorten >| option allows you to shorten the end on the line
  \emph{additionally} by the given distance. This option can also be
  useful if you have not specified an arrow tip at all.

\begin{codeexample}[]
\begin{tikzpicture}[line width=20pt]
  \useasboundingbox (0,-1.5) rectangle (3.5,1.5);
  \draw[red]                     (0,0) -- (3,0);
  \draw[-to,shorten >=10pt,gray] (0,0) -- (3,0);
\end{tikzpicture}
\end{codeexample}
\end{key}


\begin{key}{/tikz/shorten <=\meta{dimension}}
  Works like |shorten >|, but for the start.
\end{key}



\subsubsection{Graphic Parameters: Double Lines and Bordered Lines}

\begin{key}{/tikz/double=\meta{core color} (default white)}
  This option causes ``two'' lines to be drawn instead of a single
  one. However, this is not what really happens. In reality, the path
  is drawn twice. First, with the normal drawing color, secondly with
  the \meta{core color}, which is normally |white|. Upon the second
  drawing, the line width is reduced. The net effect is that it
  appears as if two lines had been drawn and this works well even with
  complicated, curved paths:

\begin{codeexample}[]
\tikz \draw[double]
  plot[smooth cycle] coordinates{(0,0) (1,1) (1,0) (0,1)};
\end{codeexample}

  You can also use the doubling option to create an effect in which a
  line seems to have a certain ``border'':

\begin{codeexample}[]
\begin{tikzpicture}
  \draw (0,0) -- (1,1);
  \draw[draw=white,double=red,very thick] (0,1) -- (1,0);
\end{tikzpicture}
\end{codeexample}
\end{key}

\begin{key}{/tikz/double distance=\meta{dimension} (initially 0.6pt)}
  Sets the distance the ``two'' lines are spaced apart. In reality,
  this is the thickness of the line that is used 
  to draw the path for the second time. The thickness of the
  \emph{first} time the path is drawn is twice the normal line width
  plus the given \meta{dimension}. As a side-effect, this option
  ``selects'' the |double| option.

\begin{codeexample}[]
\begin{tikzpicture}
  \draw[very thick,double]              (0,0) arc (180:90:1cm);
  \draw[very thick,double distance=2pt] (1,0) arc (180:90:1cm);
  \draw[thin,double distance=2pt]       (2,0) arc (180:90:1cm);
\end{tikzpicture}
\end{codeexample}
\end{key}


  




\subsection{Filling a Path}
\label{section-rules}
To fill a path, use the following option:
\begin{key}{/tikz/fill=\meta{color} (default \normalfont is scope's color setting)}
  This option causes the path to be filled. All unclosed parts of the
  path are first closed, if necessary. Then, the area enclosed by the
  path is filled with the current filling color, which is either the
  last color set using the general |color=| option or the optional
  color \meta{color}. For self-intersection paths and for paths
  consisting of several closed areas, the ``enclosed area'' is
  somewhat complicated to define and two different definitions exist,
  namely the nonzero winding number rule and the even odd rule, see
  the explanation of these options, below.

  Just as for the |draw| option, setting \meta{color} to |none|
  disables filling locally.

\begin{codeexample}[]
\begin{tikzpicture}
  \fill (0,0) -- (1,1) -- (2,1);
  \fill (4,0) circle (.5cm)  (4.5,0) circle (.5cm);
  \fill[even odd rule] (6,0) circle (.5cm)  (6.5,0) circle (.5cm);
  \fill (8,0) -- (9,1) -- (10,0) circle (.5cm);
\end{tikzpicture}
\end{codeexample}

  If the |fill| option is used together with the |draw| option (either
  because both are given as options or because a |\filldraw| command
  is used), the path is filled \emph{first}, then the path is drawn
  \emph{second}. This is especially useful if different colors are
  selected for drawing and for filling. Even if the same color is
  used, there is a difference between this command and a plain 
  |fill|: A ``filldrawn'' area will be slightly larger than a filled
  area because of the thickness of the ``pen.''

\begin{codeexample}[]
\begin{tikzpicture}[fill=examplefill,line width=5pt]
  \filldraw (0,0) -- (1,1) -- (2,1);
  \filldraw (4,0) circle (.5cm)  (4.5,0) circle (.5cm);
  \filldraw[even odd rule] (6,0) circle (.5cm)  (6.5,0) circle (.5cm);
  \filldraw (8,0) -- (9,1) -- (10,0) circle (.5cm);
\end{tikzpicture}
\end{codeexample}
\end{key}



\subsubsection{Graphic Parameters: Fill Pattern}

\label{section-fill-pattern}
Instead of filling a path with a single solid color, it is also
possible to fill it with a \emph{tiling pattern}. Imagine a small tile
that contains a simple picture like a star. Then these tiles are
(conceptually) repeated infinitely in all directions, but clipped
against the path.

Tiling patterns come in two variants: \emph{inherently 
  colored patterns} and \emph{form-only patterns}. An inherently colored
pattern is, say, a red star with a black border and will always look
like this. A form-only pattern may have a different color each time
it is used, only the form of the pattern will stay the same. As such,
form-only patterhns do not have any colors of their own, but when it
is used the current \emph{pattern color} is used as its color.

Patterns are not overly flexible. In particular, it is not possible to
change the size or orientation of a pattern without declaring a new
pattern. For complicated case, it may be easier to use two nested
|\foreach| statements to simulate a pattern, but patterns are rendered
\emph{much} more quickly than simulated ones.

\begin{key}{/tikz/pattern=\meta{name} (default \normalfont is scope's pattern)}
  This option causes the path to be filled with a pattern. If the
  \meta{name} is given, this pattern is used, otherwise the pattern
  set in the enclosing scope is used. As for the |draw| and |fill|
  options, setting \meta{name} to |none| disables filling locally.

  The pattern works like a fill color. In particular, setting a new
  fill color will fill the path with a solid color once more.

  Strangely, no \meta{name}s are permissible by default. You need to
  load for instance |pgflibrarypatterns|, see
  Section~\ref{section-library-patterns}, to install predefined
  patterns.  
  
\begin{codeexample}[]
\begin{tikzpicture}
  \draw[pattern=dots] (0,0) circle (1cm);
  \draw[pattern=fivepointed stars] (0,0) rectangle (3,1);
\end{tikzpicture}
\end{codeexample}
\end{key}

\begin{key}{/tikz/pattern color=\meta{color}}
  This option is used to set the color to be used for form-only
  patterns. This option has no effect on inherently colored patterns. 
  
\begin{codeexample}[]
\begin{tikzpicture}
  \draw[pattern color=red,pattern=fivepointed stars]  (0,0) circle (1cm);
  \draw[pattern color=blue,pattern=fivepointed stars] (0,0) rectangle (3,1);
\end{tikzpicture}
\end{codeexample}

\begin{codeexample}[]
\begin{tikzpicture}
  \def\mypath{(0,0) -- +(0,1) arc (180:0:1.5cm) -- +(0,-1)}
  \fill   [red]                                \mypath;
  \pattern[pattern color=white,pattern=bricks] \mypath;
\end{tikzpicture}
\end{codeexample}
\end{key}


\subsubsection{Graphic Parameters: Interior Rules}

The following two options can be used to decide how interior points
should be determined:
\begin{key}{/tikz/nonzero rule}
  If this rule is used (which is the default), the following method is
  used to determine whether a given point is ``inside'' the path: From
  the point, shoot a ray in some direction towards infinity (the
  direction is chosen such that no strange borderline cases
  occur). Then the ray may hit the path. Whenever it hits the path, we
  increase or decrease a counter, which is initially zero. If the ray
  hits the path as the path goes ``from left to right'' (relative to
  the ray), the counter is increased, otherwise it is decreased. Then,
  at the end, we check whether the counter is nonzero (hence the
  name). If so, the point is deemed to lie ``inside,'' otherwise it is
  ``outside.'' Sounds complicated? It is.

\begin{codeexample}[]
\begin{tikzpicture}
  \filldraw[fill=examplefill]
  % Clockwise rectangle
  (0,0) -- (0,1) -- (1,1) -- (1,0) -- cycle
  % Counter-clockwise rectangle
  (0.25,0.25) -- (0.75,0.25) -- (0.75,0.75) -- (0.25,0.75) -- cycle;

  \draw[->] (0,1) -- (.4,1);
  \draw[->] (0.75,0.75) -- (0.3,.75);

  \draw[->] (0.5,0.5) -- +(0,1) node[above] {crossings: $-1+1 = 0$};

  \begin{scope}[yshift=-3cm]
    \filldraw[fill=examplefill]
    % Clockwise rectangle
    (0,0) -- (0,1) -- (1,1) -- (1,0) -- cycle
    % Clockwise rectangle
    (0.25,0.25) -- (0.25,0.75) -- (0.75,0.75) -- (0.75,0.25) -- cycle;

    \draw[->] (0,1) -- (.4,1);
    \draw[->] (0.25,0.75) -- (0.4,.75);
      
    \draw[->] (0.5,0.5) -- +(0,1) node[above] {crossings: $1+1 = 2$};
  \end{scope}
\end{tikzpicture}
\end{codeexample}
\end{key}

\begin{key}{/tikz/even odd rule}
  This option causes a different method to be used for determining the
  inside and outside of paths. While it is less flexible, it turns out
  to be more intuitive.

  With this method, we also shoot rays from the point for which we
  wish to determine whether it is inside or outside the filling
  area. However, this time we only count how often we ``hit'' the path
  and declare the point to be ``inside'' if the number of hits is odd.

  Using the even-odd rule, it is easy to ``drill holes'' into a path.
  
\begin{codeexample}[]
\begin{tikzpicture}
  \filldraw[fill=examplefill,even odd rule]
    (0,0) rectangle (1,1) (0.5,0.5) circle (0.4cm);
  \draw[->] (0.5,0.5) -- +(0,1) [above] node{crossings: $1+1 = 2$};
\end{tikzpicture}
\end{codeexample}
\end{key}



\subsubsection{Graphic Parameters: Fill Opacity}

\label{section-fill-opacity}
Analogously to the |draw opacity|, you can also set the filling
opacity. Please see Section~\ref{section-tikz-transparency} for more
details. 


\subsection{Shading a Path}

You can shade a path using the |shade| option. A shading is like a
filling, only the shading changes its color smoothly from one color to
another.

\begin{key}{/tikz/shade}
  Causes the path to be shaded using the currently selected shading
  (more on this later). If this option is used together with the
  |draw| option, then the path is first shaded, then drawn.

  It is not an error to use this option together with the |fill|
  option, but it makes no sense.

\begin{codeexample}[]
\tikz \shade (0,0) circle (1ex);
\end{codeexample}

\begin{codeexample}[]
\tikz \shadedraw (0,0) circle (1ex);
\end{codeexample}
\end{key}

For some shadings it is not really clear how they can ``fill'' the
path. For example, the |ball| shading normally looks like this: \tikz
\shade[shading=ball] (0,0) circle (0.75ex);. How is this supposed to
shade a rectangle? Or a triangle?

To solve this problem, the predefined shadings like |ball| or |axis|
fill a large rectangle completely in a sensible way. Then, when the
shading is used to ``shade'' a path, what actually happens is that the
path is temporarily used for clipping and then the rectangular shading
is drawn, scaled and shifted such that all parts of the path are
filled.


\subsubsection{Choosing a Shading Type}

The default shading is a smooth transition from gray
to white and from above to bottom. However, other shadings are also
possible, for example a shading that will sweep a color from the
center to the corners outward. To choose the shading, you can use the
|shading=| option, which will also automatically invoke the |shade|
option. Note that this does \emph{not} change the shading color, only
the way the colors sweep. For changing the colors, other options are
needed, which are explained below.

\begin{key}{/tikz/shading=\meta{name}}
  This selects a shading named \meta{name}. The following shadings are
  predefined:
  \begin{itemize}
  \item \declare{|axis|}
    This is the default shading in which the color changes gradually
    between three horizontal lines. The top line is at the top
    (uppermost) point of the path, the middle is in the middle, the
    bottom line is at the bottom of the path.

\begin{codeexample}[]
\tikz \shadedraw [shading=axis] (0,0) rectangle (1,1);
\end{codeexample}

    The default top color is gray, the default bottom color is white,
    the default middle is the ``middle'' of these two.
  \item \declare{|radial|}
    This shading fills the path with a gradual sweep from a certain
    color in the middle to another color at the border. If the path is
    a circle, the outer color will be reached exactly at the
    border. If the shading is not a circle, the outer color will
    continue a bit towards the corners. The default inner color is
    gray, the default outer color is white.

\begin{codeexample}[]
\tikz \shadedraw [shading=radial] (0,0) rectangle (1,1);
\end{codeexample}
  \item \declare{|ball|}
    This shading fills the path with a shading that ``looks like a
    ball.'' The default ``color'' of the ball is blue (for no
    particular reason).

\begin{codeexample}[]
\tikz \shadedraw [shading=ball] (0,0) rectangle (1,1);
\end{codeexample}

\begin{codeexample}[]
\tikz \shadedraw [shading=ball] (0,0) circle (.5cm);
\end{codeexample}
  \end{itemize}

  \begin{key}{/tikz/shading angle=\meta{degrees} (initially 0)}
    This option rotates the shading (not the path!) by the given
    angle. For example, we can turn a top-to-bottom axis shading into a
    left-to-right shading by rotating it by $90^\circ$.

\begin{codeexample}[]
\tikz \shadedraw [shading=axis,shading angle=90] (0,0) rectangle (1,1);
\end{codeexample}
  \end{key}
\end{key}

You can also define new shading types yourself. However, for this, you
need to use the basic layer directly, which is, well, more basic and
harder to use. Details on how to create a shading appropriate for
filling paths are given in Section~\ref{section-shading-a-path}.



\subsubsection{Choosing a Shading Color}

The following options can be used to change the colors used for
shadings. When one of these options is given, the |shade| option is
automatically selected and also the ``right'' shading.

\begin{key}{/tikz/top color=\meta{color}}
  This option prescribes the color to be used at the top in an |axis|
  shading. When this option is given, several things happen:
  \begin{enumerate}
  \item
    The |shade| option is selected.
  \item
    The |shading=axis| option is selected.
  \item
    The middle color of the axis shading is set to the average of the
    given top color \meta{color} and of whatever color is currently
    selected for the bottom.
  \item
    The rotation angle of the shading is set to 0.
  \end{enumerate}

\begin{codeexample}[]
\tikz \draw[top color=red] (0,0) rectangle (2,1);
\end{codeexample}
\end{key}  

\begin{key}{/tikz/bottom color=\meta{color}}
  This option works like |top color|, only for the bottom color.
\end{key}

\begin{key}{/tikz/middle color=\meta{color}}
  This option specifies the color for the middle of an axis
  shading. It also sets the |shade| and |shading=axis| options, but it
  does not change the rotation angle.

  \emph{Note:} Since both |top color| and |bottom color| change the
  middle color, this option should be given \emph{last} if all of
  these options need to be given:

\begin{codeexample}[]
\tikz \draw[top color=white,bottom color=black,middle color=red]
  (0,0) rectangle (2,1);
\end{codeexample}  
\end{key}

\begin{key}{/tikz/left color=\meta{color}}
  This option does exactly the same as |top color|, except that the
  shading angle is set to $90^\circ$.
\end{key}

\begin{key}{/tikz/right color=\meta{color}}
  Works like |left color|.
\end{key}

\begin{key}{/tikz/inner color=\meta{color}}
  This option sets the color used at the center of a |radial|
  shading. When this option is used, the |shade| and |shading=radial|
  options are set.
  
\begin{codeexample}[]
\tikz \draw[inner color=red] (0,0) rectangle (2,1);
\end{codeexample}
\end{key}

\begin{key}{/tikz/outer color=\meta{color}}
  This option sets the color used at the border and outside of a
  |radial| shading.
  
\begin{codeexample}[]
\tikz \draw[outer color=red,inner color=white]
  (0,0) rectangle (2,1);
\end{codeexample}
\end{key}


\begin{key}{/tikz/ball color=\meta{color}}
  This option sets the color used for the ball shading. It sets the
  |shade| and |shading=ball| options. Note that the ball will never
  ``completely'' have the color \meta{color}. At its ``highlight'' spot
  a certain amount of white is mixed in, at the border a certain
  amount of black. Because of this, it also makes sense to say
  |ball color=white| or |ball color=black|

\begin{codeexample}[]
\begin{tikzpicture}
  \shade[ball color=white] (0,0) circle (2ex);
  \shade[ball color=red] (1,0) circle (2ex);
  \shade[ball color=black] (2,0) circle (2ex);
\end{tikzpicture}
\end{codeexample}
\end{key}



\subsection{Establishing a Bounding Box}

\pgfname\ is reasonably good at keeping track of the size of your picture
and reserving just the right amount of space for it in the main
document. However, in some cases you may want to say things like
``do not count this for the picture size'' or ``the picture is
actually a little large.'' For this you can use the option
|use as bounding box| or the command |\useasboundingbox|, which is just
a shorthand for |\path[use as bounding box]|.

\begin{key}{/tikz/use as bounding box}
  Normally, when this option is given on a path, the bounding box of
  the present path is used to determine the size of the picture and
  the size of all \emph{subsequent} paths are
  ignored. However, if there were previous path operations that have
  already established a larger bounding box, it will not be made
  smaller by this operation.

  In a sense, |use as bounding box| has the same effect as clipping
  all subsequent drawing against the current path---without actually
  doing the clipping, only making \pgfname\ treat everything as if it
  were clipped.

  The first application of this option is to have a |{tikzpicture}|
  overlap with the main text:

\begin{codeexample}[]
Left of picture\begin{tikzpicture}
  \draw[use as bounding box] (2,0) rectangle (3,1);
  \draw (1,0) -- (4,.75);
\end{tikzpicture}right of picture.
\end{codeexample}

  In a second application this option can be used to get better
  control over the white space around the picture:
  
\begin{codeexample}[]
Left of picture
\begin{tikzpicture}
  \useasboundingbox (0,0) rectangle (3,1);
  \fill (.75,.25) circle (.5cm);
\end{tikzpicture}
right of picture.
\end{codeexample}

  Note: If this option is used on a path inside a \TeX\ group (scope),
  the effect ``lasts'' only till the end of the scope. Again, this
  behavior is the same as for clipping.
\end{key}

There is a node that allows you to get the size of the current
bounding box. The |current bounding box| node has the |rectangle|
shape and its size is always the size of the current 
bounding box.

Similarly, the |current path bounding box| node has the |rectangle|
hape and the size of the bounding box of the current path.


\begin{codeexample}[]
\begin{tikzpicture}
  \draw[red] (0,0) circle (2pt);
  \draw[red] (2,1) circle (3pt);

  \draw (current bounding box.south west) rectangle
        (current bounding box.north east);

  \draw[red] (3,-1) circle (4pt);

  \draw[thick] (current bounding box.south west) rectangle
               (current bounding box.north east);
\end{tikzpicture}
\end{codeexample}



\subsection{Clipping and Fading (Soft Clipping)}

\emph{Clipping path} means that all painting on the page is restricted
to a certain area. This area need not be rectangular, rather an
arbitrary path can be used to specify this area. The |clip| option,
explained below, is used to specify the region that is to be used for
clipping.

A \emph{fading} (a term that I propose, fadings are commonly known
as soft masks, transparency masks, opacity masks or soft clips) is
similar to clipping, but a fading allows parts of the picture to be
only ``half clipped.'' This means that a fading can specify that newly
painted pixels should be partly transparent. The specification
and handling of fadings is a bit complex and it is detailed in
Section~\ref{section-tikz-transparency}, which is devoted to
transparency in general.

\begin{key}{/tikz/clip}
  This option causes all subsequent drawings to be clipped against the
  current path and the size of subsequent paths will not be important
  for the picture size.  If you clip against a self-intersecting path,
  the even-odd rule or  the nonzero winding number rule is used to
  determine whether a point is inside or outside the clipping region.

  The clipping path is a graphic state parameter, so it will be reset
  at the end of the current scope. Multiple clippings accumulate, that
  is, clipping is always done against the intersection of all clipping
  areas that have been specified inside the current scopes. The only
  way of enlarging the clipping area is to end a |{scope}|.

\begin{codeexample}[]
\begin{tikzpicture}
  \draw[clip] (0,0) circle (1cm);
  \fill[red] (1,0) circle (1cm);
\end{tikzpicture}
\end{codeexample}

  It  is usually a \emph{very} good idea to apply the |clip| option only
  to the first path command in a scope. 

  If you ``only wish to clip'' and do not wish to draw anything, you can
  use the |\clip| command, which is a shorthand for |\path[clip]|.

\begin{codeexample}[]
\begin{tikzpicture}
  \clip (0,0) circle (1cm);
  \fill[red] (1,0) circle (1cm);
\end{tikzpicture}
\end{codeexample}

  To keep clipping local, use |{scope}| environments as in the
  following example:

\begin{codeexample}[]
\begin{tikzpicture}
  \draw (0,0) -- ( 0:1cm);
  \draw (0,0) -- (10:1cm);
  \draw (0,0) -- (20:1cm);
  \draw (0,0) -- (30:1cm);
  \begin{scope}[fill=red]
    \fill[clip] (0.2,0.2) rectangle (0.5,0.5);
    
    \draw (0,0) -- (40:1cm);
    \draw (0,0) -- (50:1cm);
    \draw (0,0) -- (60:1cm);
  \end{scope}
  \draw (0,0) -- (70:1cm);
  \draw (0,0) -- (80:1cm);
  \draw (0,0) -- (90:1cm);
\end{tikzpicture}
\end{codeexample}

  There is a slightly annoying catch: You cannot specify certain graphic
  options for the command used for clipping. For example, in the above
  code we could not have moved the |fill=red| to the |\fill|
  command. The reasons for this have to do with the internals of the
  \pdf\ specification. You do not want to know the details. It is best
  simply not to specify any options for these 
  commands. 
\end{key}



\subsection{Doing Multiple Actions on a Path}

If more than one of the basic actions like drawing, clipping and
filling are requested, they are automatically applied in a sensible
order: First, a path is filled, then drawn, and then clipped (although
it took Apple two mayor revisions of their operating system to get
this right\dots). Sometimes, however, you need finer control over what
is done with a path. For instance, you might wish to first fill a path
with a color, then repaint the path with a pattern and then repaint it
with yet another pattern. In such cases you can use the following two
options:

\begin{key}{/tikz/preactions=\meta{options}}
  This option can be given to a |\path| command (or to derived
  commands like |\draw| which internally call |\path|). Similarly to
  options like |draw|, this option only has an effect when given to a
  |\path| or as part of the options of a |node|; as an option to a
  |{scope}| it has no effect.

  When this option is used on a |\path|, the effect is the following:
  When the path has been completely constructed and is about to be
  used, a scope is created. Inside this scope, the path is used but
  not with the original path options, but with \meta{options}
  instead. Then, the path is used in the usual manner. In other words,
  the path is used twice: Once with \meta{options} in force and then
  again with the normal path options in force.

  Here is an example in which the path consists of a rectangle. The
  main action is to draw this path in red (which is why we see a red
  rectangle). However, the preaction is to draw the path in blue,
  which is why we see a blue rectangle behind the red rectangle.
\begin{codeexample}[]
\begin{tikzpicture}
  \draw[help lines] (0,0) grid (3,2);

  \draw
    [preaction={draw,line width=4mm,blue}]
    [line width=2mm,red] (0,0) rectangle (2,2);
\end{tikzpicture}
\end{codeexample}

  Note that when the preactions are preformed, then the path is
  already ``finished.'' In particular, applying a coordinate
  transformation to the path has no effect. By comparison, applying a
  canvas transformation does have an effect. Let us use this to add a
  ``shadow'' to a path. For this, we use the preaction to fill the
  path in gray, shifted a bit to the right and down:

\begin{codeexample}[]
\begin{tikzpicture}
  \draw[help lines] (0,0) grid (3,2);
  \draw
    [preaction={fill=black,opacity=.5,
                transform canvas={xshift=1mm,yshift=-1mm}}]
    [fill=red] (0,0) rectangle (1,2)
               (1,2) circle (5mm);
\end{tikzpicture}
\end{codeexample}

  Naturally, you would normally create a style |shadow| that contains
  the above code. The shadow library, see
  Section~\ref{section-libs-shadows}, contains predefined shadows of
  this kind.

  It is possible to use the |preaction| option multiple times. In this
  case, for each use of the |preaction| option, the path is used again
  (thus, the \meta{options} do not accumulate in a single usage of the
  path). The path is used in the order of |preaction| options given.

  In the following example, we use one |preaction| to add a shadow and
  another to provide a shading, while the main action is to use a
  pattern. 
\begin{codeexample}[]
\begin{tikzpicture}
  \draw[help lines] (0,0) grid (3,2);
  \draw [pattern=fivepointed stars]
    [preaction={fill=black,opacity=.5,
                transform canvas={xshift=1mm,yshift=-1mm}}]
    [preaction={top color=blue,bottom color=white}]
               (0,0) rectangle (1,2)
               (1,2) circle (5mm);
\end{tikzpicture}
\end{codeexample}

  A complicated application is shown in the following example, where
  the path is used several times with different fadings and shadings
  to create a special visual effect:
\begin{codeexample}[]
\begin{tikzpicture}
  [
    % Define an interesting style
    button/.style={
      % First preaction: Fuzzy shadow
      preaction={fill=black,path fading=circle with fuzzy edge 20 percent,
                 opacity=.5,transform canvas={xshift=1mm,yshift=-1mm}},
      % Second preaction: Background pattern
      preaction={pattern=#1,
                 path fading=circle with fuzzy edge 15 percent},
      % Third preaction: Make background shiny
      preaction={top color=white,
                 bottom color=black!50,
                 shading angle=45,
                 path fading=circle with fuzzy edge 15 percent,
                 opacity=0.2},
      % Fourth preaction: Make edge especially shiny
      preaction={path fading=fuzzy ring 15 percent,
                 top color=black!5,
                 bottom color=black!80,
                 shading angle=45},
      inner sep=2ex
    },
    button/.default=horizontal lines light blue,
    circle
  ]

  \draw [help lines] (0,0) grid (4,3);

  \node [button] at (2.2,1) {\Huge Big};
  \node [button=crosshatch dots light steel blue,
         text=white] at (1,1.5) {Small};
\end{tikzpicture}
\end{codeexample}
\end{key}

\begin{key}{/tikz/postaction=\meta{options}}
  The postactions work in the same way as the preactions, only they
  are applied \emph{after} the main action has been taken. Like
  preactions, multiple |postaction| options may be given to a |\path|
  command, in which case the path is reused several times, each time
  with a different set of options in force.

  If both pre- and postactions are specified, then the preactions are
  taken first, then the main action, and then the post actions.

  In the first example, we use a postaction to draw the path, after it
  has already been drawn:
\begin{codeexample}[]
\begin{tikzpicture}
  \draw[help lines] (0,0) grid (3,2);

  \draw
    [postaction={draw,line width=2mm,blue}]
    [line width=4mm,red,fill=white] (0,0) rectangle (2,2);
\end{tikzpicture}
\end{codeexample}

  In another example, we use a postaction to ``colorzie'' a path: 

\begin{codeexample}[]
\begin{tikzpicture}
  \draw[help lines] (0,0) grid (3,2);
  \draw
    [postaction={path fading=south,fill=white}]
    [postaction={path fading=south,fading angle=45,fill=blue,opacity=.5}]
    [left color=black,right color=red,draw=white,line width=2mm]
               (0,0) rectangle (1,2)
               (1,2) circle (5mm);
\end{tikzpicture}
\end{codeexample}
\end{key}



\subsection{Decorating and Morphing a Path}

Before a path is used, it is possible to first ``decorate'' and/or
``morph'' it. Morphing means that the path is replaced by another path
that slightly varied. Such morphings are a special case
of the more general ``decorations'' described in detail in
Section~\ref{section-tikz-decorations}. For instance, in the following
example the path is drawn twice: Once normally and then in a morphed
(=decorated) manner. 
\begin{codeexample}[]
\begin{tikzpicture}
  \draw (0,0) rectangle (3,2);
  \draw [red, decorate, decoration=zigzag]
        (0,0) rectangle (3,2);
\end{tikzpicture}
\end{codeexample}

Naturally, we could have combined this into a single command using
pre- or postaction. It is also possible to deform shapes:
\begin{codeexample}[]
\begin{tikzpicture}
  \node [circular drop shadow={shadow scale=1.05},minimum size=3.13cm,
         decorate, decoration=zigzag,
         fill=blue!20,draw,thick,circle] {Hello!};
\end{tikzpicture}
\end{codeexample}



���}	� �
����DEV��INO��SYN��SV~�pgfmanual-en-tikz-shapes.tex�J��JӀ�J�j��fP�_.j�

���}	� �
����DEV��INO��SYN��SV~�pgfmanual-en-tikz-matrices.tex�J��Jр�J�j��^P�_.j�

���}	� �
����DEV��INO��SYN��SV~�pgfmanual-en-tikz-trees.tex�J��JՀ�J�j��lP�_.j�
% Copyright 2007 by Till Tantau
%
% This file may be distributed and/or modified
%
% 1. under the LaTeX Project Public License and/or
% 2. under the GNU Free Documentation License.
%
% See the file doc/generic/pgf/licenses/LICENSE for more details.


\section{Plots of Functions}

\label{section-tikz-plots}

\subsection{When Should One Use \tikzname\ for Generating Plots? }

\label{section-why-pgname-for-plots}

There exist many powerful programs that produce plots, examples are
\textsc{gnuplot} or \textsc{mathematica}. These programs can produce
two different kinds of output: First, they can output a complete plot
picture in a certain format (like \pdf) that includes all low-level
commands necessary for drawing the complete plot (including axes and
labels). Second, they can usually also produce ``just plain data'' in
the form of a long list of coordinates. Most of the powerful programs
consider it a to be ``a bit boring'' to just output tabled data and
very much prefer to produce fancy pictures. Nevertheless, when coaxed,
they can also provide the plain data.

\emph{Note that is often not necessary to use \tikzname\ for plots.}
Programs like \textsc{gnuplot} can produce very sophisticated plots
and it is usually much easier to simply include these plots as a
finished \textsc{pdf} or PostScript graphics.

However, there are a number of reasons why you may wish to invest time
and energy into mastering the \pgfname\ commands for creating plots:

\begin{itemize}
\item
  Virtually all plots produced by ``external programs'' use different
  fonts from the one used in your document.
\item
  Even worse, formulas will look totally different, if they can be
  rendered at all.
\item
  Line width will usually be too large or too small.
\item
  Scaling effects upon inclusion can create a mismatch between sizes
  in the plot and sizes in the text.
\item
  The automatic grid generated by most programs is mostly
  distracting. 
\item
  The automatic ticks generated by most programs are cryptic
  numerics. (Try adding a tick reading ``$\pi$'' at the right point.)
\item
  Most programs make it very easy to create ``chart junk'' in a most
  convenient fashion.  All show, no content.
\item
  Arrows and plot marks will almost never match the arrows used in the
  rest of the document.
\end{itemize}

The above list is not exhaustive, unfortunately.


\subsection{The Plot Path Operation}

The |plot| path operation can be used to append a line or curve to the path
that goes through a large number of coordinates. These coordinates are
either given in a simple list of coordinates, read from some file, or
they are computed on the fly.

The syntax of the |plot| comes in different versions.

\begin{pathoperation}{--plot}{\meta{further arguments}}
  This operation plots the curve through the coordinates specified in
  the \meta{further arguments}. The current (sub)path is simply
  continued, that is, a line-to operation to the first point of the
  curve is implicitly added. The details of the \meta{further
    arguments}  will be explained in a moment.
\end{pathoperation}

\begin{pathoperation}{plot}{\meta{further arguments}}
  This operation plots the curve through the coordinates specified in
  the \meta{further arguments} by first ``moving'' to the first
  coordinate of the curve.
\end{pathoperation}

The \meta{further arguments} are used in three different ways to
specifying the coordinates of the points to be plotted:

\begin{enumerate}
\item
  \opt{|--|}|plot|\oarg{local options}\declare{|coordinates{|\meta{coordinate
    1}\meta{coordinate 2}\dots\meta{coordinate $n$}|}|}
\item
  \opt{|--|}|plot|\oarg{local options}\declare{|file{|\meta{filename}|}|}
\item
  \opt{|--|}|plot|\oarg{local options}\declare{\meta{coordinate expression}}
\item
  \opt{|--|}|plot|\oarg{local options}\declare{|function{|\meta{gnuplot formula}|}|}
\end{enumerate}

These different ways are explained in the following.


\subsection{Plotting Points Given Inline}

In the first two cases, the points are given directly in the \TeX-file
as in the following example:

\begin{codeexample}[]
\tikz \draw plot coordinates {(0,0) (1,1) (2,0) (3,1) (2,1) (10:2cm)};
\end{codeexample}

Here is an example showing the difference between |plot| and |--plot|:

\begin{codeexample}[]
\begin{tikzpicture}
  \draw (0,0) -- (1,1) plot coordinates {(2,0)  (4,0)};
  \draw[color=red,xshift=5cm]
        (0,0) -- (1,1) -- plot coordinates {(2,0)  (4,0)};
\end{tikzpicture}
\end{codeexample}


\subsection{Plotting Points Read From an External File}

The second way of specifying points is to put them in an external
file named \meta{filename}. Currently, the only file format that
\tikzname\ allows is the following: Each line of the \meta{filename}
should contain one line starting with two numbers, separated by a
space. Anything following the two numbers on the line is
ignored. Also, lines starting with a |%| or a |#| are ignored as well
as empty lines. (This is exactly the format that \textsc{gnuplot}
produces when you say |set terminal table|.) If necessary, more
formats will be supported in the future, but it is usually easy to
produce a file containing data in this form.

\begin{codeexample}[]
\tikz \draw plot[mark=x,smooth] file {plots/pgfmanual-sine.table};
\end{codeexample}

The file |plots/pgfmanual-sine.table| reads:
\begin{codeexample}[code only]
#Curve 0, 20 points
#x y type
0.00000 0.00000  i
0.52632 0.50235  i
1.05263 0.86873  i
1.57895 0.99997  i
...
9.47368 -0.04889  i
10.00000 -0.54402  i
\end{codeexample}
It was produced from the following source, using |gnuplot|:
\begin{codeexample}[code only]
set terminal table
set output "../plots/pgfmanual-sine.table"
set format "%.5f"
set samples 20
plot [x=0:10] sin(x)
\end{codeexample}

The \meta{local options} of the |plot| operation are local to each
plot and do not affect other plots ``on the same path.'' For example,
|plot[yshift=1cm]| will locally shift the plot 1cm upward. Remember,
however, that most options can only be applied to paths as a
whole. For example, |plot[red]| does not have the effect of making the
plot red. After all, you are trying to ``locally'' make part of the
path red, which is not possible.

\subsection{Plotting a Function}
\label{section-tikz-plot}

When you plot a function, the coordinates of the plot data can be
computed by evaluating a mathematical expression. Since \pgfname\
comes with a mathematical engine, you can specify this expression and
then have \tikzname\ produce the desired coordinates for you,
automatically.

Since this case is quite common when plotting a function, the syntax
is easy: Following the |plot| command and its local options, you
directly provide a \meta{coordinate expression}. It looks like a
normal coordinate, but inside you may use a special macro, which is
|\x| by default, but this can be changed using the |variable|
option. The \meta{coordinate expression} is then evaluated for
different values for |\x| and the resulting coordinates are plotted.

Note that you will often have to put the $x$- or $y$-coordinate inside
braces, namely whenever you use an expression involving a paranthesis.

The following options influence how the \meta{coordinate expression}
is evaluated:
\begin{key}{/tikz/variable=\meta{macro} (initially x)}
  Sets the macro whose value is set to the different values when
  \meta{coordinate expression} is evaluated.
\end{key}

\begin{key}{/tikz/samples=\meta{number} (initially 25)}
  Sets the number of samples used in the plot.
\end{key}

\begin{key}{/tikz/domain=\meta{start}|:|\meta{end} (initially -5:5)}
  Sets the domain between which the samples are taken.
\end{key}

\begin{key}{/tikz/samples at=\meta{sample list}}
  This option specifies a list of positions for which the
  variable should be evaluated. For instance, you can say
  |samples at={1,2,8,9,10}| to have the variable evaluated exactly for
  values $1$, $2$, $8$, $9$, and $10$. You can use the |\foreach|
  syntax, so you can use |...| inside the \meta{sample list}.

  When this option is used, the |samples| and |domain| option are
  overruled. The other ways round, setting either |samples| or
  |domain| will overrule this option.
\end{key}

\begin{codeexample}[]
\begin{tikzpicture}[domain=0:4]
  \draw[very thin,color=gray] (-0.1,-1.1) grid (3.9,3.9);
  
  \draw[->] (-0.2,0) -- (4.2,0) node[right] {$x$};
  \draw[->] (0,-1.2) -- (0,4.2) node[above] {$f(x)$};
  
  \draw[color=red]    plot (\x,\x)             node[right] {$f(x) =x$};
  \draw[color=blue]   plot (\x,{sin(\x r)})    node[right] {$f(x) = \sin x$};
  \draw[color=orange] plot (\x,{0.05*exp(\x)}) node[right] {$f(x) = \frac{1}{20} \mathrm e^x$};
\end{tikzpicture}
\end{codeexample}

\begin{codeexample}[]
\tikz \draw[scale=0.5,domain=-3.141:3.141,smooth,variable=\t]
  plot ({\t*sin(\t r)},{\t*cos(\t r)});
\end{codeexample}

\begin{codeexample}[]
\tikz \draw[domain=0:360,smooth,variable=\t]
  plot ({sin(\t)},\t/360,{cos(\t)});
\end{codeexample}


\subsection{Plotting a Function Using Gnuplot}
\label{section-tikz-gnuplot}

Often, you will want to plot points that are given via a function like
$f(x) = x \sin x$. Unfortunately, \TeX\ does not really have enough
computational power to generate the points on such a function
efficiently (it is a text processing program, after all). However,
if you allow it, \TeX\ can try to call external programs that can
easily produce the necessary points. Currently, \tikzname\ knows how to
call \textsc{gnuplot}.

When \tikzname\ encounters your operation
|plot[id=|\meta{id}|] function{x*sin(x)}| for 
the first time, it will create a file called
\meta{prefix}\meta{id}|.gnuplot|, where \meta{prefix} is |\jobname.| by
default, that is, the name of you main |.tex| file. If no \meta{id} is
given, it will be empty, which is alright, but it is better when each
plot has a unique \meta{id} for reasons explained in a moment. Next,
\tikzname\ writes some initialization code into this file followed by
|plot x*sin(x)|. The initialization code sets up things 
such that the |plot| operation will write the coordinates into another
file called \meta{prefix}\meta{id}|.table|. Finally, this table file
is read as if you had said |plot file{|\meta{prefix}\meta{id}|.table}|. 

For the plotting mechanism to work, two conditions must be met:
\begin{enumerate}
\item
  You must have allowed \TeX\ to call external programs. This is often
  switched off by default since this is a security risk (you might,
  without knowing, run a \TeX\ file that calls all sorts of ``bad''
  commands). To enable this ``calling external programs'' a command
  line option must be given to the \TeX\ program. Usually, it is
  called something like |shell-escape| or |enable-write18|. For
  example, for my |pdflatex| the option |--shell-escape| can be
  given.
\item
  You must have installed the |gnuplot| program and \TeX\ must find it
  when compiling your file.
\end{enumerate}

Unfortunately, these conditions will not always be met. Especially if
you pass some source to a coauthor and the coauthor does not have
\textsc{gnuplot} installed, he or she will have trouble compiling your
files.

For this reason, \tikzname\ behaves differently when you compile your
graphic for the second time: If upon reaching
|plot[id=|\meta{id}|] function{...}| the file \meta{prefix}\meta{id}|.table|
already exists \emph{and} if the \meta{prefix}\meta{id}|.gnuplot| file
contains what \tikzname\ thinks that it ``should'' contain, the |.table|
file is immediately read without trying to call a |gnuplot|
program. This approach has the following advantages: 
\begin{enumerate}
\item
  If you pass a bundle of your |.tex| file and all |.gnuplot| and
  |.table| files to someone else, that person can \TeX\ the |.tex|
  file without having to have |gnuplot| installed.
\item
  If the |\write18| feature is switched off for security reasons (a
  good idea), then, upon the first compilation of the |.tex| file, the
  |.gnuplot| will still be generated, but not the |.table|
  file. You can then simply call |gnuplot| ``by hand'' for each
  |.gnuplot| file, which will produce all necessary |.table| files.
\item
  If you change the function that you wish to plot or its
  domain, \tikzname\ will automatically try to regenerate the |.table|
  file.
\item
  If, out of laziness, you do not provide an |id|, the same |.gnuplot|
  will be used for different plots, but this is not a problem since
  the |.table| will automatically be regenerated for each plot
  on-the-fly. \emph{Note: If you intend to share your files with
  someone else, always use an id, so that the file can by typeset
  without having \textsc{gnuplot} installed.} Also, having unique ids
  for each plot will improve compilation speed since no external
  programs need to be called, unless it is really necessary.
\end{enumerate}

When you use |plot function{|\meta{gnuplot formula}|}|, the \meta{gnuplot
  formula} must be given in the |gnuplot| syntax, whose details are
beyond the scope of this manual. Here is the ultra-condensed
essence: Use |x| as the variable and use the C-syntax for normal
plots, use |t| as the variable for parametric plots. Here are some examples:

\begin{codeexample}[]
\begin{tikzpicture}[domain=0:4]
  \draw[very thin,color=gray] (-0.1,-1.1) grid (3.9,3.9);
  
  \draw[->] (-0.2,0) -- (4.2,0) node[right] {$x$};
  \draw[->] (0,-1.2) -- (0,4.2) node[above] {$f(x)$};
  
  \draw[color=red]    plot[id=x]   function{x}           node[right] {$f(x) =x$};
  \draw[color=blue]   plot[id=sin] function{sin(x)}      node[right] {$f(x) = \sin x$};
  \draw[color=orange] plot[id=exp] function{0.05*exp(x)} node[right] {$f(x) = \frac{1}{20} \mathrm e^x$};
\end{tikzpicture}
\end{codeexample}


The plot in influenced by the following options: First, the options
|samples| and |domain| explained earlier. Second, there are some more
specialized options.

\begin{key}{/tikz/parametric=\meta{boolena} (default true)}
  Sets whether the plot is a parametric plot. If true, then |t| must
  be used instead of |x| as the parameter and two comma-separated
  functions must be given in the \meta{gnuplot formula}. An example is
  the following:
\begin{codeexample}[]
\tikz \draw[scale=0.5,domain=-3.141:3.141,smooth]
  plot[parametric,id=parametric-example] function{t*sin(t),t*cos(t)};
\end{codeexample}
\end{key}

\begin{key}{/tikz/id=\meta{id}}
  Sets the identifier of the current plot. This should be a unique
  identifier for each plot (though things will also work if it is not,
  but not as well, see the explanations above). The \meta{id} will be
  part of a filename, so it should not contain anything fancy like |*|
  or |$|.%$
\end{key}

\begin{key}{/tikz/prefix=\meta{prefix}}
  The \meta{prefix} is put before each plot file name. The default is
  |\jobname.|, but 
  if you have many plots, it might be better to use, say |plots/| and
  have all plots placed in a directory. You have to create the
  directory yourself.
\end{key}

\begin{key}{/tikz/raw gnuplot}
  This key causes the \meta{gnuplot formula} to be passed on to
  \textsc{gnuplot} without setting up the samples or the |plot|
  operation. Thus, you could write
\begin{codeexample}[code only]
plot[raw gnuplot,id=raw-example] function{set samples 25; plot sin(x)}
\end{codeexample}
  This can be 
  useful for complicated things that need to be passed to
  \textsc{gnuplot}. However, for really complicated situations you
  should create a special external generating \textsc{gnuplot} file
  and use the |file|-syntax to include the table ``by hand.''
\end{key}

The following styles influence the plot:
\begin{stylekey}{/tikz/every plot (initially \normalfont empyt)}
  This style is installed in each plot, that is, as if you always said
\begin{codeexample}[code only]
  plot[every plot,...]
\end{codeexample}
 This is most useful for globally setting a prefix for all plots by saying:
\begin{codeexample}[code only]
\tikzset{every plot/.style={prefix=plots/}}
\end{codeexample}
\end{stylekey}



\subsection{Placing Marks on the Plot}

As we saw already, it is possible to add \emph{marks} to a plot using
the |mark| option. When this option is used, a copy of the plot
mark is placed on each point of the plot. Note that the marks are
placed \emph{after} the whole path has been drawn/filled/shaded. In
this respect, they are handled like text nodes. 

In detail, the following options govern how marks are drawn:
\begin{key}{/tikz/mark=\meta{mark mnemonic}}
  Sets the mark to a mnemonic that has previously been defined using
  the |\pgfdeclareplotmark|. By default, |*|, |+|, and |x| are available,
  which draw a filled circle, a plus, and a cross as marks. Many more
  marks become available when the library |pgflibraryplotmarks| is
  loaded. Section~\ref{section-plot-marks} lists the available plot
  marks.

  One plot mark is special: the |ball| plot mark is available only
  it \tikzname. The |ball color| determines the balls's color. Do not use
  this option with a large number of marks since it will take very long
  to render in PostScript.
  
  \begin{tabular}{lc}
    Option & Effect \\\hline \vrule height14pt width0pt
    \plotmarkentrytikz{ball}
  \end{tabular}
\end{key}

\begin{key}{/tikz/mark repeat=\meta{r}}
  This option tells \tikzname\ that only every $r$th mark should be
  drawn.
  
\begin{codeexample}[]
\tikz \draw plot[mark=x,mark repeat=3,smooth] file {plots/pgfmanual-sine.table};
\end{codeexample}
\end{key}

\begin{key}{/tikz/mark phase=\meta{p}}
  This option tells \tikzname\ that the first mark to be draw should
  be the $p$th, followed by the $(p+r)$th, then the $(p+2r)$th, and so
  on.
  
\begin{codeexample}[]
\tikz \draw plot[mark=x,mark repeat=3,mark phase=6,smooth] file {plots/pgfmanual-sine.table};
\end{codeexample}
\end{key}

\begin{key}{/tikz/mark indices=\meta{list}}
  This option allows you to specify explicitly the indices at which a
  mark should be placed. Counting starts with 1. You can use the
  |\foreach| syntax, that is, |...| can be used.
    
\begin{codeexample}[]
\tikz \draw plot[mark=x,mark indices={1,4,...,10,11,12,...,16,20},smooth]
  file {plots/pgfmanual-sine.table};
\end{codeexample}
\end{key}

\begin{key}{/tikz/mark size=\meta{dimension}}
  Sets the size of the plot marks. For circular plot marks,
  \meta{dimension} is the radius, for other plot marks
  \meta{dimension} should be about half the width and height.

  This option is not really necessary, since you achieve the same
  effect by specifying |scale=|\meta{factor} as a local option, where
  \meta{factor} is the quotient of the desired size and the default
  size. However, using |mark size| is a bit faster and more natural. 
\end{key}

\begin{key}{/tikz/mark options=\meta{options}}
  These options are applied to marks when they are drawn. For example,
  you can scale (or otherwise transform) the plot mark or set its
  color. 
\begin{codeexample}[]
\tikz \fill[fill=blue!20]
  plot[mark=triangle*,mark options={color=blue,rotate=180}]
    file{plots/pgfmanual-sine.table} |- (0,0);
\end{codeexample}
\end{key}



\subsection{Smooth Plots, Sharp Plots, and Comb Plots}

There are different things the |plot| operation can do with the points
it reads from a file or from the inlined list of points. By default,
it will connect these points by straight lines. However, you can also
use options to change the behavior of |plot|.

\begin{key}{/tikz/sharp plot}
  This is the default and causes the points to be connected by
  straight lines. This option is included only so that you can
  ``switch back'' if you ``globally'' install, say, |smooth|.
\end{key}

\begin{key}{/tikz/smooth}
  This option causes the points on the path to be connected using a
  smooth curve:

\begin{codeexample}[]
\tikz\draw plot[smooth] file{plots/pgfmanual-sine.table};
\end{codeexample}

  Note that the smoothing algorithm is not very intelligent. You will
  get the best results if the bending angles are small, that is, less
  than about $30^\circ$ and, even more importantly, if the distances
  between points are about the same all over the plotting path.
\end{key}

\begin{key}{/tikz/tension=\meta{value}}
  This option influences how ``tight'' the smoothing is. A lower value
  will result in sharper corners, a higher value in more ``round''
  curves. A value of $1$ results in a circle if four points at
  quarter-positions on a circle are given. The default is $0.55$. The
  ``correct'' value depends on the details of plot.
  
\begin{codeexample}[]
\begin{tikzpicture}[smooth cycle]
  \draw                 plot[tension=0.2]
    coordinates{(0,0) (1,1) (2,0) (1,-1)};
  \draw[yshift=-2.25cm] plot[tension=0.5]
    coordinates{(0,0) (1,1) (2,0) (1,-1)};
  \draw[yshift=-4.5cm]  plot[tension=1]
    coordinates{(0,0) (1,1) (2,0) (1,-1)};
\end{tikzpicture}
\end{codeexample}
\end{key}

\begin{key}{/tikz/smooth cycle}
  This option causes the points on the path to be connected using a
  closed smooth curve. 

\begin{codeexample}[]
\tikz[scale=0.5]
  \draw plot[smooth cycle] coordinates{(0,0) (1,0) (2,1) (1,2)}
        plot               coordinates{(0,0) (1,0) (2,1) (1,2)} -- cycle;
\end{codeexample}
\end{key}

\begin{key}{/tikz/ycomb}
  This option causes the |plot| operation to interpret the plotting
  points differently. Instead of connecting them, for each point of
  the plot a straight line is added to the path from the $x$-axis to the point,
  resulting in a sort of ``comb'' or ``bar diagram.''

\begin{codeexample}[]
\tikz\draw[ultra thick] plot[ycomb,thin,mark=*] file{plots/pgfmanual-sine.table};
\end{codeexample}

\begin{codeexample}[]
\begin{tikzpicture}[ycomb]
  \draw[color=red,line width=6pt]
    plot coordinates{(0,1) (.5,1.2) (1,.6) (1.5,.7) (2,.9)};
  \draw[color=red!50,line width=4pt,xshift=3pt]
    plot coordinates{(0,1.2) (.5,1.3) (1,.5) (1.5,.2) (2,.5)};
\end{tikzpicture}
\end{codeexample}
\end{key}


\begin{key}{/tikz/xcomb}
  This option works like |ycomb| except that the bars are horizontal. 

\begin{codeexample}[]
\tikz \draw plot[xcomb,mark=x] coordinates{(1,0) (0.8,0.2) (0.6,0.4) (0.2,1)};
\end{codeexample}
\end{key}


\begin{key}{/tikz/polar comb}
  This option causes a line from the origin to the point to be added
  to the path for each plot point.

\begin{codeexample}[]
\tikz \draw plot[polar comb,
     mark=pentagon*,mark options={fill=white,draw=red},mark size=4pt]
   coordinates {(0:1cm) (30:1.5cm) (160:.5cm) (250:2cm) (-60:.8cm)};
\end{codeexample}
\end{key}


\begin{key}{/tikz/only marks}
  This option causes only marks to be shown; no path segments are
  added to the actual path. This can be useful for quickly adding some
  marks to a path.

\begin{codeexample}[]
\tikz \draw (0,0) sin (1,1) cos (2,0)
  plot[only marks,mark=x] coordinates{(0,0) (1,1) (2,0) (3,-1)};
\end{codeexample}
\end{key}


% Copyright 2006 by Till Tantau
%
% This file may be distributed and/or modified
%
% 1. under the LaTeX Project Public License and/or
% 2. under the GNU Free Documentation License.
%
% See the file doc/generic/pgf/licenses/LICENSE for more details.


\section{Transparency}

\label{section-tikz-transparency}


\subsection{Overview}

Normally, when you paint something using any of \tikzname's commands
(this includes stroking, filling, shading, patterns, and images), the
newly painted objects totally obscure whatever was painted earlier in
the same area.

You can change this behaviour by using something that can be thought
of as ``(semi)transparent colors.'' Such colors do not completely
obscure the background, rather they blend the background with the new
color. At first sight, using such semitransparent colors might seem quite
straightforward, but the math going on in the background is quite
involved and the correct handling of transparency fills some 64 pages
in the PDF specification. 

In the present section, we start with the different ways of specifying
``how transparent'' newly drawn objects should be. The simplest way is
to just specify a percentage like ``60\% transparent.'' A much more
general way is to use something that I call a \emph{fading,} also
known as a soft mask or a mask.

At the end of the section we adress the problem of creating so-called
\emph{transparency groups}. This problem arises when you paint over a
position several times with a semitransparent color. Sometimes you
want the effect to accumulate, sometimes you do not.

\emph{Note:} Transparency is best supported by the pdf\TeX\
driver. The \textsc{svg} driver also has some support. For PostScript
output, opacity is rendered correctly only with the most recent
versions of GhostScript. Printers and other programs will typically
ignore the opacity setting. 



\subsection{Specifying a Uniform Opacity}

Specifying a stroke and/or fill opacity is quite easy using the
following options.


\begin{key}{/tikz/draw opacity=\meta{value}}
  This option sets ``how transparent'' lines should be. A value of |1|
  means ``fully opaque'' or ``not transparent at all,'' a value of |0|
  means ``fully transparent'' or ``invisible.'' A value of |0.5|
  yields lines that are semitransparent.

  Note that when you use PostScript as your output format,
  this option works only with recent versions of GhostScript.
   
\begin{codeexample}[]
\begin{tikzpicture}[line width=1ex]
  \draw (0,0) -- (3,1);
  \filldraw [fill=examplefill,draw opacity=0.5] (1,0) rectangle (2,1);
\end{tikzpicture}
\end{codeexample}
\end{key}

Note that the |draw opacity| options only sets the opacity of drawn
lines. The opacity of fillings is set using the option
|fill opacity| (documented in Section~\ref{section-fill-opacity}. The
option |opacity| sets both at the same time. 

\begin{key}{/tikz/opacity=\meta{value}}
  Sets both the drawing and filling opacity to \meta{value}.

  The following predefined styles make it easier to use this option:
  \begin{stylekey}{/tikz/transparent}
    Makes everything totally transparent and, hence, invisible.

\begin{codeexample}[]
\tikz{\fill[red]             (0,0)   rectangle (1,0.5);
      \fill[transparent,red] (0.5,0) rectangle (1.5,0.25); }
\end{codeexample}
  \end{stylekey}

  \begin{stylekey}{/tikz/ultra nearly transparent}
    Makes everything, well, ultra nearly transparent.

\begin{codeexample}[]
\tikz{\fill[red]                      (0,0)   rectangle (1,0.5);
      \fill[ultra nearly transparent] (0.5,0) rectangle (1.5,0.25); }
\end{codeexample}
  \end{stylekey}

  \begin{stylekey}{/tikz/very nearly transparent}
\begin{codeexample}[]
\tikz{\fill[red]                     (0,0)   rectangle (1,0.5);
      \fill[very nearly transparent] (0.5,0) rectangle (1.5,0.25); }
\end{codeexample}
  \end{stylekey}

  \begin{stylekey}{/tikz/nearly transparent}
\begin{codeexample}[]
\tikz{\fill[red]                (0,0)   rectangle (1,0.5);
      \fill[nearly transparent] (0.5,0) rectangle (1.5,0.25); }
\end{codeexample}
  \end{stylekey}

  \begin{stylekey}{/tikz/semitransparent} 
\begin{codeexample}[]
\tikz{\fill[red]             (0,0)   rectangle (1,0.5);
      \fill[semitransparent] (0.5,0) rectangle (1.5,0.25); }
\end{codeexample}
  \end{stylekey}

  \begin{stylekey}{/tikz/nearly opaque}   
\begin{codeexample}[]
\tikz{\fill[red]           (0,0)   rectangle (1,0.5);
      \fill[nearly opaque] (0.5,0) rectangle (1.5,0.25); }
\end{codeexample}
  \end{stylekey}
 
  \begin{stylekey}{/tikz/very nearly opaque} 
\begin{codeexample}[]
\tikz{\fill[red]                (0,0)   rectangle (1,0.5);
      \fill[very nearly opaque] (0.5,0) rectangle (1.5,0.25); }
\end{codeexample}
  \end{stylekey}

  \begin{stylekey}{/tikz/ultra nearly opaque}
\begin{codeexample}[]
\tikz{\fill[red]                 (0,0)   rectangle (1,0.5);
      \fill[ultra nearly opaque] (0.5,0) rectangle (1.5,0.25); }
\end{codeexample}
  \end{stylekey}

  \begin{stylekey}{/tikz/opaque}
    This yields completely opaque drawings, which is the default.
\begin{codeexample}[]
\tikz{\fill[red]    (0,0)   rectangle (1,0.5);
      \fill[opaque] (0.5,0) rectangle (1.5,0.25); }
\end{codeexample}
  \end{stylekey}
\end{key}


\begin{key}{/tikz/fill opacity=\meta{value}}
  This option sets the opacity of fillings. In addition to filling
  operations, this opacity also applies to text and images.

  Note, again, that when you use PostScript as your output format,
  this option works only with recent versions of GhostScript.
  
\begin{codeexample}[]
\begin{tikzpicture}[thick,fill opacity=0.5]
  \filldraw[fill=red]   (0:1cm)    circle (12mm);
  \filldraw[fill=green] (120:1cm)  circle (12mm);
  \filldraw[fill=blue]  (-120:1cm) circle (12mm);
\end{tikzpicture}
\end{codeexample}

\begin{codeexample}[]
\begin{tikzpicture}
  \fill[red] (0,0) rectangle (3,2);

  \node                   at (0,0) {\huge A};
  \node[fill opacity=0.5] at (3,2) {\huge B};
\end{tikzpicture}
\end{codeexample}
\end{key}

\begin{key}{/tikz/text opacity=\meta{value}}
  Sets the opacity of text labels, overriding the |fill opacity| setting. 
\begin{codeexample}[]
\begin{tikzpicture}[every node/.style={fill,draw}]
  \draw[line width=2mm,blue!50,line cap=round] (0,0) grid (3,2);

  \node[opacity=0.5] at (1.5,2) {Upper node};
  \node[draw opacity=0.8,fill opacity=0.2,text opacity=1]
    at (1.5,0) {Lower node};
\end{tikzpicture}
\end{codeexample}
\end{key}


Note the following effect: If you setup a certain opacity for stroking
or filling and you stroke or fill the same area twice, the effect
accumulates:

\begin{codeexample}[]
\begin{tikzpicture}[fill opacity=0.5]
  \fill[red] (0,0) circle (1);
  \fill[red] (1,0) circle (1);
\end{tikzpicture}
\end{codeexample}

Often, this is exactly what you intend, but not always. You can use
transparency groups, see the end of this section, to change this.


\subsection{Fadings}

For complicated graphics, uniform transparency settings are not always
sufficient. Suppose, for instance, that while you paint a picture, you
want the transparency to vary smoothly from completely opaque to
completely transparent. This is a ``shading-like'' transparency. For
such a form of transparency I will use the term \emph{fading} (as a
noun). They are also known as \emph{soft masks}, \emph{opacity masks},
\emph{masks}, or \emph{soft clips}.


\subsubsection{Creating Fadings}

How do we specify a fading? This is a bit of an art since the
underlying mechanism is quite powerful, but a bit difficult to use.

Let us start with a bit of terminology. A \emph{fading} specifies for
each point of an area to transparency of the point. This transparency
can by any number between 0 and 1. A \emph{fading picture} is a normal
graphic that, in a way to be described in a moment, determines the
transparency of points inside the fading. Each fading has an
underlying fading picture. 

The fading picture is a normal graphic drawn using any of the normal
graphic drawing commands. A fading and its fading picture are related
as follows: Given any point of the fading, the transparency of this
point is determined by the lumonisity of the fading picture at the
same position. The luminosity of a point determines ``how bright'' the
point is. The brighter the point in the fading picture, the more
opaque is the point in the fading. In particular, a white point of the
fading picture is completely opaque in the fading and a black point of
the fading picture is completely transparent in the fading. (The
background of the fading picture is always transparent in the fading
as if the background where black.)

It is rather counter-intuitive that a \emph{white} pixel of the fading
picture will be \emph{opaque} in the fading and a \emph{black} pixel
will be \emph{transparent}. For this reason, \tikzname\ defines a
color called |transparent| that is the same as |black|. The nice thing
about this definition is that the color
|transparent!|\meta{percentage} in the fading picture yields a
pixel that is \meta{percantage} per cent transparent in the fading. 

Turning a fading picture into a normal picture is achieved using the
following commands, which are \emph{only defined in the library},
namely the library |fadings|. So, to use them, you have to say
|\usetikzlibrary{fadings}| first.

\begin{environment}{{tikzfadingfrompicture}\oarg{options}}
  This command works like a |{tikzpicture}|, only the picture is not
  shown, but instead a fading is defined based on this picture. To set
  the name of the picture, use the |name| option (which is normally
  used to set the name of a node).
  \begin{key}{/tikz/name=\marg{name}}
    Use this option with the |{tikzfadingfrompicture}| environment to
    set the name of the fading. You \emph{must} provide this option.
  \end{key}
  
  The following shading is 2cm by 2cm and changes gets more and more
  transparent from left to right, but is 50\% transparent for a large
  circle in the middle.
\begin{codeexample}[]
\begin{tikzfadingfrompicture}[name=fade right]
  \shade[left color=transparent!0,
         right color=transparent!100] (0,0) rectangle (2,2);
  \fill[transparent!50] (1,1) circle (0.7);
\end{tikzfadingfrompicture}

% Now we use the fading in another picture:
\begin{tikzpicture}
  % Background
  \fill [black!20] (-1.2,-1.2) rectangle (1.2,1.2);
  \pattern [pattern=checkerboard,pattern color=black!30]
                   (-1.2,-1.2) rectangle (1.2,1.2);
  
  \fill [path fading=fade right,red] (-1,-1) rectangle (1,1);
\end{tikzpicture}
\end{codeexample}
  In the next example we create a fading picture that contains some
  text. When the fading is used, we only see the shading ``through
  it.'' 
\begin{codeexample}[]
\begin{tikzfadingfrompicture}[name=tikz]
  \node [text=transparent!20]
    {\fontfamily{ptm}\fontsize{45}{45}\bfseries\selectfont Ti\emph{k}Z};
\end{tikzfadingfrompicture}

% Now we use the fading in another picture:
\begin{tikzpicture}
  \fill [black!20] (-2,-1) rectangle (2,1);
  \pattern [pattern=checkerboard,pattern color=black!30]
                   (-2,-1) rectangle (2,1);

  \shade[path fading=tikz,fit fading=false,
         left color=blue,right color=black]
    (-2,-1) rectangle (2,1);
\end{tikzpicture}
\end{codeexample}
\end{environment}

\begin{plainenvironment}{{tikzfadingfrompicture}\oarg{options}}
  The plain\TeX\ version of the environment.
\end{plainenvironment}

\begin{contextenvironment}{{tikzfadingfrompicture}\oarg{options}}
  The Con\TeX t version of the environment.
\end{contextenvironment}

\begin{command}{\tikzfading\oarg{options}}
  This command is used to define a fading similarly to that way a
  shading is defined. In the \meta{options} you should 
  \begin{enumerate}
  \item use the |name=|\meta{name} option to set a name for the fading,
  \item use the |shading| option to set the name of the shading that
    you wish to use,
  \item extra options for setting the colors of the shading (typically
    you will set them to the color |transparent!|\meta{percentage}).
  \end{enumerate}
  Then, a new fading named \meta{name} will be created based on the
  shading.

\begin{codeexample}[]
\tikzfading[name=fade right,
            left color=transparent!0,
            right color=transparent!100]

% Now we use the fading in another picture:
\begin{tikzpicture}
  % Background
  \fill [black!20] (-1.2,-1.2) rectangle (1.2,1.2);
  \path [pattern=checkerboard,pattern color=black!30]
                   (-1.2,-1.2) rectangle (1.2,1.2);

  \fill [red,path fading=fade right] (-1,-1) rectangle (1,1);
\end{tikzpicture}
\end{codeexample}  

\begin{codeexample}[]
\tikzfading[name=fade out,
            inner color=transparent!0,
            outer color=transparent!100]

% Now we use the fading in another picture:
\begin{tikzpicture}
  % Background
  \fill [black!20] (-1.2,-1.2) rectangle (1.2,1.2);
  \path [pattern=checkerboard,pattern color=black!30]
                   (-1.2,-1.2) rectangle (1.2,1.2);

  \fill [blue,path fading=fade out] (-1,-1) rectangle (1,1);
\end{tikzpicture}
\end{codeexample}  
\end{command}



\subsubsection{Fading a Path}

Aa fading specifies for each pixel of a certain area how transparent
this pixel will be. The following options are used to install such a
fading for the current scope or path. 

\pgfdeclarefading{fade down}{%
  \tikzset{top color=pgftransparent!0,bottom color=pgftransparent!100}
  \pgfuseshading{axis}
}
\pgfdeclarefading{fade inside}{%
  \tikzset{inner color=pgftransparent!90,outer color=pgftransparent!30}
  \pgfuseshading{radial}
}

\begin{key}{/tikz/path fading=\meta{name} (default \normalfont scope's setting)}
  This option tells \tikzname\ that the current path should be faded
  with the fading \meta{name}. If no \meta{name} is given, the 
  \meta{name} set for the whole scope is used. Similarly to options
  like |draw| or |fill|, this option is reset for each path, so you
  have to add it to each path that should be faded. You can also
  specify |none| as \meta{name}, in which case fading for the path
  will be switched off in case it has been switched on by previous
  options or styles.
\begin{codeexample}[]
\begin{tikzpicture}[path fading=south]
  % Checker board
  \fill [black!20] (0,0) rectangle (4,3);
  \pattern [pattern=checkerboard,pattern color=black!30]
                   (0,0) rectangle (4,3);

  \fill [color=blue]                   (0.5,1.5) rectangle +(1,1);
  \fill [color=blue,path fading=north] (2.5,1.5) rectangle +(1,1);

  \fill [color=red,path fading]        (1,0.75) ellipse (.75 and .5);
  \fill [color=red]                    (3,0.75) ellipse (.75 and .5);
\end{tikzpicture}
\end{codeexample}

  \begin{key}{/tikz/fit fading=\meta{boolean} (default true, initially true)}
    When set to |true|, the fading is shifted and resized (in exactly
    the same way as a shading) so that is covers the current
    path. When set to |false|, the fading is only shifted so that it
    is centered on the path's center, but it is not resized. This can
    be useful for special-purpose fadings, for instance when you use a
    fading to ``punsh out'' something.
  \end{key}

  \begin{key}{/tikz/fading transform=\meta{transformation options}}
    The \meta{transformation options} are applied to the fading before
    it is used. For instance, if \meta{transformation options} is set
    to |rotate=90|, the fading is rotated by 90 degrees.
\begin{codeexample}[]
\begin{tikzpicture}[path fading=fade down]
  % Checker board
  \fill [black!20] (0,0) rectangle (4,1.5);
  \path [pattern=checkerboard,pattern color=black!30] (0,0) rectangle (4,1.5);

  \fill [red,path fading,fading transform={rotate=90}]
    (1,0.75) ellipse (.75 and .5);
  \fill [red,path fading,fading transform={rotate=30}]
    (3,0.75) ellipse (.75 and .5);
\end{tikzpicture}
\end{codeexample}
  \end{key}
  
  \begin{key}{/tikz/fading angle=\meta{degree}}
    A shortcut for |fading transform={rotate=|\meta{degree}|}|.
  \end{key}

  Note that you can ``fade just about anything.'' In particular, you
  can fade a shading.
  
\begin{codeexample}[]
\begin{tikzpicture}
  % Checker board
  \fill [black!20] (0,0) rectangle (4,4);
  \path [pattern=checkerboard,pattern color=black!30] (0,0) rectangle (4,4);

  \shade [ball color=blue,path fading=south] (2,2) circle (1.8);
\end{tikzpicture}
\end{codeexample}

  The |fade inside| of the following example more transparent in the middle than on the
  outside.

\begin{codeexample}[]
\tikzfading[name=fade inside,
            inner color=transparent!80,
            outer color=transparent!30]    
\begin{tikzpicture}
  % Checker board
  \fill [black!20] (0,0) rectangle (4,4);
  \path [pattern=checkerboard,pattern color=black!30] (0,0) rectangle (4,4);

  \shade [ball color=red] (3,3) circle (0.8);
  \shade [ball color=white,path fading=fade inside] (2,2) circle (1.8);
\end{tikzpicture}
\end{codeexample}

  Note that adding the |path fading| option to a node fades the
  (background) path, not the text itself. To fade the text, you need
  to use a scope fading (see below).
\end{key}

Note that using fadings in conjunction with patterns can create
visually rather pleasing effects:
\begin{codeexample}[]
\tikzfading[name=middle,
            top color=transparent!50,
            bottom color=transparent!50,
            middle color=transparent!20]
\begin{tikzpicture}
  \node      [circle,circular drop shadow,
              pattern=horizontal lines dark blue,
              path fading=south,
              minimum size=3.6cm] {};
  \pattern   [path fading=north,
              pattern=horizontal lines dark gray]
    (0,0) circle (1.8cm);
  \pattern   [path fading=middle,
              pattern=crosshatch dots light steel blue]
    (0,0) circle (1.8cm);
\end{tikzpicture}
\end{codeexample}


\subsubsection{Fading a Scope}

In addtion to fading individual paths, you may also wish to ``fade a
scope,'' that is, you may wish to install a fading that is used
globally to specify the transparency for all objects drawn inside a
scope. This effect can also be thought of as a ``soft clip'' and it
works in a similar way: You add the |scope fading| option to a path in
a scope -- typically the first one -- and then all subsequent drawings
in the scope are faded. You will use a |transparency group| in
conjunction, see the end of this section.

\begin{key}{/tikz/scope fading=\meta{fading}}
  In principle, this key works in excatly the same way as the
  |path fading| key. The only difference is, that the effect of the
  fading will persist after the current path till the end of the
  scope. Thus, the \meta{fading} is applied to all subsequent drawings
  in the current scope, not just to the current path. In this regard,
  the option works very much like the |clip| option. (Note, however,
  that, unlike the |clip| option, fadings to not accumulate unless a
  transparency group is used.)

  The keys |fit fading| and |fading transform| have the same effect as
  for |path fading|. Also that, just as for |path fading|, providing
  the |scope fading| option with a |{scope}| only sets the name of the
  fading to be used. You have to explicitly provide the |scope fading|
  with a path to actually install a fading.
  
\begin{codeexample}[]
\begin{tikzpicture}
  \fill [black!20] (-2,-2) rectangle (2,2);
  \pattern [pattern=checkerboard,pattern color=black!30]
                   (-2,-2) rectangle (2,2);

  % The bounding box of the shading:
  \draw [red] (-50bp,-50bp) rectangle (50bp,50bp);

  \path [scope fading=south,fit fading=false] (0,0);
  % fading is centered at its natural size

  \fill[red]   ( 90:1) circle (1);
  \fill[green] (210:1) circle (1);
  \fill[blue]  (330:1) circle (1);
\end{tikzpicture}
\end{codeexample}

  In the following example we resize the fading to the size of the
  whole picture:
\begin{codeexample}[]
\begin{tikzpicture}
  \fill [black!20] (-2,-2) rectangle (2,2);
  \pattern [pattern=checkerboard,pattern color=black!30]
                   (-2,-2) rectangle (2,2);

  \path [scope fading=south] (-2,-2) rectangle (2,2);

  \fill[red]   ( 90:1) circle (1);
  \fill[green] (210:1) circle (1);
  \fill[blue]  (330:1) circle (1);
\end{tikzpicture}
\end{codeexample}

  Scope fadings are also needed if you wish to fade a node.
\begin{codeexample}[]
\tikz \node [scope fading=south,fading angle=45,text width=3.5cm]
{
  This is some text that will fade out as we go right
  and down. It is pretty hard to achieve this effect in
  other ways.
};    
\end{codeexample}

\end{key}


\subsection{Transparency Groups}

Consider the following cross and sign. They ``look wrong'' because we
can see how they were constructed, while this is not really part of
the desired effect. 

\begin{codeexample}[]
\begin{tikzpicture}[opacity=.5]
  \draw [line width=5mm] (0,0) -- (2,2);
  \draw [line width=5mm] (2,0) -- (0,2);
\end{tikzpicture}
\end{codeexample}

\begin{codeexample}[]
\begin{tikzpicture}
  \node at (0,0) [forbidden sign,line width=2ex,draw=red,fill=white] {Smoking};

  \node [opacity=.5]
        at (2,0) [forbidden sign,line width=2ex,draw=red,fill=white] {Smoking};
\end{tikzpicture}
\end{codeexample}

Transparency groups are used to render them correctly:

\begin{codeexample}[]
\begin{tikzpicture}[opacity=.5]
  \begin{scope}[transparency group]
    \draw [line width=5mm] (0,0) -- (2,2);
    \draw [line width=5mm] (2,0) -- (0,2);
  \end{scope}
\end{tikzpicture}
\end{codeexample}

\begin{codeexample}[]
\begin{tikzpicture}
  \node at (0,0) [forbidden sign,line width=2ex,draw=red,fill=white] {Smoking};

  \begin{scope}[opacity=.5,transparency group]
    \node at (2,0) [forbidden sign,line width=2ex,draw=red,fill=white]
      {Smoking};
  \end{scope}
\end{tikzpicture}
\end{codeexample}

\begin{key}{/tikz/transparency group}
  This option can be given to a |scope|. It will have the following
  effect: The scope's contents is stroked/filled
  ``ignoring any outside transparency.'' This means, all previous
  transparency settings are ignored (you can still set transparency
  inside the group, but never mind). For instance, in the forbidden
  sign example, the whole sign is first painted (conceptually) like
  the image on the left hand side. Note that some pixels of the sign
  are painted multiple times (up to three times), but only the last
  color ``wins.''

  Then, when the scope is finished, it is painted as a whole. The  
  \emph{fill} transparency settings are now applied to the resulting
  picutre. For instance, the pixel that has been painted three times
  is just red at the end, so this red color will be blended with
  whatever is ``behind'' the group on the page.

  Note that, depending on the driver, it is possible to directly put
  objects in a transparency group that lie outside the picture. This
  has to do with internal bounding box computations.
  Section~\ref{section-transparency} explains how to sidestep this
  problem.   
\end{key}



%%% Local Variables: 
%%% mode: latex
%%% TeX-master: "pgfmanual"
%%% End: 


��!�}	� �
����DEV��INO��SYN��SV~�pgfmanual-en-tikz-decorations.tex�J��JЀ�J�j��ZP�_.j�
% Copyright 2006 by Till Tantau
%
% This file may be distributed and/or modified
%
% 1. under the LaTeX Project Public License and/or
% 2. under the GNU Free Documentation License.
%
% See the file doc/generic/pgf/licenses/LICENSE for more details.

\section{Transformations}

\pgfname\ has a powerful transformation mechanism that is similar to
the transformation capabilities of \textsc{metafont}. The present
section explains how you can access it in \tikzname.


\subsection{The Different Coordinate Systems}

It is a long process from  a coordinate like, say, $(1,2)$ or
$(1\mathrm{cm},5\mathrm{pt})$, to the position a point is finally
placed on the display or paper. In order to find out where the point
should go, it is constantly ``transformed,'' which means that it is
mostly shifted around and possibly rotated, slanted, scaled, and
otherwise mutilated. 

In detail, (at least) the following transformations are applied to a
coordinate like $(1,2)$ before a point on the screen is chosen:
\begin{enumerate}
\item
  \pgfname\ interprets a coordinate like $(1,2)$  in its
  $xy$-coordinate system as ``add the current $x$-vector once and the
  current $y$-vector twice to obtain the new point.''
\item
  \pgfname\ applies its coordinate transformation matrix to the
  resulting coordinate. This yields the final position of the point 
  inside the picture.
\item
  The backend driver (like |dvips| or |pdftex|) adds transformation
  commands such the coordinate is shifted to the correct position in
  \TeX's page coordinate system.
\item
  \textsc{pdf} (or PostScript) apply the canvas transformation
  matrix to the point, which can once more change the position on the
  page. 
\item
  The viewer application or the printer applies the device
  transformation matrix to transform the coordinate to its final pixel
  coordinate on the screen or paper.  
\end{enumerate}

In reality, the process is even more involved, but the above should
give the idea: A point is constantly transformed by changes of the
coordinate system.

In \tikzname, you only have access to the first two coordinate systems:
The $xy$-coordinate system and the coordinate transformation matrix
(these will be explained later). \pgfname\ also allows you to change
the canvas transformation matrix, but you have to use commands of
the core layer directly to do so and you ``better know what you are
doing'' when you do this. The moment you start modifying the
canvas matrix, \pgfname\ immediately looses track of all
coordinates and shapes, anchors, and bounding box computations will no
longer work.


\subsection{The XY- and XYZ-Coordinate Systems}
\label{section-xyz}

The first and easiest coordinate systems are \pgfname's $xy$- and
$xyz$-coordinate systems. The idea is very simple: Whenever you
specify a coordinate like |(2,3)| this means $2v_x + 3v_y$, where
$v_x$ is the current \emph{$x$-vector} and $v_y$ is the current
\emph{$y$-vector}. Similarly, the coordinate |(1,2,3)| means $v_x +
2v_y + 3v_z$.

Unlike other packages, \pgfname\ does not insist that $v_x$ actually
has a $y$-component of $0$, that is, that it is a horizontal
vector. Instead, the $x$-vector can point anywhere you
want. Naturally, \emph{normally} you will want the $x$-vector to point
horizontally.

One undesirable effect of this flexibility is that it is not possible
to provide mixed coordinates as in $(1,2\mathrm{pt})$. Life is hard.

To change the $x$-, $y$-, and $z$-vectors, you can use the following
options:

\begin{key}{/tikz/x=\meta{value} (initially 1cm)}
  If \meta{value} is a dimension, the $x$-vector of
  \pgfname's $xyz$-coordinate system is setup to point 
  \meta{value} to the right, that is, to $(\meta{value},0pt)$.

\begin{codeexample}[]
\begin{tikzpicture}
  \draw                  (0,0)   -- +(1,0);
  \draw[x=2cm,color=red] (0,0.1) -- +(1,0);
\end{tikzpicture}
\end{codeexample}    

\begin{codeexample}[]
\tikz \draw[x=1.5cm] (0,0) grid (2,2);
\end{codeexample}    

  The last example shows that the size of steppings in grids, just like
  all other dimensions, are not affected by the $x$-vector. After all,
  the $x$-vector is only used to determine the coordinate of the upper
  right corner of the grid.

  If \meta{value} is a coordinate, the $x$-vector of
  \pgfname's $xyz$-coordinate system to the specified coordinate. If
  \meta{value} contains a comma, it must be put in braces. 

\begin{codeexample}[]
\begin{tikzpicture}
  \draw                            (0,0) -- (1,0);
  \draw[x={(2cm,0.5cm)},color=red] (0,0) -- (1,0);
\end{tikzpicture}
\end{codeexample}

  You can use this, for example, to exchange the meaning of the $x$- and
  $y$-coordinate.

\begin{codeexample}[]
\begin{tikzpicture}[smooth]
  \draw plot coordinates{(1,0) (2,0.5) (3,0) (3,1)};
  \draw[x={(0cm,1cm)},y={(1cm,0cm)},color=red]
        plot coordinates{(1,0) (2,0.5) (3,0) (3,1)};
\end{tikzpicture}
\end{codeexample}
\end{key}

\begin{key}{/tikz/y=\meta{value} (initially 1cm)}
  Works like the |x=| option, only if \meta{value} is a dimension, the
  resulting vector points to $(0,\meta{value})$.
\end{key}

\begin{key}{/tikz/z=\meta{value} (initially \normalfont$-\sqrt{2}$cm)}
  Works like the |y=| option, but now a dimension is means the point
  $(\meta{value},\meta{value})$.

\begin{codeexample}[]
\begin{tikzpicture}[z=-1cm,->,thick]
  \draw[color=red] (0,0,0) -- (1,0,0);
  \draw[color=blue] (0,0,0) -- (0,1,0);
  \draw[color=orange] (0,0,0) -- (0,0,1);
\end{tikzpicture}
\end{codeexample}
\end{key}



\subsection{Coordinate Transformations}

\pgfname\ and \tikzname\ allow you to specify \emph{coordinate
  transformations}. Whenever you specify a coordinate as in |(1,0)| or
|(1cm,1pt)| or |(30:2cm)|, this coordinate is first
``reduced'' to a position of the form ``$x$ points to the right and
  $y$ points upwards.'' For example, |(1in,5pt)| is reduced to
``$72\frac{72}{100}$ points to the right and 5 points upwards'' and
|(90:100pt)| means ``0pt to the right and 100 points upwards.''

The next step is to apply the current \emph{coordinate transformation
  matrix} to the coordinate. For example, the coordinate
transformation matrix might currently be set such that it adds a
certain constant to the $x$ value. Also, it might be setup such that
it, say, exchanges the $x$ and $y$ value. In general, any
``standard'' transformation like translation, rotation, slanting, or
scaling or any combination thereof is possible. (Internally, \pgfname\
keeps track of a coordinate transformation matrix very much like the
concatenation matrix used by \textsc{pdf} or PostScript.)

\begin{codeexample}[]
\begin{tikzpicture}
  \draw[help lines] (0,0) grid (3,2);
  \draw (0,0) rectangle (1,0.5);
  \begin{scope}[xshift=1cm]
    \draw             [red]    (0,0) rectangle (1,0.5);
    \draw[yshift=1cm] [blue]   (0,0) rectangle (1,0.5);
    \draw[rotate=30]  [orange] (0,0) rectangle (1,0.5);
  \end{scope}
\end{tikzpicture}
\end{codeexample}

The most important aspect of the coordinate transformation matrix is
\emph{that it applies to coordinates only!} In particular, the
coordinate transformation has no effect on things like the line width
or the dash pattern or the shading angle. In certain cases, it is not
immediately clear whether the coordinate transformation matrix
\emph{should} apply to a certain dimension. For example, should the
coordinate transformation matrix apply to grids? (It does.) And what
about the size of arced corners? (It does not.) The general rule is
``If there is no `coordinate' involved, even `indirectly,' the matrix
is not applied.'' However, sometimes, you simply have to try or look
it up in the documentation whether the matrix will be applied.

Setting the matrix cannot be done directly. Rather, all you can do is
to ``add'' another transformation to the current matrix. However, all
transformations are local to the current \TeX-group. All
transformations are added using graphic options, which are described
below.

Transformations apply immediately when they are encountered ``in the
middle of a path'' and they apply only to the coordinates on the path
following the transformation option. 

\begin{codeexample}[]
\tikz \draw (0,0) rectangle (1,0.5) [xshift=2cm] (0,0) rectangle (1,0.5);
\end{codeexample}

A final word of warning: You should refrain from using ``aggressive''
transformations like a scaling of a factor of 10000. The reason is
that all transformations are done using \TeX, which has a fairly low
accuracy. Furthermore, in certain situations it is necessary that
\tikzname\ \emph{inverts} the current transformation matrix and this will
fail if the transformation matrix is badly conditioned or even
singular (if you do not know what singular matrices are, you are blessed).   

\begin{key}{/tikz/shift={\ttfamily\char`\{}\meta{coordinate}{\ttfamily\char`\}}}
  Adds the \meta{coordinate} to all coordinates.
\begin{codeexample}[]
\begin{tikzpicture}
  \draw[help lines] (0,0) grid (3,2);
  \draw                       (0,0) -- (1,1) -- (1,0);
  \draw[shift={(1,1)},blue]   (0,0) -- (1,1) -- (1,0);
  \draw[shift={(30:1cm)},red] (0,0) -- (1,1) -- (1,0);
\end{tikzpicture}
\end{codeexample}
\end{key}

\begin{key}{/tikz/shift only}
  This option does not take any parameter. Its effect is to cancel all
  current transformations except for the shifting. This means that the
  origin will remain where it is, but any rotation around the origin
  or scaling relative to the origin or skewing will no longer have an
  effect.

  This option is useful in situtations where a complicated
  transformation is used to ``get to a position,'' but you then wish
  to draw something ``normal'' at this position. 

\begin{codeexample}[]
\begin{tikzpicture}
  \draw[help lines] (0,0) grid (3,2);
  \draw                                      (0,0) -- (1,1) -- (1,0);
  \draw[rotate=30,xshift=2cm,blue]           (0,0) -- (1,1) -- (1,0);
  \draw[rotate=30,xshift=2cm,shift only,red] (0,0) -- (1,1) -- (1,0);
\end{tikzpicture}
\end{codeexample}
\end{key}

\begin{key}{/tikz/xshift=\meta{dimension}}
  Adds \meta{dimension} to the $x$ value of all coordinates.  
\begin{codeexample}[]
\begin{tikzpicture}
  \draw[help lines] (0,0) grid (3,2);
  \draw                   (0,0) -- (1,1) -- (1,0);
  \draw[xshift=2cm,blue]  (0,0) -- (1,1) -- (1,0);
  \draw[xshift=-10pt,red] (0,0) -- (1,1) -- (1,0);
\end{tikzpicture}
\end{codeexample}
\end{key}

\begin{key}{/tikz/yshift=\meta{dimension}}
  Adds \meta{dimension} to the $y$ value of all coordinates.
\end{key}

\begin{key}{/tikz/scale=\meta{factor}}
  Multiplies all coordinates by the given \meta{factor}. The
  \meta{factor} should not be excessively large in absolute terms or
  very near to zero.
\begin{codeexample}[]
\begin{tikzpicture}
  \draw[help lines] (0,0) grid (3,2);
  \draw               (0,0) -- (1,1) -- (1,0);
  \draw[scale=2,blue] (0,0) -- (1,1) -- (1,0);
  \draw[scale=-1,red] (0,0) -- (1,1) -- (1,0);
\end{tikzpicture}
\end{codeexample}
\end{key}

\begin{key}{/tikz/scale around={\ttfamily\char`\{}\meta{factor}|:|\meta{coordinate}{\ttfamily\char`\}}}
  Scales the coordinate system by \meta{factor}, put with the ``origin
  of scaling'' centered on \meta{coordinate} rather than the origin. 
\begin{codeexample}[]
\begin{tikzpicture}
  \draw[help lines] (0,0) grid (3,2);
  \draw                             (0,0) -- (1,1) -- (1,0);
  \draw[scale=2,blue]               (0,0) -- (1,1) -- (1,0);
  \draw[scale around={2:(1,1)},red] (0,0) -- (1,1) -- (1,0);
\end{tikzpicture}
\end{codeexample}
\end{key}

\begin{key}{/tikz/xscale=\meta{factor}}
  Multiplies only the $x$-value of all coordinates by the given
  \meta{factor}. 
\begin{codeexample}[]
\begin{tikzpicture}
  \draw[help lines] (0,0) grid (3,2);
  \draw                (0,0) -- (1,1) -- (1,0);
  \draw[xscale=2,blue] (0,0) -- (1,1) -- (1,0);
  \draw[xscale=-1,red] (0,0) -- (1,1) -- (1,0);
\end{tikzpicture}
\end{codeexample}
\end{key}

\begin{key}{/tikz/yscale=\meta{factor}}
  Multiplies only the $y$-value of all coordinates by \meta{factor}.
\end{key}

\begin{key}{/tikz/xslant=\meta{factor}}
  Slants the coordinate horizontally by the given \meta{factor}:
\begin{codeexample}[]
\begin{tikzpicture}
  \draw[help lines] (0,0) grid (3,2);
  \draw                (0,0) -- (1,1) -- (1,0);
  \draw[xslant=2,blue] (0,0) -- (1,1) -- (1,0);
  \draw[xslant=-1,red] (0,0) -- (1,1) -- (1,0);
\end{tikzpicture}
\end{codeexample}
\end{key}


\begin{key}{/tikz/yslant=\meta{factor}}
  Slants the coordinate vertically by the given \meta{factor}:
\begin{codeexample}[]
\begin{tikzpicture}
  \draw[help lines] (0,0) grid (3,2);
  \draw                (0,0) -- (1,1) -- (1,0);
  \draw[yslant=2,blue] (0,0) -- (1,1) -- (1,0);
  \draw[yslant=-1,red] (0,0) -- (1,1) -- (1,0);
\end{tikzpicture}
\end{codeexample}
\end{key}


\begin{key}{/tikz/rotate=\meta{degree}}
  Rotates the coordinate system by \meta{degree}:
\begin{codeexample}[]
\begin{tikzpicture}
  \draw[help lines] (0,0) grid (3,2);
  \draw                 (0,0) -- (1,1) -- (1,0);
  \draw[rotate=40,blue] (0,0) -- (1,1) -- (1,0);
  \draw[rotate=-20,red] (0,0) -- (1,1) -- (1,0);
\end{tikzpicture}
\end{codeexample}
\end{key}

\begin{key}{/tikz/rotate around={\ttfamily\char`\{}\meta{degree}|:|\meta{coordinate}{\ttfamily\char`\}}}
  Rotates the coordinate system by \meta{degree} around the point
  \meta{coordinate}.
\begin{codeexample}[]
\begin{tikzpicture}
  \draw[help lines] (0,0) grid (3,2);
  \draw                                (0,0) -- (1,1) -- (1,0);
  \draw[rotate around={40:(1,1)},blue] (0,0) -- (1,1) -- (1,0);
  \draw[rotate around={-20:(1,1)},red] (0,0) -- (1,1) -- (1,0);
\end{tikzpicture}
\end{codeexample}
\end{key}


\begin{key}{/tikz/cm={\ttfamily\char`\{}\meta{$a$}|,|\meta{$b$}|,|\meta{$c$}|,|\meta{$d$}|,|\meta{coordinate}{\ttfamily\char`\}}}
  applies the following transformation to all coordinates: Let $(x,y)$
  be the coordinate to be transformed and let \meta{coordinate}
  specify the point $(t_x,t_y)$. Then the new coordinate is given by
  $\left(\begin{smallmatrix} a & b \\ c & d\end{smallmatrix}\right)
  \left(\begin{smallmatrix} x \\ y \end{smallmatrix}\right) +
  \left(\begin{smallmatrix} t_x \\ t_y
  \end{smallmatrix}\right)$. Usually, you do not use this option
  directly. 
\begin{codeexample}[]
\begin{tikzpicture}
  \draw[help lines] (0,0) grid (3,2);
  \draw                             (0,0) -- (1,1) -- (1,0);
  \draw[cm={1,1,0,1,(0,0)},blue]    (0,0) -- (1,1) -- (1,0);
  \draw[cm={0,1,1,0,(1cm,1cm)},red] (0,0) -- (1,1) -- (1,0);
\end{tikzpicture}
\end{codeexample}
\end{key}

\begin{key}{/tikz/reset cm}
  Completely resets the coordinate transformation matrix to the
  identity matrix. This will destroy not only the transformations
  applied in the current scope, but also all transformations inherited
  from surrounding scopes. Do not use this option, unless you really,
  really know what you are doing.
\end{key}




\subsection{Canvas Transformations}

A \emph{canvas transformation}, see
Section~\ref{section-design-transformations} for details, is best
thought of as a transformation in which the drawing canvas is
stretched or rotated. Imaging writing something on a balloon (the
canvas) and then blowing air into the balloon: Not only does the text
become larger, the thin lines also become larger. In particular, if
you scale the canvas by a factor of two, all lines are twice as
thick.

Canvas transformations should be used with great care. In most
circumstances you do \emph{not} want line widths to change in a
picture as this creates visual inconsistency.

Just as important, when
you use canvas transformations \emph{\pgfname\ looses track of
  positions of nodes and of picture sizes} since it does not take the
effect of canvas transformations into account when it computes
coordinates of nodes (you not, however, rely on this; it may change in
the future).

Finally, not that a canvas transformation always applies to a path as
a whole, it is not possible (as for coordinate transformations) to use
different transformations in different parts of a path.

In short, you should not use canvas transformations unless you really
know what you are doing.

\begin{key}{/tikz/transform canvas=\meta{options}}
  The \meta{options} should contain coordinate transformations options
  like |scale| or |xshift|. Multiple options can be given, their
  effects accumulate in the usual manner. The effect of these
  \meta{options} (immediately) changes the current canvas
  transformation matrix. The coordinate transformation matrix is not
  changed. Tracking of the picture size is (locally) switched off and
  the node coordinate will no longer be correct.
\begin{codeexample}[]
\begin{tikzpicture}
  \draw[help lines] (0,0) grid (3,2);
  \draw                                    (0,0) -- (1,1) -- (1,0);
  \draw[transform canvas={scale=2},blue]   (0,0) -- (1,1) -- (1,0);
  \draw[transform canvas={rotate=180},red] (0,0) -- (1,1) -- (1,0);
\end{tikzpicture}
\end{codeexample}
\end{key}




\part{Libraries}
\label{part-libraries}

{\Large \emph{by Till Tantau}}


\bigskip
\noindent
In this part the library packages are documented. They
provide additional predefined graphic objects like new arrow heads or
new plot marks, but also sometimes extensions of the basic \pgfname\
or \tikzname\ system. The libraries are not loaded by default since
many users will not need them.  

\medskip
\noindent
\begin{codeexample}[graphic=white]
\tikzset{
  ld/.style={level distance=#1},lw/.style={line width=#1},  
  level 1/.style={ld=4.5mm, trunk,          lw=1ex ,sibling angle=60},
  level 2/.style={ld=3.5mm, trunk!80!leaf a,lw=.8ex,sibling angle=56},
  level 3/.style={ld=2.75mm,trunk!60!leaf a,lw=.6ex,sibling angle=52},
  level 4/.style={ld=2mm,   trunk!40!leaf a,lw=.4ex,sibling angle=48},
  level 5/.style={ld=1mm,   trunk!20!leaf a,lw=.3ex,sibling angle=44},
  level 6/.style={ld=1.75mm,leaf a,         lw=.2ex,sibling angle=40},
}
\pgfarrowsdeclare{leaf}{leaf}
  {\pgfarrowsleftextend{-2pt} \pgfarrowsrightextend{1pt}}
{
  \pgfpathmoveto{\pgfpoint{-2pt}{0pt}}
  \pgfpatharc{150}{30}{1.8pt}
  \pgfpatharc{-30}{-150}{1.8pt}    
  \pgfusepathqfill
}

\newcommand{\logo}[5]
{
  \colorlet{border}{#1}
  \colorlet{trunk}{#2}
  \colorlet{leaf a}{#3}
  \colorlet{leaf b}{#4}
  \begin{tikzpicture}
    \scriptsize\scshape
    \draw[border,line width=1ex,yshift=.3cm,
          yscale=1.45,xscale=1.05,looseness=1.42]
      (1,0) to [out=90, in=0]    (0,1)  to [out=180,in=90]  (-1,0)
            to [out=-90,in=-180] (0,-1) to [out=0,  in=-90] (1,0) -- cycle;

    \coordinate (root) [grow cyclic,rotate=90]
    child {
      child [line cap=round] foreach \a in {0,1} {
        child foreach \b in {0,1} {
          child foreach \c in {0,1} {
            child foreach \d in {0,1} {
              child foreach \leafcolor in {leaf a,leaf b}
                { edge from parent [color=\leafcolor,-#5] }
        } } }
      } edge from parent [shorten >=-1pt,serif cm-,line cap=butt]
    };

    \node [text centered,text width=2cm,below] at (0pt,-.5ex)
    { \textcolor{border}{T}heoretical \\ \textcolor{border}{C}omputer \\
      \textcolor{border}{S}cience };
  \end{tikzpicture}
}
\begin{minipage}{3cm}
  \logo{green!80!black}{green!25!black}{green}{green!80}{leaf}\\
  \logo{green!50!black}{black}{green!80!black}{red!80!green}{leaf}\\
  \logo{red!75!black}{red!25!black}{red!75!black}{orange}{leaf}\\
  \logo{black!50}{black}{black!50}{black!25}{}
\end{minipage}
\end{codeexample}


���}	� �
����DEV��INO��SYN��SV~�pgfmanual-en-library-arrows.tex�J��J���J�j��P�_.j�

��!�}	� �
����DEV��INO��SYN��SV~�pgfmanual-en-library-automata.tex�J��J���J�j��P�_.j�
% Copyright 2006 by Till Tantau
%
% This file may be distributed and/or modified
%
% 1. under the LaTeX Project Public License and/or
% 2. under the GNU Free Documentation License.
%
% See the file doc/generic/pgf/licenses/LICENSE for more details.



\section{Background Library}

\label{section-tikz-backgrounds}

\begin{tikzlibrary}{backgrounds}
  This library defines ``backgrounds'' for pictures. This does not
  refer to background pictures, but rather to frames drawn around and
  behind pictures. For example, this package allows you to just add
  the |framed| option to a picture to get a rectangular box around
  your picture or |gridded| to put a grid behind your picture.
\end{tikzlibrary}

When this package is loaded, the following styles become available:
\begin{stylekey}{/tikz/show background rectangle}
  This style causes a rectangle to be drawn behind your graphic. This
  style option must be given to the |{tikzpicture}| environment or to
  the |\tikz| command.
\begin{codeexample}[]
\begin{tikzpicture}[show background rectangle]
  \draw (0,0) ellipse (10mm and 5mm);
\end{tikzpicture}
\end{codeexample}
  The size of the background rectangle is determined as follows:
  We start with the bounding box of the picture. Then, a certain
  separator distance is added on the sides. This distance can be
  different for the $x$- and $y$-directions and can be set using the
  following options:
  \begin{key}{/tikz/inner frame xsep=\meta{dimension} (initially 1ex)}
    Sets the additional horizontal separator distance for the
    background rectangle.    
  \end{key}
  \begin{key}{/tikz/inner frame ysep=\meta{dimension} (initially 1ex)}
    Same for the vertical separator distance.
  \end{key}  
  \begin{key}{/tikz/inner frame sep=\meta{dimension}}
    Sets the horizontal and vertical separator distances
    simultaneously. 
  \end{key}
  The following two styles make setting the inner separator a bit
  easier to remember:
  \begin{stylekey}{/tikz/tight background}
    Sets the inner frame separator to 0pt. The background rectangle
    will have the size of the bounding box. 
  \end{stylekey}
  \begin{stylekey}{/tikz/loose background}
    Sets the inner frame separator to 2ex.
  \end{stylekey}  

  You can influence how the background rectangle is rendered by setting
  the following style:
  \begin{stylekey}{/tikz/background rectangle (initially draw)}
    This style dictates how the background rectangle is drawn or
    filled. The default setting causes the path of the background
    rectangle to be drawn in the usual 
    way. Setting this style to, say, |fill=blue!20| causes a light
    blue background to be added to the picture. You can also use more
    fancy settings as shown in the following example:
\begin{codeexample}[]
\begin{tikzpicture}
  [background rectangle/.style=
     {double,ultra thick,draw=red,top color=blue,rounded corners},  
   show background rectangle]
  \draw (0,0) ellipse (10mm and 5mm);
\end{tikzpicture}
\end{codeexample}
    Naturally, no one in their right mind would use the above, but
    here is a nice background: 
\begin{codeexample}[]
\begin{tikzpicture}
  [background rectangle/.style=
     {draw=blue!50,fill=blue!20,rounded corners=1ex},
   show background rectangle]
  \draw (0,0) ellipse (10mm and 5mm);
\end{tikzpicture}
\end{codeexample}
  \end{stylekey}
\end{stylekey}

\begin{stylekey}{/tikz/framed}
  This is a shorthand for |show background rectangle|.
\end{stylekey}

\begin{stylekey}{/tikz/show background grid}
  This style behaves similarly to the |show background rectangle|
  style, but it will not use a rectangle path, but a grid. The lower
  left and upper right corner of the grid is computed in the same way
  as for the background rectangle:
\begin{codeexample}[]
\begin{tikzpicture}[show background grid]
  \draw (0,0) ellipse (10mm and 5mm);
\end{tikzpicture}
\end{codeexample}
  You can influence the background grid by setting
  the following style:
  \begin{stylekey}{/tikz/background grid (initially draw,help lines)}
    This style dictates how the background grid path is drawn. 
\begin{codeexample}[]
\begin{tikzpicture}
  [background grid/.style={thick,draw=red,step=.5cm},
   show background grid]
  \draw (0,0) ellipse (10mm and 5mm);
\end{tikzpicture}
\end{codeexample}
  \end{stylekey}
  This option can be combined with the |framed| option (use the
  |framed| option first):
\begin{codeexample}[]
\tikzset{background grid/.style={thick,draw=red,step=.5cm},
         background rectangle/.style={rounded corners,fill=yellow}}
\begin{tikzpicture}[framed,gridded]
  \draw (0,0) ellipse (10mm and 5mm);
\end{tikzpicture}
\end{codeexample}
\end{stylekey}

\begin{stylekey}{/tikz/gridded}
  This is a shorthand for |show background grid|.
\end{stylekey}

\begin{stylekey}{/tikz/show background top}
  This style causes a single line to be drawn at the top of the
  background rectangle. Normally, the line coincides exactly with the
  top line of the background rectangle:
\begin{codeexample}[]
\begin{tikzpicture}[
    background rectangle/.style={fill=yellow},
    framed,show background top]
  \draw (0,0) ellipse (10mm and 5mm);
\end{tikzpicture}
\end{codeexample}
  The following option allows you to lengthen (or shorten) the line:
  \begin{key}{/tikz/outer frame xsep=\meta{dimension} (initially 0pt)}
    The \meta{dimension} is added at the left and right side of the
    line. 
\begin{codeexample}[]
\begin{tikzpicture}
  [background rectangle/.style={fill=yellow},
   framed,
   show background top,
   outer frame xsep=1ex]
  \draw (0,0) ellipse (10mm and 5mm);
\end{tikzpicture}
\end{codeexample}
  \end{key}
  \begin{key}{/tikz/outer frame ysep=\meta{dimension} (initially 0pt)}
    This option does not apply to the top line, but to the left and
    right lines, see below.
  \end{key}
  \begin{key}{/tikz/outer frame sep=\meta{dimension}}
    Sets both the $x$- and $y$-separation.
  \end{key}
\begin{codeexample}[]
\begin{tikzpicture}
  [background rectangle={fill=blue!20},
   outer frame sep=1ex,%
   show background top,%
   show background bottom,%
   show background left,%
   show background right]
  \draw (0,0) ellipse (10mm and 5mm);
\end{tikzpicture}
\end{codeexample}
  You can influence how the line is drawn grid by setting
  the following style:
  \begin{stylekey}{/tikz/background top (initially draw)}
\begin{codeexample}[]
\tikzset{background rectangle/.style={fill=blue!20},
         background top/.style={draw=blue!50,line width=1ex}}
\begin{tikzpicture}[framed,show background top]
  \draw (0,0) ellipse (10mm and 5mm);
\end{tikzpicture}
\end{codeexample}
  \end{stylekey}
\end{stylekey}

\begin{stylekey}{/tikz/show background bottom}
  Works like the style for the top line.
\end{stylekey}
\begin{stylekey}{/tikz/show background left}
  Works similarly.
\end{stylekey}
\begin{stylekey}{/tikz/show background right}
  Works similarly.
\end{stylekey}



%%% Local Variables: 
%%% mode: latex
%%% TeX-master: "pgfmanual-pdftex-version"
%%% End: 


��!�}	� �
����DEV��INO��SYN��SV~�pgfmanual-en-library-calendar.tex�J��J���J�j��P�_.j�
% Copyright 2008 by Till Tantau
%
% This file may be distributed and/or modified
%
% 1. under the LaTeX Project Public License and/or
% 2. under the GNU Free Documentation License.
%
% See the file doc/generic/pgf/licenses/LICENSE for more details.


\section{Chains}

\label{section-chains}

\begin{tikzlibrary}{chains}
  This library defines options for creating chains.
\end{tikzlibrary}


\subsection{Overview}

\emph{Chains} are sequences of nodes that are -- typically -- arranged
in an o row or a column and that are -- typically -- connected by
edges. More generally, they can be used to position nodes of a
branching network in a systematic manner. For the positioning of nodes
in rows and columns you can also use matrices, see 
Section~\ref{section-matrices}, but chains can also be used 
to describe the connections between nodes that have already been
connected using, say, matrices. Thus, it often makes sense to use
matrices for the positioning of elements and chains to describe the
connections. 



\subsection{Starting and Continuing a Chain}

Typically, you construct one chain at a time, but it is
permissible to have construct multiple chains simultaneously. In this
case, the chains must be named differently and you must specify for
each node which chain it belongs to.

The first step toward creating a chain is to use the |start chain|
option.

\begin{key}{/tikz/start chain=\opt{\meta{chain name}}\opt{\meta{direction}}}
  This key should, but need not, be given as an option to a scope
  enclosing all nodes of the chain. Typically, this will be a |scope|
  or the whole |tikzpicture|, but it might just be a path on which all
  nodes of the chain are found.
  If no \meta{chain name} is given, the default value |chain| will be
  used instead.

  The key starts a chain named \meta{chain name} and makes it
  \emph{active}, which means that is currently being constructed. The
  |start chain| can be issued only once to activate a chain, inside a
  scope in which a chain is active you cannot use this option once
  more (for the same chain name). The chain stops being active at the
  end of the scope in which the |start chain| command was given.

  Although chains are only locally active (that is, active inside the
  scope the |start chain| command was issued), the information
  concerning the chains is stored globally and it is possible
  to \emph{continue} a chain after a scope has ended. For this, the
  |continue chain| option can be used, which allows you to reactivate
  an existing chain in another scope.

  The \meta{direction} is used to determine the placement rule for
  nodes on the chain. If it is omitted, the current value of the
  following key is used:
  \begin{key}{/tikz/chain default direction=\meta{direction}
      (initially going right)}
    This \meta{direction} is used in a |chain| option, if no other
    \meta{direction} is specified.
  \end{key}

  The \meta{direction} can have two different forms:
  \declare{|going |\meta{options}} or
  \declare{|placed |\meta{options}}. The effect of these rules will be
  explained in the description of the |on chain| option. Right now,
  just remember that the \meta{direction} you provide with the |chain|
  option applies to the whole chain.
  
  Other than this, this key has no further effect. In particular, to
  place nodes on the chain, you must use the |on chain| option,
  described next.
\begin{codeexample}[]
\begin{tikzpicture}[start chain]
  % The chain is called just "chain"
  \node [on chain] {A};
  \node [on chain] {B};
  \node [on chain] {C};
\end{tikzpicture}
\end{codeexample}

\begin{codeexample}[]
\begin{tikzpicture}
  % Same as above, using the scope shorthand
  { [start chain]
    \node [on chain] {A};
    \node [on chain] {B};
    \node [on chain] {C};
  }
\end{tikzpicture}
\end{codeexample}

\begin{codeexample}[]
\begin{tikzpicture}[start chain=1 going right,
                    start chain=2 going below,
                    node distance=5mm,
                    every node/.style=draw]
  \node [on chain=1] {A};
  \node [on chain=1] {B};
  \node [on chain=1] {C};

  \node [on chain=2] at (0.5,-.5) {0};
  \node [on chain=2] {1};
  \node [on chain=2] {2};
  
  \node [on chain=1] {D}; 
\end{tikzpicture}
\end{codeexample}
\end{key}

\begin{key}{/tikz/continue chain=\opt{\meta{chain name}}\opt{\meta{direction}}}
  This option allows you to (re)activate an existing chain and to
  possibly change the default direction. If the |chain name| is
  missing, the name of the innermost activated chain is used. If no
  chain is activated, |chain| is used.

  Let us have a look at the two different applications of this
  option. The first is to change the direction of a chain as it is
  begin constructed. For this, just give this option somewhere inside
  the scope of the chain.
\begin{codeexample}[]
\begin{tikzpicture}[start chain=going right,node distance=5mm]
  \node [draw,on chain] {Hello};
  \node [draw,on chain] {World};
  \node [draw,continue chain=going below,on chain] {,};
  \node [draw,on chain] {this};
  \node [draw,on chain] {is};
\end{tikzpicture}
\end{codeexample}

  The second application is to reactivate a chain after it ``has
  already been closed down.''

\begin{codeexample}[]
\begin{tikzpicture}[node distance=5mm,
                    every node/.style=draw]
  { [start chain=1]
    \node [on chain] {A};
    \node [on chain] {B};
    \node [on chain] {C};
  }

  { [start chain=2 going below]
    \node [on chain=2] at (0.5,-.5) {0};
    \node [on chain=2] {1};
    \node [on chain=2] {2};
  }

  { [continue chain=1]
    \node [on chain] {D};
  }
\end{tikzpicture}
\end{codeexample}
\end{key}


\subsection{Nodes on a Chain}

\begin{key}{/tikz/on chain=\opt{\meta{chain name}}\opt{\meta{direction}}}
  This key should be given as an option to a node. When the option is
  used, the \meta{chain name} must be the name of a chain that has
  been started using the |start chain| option. If \meta{chain name} is
  the empty string, the current value of the innermost activated chain
  is used. If this option is used several times for a node, only the
  last invocation ``wins.'' (To place a node on several chains, use
  the |\chainin| command repeatedly.)

  The \meta{direction} part is optional. If present it sets the
  direction used for this node, otherwise the \meta{direction}
  that was given to the original |start chain| option is used (or of
  the last |continue chain| option, which allows you to change this
  default). 

  The effects of this option are the following:
  \begin{enumerate}
  \item An internal counter (there is one, local, counter
    for each chain) is increased. This counter reflects the current
    number of the node in the chain, where the first node is node 1,
    the second is node 2, and so on.

    This value of this internal counter is globally stored in the
    macro \declare{|\tikzchaincount|}.
  \item If the node does not yet have a name, (having been given using
    the |name| option or the name-syntax), the name of the node is set to
    \meta{chain name}|-|\meta{value of the internal chain
      counter}. For instance, if the chain is called |nums|, the first
    node would be named |nums-1|, the second |nums-2|, and so on. For
    the default chain name |chain|, the first node is named |chain-1|,
    the second |chain-2|, and so on.
  \item Independently of whether the name has been provided
    automatically or via the |name| option, the name of the node is
    globally stored in the macro \declare{|\tikzchaincurrent|}.
  \item Except for the first node, the macro
    \declare{|\tikzchainprevious|} is now globally set to the name of
    the node of the previous node on the chain. For the first node of
    the chain, this macro is globally set to the empty string.
  \item Except possibly for the first node of the chain, the placement
    rule is now executed. The placement rule is just a \tikzname\ option
    that is applied automatically to each node on the chain. Depending
    on the form of the \meta{direction} parameter (either the locally
    given one or the one given to the |start chain| option), different
    things happen.

    First, it makes a difference whether the \meta{direction} starts
    with |going| or with |placed|. The difference is that in the first
    case, the placement rule is not applied to the first node of the
    chain, while in the second case the placement rule is applied also
    to this first node. The idea is that a |going|-direciton indicates
    that we are ``going somewhere relative to the previous node''
    whereas a |placed| indicates that we are ``placing nodes according
    to their number.''

    Independently of which form is used, the \meta{text} inside
    \meta{direction} that follows |going| or |placed| (separated by a
    compulsory space) can have two different effects:
    \begin{enumerate}
    \item If it contains an equal sign, then this \meta{text} is used
      as the placement rule, that is, it is simply executed.  
    \item If it does not contain an equal sign, then
      \meta{text}|=of \tikzchainprevious| is used as the placement
      rule. 
    \end{enumerate}

    Note that in the first case, inside the \meta{text} you have
    access to |\tikzchainprevious| and |\tikzchaincount| for doing
    your positioning calculations.
\begin{codeexample}[]
\begin{tikzpicture}[start chain=circle placed {at=(\tikzchaincount*30:1.5)}]
  \foreach \i in {1,...,10}
    \node [on chain] {\i};
  
  \draw (circle-1) -- (circle-10);
\end{tikzpicture}
\end{codeexample}
  \item
    The following style is executed:
    \begin{stylekey}{/tikz/every on chain}
      This key is executed for every node of a chain, including the
      first one. 
    \end{stylekey}
  \end{enumerate}

  Recall that the standard replacement rule has a form like
  |right=of (\tikzchainprevious)|. This means that each 
  new node is placed to the right of the previous one, spaced by the
  current value of |node distance|.
\begin{codeexample}[]
\begin{tikzpicture}[start chain,node distance=5mm]
  \node [draw,on chain] {};
  \node [draw,on chain] {Hallo};
  \node [draw,on chain] {Welt};
\end{tikzpicture}
\end{codeexample}

  The optional \meta{direction} allows us to temporarily change the
  direction in the middle of a chain:
\begin{codeexample}[]
\begin{tikzpicture}[start chain,node distance=5mm]
  \node [draw,on chain] {Hello};
  \node [draw,on chain] {World};
  \node [draw,on chain=going below] {,};
  \node [draw,on chain] {this};
  \node [draw,on chain] {is};
\end{tikzpicture}
\end{codeexample}

  You can also use more complicated computations in the \meta{direction}:
\begin{codeexample}[]
\begin{tikzpicture}[start chain=going {at=(\tikzchainprevious),shift=(30:1)}]
  \draw [help lines] (0,0) grid (3,2);
  \node [draw,on chain] {1};
  \node [draw,on chain] {Hello};
  \node [draw,on chain] {World};
  \node [draw,on chain] {.};
\end{tikzpicture}
\end{codeexample}
\end{key}

For each chain, two special ``pseudo nodes'' are created.

\begin{predefinednode}{\meta{chain name}-begin}
  This node is the same as the first node on the chain. It is only
  defined after a first node has been defined.
\end{predefinednode}

\begin{predefinednode}{\meta{chain name}-end}
  This node is the same as the (currently) last node on the chain. As
  the chain is extended, this node changes.
\end{predefinednode}

The |on chain| option can also be used, in conjunction with
|late options|, to add an already existing node to a chain. The
following command, which is only defined inside scopes where a
|start chain| option is present, simplifies this process.

\begin{command}{\chainin |(|\meta{existing name}|)| \opt{\oarg{options}}}
  This command makes it easy to add a node to chain that has already
  been constructed. This node may even be part of a another chain.

  When you say |\chainin (some node);|, the node |some node| must
  already exist. It will then be made part of the current chain. This
  does not mean that the node can be changed (it is already
  constructed, after all), but the |join| option can be used to join
  |some node| to the previous last node on the chain and subsequent
  nodes will be placed relative to |some node|.

  It is permissible to give the |on chain| option inside the
  \meta{options} in order to specify on which chain the node should be
  put. 
  
  This command is just a shortcut for
\begin{quote}
|\path (|\meta{existing name}|) [late options={on chain,every chain in,|\meta{options}|}]|
\end{quote}
  In particular, it is possible to continue to path after a |\chainin|
  command, though that does not seem very useful.
  
\begin{codeexample}[]
\begin{tikzpicture}[node distance=5mm,
                    every node/.style=draw,every join/.style=->]
  \draw [help lines] (0,0) grid (3,2);

  \node[red] (existing) at (0,2) {existing};

  { [start chain]
    \node [draw,on chain,join] {Hello};
    \node [draw,on chain,join] {World};
    \chainin (existing) [join];
    \node [draw,on chain,join] {this};
    \node [draw,on chain,join] {is};
  } 
\end{tikzpicture}
\end{codeexample}

  Here is an example where nodes are positioned using a matrix and
  then connected using a chain
{\catcode`\|=12  
\begin{codeexample}[]
\begin{tikzpicture}[every node/.style=draw]
  \matrix [matrix of nodes,column sep=5mm,row sep=5mm]
  {
    |(a)|           World & |(b) [circle]|             peace \\
    |(c)|           be    & |(d) [isosceles triangle]| would \\
    |(e) [ellipse]| great & |(f)|                      ! \\
  };
  
  { [start chain,every on chain/.style={join=by ->}]
    \chainin (a);
    \chainin (b);
    \chainin (d);
    \chainin (c);
    \chainin (e);
    \chainin (f);
  } 
\end{tikzpicture}
\end{codeexample}
}
\end{command}



\subsection{Joining Nodes on a Chain}

\begin{key}{/tikz/join=\opt{|with |\meta{with} }\opt{|by |\meta{options}}}
  When this key is given to any node on a chain (except possibly for
  the first node), an |edge| command is added after the node. The
  |with| part specifies which node should be used for the start point
  of the edge; if the |with| part is omitted, the |\tikzchainprevious|
  is used. This |edge| command gets the \meta{options} as parameter
  and the current node as its target. If there is no
  previous node and no |with| is given, no |edge| command gets
  executed.  
  \begin{stylekey}{/tikz/every join}
    This style is executed each time this command is used.
  \end{stylekey}

  Note that is makes sense to call this option several times for a
  node, in order to connect it to several nodes. This is especially
  useful for joining in branches, see the next section.
\begin{codeexample}[]
\begin{tikzpicture}[start chain,node distance=5mm,
                    every join/.style={->,red}]
  \node [draw,on chain,join] {};
  \node [draw,on chain,join] {Hallo};
  \node [draw,on chain,join] {Welt};
  \node [draw,on chain=going below,
         join,join=with chain-1 by {blue,<-}] {foo};
\end{tikzpicture}
\end{codeexample}
\end{key}


\subsection{Branches}

A \emph{branch} is a chain that (typically only temporarily) extends
an existing chain. The idea is the following: Suppose we are
constructing a chain and at some node |x| there is a fork. In this
case, one (or even more) branches starts at this fork. For each branch
a chain is created, but the first node on this chain should
be~|x|. For this, it is useful to use |\chainin| on the node |x| to
make it part of the different branch chains and to name the branch
chains in some way that reflects the name of the main chain.

The |start branch| option provides a shorthand for doing exactly what
was just described.

\begin{key}{/tikz/start branch=\meta{branch name}\opt{\meta{direction}}}
  This key is used in the same manner as the |start chain| command,
  however, the effect is slightly different:
  \begin{itemize}
  \item This option may only be used if some chain is already active
    and there is a (last) node on this chain. Let us call this node
    the \meta{fork node}.
  \item The chain is not just called \meta{branch name}, but
    \meta{current chain}|/|\meta{branch name}. For instance, if the
    \meta{fork node} is part of the chain called |trunk| and the
    \meta{branch name} is set to |left|, the complete chain name of
    the branch is |trunk/left|. The \meta{branch name} must be given,
    there is no default value.
  \item The \meta{fork node} is automatically ``chained into'' the
    branch chain as its first node. Thus, for the first node on the
    branch that you provide, the |join| option will cause it to be
    connected to the fork node.
  \end{itemize}
\begin{codeexample}[]
\begin{tikzpicture}[every on chain/.style=join,every join/.style=->,
                    node distance=2mm and 1cm]
  { [start chain=trunk]
    \node [on chain] {A};
    \node [on chain] {B};

    { [start branch=numbers going below]
      \node [on chain] {1};
      \node [on chain] {2};
      \node [on chain] {3};
    }
    { [start branch=greek going above]
      \node [on chain] {$\alpha$};
      \node [on chain] {$\beta$};
      \node [on chain] {$\gamma$};
    }
    
    \node [on chain,join=with trunk/numbers-end,join=with trunk/greek-end] {C};
    { [start branch=symbols going below]
      \node [on chain] {$\star$};
      \node [on chain] {$\circ$};
      \node [on chain] {$\int$};
    }
  }
\end{tikzpicture}
\end{codeexample}
\end{key}

\begin{key}{/tikz/continue branch=\meta{branch name}\opt{\meta{direction}}}
  This option works like the |continue chain| option, only 
  \meta{current chain}|/|\meta{branch name} is used as the chain name,
  rather than just \meta{branch name}.
\begin{codeexample}[]
\begin{tikzpicture}[every on chain/.style=join,every join/.style=->,
                    node distance=2mm and 1cm]
  { [start chain=trunk]
    \node [on chain] {A};
    \node [on chain] {B};
    { [start branch=numbers going below] } % just a declaration,
    { [start branch=greek   going above] } % we will come back later
    \node [on chain] {C};

    % Now come the branches...
    { [continue branch=numbers]
      \node [on chain] {1};
      \node [on chain] {2};
    }
    { [continue branch=greek]
      \node [on chain] {$\alpha$};
      \node [on chain] {$\beta$};
    }
  }
\end{tikzpicture}
\end{codeexample}
\end{key}


��$�}	� �
����DEV��INO��SYN��SV~�pgfmanual-en-library-decorations.tex�J��J���J�j��P�_.j�

���}	� �
����DEV��INO��SYN��SV~�pgfmanual-en-library-er.tex�J��J���J�j��P�_.j�
% Copyright 2006 by Till Tantau
%
% This file may be distributed and/or modified
%
% 1. under the LaTeX Project Public License and/or
% 2. under the GNU Free Documentation License.
%
% See the file doc/generic/pgf/licenses/LICENSE for more details.


\section{Fading Library}
\label{section-library-fadings}

\begin{pgflibrary}{fadings}
  The package defines a number of fadings, see
  Section~\ref{section-tikz-transparency} for an introduction.  The
  \tikzname\ version defines special \tikzname\ commands for creating
  fadings. These commands are explained in
  Section~\ref{section-tikz-transparency}.   
\end{pgflibrary}

\newcommand\fadingindex[1]{%
  \index{#1@\protect\texttt{#1} fading}%
  \index{Fadings!#1@\protect\texttt{#1}}%
  \texttt{#1}& 
  \begin{tikzpicture}[baseline=5mm-.5ex]
    \fill [black!20] (0,0) rectangle (1,1);
    \path [pattern=checkerboard,pattern color=black!30] (0,0) rectangle (1,1);

    \fill [path fading=#1,blue] (0,0) rectangle (1,1);
  \end{tikzpicture} \\[4.5mm]
}

\noindent
\begin{tabular}{ll}
  \emph{Fading name} & \emph{Example (solid blue faded on checkerboard)} \\[1mm]
  \fadingindex{west}  
  \fadingindex{east}  
  \fadingindex{north}  
  \fadingindex{south} 
  \fadingindex{circle with fuzzy edge 10 percent} 
  \fadingindex{circle with fuzzy edge 15 percent} 
  \fadingindex{circle with fuzzy edge 20 percent} 
  \fadingindex{fuzzy ring 15 percent} 
\end{tabular}


%%% Local Variables: 
%%% mode: latex
%%% TeX-master: "pgfmanual-pdftex-version"
%%% End: 


���}	� �
����DEV��INO��SYN��SV~�pgfmanual-en-library-fit.tex�J��J���J�j��"P�_.j�

��!�}	� �
����DEV��INO��SYN��SV~�pgfmanual-en-library-matrices.tex�J��J���J�j��&P�_.j�
% Copyright 2006 by Till Tantau
%
% This file may be distributed and/or modified
%
% 1. under the LaTeX Project Public License and/or
% 2. under the GNU Free Documentation License.
%
% See the file doc/generic/pgf/licenses/LICENSE for more details.

\section{Mindmap Drawing Library}

\begin{tikzlibrary}{mindmap}
  This packages provides styles for drawing mindmap diagrams.
\end{tikzlibrary}

\subsection{Overview}

This library is intended to make the creation of mindmaps or concept
maps easier. A \emph{mindmap} is a graphical representation of a
concept together 
with related concepts and annotations. Mindmaps are, essentially,
trees, possibly with a few extra edges added, but they are usually
drawn in a special way: The root concept is placed in the middle of
the page and is drawn as a huge circle, ellipse, or cloud. The related
concepts then ``leave'' this root concept via branch-like tendrils.

The mindmap library of \tikzname\ produces mindmaps that look a bit
different from the standard mindmaps: While the big root concept is
still a circle, related concepts are also depicted as (smaller)
circles. The related concepts are linked to the root concept via
organic-looking connections. The overall effect is visually rather
pleasing, but readers may not immediately think of a mindmap when they
see a picture created with this library.

Although it is not strictly necessary, you will usually create
mindmaps using \tikzname's tree mechanism and some of the styles and
macros of the package work best when used inside trees. However, it is
still possible and sometimes necessary to treat parts of a mindmap as
a graph with arbitrary edges and this is also possible.


\subsection{The Mindmap Style}

Every mindmap should be put in a scope or a picture where the
|mindmap| style is used. This style installs some internal settings.

\begin{stylekey}{/tikz/mindmap}
  Use this style with all pictures or at least scopes that contain a
  mindmap. It installs a whole bunch of settings that are useful for
  drawing mindmaps. 
\begin{codeexample}[]
\tikz[mindmap,concept color=red!50]
  \node [concept] {Root concept}
    child[grow=right] {node[concept] {Child concept}};
\end{codeexample}
  The sizes of concepts are predefined in such a way that a
  medium-size mindmap will fit on an A4 page (more or less).  
  \begin{stylekey}{/tikz/every mindmap}
    This style is included by the |mindmap| style. Change this style to
    add special settings to your mindmaps.
\begin{codeexample}[]
\tikz[large mindmap,concept color=red!50]
  \node [concept] {Root concept}
    child[grow=right] {node[concept] {Child concept}};
\end{codeexample}
  \end{stylekey}
\end{stylekey}

\begin{stylekey}{/tikz/large mindmap}
  This style includes the |mindmap| style, but additionally changes
  the default size of concepts and of distances so that a medium-sized
  mindmap will fit on an A3 page (A3 pages are twice as large as A4
  pages).
\end{stylekey}
\begin{stylekey}{/tikz/huge mindmap}
  This style causes conepts to be even bigger and it is best used with
  A2 paper and above.
\end{stylekey}


\subsection{Concepts Nodes}

The basic entities of mindmaps are called \emph{concepts} in
\tikzname. A concept is a node of style |concept| and it must be
circular for some of the connection macros to work.


\subsubsection{Isolated Concepts}

The following styles influence how isolated concepts are rendered:

\begin{stylekey}{/tikz/concept}
  This style should be used with all nodes that are concepts, although
  some styles like |extra concept| install this style automatically.

  Bascially, this style makes the concept node circular and installs a
  uniform color called |concept color|, see below. Additionally, the
  style |every concept| is called.
\begin{codeexample}[]
\tikz[mindmap,concept color=red!50] \node [concept] {Some concept};
\end{codeexample}

  \begin{stylekey}{/tikz/every concept}
    In order to change the appearance of concept nodes, you should
    change this style. Note, however, that the color of a concept should
    be uniform for some of the connection bar stuff to work, so you
    should not change the color or the draw/fill state of concepts using
    this option. It is mostly useful for changing the text color and
    font.
  \end{stylekey}

  \begin{key}{/tikz/concept color=\meta{color}}
    This option tells \tikzname\ which color should be used for filling
    and stroking concepts. The difference between this option and just
    setting |every concept| to the desired color is that this option
    allows \tikzname\ to keep track of the colors used for
    concepts. This is important when you \emph{change} the color between
    two connected concepts. In this case, \tikzname\ can automatically
    create a shading that provides a smooth transition between the old
    and the new concept color; we will come back to this in the next
    section.
  \end{key}
\end{stylekey}

\begin{stylekey}{/tikz/extra concept}
  This style is intended for concepts that are not part of the
  ``mindmap tree,'' but stand beside it. Typically, they will have a
  subdued color are be smaller. In order to have these concepts appear
  in a uniform way and in order to indicate in the code that these
  concepts are extra, you can use this style.
\begin{codeexample}[]
\begin{tikzpicture}[mindmap,concept color=blue!80]
  \node [concept]                 {Root concept};
  \node [extra concept] at (10,0) {extra concept};
\end{tikzpicture}
\end{codeexample}
  \begin{stylekey}{/tikz/every extra concept}
    Change this style to change the appearance of extra concepts.
  \end{stylekey}
\end{stylekey}


\subsubsection{Concepts in Trees}

As pointed out earlier, \tikzname\ assumes that your mindmap is build
using the |child| facilities of \tikzname. There are numerous options
that influence how concepts are rendered at the different levels of a
tree. 

\begin{stylekey}{/tikz/root concept}
  This style is used for the roots of mindmap trees. By adding
  something to this, you can change how the root of a mindmap will be
  rendered.
\begin{codeexample}[]
\tikz
  [root concept/.append style={concept color=blue!80,minimum size=3.5cm},
   mindmap]
  \node [concept] {Root concept};
\end{codeexample}

  Note that styles like |large mindmap| redefine these styles, so you
  should add something to this style only inside the picture.
\end{stylekey}

\begin{stylekey}{/tikz/level 1 concept}
  The |mindmap| style adds this style to the |level 1| style. This
  means that the first level children of a mindmap tree will use this
  style. 
\begin{codeexample}[]
\tikz
  [root concept/.append style={concept color=blue!80},
   level 1 concept/.append style={concept color=red!50},
   mindmap]
  \node [concept] {Root concept}
    child[grow=30] {node[concept] {child}}
    child[grow=0 ] {node[concept] {child}};
\end{codeexample}
\end{stylekey}
\begin{stylekey}{/tikz/level 2 concept}
  Works like |level 1 concept|, only for second level children.
\end{stylekey}
\begin{stylekey}{/tikz/level 3 concept}
  Works like |level 1 concept|.
\end{stylekey}
\begin{stylekey}{/tikz/level 4 concept}
  Works like |level 1 concept|. Note that there are not fifth and
  higher level styles, you need to modify |level 5| directly in such
  cases. 
\end{stylekey}  

\begin{key}{/tikz/concept color=\meta{color}}
  We saw already that this option is used to change the color of
  concepts. We now have a look at its effect when used on child nodes
  of a concept. Normally, this option simply changes the color of the
  children. However, when the option is given as an option to the
  |child| operation (and not to the |node| operation and also not as
  an option to all children via the |level 1| style), \tikzname\ will
  smoothly change the concept color from the parent's color to the
  color of the child concept. 

  Here is an example:
\begin{codeexample}[]
\tikz[mindmap,concept color=blue!80]
  \node [concept] {Root concept}
    child[concept color=red,grow=30] {node[concept] {Child concept}}
    child[concept color=orange,grow=0]  {node[concept] {Child concept}};
\end{codeexample}

  In order to have all children of a certain level have a different
  concept color, a tiny bit of magic is needed:
\begin{codeexample}[]
\tikz[mindmap,text=white,
      root concept/.style={concept color=blue},
      level 1 concept/.append style=
        {every child/.style={concept color=blue!50}}]
  \node [concept] {Root concept}
    child[grow=30] {node[concept] {child}}
    child[grow=0 ] {node[concept] {child}};
\end{codeexample}
\end{key}

\subsection{Connecting Concepts}

\subsubsection{Simple Connections}

The easiest way to connect two concepts is to draw a line between
them. In order to give such lines a consistent appearance, it is
recommendable to use the following style when drawing such lines:

\begin{stylekey}{/tikz/concept connection}
  This style can be used for lines between two concepts. Feel free to
  redefine this style.
\end{stylekey}

A problem arises when you need to connect concepts after the main
mindmap has been drawn. In this case you will want the connection
lines to lie \emph{behind} the main mindmap. However, you can draw the
lines only after the coordinates of the concepts have been
determined. In this case you should place the connecting lines on a
background layer as in the following example:

\begin{codeexample}[]
\begin{tikzpicture}
  [root concept/.append style={concept color=blue!20,minimum size=2cm},
   level 1 concept/.append style={sibling angle=45},
   mindmap]
  \node [concept] {Root concept}
    [clockwise from=45]
    child { node[concept] (c1) {child}}
    child { node[concept] (c2) {child}}
    child { node[concept] (c3) {child}};
  \begin{pgfonlayer}{background}
    \draw [concept connection]  (c1) edge (c2)
                                     edge (c3)
                                (c2) edge (c3);
  \end{pgfonlayer}
\end{tikzpicture}
\end{codeexample}


\subsubsection{The Circle Connection Bar Decoration}

Instead of a simple line between two concepts, you can also add a bar
between the two nodes that has slightly organic ends. These bars are
also used by default as the edges from parents in the mindmap tree.

For the drawing of the bars a special decoration is used, which is defined
in the mindmap library:

\begin{decoration}{circle connection bar}
  This decoration can be used to connect two circles. The start of the
  to-be-decorated path should lie on the border of the first circle,
  the end should lie on the border of the second circle. The following
  two decoration keys should be initialized with the sizes of the circles:
  \begin{itemize}
  \item |start radius|
  \item |end radius|
  \end{itemize}
  Furthermore, the following two decoration keys influence the decoration:
  \begin{itemize}
  \item |amplitude|
  \item |angle|
  \end{itemize}
  The decoration turns a straight line into a path that starts on the border of the
  first circle at the specified angle relative to the line connecting
  the centers of the circles. The path then changes into a rectangle
  whose thickness is given by the amplitude. Finally, the path ends
  with the same angles on the second circle. 

  Here is an example that should make this clearer:
\begin{codeexample}[]
\begin{tikzpicture}
    [decoration={start radius=1cm,end radius=.5cm,amplitude=2mm,angle=30}]
  \fill[blue!20] (0,0) circle   (1cm);
  \fill[red!20]  (2.5,0) circle (.5cm);

  \filldraw [draw=red,fill=black,
             decorate,decoration=circle connection bar] (1,0) -- (2,0);
\end{tikzpicture}
\end{codeexample}

  As can be seen, the decorated path consists of three parts and is not really
  useful for drawing. However, if you fill the decorated path only, and if you
  use the same color as for the circles, the result is better.
\begin{codeexample}[]
\begin{tikzpicture}
  [blue!50,decoration={start radius=1cm,
                       end radius=.5cm,amplitude=2mm,angle=30}]
  \fill (0,0) circle   (1cm);
  \fill  (2.5,0) circle (.5cm);

  \fill [decorate,decoration=circle connection bar] (1,0) -- (2,0);
\end{tikzpicture}
\end{codeexample}

  In the above example you may notice the small white line between the
  circles and the decorated path. This is due to rounding
  errors. Unfortunately, for larger distances, there errors can
  accumulate quite strongly, especially since \tikzname\ and \TeX\ are
  not very good at computing square roots. For this reason, it is a
  good idea to make the circles slightly larger to cover up such
  problems. When using nodes of shape |circle|, you can just add the
  |draw| option with a |line width| or one or two points (for very
  large distances you may need line width up to 4pt). 
\begin{codeexample}[]
\begin{tikzpicture}
  [blue!50,decoration={start radius=1cm,
                       end radius=.5cm,amplitude=2mm,angle=30}]
  \fill (0,0) circle   (1cm+1pt);
  \fill  (2.4,0) circle (.5cm+1pt);

  \fill [decorate,decoration=circle connection bar] (1,0) -- (1.9,0);
\end{tikzpicture}
\end{codeexample}

  Note the slightly strange |outer sep=0pt|. This is needed so that
  the decorated path lies on the border of the filled circle, not on the
  border of the stroked circle (which is slightly larger and this
  slightly larger size is exactly what we wish to use to cover up the
  rounding errors).
\end{decoration}



\subsubsection{The Circle Connection Bar To-Path}

The |circle connection bar| decoration is a bit complicated to
use. Especially specifying the radii is quite bothersome (the
amplitude and the angle can be set once and for all). For this reason,
the mindmap library defines a special to-path, that performs the
necessary computations for you.

\begin{stylekey}{/tikz/circle connection bar}
  This style installs a rather involved to-path. Unlike normal
  to-paths, this path requires that the start and the target of the
  to-path are named nodes of shape |circle| -- if this is not the case,
  this path will produce errors.

  Assuming that the start and the target are circles, the to-path will
  first compute the radii of these circles (by measuring the distance
  from the |center| anchor to some anchor on the border) and will set
  the |start circle| keys accordingly. Next, the |fill| option
  is set to the |concept color| while |draw=none| is set. The decoration is
  set to |circle conncetion bar|. Finally, the following style is
  included:
  \begin{stylekey}{/tikz/every circle connection bar}
    Redefine this sytle to change the appearance of circle connection
    bar to-paths.      
  \end{stylekey}
\begin{codeexample}[]
\begin{tikzpicture}[concept color=blue!50,blue!50,outer sep=0pt]
  \node (n1) at (0,0)   [circle,minimum size=2cm,fill,draw,thick] {};
  \node (n2) at (2.5,0) [circle,minimum size=1cm,fill,draw,thick] {};

  \path (n1) to[circle connection bar] (n2);
\end{tikzpicture}
\end{codeexample}
  Note that it is not a good idea to have more than one |to| operation
  together this the option |circle connection bar| in a single
  |\path|. Use the |edge| operation, instead, for creating multiple
  connections and this operation creates a new scope for each edge.
\end{stylekey}

In a mindmap we sometimes want colors to change from one concept color
to another. Then, the connection bar should, ideally, consist of a
smooth transition between these two colors. Getting this right using
shadings is a bit tricky if you try this ``by hand,'' so the  mindmap
library provides a special option for facilitating this procedure.

\begin{key}{/tikz/circle connection bar switch color=|from (|\meta{first
    color}|) to (|\meta{second color}|)|}
  This style works similarly to the |circle connection bar|. The only
  difference is that instead of filling the path with a single color a
  shading is used.
\begin{codeexample}[]
\begin{tikzpicture}[outer sep=0pt]
  \node (n1) at (0,0)    [circle,minimum size=2cm,fill,draw,thick,red] {};
  \node (n2) at (30:2.5) [circle,minimum size=1cm,fill,draw,thick,blue] {};

  \path (n1) to[circle connection bar switch color=from (red) to (blue)] (n2);
\end{tikzpicture}  
\end{codeexample}
\end{key}


\subsubsection{Tree Edges}

Most of the time, concepts in a mindmap are connected automatically
when the mindmap is build as a tree. The reason is that the |mindmap|
installs a |circle connection bar| path as the edge from parent
path. Also, the |mindmap| option takes care of things like setting the
correct |draw| and |outer sep| settings and some other stuff.

In detail, the |mindmap| option sets the |edge from parent path| to a
path that uses the to-path |circle connection bar| to connect the parent node
and the child node. The |concept color| option (locally) changes this
by using |circle connection bar switch color| instead with the
from-color set to the old (parent's) concept color and the to-color
set to the new (child's) concept color. This menas that when you
provide the |concept color| option to a |child| command, the color
will change from the parent's concept color to the specified color.

Here is an example of a tree build in this way:

\begin{codeexample}[]
\begin{tikzpicture}
  \path[mindmap,concept color=black,text=white]
    node[concept] {Computer Science}
    [clockwise from=0]
    child[concept color=green!50!black] {
      node[concept] {practical}
      [clockwise from=90]
      child { node[concept] {algorithms} }
      child { node[concept] {data structures} }
      child { node[concept] {pro\-gramming languages} }
      child { node[concept] {software engineer\-ing} }
    }  
    child[concept color=blue] {
      node[concept] {applied}
      [clockwise from=-30]
      child { node[concept] {databases} }
      child { node[concept] {WWW} }
    }
    child[concept color=red] { node[concept] {technical} }
    child[concept color=orange] { node[concept] {theoretical} };
\end{tikzpicture}
\end{codeexample}



\subsection{Adding Annotations}

An \emph{annotation} is some text outside a mindmap that, unlike an
extra concept, simply explains something in the mindmap. The following
style is mainly intended to help readers of the code see that a node
in an annotation node.

\begin{stylekey}{/tikz/annotation}
  This style indicates that a node is an annotation node. It includes
  the style |every annotation|, which allows you to change this style
  in a convenient fashion.
\begin{codeexample}[]

\begin{tikzpicture}
  [mindmap,concept color=blue!80,
  every annotation/.style={fill=red!20}]
  \node [concept] (root)  {Root concept};

  \node [annotation,right] at (root.east)
  {The root concept is, in general, the most important concept.};
\end{tikzpicture}
\end{codeexample}
  \begin{stylekey}{/tikz/every annotation}
    This style is included by |annotation|.
  \end{stylekey}
\end{stylekey}



%%% Local Variables: 
%%% mode: latex
%%% TeX-master: "pgfmanual-pdftex-version"
%%% End: 


�� �}	� �
����DEV��INO��SYN��SV~�pgfmanual-en-library-folding.tex�J��J���J�j��$P�_.j�
% Copyright 2006 by Till Tantau
%
% This file may be distributed and/or modified
%
% 1. under the LaTeX Project Public License and/or
% 2. under the GNU Free Documentation License.
%
% See the file doc/generic/pgf/licenses/LICENSE for more details.


\section{Pattern Library}
\label{section-library-patterns}

\begin{pgflibrary}{patterns}
  The package defines patterns for filling areas.
\end{pgflibrary}


\newcommand\patternindex[1]{%
  \index{#1@\protect\texttt{#1} pattern}%
  \index{Patterns!#1@\protect\texttt{#1}}%
  \texttt{#1}& 
  \begin{tikzpicture}[baseline=.5ex]

    % Background
    \pattern [path fading=west,pattern=checkerboard light gray]
      (0,0) rectangle (5cm,2em);
    
    \pattern [pattern=#1,pattern color=black] (0,0) rectangle +(1.5cm,2em);
    \pattern [pattern=#1,pattern color=blue] (1.75,0) rectangle +(1.5cm,2em);
    \pattern [pattern=#1,pattern color=red] (3.5,0) rectangle +(1.5cm,2em);
  \end{tikzpicture} \\[1ex]
}

\newcommand\patternindexinherentlycolored[1]{%
  \index{#1@\protect\texttt{#1} pattern}%
  \index{Patterns!#1@\protect\texttt{#1}}%
  \texttt{#1}& 
  \begin{tikzpicture}[baseline=.5ex]

    % Background
    \pattern [path fading=west,pattern=checkerboard light gray]
      (0,0) rectangle (5cm,2em);
    
    \pattern [pattern=#1,pattern color=blue] (0,0) rectangle +(5cm,2em);
  \end{tikzpicture} \\[1ex]
}

\subsection{Form-Only Patterns}

\begin{tabular}{ll}
  \emph{Pattern name} & \emph{Example (pattern in black, blue, and red
    on faded checkerboard)} \\ 
  \patternindex{horizontal lines} 
  \patternindex{vertical lines} 
  \patternindex{north east lines} 
  \patternindex{north west lines} 
  \patternindex{grid} 
  \patternindex{crosshatch} 
  \patternindex{dots} 
  \patternindex{crosshatch dots} 
  \patternindex{fivepointed stars} 
  \patternindex{sixpointed stars} 
  \patternindex{bricks}
  \patternindex{checkerboard}
\end{tabular}
  
\subsection{Inherently Colored Patterns}


\begin{tabular}{ll}
  \emph{Pattern name} & \emph{Example} \\
  \patternindexinherentlycolored{checkerboard light gray} 
  \patternindexinherentlycolored{horizontal lines light gray} 
  \patternindexinherentlycolored{horizontal lines gray} 
  \patternindexinherentlycolored{horizontal lines dark gray} 
  \patternindexinherentlycolored{horizontal lines light blue} 
  \patternindexinherentlycolored{horizontal lines dark blue} 
  \patternindexinherentlycolored{crosshatch dots gray} 
  \patternindexinherentlycolored{crosshatch dots light steel blue} 
\end{tabular}
  

%%% Local Variables: 
%%% mode: latex
%%% TeX-master: "pgfmanual-pdftex-version"
%%% End: 

% Copyright 2006 by Till Tantau
%
% This file may be distributed and/or modified
%
% 1. under the LaTeX Project Public License and/or
% 2. under the GNU Free Documentation License.
%
% See the file doc/generic/pgf/licenses/LICENSE for more details.


\section{Petri-Net Drawing Library}

\begin{tikzlibrary}{petri}
  This packages provides shapes and styles for drawing Petri nets. 
\end{tikzlibrary}



\subsection{Places}

The package defines a style for drawing places of Petri nets. 

\begin{stylekey}{/tikz/place}
  This style indicates that a node is a place of a Petri net. Usually,
  the text of the node should be empty since places do not contain any
  text. You should use the |label| option to add text outside the node
  like its name or its capacity. You should use the |tokens| options,
  explained in Section~\ref{section-tokens}, to add tokens inside the
  place.
  
\begin{codeexample}[]
\begin{tikzpicture}
  \node[place,label=above:$p_1$,tokens=2]        (p1) {};
  \node[place,label=below:$p_2\ge1$,right=of p1] (p2) {};
\end{tikzpicture}
\end{codeexample}
  
  \begin{stylekey}{/tikz/every place}
    This stype is envoked by the style |place|. To change the
    appearance of places, you can change this style.
\begin{codeexample}[]
\begin{tikzpicture}
  [every place/.style={draw=blue,fill=blue!20,thick,minimum size=9mm}]
  \node[place,tokens=7,label=above:$p_1$]  (p1) {};
  \node[place,structured tokens={3,2,9},
        label=below:$p_2\ge1$,right=of p1] (p2) {};
\end{tikzpicture}
\end{codeexample}
  \end{stylekey}
\end{stylekey}




\subsection{Transitions}

Transitions are also nodes. They should be drawn using the following
style: 

\begin{stylekey}{/tikz/transition}
  This style indicates that a node is a transition. As for places, the
  text of a transition should be empty and the |label| option should
  be used for adding labels.

  To connect a transition to places, you can use the |edge| command as
  in the following example:
  
\begin{codeexample}[]
\begin{tikzpicture}
  \node[place,tokens=2,label=above:$p_1$]        (p1) {};
  \node[place,label=above:$p_2\ge1$,right=of p1] (p2) {};

  \node[transition,below right=of p1,label=below:$t_1$] {}
    edge[pre]                 (p1)
    edge[post] node[auto] {2} (p2);
\end{tikzpicture}
\end{codeexample}
  
  \begin{stylekey}{/tikz/every transition}
    This style is envoked by the style |transition|.
  \end{stylekey}

  \begin{stylekey}{/tikz/pre}
    This style can be used with paths leading \emph{from} a transition
    \emph{to} a place to indicate that the place is in the pre-set of
    the transition. By default, this style is |<-,shorten <=1pt|, but
    feel free to redefine it.
  \end{stylekey}

  \begin{stylekey}{/tikz/post}
    This style is also used with paths leading \emph{from} a transition
    \emph{to} a place, but this time the place is in the post-set of
    the transition. Again, feel free to redefine it.
  \end{stylekey}

  \begin{stylekey}{/tikz/pre and post}
    This style is to be used to indicate that a place is both in the
    pre- and post-set of a transition.
  \end{stylekey}
\end{stylekey}


\subsection{Tokens}
\label{section-tokens}

Interestingly, the most complicated aspect of drawing Petri nets in
\tikzname\ turns out to be the placement of tokens.

Let us start with a single token. They are also nodes and there is a
simple style for typesetting them.

\begin{stylekey}{/tikz/token}
  This style indicates that a node is a token. By default, this causes
  the node to be a small black circle. Unlike places and transitions,
  it \emph{does} make sense to provide text for the token node. Such
  text will be typeset in a tiny font and in white on black
  (naturally, you can easily change this by setting the style
  |every token|).
    
\begin{codeexample}[]
\begin{tikzpicture}
  \node[place,label=above:$p_1$]             (p1) {};
  \node[token] at (p1) {};

  \node[place,label=above:$p_2$,right=of p1] (p2) {};
  \node[token] at (p2) {$y$};
\end{tikzpicture}
\end{codeexample}
  \begin{stylekey}{/tikz/every token}
    Change this style to chagne the appearance of tokens.
  \end{stylekey}
\end{stylekey}

In the above example, it is bothersome that we need an extra command
for the token node. Worse, when we have \emph{two} tokens on a node,
it is difficult to place both nodes inside the node without overlap.

The Petri net library offers a solution to this problem: The
|children are tokens| style.


\begin{stylekey}{/tikz/children are tokens}
  The idea behind this style is to use trees mechanism for placing
  tokens. Every token lying on a place is treated as a child of the
  node. Normally this would have the effect that the tokens are placed
  below the place and they would be connected to the place by an
  edge. The |children are tokens| style, however, redefines the growth
  function of trees such that it places the children next to each
  other inside (or, rather, on top) of the place node. Additionally,
  the edge from the parent node is not drawn.
\begin{codeexample}[]
\begin{tikzpicture}
  \node[place,label=above:$p_1$] {}
  [children are tokens]
  child {node [token] {1}}
  child {node [token] {2}}
  child {node [token] {3}};
\end{tikzpicture}
\end{codeexample}

  In detail, what happens is the following: Tree growth functions tell
  \tikzname\ where it should place the children of nodes. These
  functions get passed the number of children that a node has an the
  number of the child that should be placed. The special tree growth
  function for tokens has a special mapping for each possible number
  of children up to nine children. This mapping decides for each child
  where it should be placed on top of the place. For example, a single
  child is placed directly on top of the place. Two children are
  placed next to each other, separated by the |token distance|. Three
  children are placed in a triangle whose side lengths are
  |token distance|; and so on up to nine tokens. If you wish to place
  more than nice tokens on a place, you will have to write your own
  placement code.
\begin{codeexample}[]
\begin{tikzpicture}
  \node[place,label=above:$p_2$] {}
  [children are tokens]
  child {node [token] {1}}
  child {node [token,fill=red] {2}}
  child {node [token,fill=red] {2}}
  child {node [token] {1}};
\end{tikzpicture}
\end{codeexample}

  \begin{key}{/tikz/token distance=\meta{distance}}
  This specifies the distance between the centers of the tokens in the
  arrangements of the option |children are tokens|.
\begin{codeexample}[]
\begin{tikzpicture}
  \node[place,label=above:$p_3$] {}
  [children are tokens,token distance=1.1ex]
  child {node [token] {}}
  child {node [token,red] {}}
  child {node [token,blue] {}}
  child {node [token] {}};
\end{tikzpicture}
\end{codeexample}
  \end{key}
\end{stylekey}

The |children are tokens| options gives you a lot of flexibility, but
it is a bit cumbersome to use. For this reason there are some options
that help in standard situations. They all use |children are tokens|
internally, so any change to, say, the |every tokens| style will
affect how these options depict tokens.

\begin{key}{/tikz/tokens=\meta{number}}
  This option is given to a |place| node, not to a |token| node. The
  effect of this option is to add \meta{number} many child nodes to
  the place, each having the style |token|. Thus, the following two
  pieces of codes have the same effect:
\begin{codeexample}[]
\tikz
  \node[place] {}
  [children are tokens]
  child {node [token] {}}
  child {node [token] {}}
  child {node [token] {}};
\tikz
  \node[place,tokens=3] {};
\end{codeexample}
  It is legal to say |tokens=0|, no tokens are drawn in this
  case. This option does not handle ten or more tokens correctly. If
  you need this many tokens, you will have to program your own code.
\begin{codeexample}[]
\begin{tikzpicture}[every place/.style={minimum size=9mm}]

  \foreach \x/\y/\tokennumber in {0/2/1,1/2/2,2/2/3,
                                  0/1/4,1/1/5,2/1/6,
                                  0/0/7,1/0/8,2/0/9}
    \node [place,tokens=\tokennumber] at (\x,\y) {};
\end{tikzpicture}
\end{codeexample}
\end{key}

\begin{key}{/tikz/colored tokens=\meta{color list}}
  This option, which must also be given when a place node is being
  created, gets a list of colors as parameter. It will then add as
  many tokens to the place are in this list, each colored with the
  corresponding color in the list.
\begin{codeexample}[]
\tikz  \node[place,colored tokens={black,black,red,blue}] {};
\end{codeexample}
\end{key}
\begin{key}{/tikz/structured tokens=\meta{token texts}}
  This option, which must again be passed to a place, gets a list
  texts for tokens. For each text, another token will be added to the place.
\begin{codeexample}[]
\tikz  \node[place,structured tokens={$x$,$y$,$z$}] {};
\end{codeexample}
\begin{codeexample}[]
\begin{tikzpicture}[every place/.style={minimum size=9mm}]

  \foreach \x/\y/\tokennumber in {0/2/1,1/2/2,2/2/3,
                                  0/1/4,1/1/5,2/1/6,
                                  0/0/7,1/0/8,2/0/9}
    \node [place,structured tokens={1,...,\tokennumber}] at (\x,\y) {};
\end{tikzpicture}
\end{codeexample}
  If you use lots of structured tokens, consider redefining the
  |every token| style so that the tokens are larger.
\end{key}


\subsection{Examples}


\begin{codeexample}[]
\begin{tikzpicture}[yscale=-1.6,xscale=1.5,thick,
  every transition/.style={draw=red,fill=red!20,minimum size=3mm},
  every place/.style={draw=blue,fill=blue!20,minimum size=6mm}]

  \foreach \i in {1,...,6} {
    \node[place,label=left:$p_\i$] (p\i) at (0,\i) {}; 
    \node[place,label=right:$q_\i$] (q\i) at (8,\i) {};
  }
  \foreach \name/\var/\vala/\valb/\height/\x in
      {m1/m_1/f/t/2.25/3,m2/m_2/f/t/2.25/5,h/\mathit{hold}/1/2/4.5/4} {
    \node[place,label=above:{$\var = \vala$}] (\name\vala) at (\x,\height) {};
    \node[place,yshift=-8mm,label=below:{$\var = \valb$}] (\name\valb) at (\x,\height) {};
  }
  \node[token] at (p1) {};   \node[token] at (q1) {};
  \node[token] at (m1f) {};  \node[token] at (m2f) {};
  \node[token] at (h1) {};

  \node[transition] at (1.5,1.5) {}  edge [pre] (p1)  edge [post] (p2);
  \node[transition] at (1.5,2.5) {}  edge[pre] (p2)   edge[pre] (m1f)
                                     edge[post] (p3)  edge[post] (m1t);
  \node[transition] at (1.5,3.3) {}  edge [pre] (p3)  edge [post] (p4)
                                     edge [pre and post] (h1);
  \node[transition] at (1.5,3.7) {}  edge [pre] (p3)  edge [pre] (h2)
                                     edge [post] (p4) edge [post] (h1.west);
  \node[transition] at (1.5,4.3) {}  edge [pre] (p4)  edge [post] (p5)
                                     edge [pre and post] (m2f);
  \node[transition] at (1.5,4.7) {}  edge [pre] (p4)  edge [post] (p5)
                                     edge [pre and post] (h2);
  \node[transition] at (1.5,5.5) {}  edge [pre] (p5)  edge [pre] (m1t)
                                     edge [post] (p6) edge [post] (m1f);
  \node[transition] at (1.5,6.5) {}  edge [pre] (p6)  edge [post] (p1.south east);
  \node[transition] at (6.5,1.5) {}  edge [pre] (q1)  edge [post] (q2);
  \node[transition] at (6.5,2.5) {}  edge [pre] (q2)  edge [pre] (m2f)
                                     edge [post] (q3) edge [post] (m2t);
  \node[transition] at (6.5,3.3) {}  edge [pre] (q3)  edge [post] (q4)
                                     edge [pre and post] (h2);
  \node[transition] at (6.5,3.7) {}  edge [pre] (q3)  edge [pre] (h1)
                                     edge [post] (q4) edge [post] (h2.east);
  \node[transition] at (6.5,4.3) {}  edge [pre] (q4)  edge [post] (q5)
                                     edge [pre and post] (m1f);
  \node[transition] at (6.5,4.7) {}  edge [pre] (q4)  edge [post] (q5)
                                     edge [pre and post] (h1);
  \node[transition] at (6.5,5.5) {}  edge [pre] (q5)  edge [pre] (m2t)
                                     edge [post] (q6) edge [post] (m2f);
  \node[transition] at (6.5,6.5) {}  edge [pre] (q6)  edge [post] (q1.south west);
\end{tikzpicture}  
\end{codeexample}

Here is the same net once more, but with these styles changes:
\begin{codeexample}[code only]
\begin{tikzpicture}[yscale=-1.1,thin,>=stealth,
  every transition/.style={fill,minimum width=1mm,minimum height=3.5mm},
  every place/.style={draw,thick,minimum size=6mm}]
\end{codeexample}

\begin{tikzpicture}[yscale=-1.1,thin,>=stealth,
  every transition/.style={fill,minimum width=1mm,minimum height=3.5mm},
  every place/.style={draw,thick,minimum size=6mm}]

  \foreach \i in {1,...,6} {
    \node[place,label=left:$p_\i$] (p\i) at (0,\i) {}; 
    \node[place,label=right:$q_\i$] (q\i) at (8,\i) {};
  }
  \foreach \name/\var/\vala/\valb/\height/\x in
      {m1/m_1/f/t/2.25/3,m2/m_2/f/t/2.25/5,h/\mathit{hold}/1/2/4.5/4} {
    \node[place,label=above:{$\var = \vala$}] (\name\vala) at (\x,\height) {};
    \node[place,yshift=-8mm,label=below:{$\var = \valb$}] (\name\valb) at (\x,\height) {};
  }
  \node[token] at (p1) {};   \node[token] at (q1) {};
  \node[token] at (m1f) {};  \node[token] at (m2f) {};
  \node[token] at (h1) {};

  \node[transition] at (1.5,1.5) {}  edge [pre] (p1)  edge [post] (p2);
  \node[transition] at (1.5,2.5) {}  edge[pre] (p2)   edge[pre] (m1f)
                                     edge[post] (p3)  edge[post] (m1t);
  \node[transition] at (1.5,3.3) {}  edge [pre] (p3)  edge [post] (p4)
                                     edge [pre and post] (h1);
  \node[transition] at (1.5,3.7) {}  edge [pre] (p3)  edge [pre] (h2)
                                     edge [post] (p4) edge [post] (h1.west);
  \node[transition] at (1.5,4.3) {}  edge [pre] (p4)  edge [post] (p5)
                                     edge [pre and post] (m2f);
  \node[transition] at (1.5,4.7) {}  edge [pre] (p4)  edge [post] (p5)
                                     edge [pre and post] (h2);
  \node[transition] at (1.5,5.5) {}  edge [pre] (p5)  edge [pre] (m1t)
                                     edge [post] (p6) edge [post] (m1f);
  \node[transition] at (1.5,6.5) {}  edge [pre] (p6)  edge [post] (p1.south east);
  \node[transition] at (6.5,1.5) {}  edge [pre] (q1)  edge [post] (q2);
  \node[transition] at (6.5,2.5) {}  edge [pre] (q2)  edge [pre] (m2f)
                                     edge [post] (q3) edge [post] (m2t);
  \node[transition] at (6.5,3.3) {}  edge [pre] (q3)  edge [post] (q4)
                                     edge [pre and post] (h2);
  \node[transition] at (6.5,3.7) {}  edge [pre] (q3)  edge [pre] (h1)
                                     edge [post] (q4) edge [post] (h2.east);
  \node[transition] at (6.5,4.3) {}  edge [pre] (q4)  edge [post] (q5)
                                     edge [pre and post] (m1f);
  \node[transition] at (6.5,4.7) {}  edge [pre] (q4)  edge [post] (q5)
                                     edge [pre and post] (h1);
  \node[transition] at (6.5,5.5) {}  edge [pre] (q5)  edge [pre] (m2t)
                                     edge [post] (q6) edge [post] (m2f);
  \node[transition] at (6.5,6.5) {}  edge [pre] (q6)  edge [post] (q1.south west);
\end{tikzpicture}

%%% Local Variables: 
%%% mode: latex
%%% TeX-master: "pgfmanual-pdftex-version"
%%% End: 

% Copyright 2006 by Till Tantau
%
% This file may be distributed and/or modified
%
% 1. under the LaTeX Project Public License and/or
% 2. under the GNU Free Documentation License.
%
% See the file doc/generic/pgf/licenses/LICENSE for more details.


\section{Plot Handler Library}
\label{section-library-plothandlers}

\begin{pgflibrary}{plothandlers}
  This library packages defines additional plot handlers, see
  Section~\ref{section-plot-handlers} for an introduction to plot
  handlers. The additional handlers are described in the following.

  This library is loaded automatically by \tikzname.
\end{pgflibrary}


\subsection{Curve Plot Handlers}
  
\begin{command}{\pgfplothandlercurveto}
  This handler will issue a |\pgfpathcurveto| command for each point of
  the plot, \emph{except} possibly for the first. As for the line-to
  handler, what happens with the first point can be specified using
  |\pgfsetmovetofirstplotpoint| or |\pgfsetlinetofirstplotpoint|.

  Obviously, the |\pgfpathcurveto| command needs, in addition to the
  points on the path, some control points. These are generated
  automatically using a somewhat ``dumb'' algorithm: Suppose you have
  three points $x$, $y$, and $z$ on the curve such that $y$ is between
  $x$ and $z$:
\begin{codeexample}[]
\begin{tikzpicture}    
  \draw[gray] (0,0) node {x} (1,1) node {y} (2,.5) node {z};
  \pgfplothandlercurveto
  \pgfplotstreamstart
  \pgfplotstreampoint{\pgfpoint{0cm}{0cm}}
  \pgfplotstreampoint{\pgfpoint{1cm}{1cm}}
  \pgfplotstreampoint{\pgfpoint{2cm}{.5cm}}
  \pgfplotstreamend
  \pgfusepath{stroke}
\end{tikzpicture}
\end{codeexample}

  In order to determine the control points of the curve at the point
  $y$, the handler computes the vector $z-x$ and scales it by the
  tension factor (see below). Let us call the resulting vector
  $s$. Then $y+s$ and $y-s$ will be the control points around $y$. The
  first control point at the beginning of the curve will be the
  beginning itself, once more; likewise the last control point is the
  end itself.
\end{command}

\begin{command}{\pgfsetplottension\marg{value}}
  Sets the factor used by the curve plot handlers to determine the
  distance of the control points from the points they control. The
  higher the curvature of the curve points, the higher this value
  should be. A value of $1$ will cause four points at quarter
  positions of a circle to be connected using a circle. The default is
  $0.5$. 

\begin{codeexample}[]
\begin{tikzpicture}    
  \draw[gray] (0,0) node {x} (1,1) node {y} (2,.5) node {z};
  \pgfsetplottension{0.75}
  \pgfplothandlercurveto
  \pgfplotstreamstart
  \pgfplotstreampoint{\pgfpoint{0cm}{0cm}}
  \pgfplotstreampoint{\pgfpoint{1cm}{1cm}}
  \pgfplotstreampoint{\pgfpoint{2cm}{0.5cm}}
  \pgfplotstreamend
  \pgfusepath{stroke}
\end{tikzpicture}
\end{codeexample}
\end{command}


\begin{command}{\pgfplothandlerclosedcurve}
  This handler works like the curve-to plot handler, only it will
  add a new part to the current path that is a closed curve through
  the plot points.
\begin{codeexample}[]
\begin{tikzpicture}    
  \draw[gray] (0,0) node {x} (1,1) node {y} (2,.5) node {z};
  \pgfplothandlerclosedcurve
  \pgfplotstreamstart
  \pgfplotstreampoint{\pgfpoint{0cm}{0cm}}
  \pgfplotstreampoint{\pgfpoint{1cm}{1cm}}
  \pgfplotstreampoint{\pgfpoint{2cm}{0.5cm}}
  \pgfplotstreamend
  \pgfusepath{stroke}
\end{tikzpicture}
\end{codeexample}
\end{command}


\subsection{Comb Plot Handlers}

There are three ``comb'' plot handlers. There name stems from the fact
that the plots they produce look like ``combs'' (more or less).

\begin{command}{\pgfplothandlerxcomb}
  This handler converts each point in the plot stream into a line from
  the $y$-axis to the point's coordinate, resulting in a ``horizontal
  comb.''

  
\begin{codeexample}[]
\begin{tikzpicture}    
  \draw[gray] (0,0) node {x} (1,1) node {y} (2,.5) node {z};
  \pgfplothandlerxcomb
  \pgfplotstreamstart
  \pgfplotstreampoint{\pgfpoint{0cm}{0cm}}
  \pgfplotstreampoint{\pgfpoint{1cm}{1cm}}
  \pgfplotstreampoint{\pgfpoint{2cm}{0.5cm}}
  \pgfplotstreamend
  \pgfusepath{stroke}
\end{tikzpicture}
\end{codeexample}
\end{command}


\begin{command}{\pgfplothandlerycomb}
  This handler converts each point in the plot stream into a line from
  the $x$-axis to the point's coordinate, resulting in a ``vertical
  comb.''
  
  This handler is useful for creating ``bar diagrams.''
\begin{codeexample}[]
\begin{tikzpicture}    
  \draw[gray] (0,0) node {x} (1,1) node {y} (2,.5) node {z};
  \pgfplothandlerycomb
  \pgfplotstreamstart
  \pgfplotstreampoint{\pgfpoint{0cm}{0cm}}
  \pgfplotstreampoint{\pgfpoint{1cm}{1cm}}
  \pgfplotstreampoint{\pgfpoint{2cm}{0.5cm}}
  \pgfplotstreamend
  \pgfusepath{stroke}
\end{tikzpicture}
\end{codeexample}
\end{command}

\begin{command}{\pgfplothandlerpolarcomb}
  This handler converts each point in the plot stream into a line from
  the origin to the point's coordinate.
  
\begin{codeexample}[]
\begin{tikzpicture}    
  \draw[gray] (0,0) node {x} (1,1) node {y} (2,.5) node {z};
  \pgfplothandlerpolarcomb
  \pgfplotstreamstart
  \pgfplotstreampoint{\pgfpoint{0cm}{0cm}}
  \pgfplotstreampoint{\pgfpoint{1cm}{1cm}}
  \pgfplotstreampoint{\pgfpoint{2cm}{0.5cm}}
  \pgfplotstreamend
  \pgfusepath{stroke}
\end{tikzpicture}
\end{codeexample}
\end{command}

\subsection{Mark Plot Handler}

\label{section-plot-marks}

\begin{command}{\pgfplothandlermark\marg{mark code}}
  This command will execute the \meta{mark code} for some points of the
  plot, but each time the coordinate transformation matrix will be
  setup such that the origin is at the position of the point to be
  plotted. This way, if the \meta{mark code} draws a little circle
  around the origin, little circles will be drawn at some point of the
  plot.

  By default, a mark is drawn at all points of the plot. However, two
  parameters $r$ and $p$ influence this. First, only every $r$th mark
  is drawn. Second, the first mark drawn is the $p$th. These
  parameters can be influenced using the commands below.
  
\begin{codeexample}[]
\begin{tikzpicture}    
  \draw[gray] (0,0) node {x} (1,1) node {y} (2,.5) node {z};
  \pgfplothandlermark{\pgfpathcircle{\pgfpointorigin}{4pt}\pgfusepath{stroke}}
  \pgfplotstreamstart
  \pgfplotstreampoint{\pgfpoint{0cm}{0cm}}
  \pgfplotstreampoint{\pgfpoint{1cm}{1cm}}
  \pgfplotstreampoint{\pgfpoint{2cm}{0.5cm}}
  \pgfplotstreamend
  \pgfusepath{stroke}
\end{tikzpicture}
\end{codeexample}

  Typically, the \meta{code} will be |\pgfuseplotmark{|\meta{plot mark
      name}|}|, where \meta{plot mark name} is the name of a
  predefined plot mark.
\end{command}

\begin{command}{\pgfsetplotmarkrepeat\marg{repeat}}
  Sets the $r$ parameter to \meta{repeat}, that is, only every $r$th
  mark will be drawn.
\end{command}

\begin{command}{\pgfsetplotmarkphase\marg{phase}}
  Sets the $p$ parameter to \meta{phase}, that is, the first mark to
  be drawn is the $p$th, followed by the $(p+r)$th, then the
  $(p+2r)$th, and so on.
\end{command}

\begin{command}{\pgfplothandlermarklisted\marg{mark code}\marg{index list}}
  This command works similar to the previous one. However, marks will
  only be placed at those indices in the given \meta{index list}. The
  syntax for the list is the same as for the |\foreach| statement. For
  example, if you provide the list |1,3,...,25|, a mark will be placed
  only at every second point. Similarly, |1,2,4,8,16,32| yields marks
  only at those points that are powers of two.
  
\begin{codeexample}[]
\begin{tikzpicture}    
  \draw[gray] (0,0) node {x} (1,1) node {y} (2,.5) node {z};
  \pgfplothandlermarklisted
    {\pgfpathcircle{\pgfpointorigin}{4pt}\pgfusepath{stroke}}
    {1,3}
  \pgfplotstreamstart
  \pgfplotstreampoint{\pgfpoint{0cm}{0cm}}
  \pgfplotstreampoint{\pgfpoint{1cm}{1cm}}
  \pgfplotstreampoint{\pgfpoint{2cm}{0.5cm}}
  \pgfplotstreamend
  \pgfusepath{stroke}
\end{tikzpicture}
\end{codeexample}
\end{command}

\begin{command}{\pgfuseplotmark\marg{plot mark name}}
  Draws the given \meta{plot mark name} at the origin. The \meta{plot
    mark name} must previously have been declared using
  |\pgfdeclareplotmark|. 

\begin{codeexample}[]
\begin{tikzpicture}    
  \draw[gray] (0,0) node {x} (1,1) node {y} (2,.5) node {z};
  \pgfplothandlermark{\pgfuseplotmark{pentagon}}
  \pgfplotstreamstart
  \pgfplotstreampoint{\pgfpoint{0cm}{0cm}}
  \pgfplotstreampoint{\pgfpoint{1cm}{1cm}}
  \pgfplotstreampoint{\pgfpoint{2cm}{0.5cm}}
  \pgfplotstreamend
  \pgfusepath{stroke}
\end{tikzpicture}
\end{codeexample}
\end{command}

\begin{command}{\pgfdeclareplotmark\marg{plot mark name}\marg{code}}
  Declares a plot mark for later used with the |\pgfuseplotmark|
  command.

\begin{codeexample}[]
\pgfdeclareplotmark{my plot mark}
  {\pgfpathcircle{\pgfpoint{0cm}{1ex}}{1ex}\pgfusepathqstroke}  
\begin{tikzpicture}    
  \draw[gray] (0,0) node {x} (1,1) node {y} (2,.5) node {z};
  \pgfplothandlermark{\pgfuseplotmark{my plot mark}}
  \pgfplotstreamstart
  \pgfplotstreampoint{\pgfpoint{0cm}{0cm}}
  \pgfplotstreampoint{\pgfpoint{1cm}{1cm}}
  \pgfplotstreampoint{\pgfpoint{2cm}{0.5cm}}
  \pgfplotstreamend
  \pgfusepath{stroke}
\end{tikzpicture}
\end{codeexample}
\end{command}


\begin{command}{\pgfsetplotmarksize\marg{dimension}}
  This command sets the \TeX\ dimension |\pgfplotmarksize| to
  \meta{dimension}. This dimension is a ``recommendation'' for plot
  mark code at which size the plot mark should be drawn; plot mark
  code may choose to ignore this \meta{dimension} altogether. For
  circles, \meta{dimension} should  be the radius, for other shapes it
  should be about half the width/height.

  The predefined plot marks all take this dimension into account.

\begin{codeexample}[]
\begin{tikzpicture}    
  \draw[gray] (0,0) node {x} (1,1) node {y} (2,.5) node {z};
  \pgfsetplotmarksize{1ex}
  \pgfplothandlermark{\pgfuseplotmark{*}}
  \pgfplotstreamstart
  \pgfplotstreampoint{\pgfpoint{0cm}{0cm}}
  \pgfplotstreampoint{\pgfpoint{1cm}{1cm}}
  \pgfplotstreampoint{\pgfpoint{2cm}{0.5cm}}
  \pgfplotstreamend
  \pgfusepath{stroke}
\end{tikzpicture}
\end{codeexample}
\end{command}

\begin{textoken}{\pgfplotmarksize}
  A \TeX\ dimension that is a ``recommendation'' for the size of plot
  marks.
\end{textoken}

The following plot marks are predefined (the filling color has been
set to yellow):

\medskip
\begin{tabular}{lc}
  \plotmarkentry{*}
  \plotmarkentry{x}
  \plotmarkentry{+}
\end{tabular}


%%% Local Variables: 
%%% mode: latex
%%% TeX-master: "pgfmanual-pdftex-version"
%%% End: 

% Copyright 2006 by Till Tantau
%
% This file may be distributed and/or modified
%
% 1. under the LaTeX Project Public License and/or
% 2. under the GNU Free Documentation License.
%
% See the file doc/generic/pgf/licenses/LICENSE for more details.


\section{Plot Mark Library}

\begin{pgflibrary}{plotmarks}
  This library defines a number of plot marks.
\end{pgflibrary}

This library defines the following plot marks in
addition to |*|, |x|, and |+| (the filling color has been set to a
dark yellow):

{
\catcode`\|=12
\medskip
\begin{tabular}{lc}
  \plotmarkentry{-}
  \index{*vbar@\protect\texttt{\protect\myvbar} plot mark}%
  \index{Plot marks!*vbar@\protect\texttt{\protect\myvbar}}
  \texttt{\char`\\pgfuseplotmark\char`\{\declare{|}\char`\}} &
  \tikz\draw[color=black!25] plot[mark=|,mark options={fill=yellow,draw=black}]
  coordinates {(0,0) (.5,0.2) (1,0) (1.5,0.2)};\\
  \plotmarkentry{o}
  \plotmarkentry{asterisk}
  \plotmarkentry{star}
  \plotmarkentry{oplus}
  \plotmarkentry{oplus*}
  \plotmarkentry{otimes}
  \plotmarkentry{otimes*}
  \plotmarkentry{square}
  \plotmarkentry{square*}
  \plotmarkentry{triangle}
  \plotmarkentry{triangle*}
  \plotmarkentry{diamond}
  \plotmarkentry{diamond*}
  \plotmarkentry{pentagon}
  \plotmarkentry{pentagon*}
\end{tabular}
}


%%% Local Variables: 
%%% mode: latex
%%% TeX-master: "pgfmanual-pdftex-version"
%%% End: 


�� �}	� �
����DEV��INO��SYN��SV~�pgfmanual-en-library-shadows.tex�J��JĀ�J�j��2P�_.j�
% Copyright 2007 by Till Tantau and Mark Wibrow
%
% This file may be distributed and/or modified
%
% 1. under the LaTeX Project Public License and/or
% 2. under the GNU Free Documentation License.
%
% See the file doc/generic/pgf/licenses/LICENSE for more details.


\section{Shape Library}
\label{section-libs-shapes}


\subsection{Overview}

In addition to the standard shapes |rectangle|, |circle| and
|coordinate|, there exist a number of additional shapes defined in
different shape libraries. Most of these shapes have been 
contributed by Mark Wibrow. In the present section, these shapes are
described. Note that the library |shapes| is provided for
compatibility only. Please include sublibraries like
|shapes.geometric| or |shapes.misc| directly.

The appearance of shapes is influenced by numerous parameters like
|minimum height| or |inner xsep|. These general parameters are documented in
Section~\ref{section-shape-common-options} 


\subsection{Predefined Shapes}
\label{section-predefined-shapes}

The three shapes |rectangle|, |circle|, and |coordiante| are always
defined and no library needs to be loaded for them. While the
|coordinate| shape defines only the |center| anchor, the other two
shapes define a standard set of anchors.

\begin{shape}{circle}
  This shape draws a tightly fitting circle around the text. The
  following figure shows the anchors this shape defines; the anchors
  |10| and |130| are example of border anchors. 
\begin{codeexample}[]
\Huge
\begin{tikzpicture}
  \node[name=s,shape=circle,shape example] {Circle\vrule width 1pt height 2cm};
  \foreach \anchor/\placement in
    {north west/above left, north/above, north east/above right, 
     west/left, center/above, east/right, 
     mid west/right, mid/above, mid east/left, 
     base west/left, base/below, base east/right, 
     south west/below left, south/below, south east/below right, 
     text/left, 10/right, 130/above}
     \draw[shift=(s.\anchor)] plot[mark=x] coordinates{(0,0)}
       node[\placement] {\scriptsize\texttt{(s.\anchor)}};
\end{tikzpicture}
\end{codeexample}
\end{shape}

\begin{shape}{rectangle}
  This shape, which is the standard, is a rectangle around the
  text. The inner   and outer separations (see
  Section~\ref{section-shape-seps}) influence the white space around
  the text. The following figure shows the anchors this
  shape defines; the anchors |10| and |130| are example of border anchors.
\begin{codeexample}[]
\Huge
\begin{tikzpicture}
  \node[name=s,shape=rectangle,shape example] {Rectangle\vrule width 1pt height 2cm};
  \foreach \anchor/\placement in
    {north west/above left, north/above, north east/above right, 
     west/left, center/above, east/right, 
     mid west/right, mid/above, mid east/left, 
     base west/left, base/below, base east/right, 
     south west/below left, south/below, south east/below right, 
     text/left, 10/right, 130/above}
     \draw[shift=(s.\anchor)] plot[mark=x] coordinates{(0,0)}
       node[\placement] {\scriptsize\texttt{(s.\anchor)}};
\end{tikzpicture}
\end{codeexample}
\end{shape}



\subsection{Geometric Shapes}

\begin{pgflibrary}{shapes.geometric}
  This library defines different shapes that correspond to basic
  geometric objects like ellipses or polygons.
\end{pgflibrary}


\begin{shape}{diamond}
  This shape is a diamond tightly fitting the text box. The ratio
  between width and height is 1 by default, but can be changed by
  setting the shape aspect ratio using the following \pgfname{}
  key (to use this key in \tikzname{} simply remove the 
  \declare{|/pgf/|} path). 

  \begin{key}{/pgf/aspect=\meta{value} (initially 1.0)}
    The aspect is a recommendation for the quotient of the width and 
    the height of a shape. This key calls the macro
    |\pgfsetshapeaspect|.
  \end{key}
  
  The following figure shows the anchors this
  shape defines; the anchors |10| and |130| are example of border
  anchors.
  
\begin{codeexample}[]
\Huge
\begin{tikzpicture}
  \node[name=s,shape=diamond,shape example] {Diamond\vrule width 1pt height 2cm};
  \foreach \anchor/\placement in
    {north west/above left, north/above, north east/above right, 
     west/left, center/above, east/right, 
     mid/above, 
     base/below,  
     south west/below left, south/below, south east/below right, 
     text/left, 10/right, 130/above}
     \draw[shift=(s.\anchor)] plot[mark=x] coordinates{(0,0)}
       node[\placement] {\scriptsize\texttt{(s.\anchor)}};
\end{tikzpicture}
\end{codeexample}
\end{shape}

\begin{shape}{ellipse}
  This shape is an ellipse tightly fitting the text box, if no inner
  separation is given. The following figure shows the anchors this
  shape defines; the anchors |10| and |130| are example of border anchors.
\begin{codeexample}[]
\Huge
\begin{tikzpicture}
  \node[name=s,shape=ellipse,shape example] {Ellipse\vrule width 1pt height 2cm};
  \foreach \anchor/\placement in
    {north west/above left, north/above, north east/above right, 
     west/left, center/above, east/right, 
     mid west/right, mid/above, mid east/left, 
     base west/left, base/below, base east/right, 
     south west/below left, south/below, south east/below right, 
     text/left, 10/right, 130/above}
     \draw[shift=(s.\anchor)] plot[mark=x] coordinates{(0,0)}
       node[\placement] {\scriptsize\texttt{(s.\anchor)}};
\end{tikzpicture}
\end{codeexample}
\end{shape}





\begin{shape}{trapezium}
  This shape is a trapezium, that is, a quadrilateral with a single
  pair of parallel lines (this can sometimes be known as a trapezoid).
  The trapezium shape supports the rotation of the shape border, as 
  described in Section~\ref{section-rotating-shape-borders}. 
  
  The lower internal angles at the lower corners of the trapezium can 
  be specified independently, and the resulting extensions are in 
  addition to the natural dimensions of the node contents (which
  includes any |inner sep|.
	
\begin{codeexample}[]
\begin{tikzpicture}
   \tikzstyle{every node}=[trapezium, draw]
   \node at (0,2) {A};
   \node[trapezium left angle=75, trapezium right angle=45pt]
         at (0,1) {B};
   \node[trapezium left angle=120, trapezium right angle=60pt]
         at (0,0) {C};
\end{tikzpicture}
\end{codeexample}

         
  The \pgfname{} keys to set the lower internal angles of the trapezium 
  are shown below. 
  To use these keys in \tikzname, simply remove the \declare{|/pgf/|} path.
	
  \begin{key}{/pgf/trapezium left angle=\meta{angle} (initially 60)}
    Set the lower internal angle of the left side. 
  \end{key}
   
  \begin{key}{/pgf/trapezium right angle=\meta{angle} (initially 60)}
    Set the lower internal angle of the right side. 
  \end{key}
  
  \begin{stylekey}{/pgf/trapezium angle=\meta{angle}}
    This key stores no value itself, but sets the value of the
    previous two keys to \meta{angle}. 
  \end{stylekey}
     
  Regardless of the rotation of the shape border, the width
  and height of the trapezium are as follows:

\begin{codeexample}[]
\begin{tikzpicture}[>=stealth, every node/.style={text=black}, 
    shape border uses incircle, shape border rotate=60]
  \node [trapezium, fill=gray!25, minimum width=2cm] (t) {};
  \draw [red, <->] (t.bottom left corner) -- (t.bottom right corner) 
    node [midway, below right] {width};
  \draw [red, <->] (t.top side) -- (t.bottom side) 
    node [at start, above] {height};
\end{tikzpicture}
\end{codeexample}

  \begin{key}{/pgf/trapezium stretches=\meta{boolean} (default true)}
    This key controls whether \pgfname{} allows the width and the height
    of the trapezium to be enlarged independently, 
    when considering any minimum size specification. This is initially 
    |false|,  ensuring that the shape ``looks the same but bigger'' when 
    enlarged.
  
\begin{codeexample}[]
\tikzset{my node/.style={trapezium, fill=#1!20, draw=#1!75, text=black}}
\begin{tikzpicture}
  \draw [help lines] grid (3,2);
  \node [my node=red]                                      {A};
  \node [my node=green, minimum height=1.5cm] at (1, 1.25) {B};
  \node [my node=blue,  minimum width=1.5cm]  at (2, 0)    {C};
\end{tikzpicture}
\end{codeexample} 

    By setting \meta{boolean} to |true|, the trapezium can be stretched
    horizontally or vertically.
  
\begin{codeexample}[]
\tikzset{my node/.style={trapezium, fill=#1!20, draw=#1!75, text=black}}
\begin{tikzpicture}
\tikzset{trapezium stretches=true}
  \draw [help lines] grid (3,2);
  \node [my node=red]                                      {A};
  \node [my node=green, minimum height=1.5cm] at (1, 1.25) {B};
  \node [my node=blue,  minimum width=1.5cm]  at (2, 0)    {C};
\end{tikzpicture}
\end{codeexample}
\end{key}

  \begin{key}{/pgf/trapezium stretches body=\meta{boolean} (default true)}
    This is similar to the |trapezium stretches| key execept that
    when \meta{boolean} is |true|, \pgfname{} enlarges only the body 
    of the trapezium when applying minimum width.
  
\begin{codeexample}[]
\tikzset{my node/.style={trapezium, fill=#1!20, draw=#1!75, text=black}}
\begin{tikzpicture}
  \draw [help lines] grid (3,2);
  \node [my node=red]                      at (1.5,.25)  {A};
  \node [my node=green, minimum width=3cm, trapezium stretches] 
    at (1.5,1)    {B};
  \node [my node=blue,  minimum width=3cm, trapezium stretches body] 
    at (1.5,1.75) {C};
\end{tikzpicture}
\end{codeexample}
  \end{key}

  The anchors for the trapezium are shown below. The anchor |160| is an
  example of a border anchor.

\begin{codeexample}[]
\Huge
\begin{tikzpicture}
  \node[name=s, shape=trapezium, shape example, inner sep=1cm] 
    {Trapezium\vrule width 1pt height 2cm};
  \foreach \anchor/\placement in
    {bottom left corner/below, top right corner/right, 
     top left corner/left,     bottom right corner/below,
     bottom side/below,        left side/left, 
     right side/right,         top side/above,
     center/above,   text/below,      mid/right,       base/below, 
     mid west/right, base west/below, mid east/left,   base east/below, 
     west/above,     east/above,      north/below,     south/above,
     north west/above, north east/above, 
     south west/below, south east/below, 160/above}    
  \draw[shift=(s.\anchor)] plot[mark=x] coordinates{(0,0)}
    node[\placement] {\scriptsize\texttt{(s.\anchor)}};
\end{tikzpicture}
\end{codeexample}  
\end{shape}


\begin{shape}{semicircle}
  This shape is a semicircle, which tightly fits the node contents.
  This shape supports the rotation of the shape border, as described in 
  Section~\ref{section-rotating-shape-borders}.
  The anchors for the |semicircle| shape are shown below. 
  Anchor |30| is an example of a border anchor.
	
\begin{codeexample}[]
\Huge
\begin{tikzpicture}
  \node[name=s,shape=semicircle,shape border rotate=0,shape example, inner sep=1cm] 
  	{Semicircle\vrule width 1pt height 2cm};
  \foreach \anchor/\placement in
    {apex/above,      arc start/below, arc end/below,  chord center/below,
     center/above,    base/below,      mid/right,      text/left,
     base west/below, base east/below, mid west/left, mid east/right, 
     north/below,     south/above,     east/above,     west/above,
     north west/above left, north east/above right,
     south west/below,      south east/below, 30/right}
     \draw[shift=(s.\anchor)] plot[mark=x] coordinates{(0,0)}
       node[\placement] {\scriptsize\texttt{(s.\anchor)}};
\end{tikzpicture}
\end{codeexample}
\end{shape}





\begin{shape}{regular polygon}
  This shape is a regular polygon, which, by default, is drawn so that 
  a side (rather than a corner) is always at the bottom. 
  This shape supports the rotation as described in 
  Section~\ref{section-rotating-shape-borders}, but the border of the 
  polygon is \emph{always} constructed using the incircle, whose
  radius is calculated to tightly fit the node contents (including
  any |inner sep|).
  
\begin{codeexample}[]
\begin{tikzpicture}
  \foreach \a in {3,...,7}{
    \draw[red, dashed] (\a*2,0)  circle(0.5cm);
    \node[regular polygon, regular polygon sides=\a, draw,
     inner sep=0.3535cm] at (\a*2,0) {};
   }  
\end{tikzpicture}
\end{codeexample}	
	
  If the node is enlarged to any specified minimum size, 
  this is interpreted as the diameter of the the 
  circumcircle, that is, the circle that passes through all the 
  corners of the polygon border.

\begin{codeexample}[]
\begin{tikzpicture}
  \foreach \a in {3,...,7}{
    \draw[blue, dashed] (\a*2,0)  circle(0.5cm);
    \node[regular polygon, regular polygon sides=\a, minimum size=1cm, draw] at (\a*2,0) {};
   }  
\end{tikzpicture}
\end{codeexample}	

  There is a \pgfname{} key to set the number of sides for the regular
  polygon.
  To use this key in \tikzname, simply remove the \declare{|/pgf/|} path.
	
  \begin{key}{/pgf/regular polygon sides=\meta{integer} (initially 5)}
  \end{key}
  
  The anchors for a regular polygon shape are shown below.  
  The anchor |75| is an example of a border anchor.
  
\begin{codeexample}[]
\Huge
\begin{tikzpicture}
  \node[name=s, shape=regular polygon, shape example, inner sep=.5cm] 
    {Regular Polygon\vrule width 1pt height 2cm};
  \foreach \anchor/\placement in
    {corner 1/above, corner 2/above, corner 3/left, corner 4/right, corner 5/above, 
     side 1/above,   side 2/left,    side 3/below,  side 4/right,   side 5/above,  
     center/above, text/left,  mid/right,   base/below, 75/above,
     west/above,   east/above, north/below, south/above,
     north east/below, south east/above, north west/below, south west/above}
  \draw[shift=(s.\anchor)] plot[mark=x] coordinates{(0,0)}
    node[\placement] {\scriptsize\texttt{(s.\anchor)}};
\end{tikzpicture}
\end{codeexample}

\end{shape}

\begin{shape}{star}
  This shape is a star, which by default (minus any transformations) is
  drawn with the first point pointing upwards.  
  This shape supports the rotation as described in 
  Section~\ref{section-rotating-shape-borders}, but the border of the 
  star is \emph{always} constructed using the incircle.
  
  A star should be thought of as having an set of ``inner points'' and
  and ``outer points''. 
  The inner points of the border are based on the radius of the circle
  which tightly fits the node contents, and the outer points are based
  on the circumcircle, the circle that passes through every outer
  point.
  Any specified minimum size, width or height, is interpreted as the 
  diameter of the circumcircle.
 
\begin{codeexample}[]
\begin{tikzpicture}
   \draw [help lines]   (0,0) grid (2,2);
   \draw [blue, dashed]  (1,1) circle(1cm);
   \draw [red, dashed] (1,1) circle(.5cm);
   \node [star, star point height=.5cm, minimum size=2cm, draw] 
       at (1,1) {S};
\end{tikzpicture}
\end{codeexample} 
  
  The \pgfname{} keys to set the number of star points, and the height
  of the star points, are shown below. To use these keys in \tikzname,
  simply remove the \declare{|/pgf/|} path.
  
  \begin{key}{/pgf/star points=\meta{integer} (initially 5)}
    Sets the number of points for the star.
  \end{key}
  
  \begin{key}{/pgf/star point height=\meta{distance} (initially .5cm)}
    Sets the height of the star points. This is the distance between the
    inner point and outer point radii. If the star is enlarged to some
    specified minimum size, the inner radius is increased to maintain
    the point height.	
  \end{key}
  
  \begin{key}{/pgf/star point ratio=\meta{number} (initially 1.5)}
    Sets the ratio between the inner point and outer point radii.		
    If the star is enlarged to some specified minimum size, the
    inner radius is increased to maintain the ratio.	
  \end{key}

	The inner and outer points form the principle anchors for the star,
   as shown below (anchor |75| is an example of a border anchor).
  
  \begin{codeexample}[]
\Huge
\begin{tikzpicture}
  \node[name=s, shape=star, star points=5, star point ratio=1.65, shape example, inner sep=1.5cm] 
    {Star\vrule width 1pt height 2cm};
  \foreach \anchor/\placement in
     {inner point 1/above, inner point 2/above, inner point 3/below, inner point 4/right, 
      inner point 5/above, outer point 1/above, outer point 2/above, outer point 3/left,  
      outer point 4/right, outer point 5/above,
      center/above, text/left,  mid/right,   base/below, 75/above,
     	west/above,   east/above, north/below, south/above,
     	north east/below, south east/above, north west/below, south west/above}
  \draw[shift=(s.\anchor)] plot[mark=x] coordinates{(0,0)}
    node[\placement] {\scriptsize\texttt{(s.\anchor)}};
\end{tikzpicture}
\end{codeexample}
\end{shape}





\begin{shape}{isosceles triangle}
  This shape is an isosceles triangle, which supports the rotation of 
  the shape border, as described in 
  Section~\ref{section-rotating-shape-borders}. The angle of rotation
  determines the direction in which the apex of the triangle points
  (provided no other transformations are applied). However, regardless
  of the rotation of the shape border, the width and height are 
  always considered as follows:
	
\begin{codeexample}[]
\begin{tikzpicture}[>=stealth, every node/.style={text=black},
    shape border uses incircle, shape border rotate=-30]
  \node [isosceles triangle, fill=gray!25, minimum width=1.5cm] (t) {};
  \draw [red, <->] (t.left corner) -- (t.right corner)
    node [midway, above left] {width};
  \draw [red, <->] (t.apex) -- (t.lower side)
    node [midway, above right] {height};
\end{tikzpicture}
\end{codeexample}

	There are \pgfname{} keys to customise this shape. 
	To use these keys in \tikzname, simply remove the \declare{|/pgf/|} 
	path.
    
  \begin{key}{/pgf/isosceles triangle apex angle=\meta{angle} (initially 45)}
    Sets the angle of the apex of the isosceles triangle. 
  \end{key}

\begin{key}{/pgf/isosceles triangle stretches=\meta{boolean} (default true)}
  
	By default \meta{boolean} is |false|. This means, that when applying
	any minimum width or minimum height requirements, increasing the 
	height will increase the width (and	vice versa), in order to keep the
	apex angle the same.
   
\begin{codeexample}[]
\begin{tikzpicture}[paint/.style={draw=#1!75, fill=#1!20}]
  \tikzset{every node/.style={isosceles triangle, draw, inner sep=0pt, 
    anchor=left corner, shape border rotate=90}}
  \draw[help lines] grid(4,2);
  \foreach \a/\c in {1.5/blue, 1/green, 0.5/red}{
    \node[paint=\c, minimum height=\a cm] at (0,0) {};
    \node[paint=\c, minimum width=\a cm] at (2,0) {};
  }
\end{tikzpicture}
\end{codeexample}	

	However, by setting \meta{boolean} to |true|, minimum width and 
	height can be applied	independently.
	
\begin{codeexample}[]
\begin{tikzpicture}[paint/.style={draw=#1!75, fill=#1!20}]
  \tikzset{every node/.style={isosceles triangle, draw, inner sep=0pt, 
     anchor=south, shape border rotate=90, isosceles triangle stretches}}
  \draw[help lines] grid(4,2);
  \foreach \a/\c in {1.5/blue, 1/green, 0.5/red}{
    \node[paint=\c, minimum height=\a cm, minimum width=1.5cm] at (0.75,0) {};
    \node[paint=\c, minimum width=\a cm, minimum height=1.5cm] at (3,0)    {};
  }
\end{tikzpicture}
\end{codeexample}	
\end{key}
	
   The anchors for the |isosceles triangle| are shown below 
   (anchor |150| is an	example of a border anchor). Note that,
   somewhat confusingly, the anchor names such as |left side| and
   |right corner| are named as if the triangle is rotated to 
   90 degrees. Note also that the |center| anchor
   does not	necessarily correspond to any kind of geometric center.
	
\begin{codeexample}[]
\Huge
\begin{tikzpicture}
  \node[name=s, shape=isosceles triangle, shape example, inner xsep=1cm]
    {Isosceles Triangle\vrule width 1pt height 2cm};
  \foreach \anchor/\placement in
    {apex/above,      left corner/right, right corner/right,
     left side/above, right side/below,  lower side/right,    
     center/above,    text/right,        150/above,
     mid/right,       mid west/above,    mid east/right,
     base/below,      base west/below,   base east/below,
     west/above, east/below, north/below, south/above,
     north west/below, north east/below, 
     south west/above, south east/above}  
  \draw[shift=(s.\anchor)] plot[mark=x] coordinates{(0,0)}
    node[\placement] {\scriptsize\texttt{(s.\anchor)}};
\end{tikzpicture}
\end{codeexample} 
\end{shape}






\par\leavevmode
\begin{shape}{kite}

	This shape is a kite, which supports the rotation of the shape border, 
	as described in Section~\ref{section-rotating-shape-borders}. 
	There are \pgfname{} keys to specify the upper and lower vertex angles
	of the kite. 
	To use these keys in \tikzname, simply remove the \declare{|/pgf/|} 
	path.
	
	\begin{key}{/pgf/kite upper vertex angle=\meta{angle} (initially 120)}
	Set the upper internal angle of the kite.
	\end{key}
	
	\begin{key}{/pgf/kite lower vertex angle=\meta{angle} (initially 60)}
	Set the lower internal angle of the kite.
	\end{key}
	
	\begin{key}{/pgf/kite vertex angles=\meta{angle specification}}
		This key sets the keys for both the upper and lower vertex angles
		(it stores no value itself).
	   \meta{angle specification} can be pair of angles in the form
	   \meta{upper angle} |and| \meta{lower angle}, or a single angle.
	   In this latter case, both the upper and lower vertex angles will 
	   be the same.
	\end{key}%
    
\begin{codeexample}[]
\begin{tikzpicture}
  \tikzstyle{every node}=[kite, draw]
  \node[kite upper vertex angle=135, kite lower vertex angle=70] at (0,0) {A};
  \node[kite vertex angles=90 and 45] at (1,0) {B};
  \node[kite vertex angles=60]        at (2,0) {C};
\end{tikzpicture}
\end{codeexample}


	The anchors for the |kite| are shown below. Anchor |110| is an 
	example of a border anchor.
	
\begin{codeexample}[]
\Huge
\begin{tikzpicture}
  \node[name=s, shape=kite, shape example, inner sep=1.5cm] 
    {Kite\vrule width 1pt height 2cm};
  \foreach \anchor/\placement in
    {upper vertex/above, left vertex/above,    lower vertex/below, 
     right vertex/above, upper left side/above, upper right side/above,
     lower left side/below, lower right side/below,
     center/above,   text/left,       mid/right,        base/below, 
     mid west/left,  base west/below, mid east/right,   base east/below,
     west/above,     east/above,      north/below,     south/above,
     north west/left, north east/right, 
     south west/above, south east/above, 110/above}  
  \draw[shift=(s.\anchor)] plot[mark=x] coordinates{(0,0)}
    node[\placement] {\scriptsize\texttt{(s.\anchor)}};
\end{tikzpicture}
\end{codeexample}
\end{shape}


\begin{shape}{dart}


	This shape is a dart (which can also be known as an arrowhead or
	concave kite). This shape supports the rotation of the shape border, 
	as described in Section~\ref{section-rotating-shape-borders}. 
	The angle of the border rotation determines the direction in which 
	the dart points (unless other transformations have been applied).
	
	There are \pgfname{} keys to set the 
	angle for the `tip' of the dart and the angle between the `tails'
	of the dart. 
	To use these keys in \tikzname, simply remove the \declare{|/pgf/|} 
	path.

\begin{codeexample}[]
\begin{tikzpicture}
   \node[dart, draw, gray, shape border uses incircle, shape border rotate=45] 
       (d) {dart};
   \draw [<->] (d.tip)++(202.5:.5cm) arc(202.5:247.5:.5cm);
   \node [left=.5cm] at (d.tip) {tip angle};
   \draw [<->] (d.tail center)++(157.5:.5cm) arc(157.5:292.5:.5cm);
   \node [right] at (d.tail center) {tail angle};
\end{tikzpicture}
\end{codeexample}

	\begin{key}{/pgf/dart tip angle=\meta{angle} (initially 45)}
		Set the angle at the tip of the dart.
	\end{key}
	
	\begin{key}{/pgf/dart tail angle=\meta{angle} (initially 135)}
		Set the angle between the tails of the dart.
	\end{key}
		
	The anchors for the |dart| shape are shown below (note that the 
	shape is rotated 90 degrees anti-clockwise). Anchor |110| is an 
	example of a border anchor.
\begin{codeexample}[]
\Huge
\begin{tikzpicture}
  \node[name=s, shape=dart, shape border rotate=90, shape example, inner sep=1.25cm] 
    {Dart\vrule width 1pt height 2cm};
  \foreach \anchor/\placement in
    {tip/above,       tail center/below, right tail/below, 
     left tail/below, right tail/below,  left side/left,   right side/right,
     center/above,    text/left,         mid/right,        base/below, 
     mid west/left,   base west/below,   mid east/right,   base east/below,
     west/above,      east/above,        north/below,      south/above,
     north west/left, north east/right,  south west/above, south east/above,
     110/above}    
  \draw[shift=(s.\anchor)] plot[mark=x] coordinates{(0,0)}
    node[\placement] {\scriptsize\texttt{(s.\anchor)}};
\end{tikzpicture}
\end{codeexample}
\end{shape}




\begin{shape}{circular sector}

	This shape is a circular sector (which can also be known as a
	wedge).
	This shape supports the rotation of the shape border, 
	as described in Section~\ref{section-rotating-shape-borders}. 
	The angle of the border rotation determines the direction in which 
	the `apex' of the sector points (unless other transformations have 
	been applied).
	
\begin{codeexample}[]
\begin{tikzpicture}
	\tikzstyle{every node}=[circular sector, shape border uses incircle, draw];
   \node at (0,0) {A};
   \node [shape border rotate=30] at (1.5,0) {A};
\end{tikzpicture}
\end{codeexample}

	There is a \pgfname{} key to set the central angle of the sector, 
	which is expected to be less than 180 degrees. 
	To use this key in \tikzname,	simply remove the \declare{|/pgf/|} 
	path.
	
	\begin{key}{/pgf/circular sector angle=\meta{angle} (initially 60)}
		Set the central angle of the sector. 
	\end{key}
	
	The anchors for the circular sector shape are shown below.
	Anchor |30| is an example of a border anchor.
	
\begin{codeexample}[]
\Huge
\begin{tikzpicture}
  \node[name=s,shape=circular sector,  style=shape example, inner sep=1cm] 
  	{Circular Sector\vrule width 1pt height 2cm};
  \foreach \anchor/\placement in
   {sector center/above, arc start/below, arc end/below, arc center/below,
    center/above,        base/below,      mid/right,     text/below,
    north/below,         south/above,     east/below,    west/above,
    north west/above left, north east/above right,
    south west/below,      south east/below, 30/right}
     \draw[shift=(s.\anchor)] plot[mark=x] coordinates{(0,0)}
       node[\placement] {\scriptsize\texttt{(s.\anchor)}};
\end{tikzpicture}
\end{codeexample}
\end{shape}




\begin{shape}{cylinder}
	This shape is a 2-dimensional representation of a cylinder, which 
	supports the rotation of the shape border as described in
	Section~\ref{section-rotating-shape-borders}.

\begin{codeexample}[]
\begin{tikzpicture}
  \node[cylinder, draw, shape aspect=.5] {ABC};
\end{tikzpicture}
\end{codeexample}
		
	Regardless the rotation of the shape border, the height is always the
	distance between the curved ends, and the width is always the	
	distance between the straight sides. 

\begin{codeexample}[]
\begin{tikzpicture}[>=stealth]
  \node [cylinder, gray!50, rotate=30, draw, 
    minimum height=2cm, minimum width=1cm] (c) {Cylinder};
  \draw[red, <->] (c.top)   -- (c.bottom) 
    node [at end, below, black]   {height};
  \draw[red, <->] (c.north) -- (c.south) 
    node [at start, above, black] {width};
\end{tikzpicture}
\end{codeexample}

	Enlarging the shape to some minimum height will stretch only the body
	of the cylinder. By contrast, enlarging the shape to some minimum 
	width will stretch the curved ends.
	
\begin{codeexample}[]
\begin{tikzpicture}[>=stealth, shape aspect=.5]
  \tikzset{every node/.style={cylinder, shape border rotate=90, draw}}
  \node [minimum height=1.5cm]            {A};  
  \node [minimum width=1.5cm]  at (1.5,0) {B};  
\end{tikzpicture}
\end{codeexample}

  There are various keys to customize this shape (to use \pgfname{}
  keys in \tikzname{}, simply remove the \declare{|/pgf/|} path).
  
\begin{key}{/pgf/aspect=\meta{value} (initially 1.0)}
  The aspect is a recommendation for the quotient of the radii of
  the cylinder end. This may be ignored if the shape is enlarged
  to some minimum width.

\begin{codeexample}[]
\begin{tikzpicture}[>=stealth]
  \tikzset{every node/.style={cylinder, shape border rotate=90, draw}}
  \node [aspect=1.0]           {A};  
  \node [aspect=0.5]  at (1,0) {B};  
  \node [aspect=0.25] at (2,0) {C};  
\end{tikzpicture}
\end{codeexample}

\end{key}

\begin{key}{/pgf/cylinder uses custom fill=\meta{boolean} (default true)}
	This enables the use of a custom fill for the body and the end of 
	the cylinder. The background path for the shape should not be 
	filled (e.g., in \tikzname{}, the |fill| option for the node must 
	be implicity or explicitly set to |none|).
  Internally, this key sets the \TeX-if 
  |\ifpgfcylinderusescustomfill| appropriately.
\end{key}

\begin{codeexample}[]
\begin{tikzpicture}[>=stealth, aspect=0.5]
  \node [cylinder, cylinder uses custom fill, cylinder end fill=red!50,
         cylinder body fill=red!25] {Cylinder};  
\end{tikzpicture}
\end{codeexample}

\begin{key}{/pgf/cylinder end fill=\meta{color} (initially white)}
	Set the color for the end of the cylinder.
\end{key}
\begin{key}{/pgf/cylinder body fill=\meta{color} (initially white)}
	Set the color for the body of the cylinder.
\end{key}


  The anchors this shape are shown below (anchor |160| is an
	example of a border anchor). Note the the cylinder shape does not 
	distinguish between |outer xsep| and |outer ysep|. Only the larger 
	of the two values is used for the shape. Note also the difference 
	between the |center| and |shape center| anchors: |center| is the
	center of the cylinder body and also the center of rotation. 
	The |shape center| is the center of the shape which includes the 
	2-dimensional representation of the cylinder top.	
	 

\begin{codeexample}[]
\Huge
\begin{tikzpicture}
  \node[name=s, shape=cylinder, shape example, aspect=.5, inner xsep=3cm,
        inner ysep=1cm] {Cylinder\vrule width 1pt height 2cm};
  \foreach \anchor/\placement in
    {before top/above,    top/above,       after top/below,
     before bottom/below, bottom/above,    after bottom/above,
     mid/right,           mid west/right,  mid east/left,  
     base/below,          base west/below, base east/below,
     center/above,        text/above,      shape center/right, 
     west/right, east/left, north/above, south/below,
     north west/below, north east/above, 
     south west/above, south east/below, 160/above}    
  \draw[shift=(s.\anchor)] plot[mark=x] coordinates{(0,0)}
    node[\placement] {\scriptsize\texttt{(s.\anchor)}};
\end{tikzpicture}
\end{codeexample}  


\end{shape}






\subsection{Symbol Shapes}

\begin{pgflibrary}{shapes.symbols}
  This library defines shapes that can be used for drawing symbols
  like a forbidden sign or a cloud.
\end{pgflibrary}



\begin{shape}{forbidden sign}
  This shape places the node inside a circle with a diagonal from the
  lower left to the upper right added. The circle is part of the
  background, the diagonal line part of the foreground path; thus, the
  diagonal line is on top of the text.
  
\begin{codeexample}[]
\begin{tikzpicture}
  \node [forbidden sign,line width=1ex,draw=red,fill=white] {Smoking};
\end{tikzpicture}
\end{codeexample}

  The shape inherits all anchors from the |circle| shape, see also the
  following figure:
\begin{codeexample}[]
\Huge
\begin{tikzpicture}
  \node[name=s,shape=forbidden sign,shape example] {Forbidden\vrule width 1pt height 2cm};
  \foreach \anchor/\placement in
    {north west/above left, north/above, north east/above right, 
     west/left, center/above, east/right, 
     mid west/right, mid/above, mid east/left, 
     base west/left, base/below, base east/right, 
     south west/below left, south/below, south east/below right, 
     text/left, 10/right, 130/above}
     \draw[shift=(s.\anchor)] plot[mark=x] coordinates{(0,0)}
       node[\placement] {\scriptsize\texttt{(s.\anchor)}};
\end{tikzpicture}
\end{codeexample}
\end{shape}


\begin{shape}{cloud}

	This shape is a cloud, drawn to tightly fit the node contents 
	(strictly speaking, using an ellipse which tightly fits the node
	contents -- including any |inner sep|). 
	
\begin{codeexample}[]
\begin{tikzpicture}
  \node[cloud, draw, fill=gray!20, aspect=2] {ABC};
  \node[cloud, draw, fill=gray!20] at (1.5,0) {D};
\end{tikzpicture}
\end{codeexample}

	A cloud should be thought of as having a number of ``puffs'', which
	are the individual arcs drawn around the border. There are \pgfname{}
	keys to specify how the cloud is drawn (to use these keys in 
	\tikzname{}, simply remove the \declare{|/pgf/|} path).
	
	\begin{key}{/pgf/cloud puffs=\meta{integer} (initially 10)}
	  Set the number of puffs for the cloud.
	\end{key}
	
	\begin{key}{/pgf/cloud puff arc=\meta{angle} (initially 135)}
	  Set the length of the puff arc (in degrees). A shorter arc can 
	  produce better looking joins between puffs for larger line widths.
	\end{key}
	
	Like the diamond shape, the cloud shape also uses the 
	\declare{|aspect|} key, to determine the ratio of the width and the 
	height of the cloud. However there may be cirumstances where it may
	be undesirable to continually specify the |aspect| for the cloud.
	Therefore, the following key is implemented:
	
	\begin{key}{/pgf/cloud ignores aspect=\meta{boolean} (default true)}
		Instruct \pgfname{} to ignore the |aspect| key. Internally, the
		\TeX-if |\ifpgfcloudignoresaspect| is set appropriately. The initial
		value is |false|.

\begin{codeexample}[]
\begin{tikzpicture}[aspect=1, every node/.style={cloud, cloud puffs=11, draw}]
  \node [fill=gray!20]                                {rain};
  \node [cloud ignores aspect, fill=white] at (1.5,0) {snow};
\end{tikzpicture}
\end{codeexample}	
	
	\end{key}
	
	
	Any minimum size requirements are applied to the ``circum-ellipse'',
	which is the ellipse which passes through all the midpoints of the
	puff arcs. These requirements are considered \emph{after} any 
	aspect specification is applied.
	
\begin{codeexample}[]
\begin{tikzpicture}
  \draw [help lines] grid (3,2);
  \draw [blue, dashed] (1.5, 1) ellipse (1.5cm and 1cm);
  \node [cloud, cloud puffs=9, draw, minimum width=3cm, minimum height=2cm] 
    at (1.5, 1) {};
\end{tikzpicture}
\end{codeexample}
	
  The anchors for the cloud shape are shown below for a cloud with
  eleven puffs. Anchor 70 is an example of a border anchor. 
  
\begin{codeexample}[]
\Huge
\begin{tikzpicture}
  \node[name=s, shape=cloud, style=shape example, cloud puffs=11, aspect=1.5,
       cloud puff arc=120,inner ysep=1cm] {Cloud\vrule width 1pt height 2cm};
  \foreach \anchor/\placement in
   {puff 1/above, puff 2/above,  puff 3/above,  puff 4/below, 
    puff 5/left,  puff 6/below,  puff 7/below,  puff 8/right,
    puff 9/below, puff 10/above, puff 11/above, 70/right,
    center/above, base/below,    mid/right,     text/left, 
    north/below,  south/below,   east/above,    west/above,
    north west/left,             north east/right, 
    south west/below,            south east/below}
     \draw[shift=(s.\anchor)] plot[mark=x] coordinates{(0,0)}
       node[\placement] {\scriptsize\texttt{(s.\anchor)}};
\end{tikzpicture}
\end{codeexample}
\end{shape} 






\begin{shape}{starburst}

	This shape is a randomly generated eliptical star,
	which supports the rotating of the shape border as described in 
	Section~\ref{section-rotating-shape-borders}. 
\begin{codeexample}[]
\begin{tikzpicture}
  \node[starburst, fill=yellow, draw=red, line width=2pt] {\bf BANG!};
\end{tikzpicture}
\end{codeexample}	
	Like the |star| shape, the starburst should be thought of as having a set
	of inner points and outer points. The inner points lie on the ellipse
	which tightly fits the node contents (including any |inner sep|).
	
	Using a specified `starburst point height' value, the outer points
	are generated randomly between this value and one quarter of this 
	value. For a given starburst shape the angle between each point is 
	fixed, and is determined by the number of points specified for
	the starburst.
	
	It is important to note that, whilst the maximum possible point 
	height is used to calculate minimum width or height requirements, 
	the outer points are randomly generated, so there is (unfortunately) 
	no guarantee that any such requirements will be fully met. 
	
\begin{codeexample}[]
\begin{tikzpicture}
  \draw[help lines] grid(3,2);
  \node[starburst, draw, minimum width=3cm, minimum height=2cm] 
    at (1.5, 1) {\bf BOOM!};
\end{tikzpicture}
\end{codeexample}

	There are \pgfname{} keys to control the drawing of the starburst
	shape. To use these keys in \tikzname,	simply remove the 
	\declare{|/pgf/|}	path.

	\begin{key}{/pgf/starburst points=\meta{integer} (initially 17)}
		Set the number of points for the starburst.
	\end{key}
	\begin{key}{/pgf/starburst point height=\meta{length} (initially .5cm)}
      Set the \emph{maximum} distance between the inner point radius  
      and the outer point radius.
	\end{key}
	
	\begin{key}{/pgf/random starburst=\meta{integer} (initially 100)}
      Set the seed for the random number generator for creating the
      starburst.  The maximum value for \meta{integer} is |16383|.
      If \meta{integer}|=0|, the random number generator will not be 
      used, and the maximum point height will be used for all outer 
      points. If \meta{integer} is omitted, a seed will be randomly
      chosen.
	\end{key}
	
	The basic anchors for a nine point |starburst| shape are shown below. 
	Anchor |80| is an example of a border anchor.
\begin{codeexample}[]
\Huge
\begin{tikzpicture}
  \node[name=s, shape=starburst, starburst points=9, starburst point height=3.5cm, 
        style=shape example,inner sep=1cm] 
    {Starburst\vrule width 1pt height 2cm};
  \foreach \anchor/\placement in
    {outer point 1/above, outer point 2/above, outer point 3/right,
     outer point 4/above, outer point 5/below, outer point 6/above,
     outer point 7/left,  outer point 8/above, outer point 9/above,
     inner point 1/below, inner point 2/above, inner point 3/left,
     inner point 4/above, inner point 5/above, inner point 6/above,
     inner point 7/below, inner point 8/above, inner point 9/below,
     center/above, text/left,   mid/right, base/below, 80/above,
     north/below,  south/below, east/left, west/right,
     north east/below, south west/below, south east/below, north west/below}
  \draw[shift=(s.\anchor)] plot[mark=x] coordinates{(0,0)}
    node[\placement] {\scriptsize\texttt{(s.\anchor)}};
\end{tikzpicture}
\end{codeexample}
\end{shape}

\begin{shape}{signal}

	This shape is a ``signal'' or sign shape, that is, a rectangle, with
	optionally pointed sides. A signal can point ``to'' somewhere, with 
	outward points in that direction. It can also be ``from'' 
	somewhere, with inward points from that direction. The resulting 
	points extend the node contents (which include the |inner sep|).
	
\begin{codeexample}[]
\begin{tikzpicture}[every node/.style={signal, draw,  text=white}]
  \node[fill=green!65!black, signal to=east] at (0,1) {To East};
  \node[fill=red!65!black, signal from=east] at (0,0) {From East};
\end{tikzpicture}
\end{codeexample}

	There are \pgfname{} keys for drawing the signal shape (to use these
	keys in \tikzname{}, simply remove the \declare{|/pgf/|} path):
	
	\begin{key}{/pgf/signal pointer angle=\meta{angle} (initially 90)} 
		Set the angle for the pointed sides of the shape. This angle is
		maintained when enforcing any minimum size requirements, so
		any adjustment to the width will affect the height, and vice versa.
	\end{key}
	
	\begin{key}{/pgf/signal from=\meta{direction}\space\opt{and \meta{opposite direction}} (initially nowhere)} 
		Set which sides take an inward pointer (i.e., that points towards the
		center of the shape). The possible values for \meta{direction} and 
		\meta{opposide direction} are the compass point directions |north|,
		|south|, |east| and |west| (or |above|, |below|, |right| and |left|).
		An additional keyword |nowhere| can be used to reset the sides so 
		they have no pointers. When used with |signal from| key, this only 
		resets inward pointers;	used with the |signal to| key, it only 
		resets outward pointers. 
		
	\end{key}
	
	\begin{key}{/pgf/signal to=\meta{direction}\space\opt{and \meta{opposite direction}} (initially east)} 
		Set which sides take an outward pointer (i.e., that points away from 
		the	the shape). 
	\end{key}
	
	Note that \pgfname{} will ignore any instruction to use directions
	that are not opposites (so using the value |east and north|, will
	result in only |north| being assigned a pointer). This is also 
	the case if non-opposite values are used in the |signal to| and
	|signal from| keys at the same time. So, for example, it is not 
	possible for a signal to have an outward point to the left, and also
	have an inward point from below.
	
	The anchors for the signal shape are shown below. Anchor |70| is an
	example of a border anchor.

\begin{codeexample}[]
\Huge
\begin{tikzpicture}
  \node[name=s, shape=signal, signal from=west, shape example, inner sep=2cm] 
    {Signal\vrule width1pt height2cm};
  \foreach \anchor/\placement in
    {text/left,   center/above,    70/above,
     base/below,  base east/below, base west/below,
     mid/right,   mid east/above left,  mid west/above left, 
     north/above,      south/below, 
     east/above,       west/above,        
     north west/above, north east/above, 
     south west/below, south east/below}
     \draw[shift=(s.\anchor)] plot[mark=x] coordinates{(0,0)}
       node[\placement] {\scriptsize\texttt{(s.\anchor)}};
\end{tikzpicture}
\end{codeexample}

\end{shape}





\begin{shape}{tape}
	This shape is a rectangle with optional, ``bendy'' top and bottom
	sides, which tightly fits the node contents (including the 
	|inner sep|).
	
\begin{codeexample}[]
\begin{tikzpicture}
  \node[tape, draw]{ABCD};
  \node[tape, draw, tape bend top=none] at (1.5, 0) {EFGH};
\end{tikzpicture}
\end{codeexample}

  There are \pgfname{} keys to specify which sides bend and how high
  the bends are (to use these keys in \tikzname{}, simply remove the
  \declare{|/pgf/|} path):
  
  \begin{key}{/pgf/tape bend top=\meta{bend style} (initially in and out)}
  	Specify how the top side bends. The \meta{bend style} is either
  	|in and out|, |out and in| or |none| (i.e., a straight line). 
  	The bending sides are drawn in a 
  	clockwise direction, and using the bend style |in and out| will mean 
  	the side will first	bend inwards and then bend outwards. 
  	The opposite holds true for	|out and in|. 
  	
\begin{codeexample}[]
\begin{tikzpicture}[-stealth]
  \node[tape, draw, gray, minimum width=2cm](t){Tape};
  \draw [blue]([yshift=5pt] t.north west) -- ([yshift=5pt]t.north east) 
         node[midway, above, black]{in and out};
  \draw [blue]([yshift=-5pt]t.south east) -- ([yshift=-5pt]t.south west) 
         node[sloped, allow upside down, midway, above, black]{in and out};
\end{tikzpicture}
\end{codeexample}  

    This might take a bit of getting used to, but just remember that 
    when you want the bendy sides to be parallel, the sides take the 
    same bend style. It is possible for the top and bottom sides to 
    take opposite bend styles, but the author of this shape cannot 
    think of a single use for such a combination.
    
\begin{codeexample}[]
\begin{tikzpicture}
  \tikzstyle{every node}=[tape, draw]
  \node [tape bend top=out and in, tape bend bottom=out and in] {Parallel};
  \node at (2,0) [tape bend bottom=out and in]                  {Why?};
\end{tikzpicture}
\end{codeexample} 

	\end{key}
	
	\begin{key}{/pgf/tape bend bottom=\meta{bend style} (initially in and out)}
		Specify how the bottom side bends.
	\end{key}%
	
	\begin{key}{/pgf/tape bend height=\meta{length} (initially 5pt)}
		Set the total height for a side with a bend.
		
\begin{codeexample}[]
\begin{tikzpicture}[>=stealth]
  \draw [help lines] grid(3,2);
  \node [tape, fill, minimum size=2cm, red!50, tape bend top=none,
         tape bend height=1cm] at (1.5,1.5) (t) {};
  \draw [|<->|, blue] (1.5,0) -- (1.5,1) 
         node [at end, above, black]{tape bend height};
\end{tikzpicture}
\end{codeexample} 
 
	\end{key}
	
	The anchors for the tape shape are shown below. Anchor |60| is an
	example of a border anchor. Note that border anchors will snap to
	the center of convex curves (i.e. when bending in). 
	
\begin{codeexample}[]
\Huge
\begin{tikzpicture}
  \node[name=s, shape=tape, tape bend height=1cm, shape example, inner xsep=3cm] 
    {Tape\vrule width1pt height2cm};
   \foreach \anchor/\placement in
    {text/left,  center/above,    60/above, 
     base/below, base east/below, base west/below,
     mid/right,  mid east/left,   mid west/right,  
     north/above, south/below,  east/above, west/above,        
     north west/above, north east/above, 
     south west/below, south east/below}
     \draw[shift=(s.\anchor)] plot[mark=x] coordinates{(0,0)}
       node[\placement] {\scriptsize\texttt{(s.\anchor)}};
\end{tikzpicture}
\end{codeexample}

\end{shape}%



\subsection{Arrow Shapes}

\begin{pgflibrary}{shapes.arrows}
  This library defines arrow shapes. Note that an arrow shape is
  something quite different from a (normal) arrow tip: It is a shape
  that just ``happens'' to ``look like'' an arrow. In particular, you
  cannot use these shapes as arrow tips.
\end{pgflibrary}

\begin{shape}{single arrow}
	This shape is an arrow, which tightly fits the note contents 
	(including any |inner sep|). 
	This shape supports the rotation of the shape border, as 
	described in Section~\ref{section-rotating-shape-borders}. 
	The angle of rotation determines which direction the arrow
	points (provided no other rotational transformations are applied).
	
\begin{codeexample}[]
\begin{tikzpicture}[every node/.style={single arrow, draw},
    rotate border/.style={shape border uses incircle, shape border rotate=#1}]
  \node {right};
  \node at (2,0) [shape border rotate=90]{up};
  \node at (1,1) [rotate border=37, inner sep=0pt]{$37^\circ$};
\end{tikzpicture}
\end{codeexample}

	Regardless of the rotation of the arrow border, the width is 
  measured between the back ends of the arrow head, and the 
  height is measured from the arrow tip to the end of the arrow 
  tail.

\begin{codeexample}[]
\begin{tikzpicture}[>=stealth, 
    rotate border/.style={shape border uses incircle, shape border rotate=#1}]
  \node[rotate border=-30, fill=gray!25, minimum height=3cm, single arrow, 
    single arrow head extend=.5cm, single arrow head indent=.25cm] (arrow) {};
  \draw[red, <->] (arrow.before tip) -- (arrow.after tip)
    node [near end, left, black] {width};
  \draw[red, <->] (arrow.tip) -- (arrow.tail)
    node [near end, below left, black] {height};
\end{tikzpicture}
\end{codeexample}

	There are \pgfname{} keys that can be used to customize this shape (to
	use these keys in \tikzname{}, simply remove the \declare{|/pgf/|}
	path).
	
\begin{key}{/pgf/single arrow tip angle=\meta{angle} (initially 90)}
  Set the angle for the arrow tip. Enlarging the arrow to some
  minimum width may increase the the height of the shape to maintain
  this angle.
\end{key}

\begin{key}{/pgf/single arrow head extend=\meta{length} (initially .5cm)}
  This sets the distance between the tail of the arrow and the outer
  end of the arrow head. This may change if the shape is enlarged to
  some minimum width.
  
\begin{codeexample}[]
\begin{tikzpicture}
  \node[single arrow, draw, single arrow head extend=.5cm, gray!50, rotate=60] 
     (a) {Arrow};
  \draw[red, |<->|] (a.before tip) -- (a.before head) 
    node [midway, below, sloped, black] {head extend};
\end{tikzpicture}
\end{codeexample}
\end{key}

\begin{key}{/pgf/single arrow head indent=\meta{length} (initially 0cm)}
  This moves the point where the arrow head joins the shaft of the
  arrow \emph{towards} the arrow tip, by \meta{length}.
  
\begin{codeexample}[]
\begin{tikzpicture}[every node/.style={single arrow, draw=none, rotate=60}]
  \node [fill=red!50]                                           {arrow 1};
  \node [fill=blue!50, single arrow head indent=1ex] at (1.5,0) {arrow 2};
\end{tikzpicture}
\end{codeexample}
\end{key}

  The anchors for this shape are shown below (anchor |20| is an 
  example of a border anchor).
  
\begin{codeexample}[]
\Huge
\begin{tikzpicture}
  \node[name=s,shape=single arrow, shape example, single arrow head extend=1.5cm] 
    {Single Arrow\vrule width1pt height2cm};
  \foreach \anchor/\placement in
    {text/above,      center/above, 20/above,
     mid west/left,   mid/above,    mid east/above left,
     base west/below, base/below,   base east/below,
     tip/above, before tip/above, after tip/below, before head/above, 
     after head/below, after tail/above, before tail/below, tail/right,   
     north/above, south/below, east/below, west/above,
     north west/above, north east/below, south west/below, south east/above}    
     \draw[shift=(s.\anchor)] plot[mark=x] coordinates{(0,0)}
       node[\placement] {\scriptsize\texttt{(s.\anchor)}};
\end{tikzpicture}
\end{codeexample}

\end{shape}





\begin{shape}{double arrow}
  This shape is a double arrow, which tightly fits the note contents 
	(including any |inner sep|), and supports the rotation of the shape
	 border, as described in Section~\ref{section-rotating-shape-borders}. 
	 
	
\begin{codeexample}[]
\begin{tikzpicture}[every node/.style={double arrow, draw}]
  \node [double arrow, draw] {Left or Right};
\end{tikzpicture}
\end{codeexample}

  The double arrow behaves exactly like the single arrow, so you
  need to remember that the width is \emph{always} the distance
  between the back ends of the arrow heads, and the height
  is \emph{always} the the tip-to-tip distance.
  
\begin{codeexample}[]
\begin{tikzpicture}[>=stealth, 
    rotate border/.style={shape border uses incircle, shape border rotate=#1}]
  \node[rotate border=210, fill=gray!25, minimum height=3cm, double arrow, 
    double arrow head extend=.5cm, double arrow head indent=.25cm] (arrow) {};
  \draw[red, <->] (arrow.before tip 1) -- (arrow.after tip 1)
    node [near start, right, black] {width};
  \draw[red, <->] (arrow.tip 1) -- (arrow.tip 2)
    node [near end, above left, black] {height};
\end{tikzpicture}
\end{codeexample}

  The \pgfname{} keys that can be used to customize the double arrow 
  behave similarly to the keys for the single arrow (to
	use these keys in \tikzname{}, simply remove the \declare{|/pgf/|}
	path).
  
\begin{key}{/pgf/double arrow tip angle=\meta{angle} (initially 90)}
  Set the angle for the arrow tip. Enlarging the arrow to some
  minimum width may increase the the height of the shape to maintain
  this angle.
\end{key}

\begin{key}{/pgf/double arrow head extend=\meta{length} (initially .5cm)}
  This sets the distance between the shaft of the arrow and the outer
  end of the arrow heads. This may change if the shape is enlarged to
  some minimum width.
\end{key}

\begin{key}{/pgf/double arrow head indent=\meta{length} (initially 0cm)}
  This moves the point where the arrow heads join the shaft of the
  arrow \emph{towards} the arrow tips, by \meta{length}.
  \begin{codeexample}[]
\begin{tikzpicture}[every node/.style={double arrow, draw=none, rotate=-60}]
  \node [fill=red!50]                                           {arrow 1};
  \node [fill=blue!50, double arrow head indent=1ex] at (1.5,0) {arrow 2};
\end{tikzpicture}
\end{codeexample}
\end{key}


  The anchors for this shape are shown below (anchor |20| is an 
  example of a border anchor).
  
  
\begin{codeexample}[]
\Huge
\begin{tikzpicture}
  \node[name=s,shape=double arrow, double arrow head extend=1.5cm, shape example, inner xsep=2cm] 
    {Double Arrow\vrule width1pt height2cm};
  \foreach \anchor/\placement in
    {text/above, center/above, 20/above,
     mid west/above right, mid/above, mid east/above left,
     base west/below, base/below, base east/below,
     before head 1/above, before tip 1/above, tip 1/above, after tip 1/below, after head 1/below,
     before head 2/above, before tip 2/below, tip 2/above, after tip 2/above, after head 2/below,
     north/above, south/below, east/below, west/below,
     north west/below, north east/below, south west/above, south east/above}    
     \draw[shift=(s.\anchor)] plot[mark=x] coordinates{(0,0)}
       node[\placement] {\scriptsize\texttt{(s.\anchor)}};
\end{tikzpicture}
\end{codeexample}
\end{shape}




\begin{shape}{arrow box}
This shape is a rectangle with optional arrows which extend from the
four sides.

\begin{codeexample}[]
\begin{tikzpicture}
  \node[arrow box, draw] {A};
  \node[arrow box, draw, arrow box arrows={north:.5cm, west:0.75cm}] 
    at (2,0) {B};
\end{tikzpicture}
\end{codeexample}

Any minimum size requirements are applied to the main rectangle
\emph{only}. This does not pose too many problems if you wish to
accommodate the length of the arrows, as it is possible to specify 
the length of each arrow independently, from either the border of the
rectangle (the default) or the center of the rectangle.

\begin{codeexample}[]
\begin{tikzpicture}
	\tikzset{box/.style={arrow box, fill=#1}}
	\draw [help lines] grid(3,2);
  \node[box=blue!50, arrow box arrows={east:2cm}]             at (1,1.5){One};
  \node[box=red!50,  arrow box arrows={east:2cm from center}] at (1,0.5){Two};
\end{tikzpicture}
\end{codeexample}

There are various \pgfname{} keys for drawing this shape (to use these
keys in \tikzname, simply remove the \declare{/pgf/} path).

\begin{key}{/pgf/arrow box tip angle=\meta{angle} (initially 90)}
  Set the angle at the arrow tip for all four arrows. 
\end{key}

\begin{key}{/pgf/arrow box head extend=\meta{length} (initially .125cm)}
  Set the the distance the arrow head extends away from the the shaft
  of the arrow. This applies to all arrows.
\end{key}

\begin{key}{/pgf/arrow box head indent=\meta{length} (initially 0cm)}
  Move the point where the arrow head joins the shaft of the arrow
  \emph{towards} the arrow tip. This applies to all arrows.
\end{key}

\begin{key}{/pgf/arrow box shaft width=\meta{length} (initially .125cm)}
  Set the width of the shaft of all arrows.
\end{key}

\begin{key}{/pgf/arrow box north arrow=\meta{distance} (initially .5cm)}
  Set distance the north arrow extends from the node. By default this
  is from the border of the shape, but by using the additional keyword
  |from center|, the distance will be measured from the center of the
  shape. If \meta{distance} is |0pt| or a negative distance, the arrow
  will not be drawn.
\end{key}

\begin{key}{/pgf/arrow box south arrow=\meta{distance} (initially .5cm)}
	Set distance the south arrow extends from the node.
\end{key}

\begin{key}{/pgf/arrow box east arrow=\meta{distance} (initially .5cm)}
	Set distance the east arrow extends from the node.
\end{key}

\begin{key}{/pgf/arrow box west arrow=\meta{distance} (initially .5cm)}
	Set distance the west arrow extends from the node.
\end{key}

\begin{key}{/pgf/arrow box arrows={\ttfamily\char`\{}\meta{list}{\ttfamily\char`\}}}
  Set the distance that all arrows extend from the node. The 
  specification in \meta{list} consists of the four compass points
  |north|, |south|, |east| or |west|, separated by commas (so the list
  must be contained within braces). 
  The distances can be specified after each side separated by a colon 
  (e.g., |north:1cm|, or |west:5cm from center|). 
  If an item specifies no distance, the most recently specifed 
  distance will be used (at the start of the list this is |0cm|,
  so the first item in the list should specify a distance). 
  Any sides not specified will not be drawn with an arrow.
\end{key}

The anchors for this shape are shown below (unfortunately due to its
size, this example must be rotated). Anchor |75| is an example of a
border anchor.
If a side is drawn without an arrow, the anchors for that arrow should 
be considered unavailable. They are (unavoidably) defined, but default 
to the center of the appropriate side.

\begin{codeexample}[]
\Huge
\begin{tikzpicture}
  \node[shape=arrow box, shape example, inner xsep=1cm, inner ysep=1.5cm, arrow box shaft width=2cm,
    arrow box arrows={north:3.5cm from border, south, east:5cm from border, west}, 
    arrow box head extend=0.75cm, rotate=-90](s) {Arrow Box\vrule width1pt height2cm};
  \foreach \anchor/\placement in
   {center/above, text/above, mid/right, base/below, 75/above,
    mid east/right, mid west/left, base east/right, base west/left,
    north/below, south/below, east/below, west/below,
    north east/above, south east/above, south west/below, north west/below,
    north arrow tip/above,south arrow tip/above, east arrow tip/above, west arrow tip/above,
    before north arrow/above, before north arrow head/below left, before north arrow tip/above left, 
    after north arrow tip/above right, after north arrow head/below right, after north arrow/below,
    before south arrow/below, before south arrow head/above right, before south arrow tip/below right, 
    after south arrow tip/below left, after south arrow head/above left, after south arrow/above,
    before east arrow/above, before east arrow head/above right, before east arrow tip/above, 
    after east arrow tip/below, after east arrow head/below right, after east arrow/below, 
    before west arrow/below, before west arrow head/below left, before west arrow tip/below, 
    after west arrow tip/above, after west arrow head/above left, after west arrow/below}
      \draw[shift=(s.\anchor)] plot[mark=x] coordinates{(0,0)}
        node[\placement, rotate=-90] {\scriptsize\texttt{(s.\anchor)}};
\end{tikzpicture}
\end{codeexample}

\end{shape}



\subsection{Shapes with Multiple Text Parts}

\begin{pgflibrary}{shapes.multipart}
  This library defines general-purpose shapes that are composed of
  multiple (text) parts. 
\end{pgflibrary}


\begin{shape}{circle split}
  This shape is a multi-part shape consisting of a circle with a line
  in the middle. The upper part is the main part (the |text| part),
  the lower part is the |lower| part.
  
\begin{codeexample}[]
\begin{tikzpicture}
  \node [circle split,draw,double,fill=red!20]
  {
    $q_1$
    \nodepart{lower}
    $00$
  };
\end{tikzpicture}
\end{codeexample}

  The shape inherits all anchors from the |circle| shape and defines
  the |lower| anchor in addition. See also the
  following figure:
\begin{codeexample}[]
\Huge
\begin{tikzpicture}
  \node[name=s,shape=circle split,shape example] {text\nodepart{lower}lower};
  \foreach \anchor/\placement in
    {north west/above left, north/above, north east/above right, 
     west/left, center/below, east/right, 
     mid west/right, mid/above, mid east/left, 
     base west/left, base/below, base east/right, 
     south west/below left, south/below, south east/below right, 
     text/left, lower/left, 130/above}
     \draw[shift=(s.\anchor)] plot[mark=x] coordinates{(0,0)}
       node[\placement] {\scriptsize\texttt{(s.\anchor)}};
\end{tikzpicture}
\end{codeexample}
\end{shape}


\begin{shape}{ellipse split}
	This shape is a multi-part shape consisting of an ellipse with a line
  in the middle. The upper part is the main part (the |text| part),
  the lower part is the |lower| part.  
  The anchors for this shape are shown below. Anchor |60| is a border anchor.
\begin{codeexample}[]
\Huge
\begin{tikzpicture}
  \node[name=s,shape=ellipse split,shape example] {text\nodepart{lower}lower};
  \foreach \anchor/\placement in
    {center/below, text/left, lower/left, 60/above right,
     mid/above, mid east/above, mid west/above,
     base/right, base east/left, base west/right,
     north/above, south/below, east/below, west/below,
     north east/above, south east/below, south west/below, north west/above}
     \draw[shift=(s.\anchor)] plot[mark=x] coordinates{(0,0)}
       node[\placement] {\scriptsize\texttt{(s.\anchor)}};
\end{tikzpicture}
\end{codeexample}
\end{shape}


\begin{shape}{rectangle split}
  This shape is a rectangle which can optionally be split into 
  two, three or four node parts, or even used with a single node 
  part. 

\begin{codeexample}[]
\begin{tikzpicture}[every text node part/.style={text centered}]
  \node[rectangle split, rectangle split parts=3, draw, text width=2.75cm] 
    {Student
     \nodepart{second}
       age:int \\
       name:String
     \nodepart{third}
       getAge():int \\
       getName():String};
\end{tikzpicture}
\end{codeexample} 

  
  The contents of node parts which are not used are ignored. 
  Which node parts are used in each case is shown below:

\begin{codeexample}[]
\begin{tikzpicture}
  \foreach \a/\x/\y in {1/0/0, 2/1.5/0, 3/0/-1.5, 4/1.5/-1.5}
  \node[rectangle split, rectangle split parts=\a, draw, anchor=north] 
    at (\x,\y){
      text
    \nodepart{second}
      second
    \nodepart{third}
      third
    \nodepart{fourth}
      fourth};
\end{tikzpicture}
\end{codeexample} 

  There are several \pgfname{} keys to specify how the shape is
  drawn. To use these keys in \tikzname, simply remove the 
  \declare{|/pgf/|} path:
  
  \begin{key}{/pgf/rectangle split parts=\meta{number} (initially 4)}
    Split the rectangle into \meta{number} parts, 
    which should be in the range |1| to |4|.
  \end{key}
  
  \begin{key}{/pgf/rectangle split empty part height=\meta{length} (initially 1ex)}
    Set the default height for a node part box if it is empty.
  \end{key}
  
  \begin{key}{/pgf/rectangle split part align={\ttfamily\char`\{}\meta{list}{\ttfamily\char`\}} (initially center)}
  	Set the alignment of the boxes inside the node parts.
  	There should be a maximum of four entries in \meta{list}, 
  	separated by commas (so if there is more than one entry in 
  	\meta{list} it must be surrounded by braces).
  	Each entry is one of |left|,
    |right|, or |center|. If \meta{list} has less entries than 
    node parts then the remaining node parts are aligned according to 
    the last entry in the list.    
    Note that this only aligns the boxes in each part and \emph{does not} 
    affect the alignment of the contents of the boxes.
    
\begin{codeexample}[]
\def\v#1{\vrule width#1ex height2ex}
\def\x{\v2 \nodepart{second} \v5 \nodepart{third} \v2 \nodepart{fourth} \v2}
\begin{tikzpicture}[every node/.style={rectangle split, draw, text=blue!40}]
  \node[rectangle split part align={center, left, right}] at (0,0)    {\x};
  \node[rectangle split part align={center, left}]        at (1.25,0) {\x};
  \node[rectangle split part align={center}]              at (2.5,0)  {\x};
\end{tikzpicture}
\end{codeexample}
  \end{key}
   
  \begin{key}{/pgf/rectangle split draw splits=\meta{boolean} (initially true)}
  	Set whether the line or lines between node parts will be drawn.
  	Internally, this sets the \TeX-if |\ifpgfrectanglesplitdrawsplits| 
  	appropriately.
  \end{key}
  
  \begin{key}{/pgf/rectangle split use custom fill=\meta{boolean} (initially false)}
    This enables the use of a custom fill for each of the node
    parts (including the area covered by the |inner sep|). The 
    background path for the shape should not be filled (e.g., in
    \tikzname{}, the |fill|
    option for the node must be implicity or explicitly set to |none|).
    Internally, this key sets the \TeX-if 
    |\ifpgfrectanglesplitusecustomfill| appropriately.
  \end{key}
  
  \begin{key}{/pgf/rectangle split part fill={\ttfamily\char`\{}\meta{list}{\ttfamily\char`\}} (initially white)}
  	Set the custom fill color for each node part shape. 
  	There should be a maximum of four entries in \meta{list} (one 
  	for each node part), separated by commas (so if there is more than 
  	one entry in \meta{list} it must be surrounded by braces).
  	If \meta{list}  has less entries than node
    parts then the remaining node parts use the color from
    the last entry in the list. This key will automatically set
    |/pgf/rectangle split use custom fill|.
    
\begin{codeexample}[]
\begin{tikzpicture}
  \tikzset{every node/.style={rectangle split, draw, minimum width=.5cm}}
  \node[rectangle split part fill={red!50, green!50, blue!50, yellow!50}]  {};
  \node[rectangle split part fill={red!50, green!50, blue!50}] at (0.75,0) {};
  \node[rectangle split part fill={red!50, green!50}]          at (1.5,0)  {};
  \node[rectangle split part fill={red!50}]                    at (2.25,0) {};
\end{tikzpicture}
\end{codeexample}

	\end{key}
	
  The anchors for the |rectangle split| shape, are
  shown below (anchor |70| is an example of a border angle). When a 
  node part is missing (i.e., when the number of parts is less than 
  |4|), the anchors prefixed with name of that node part should be 
  considered unavailable. They are (unavoidably) defined, but default 
  to other anchor positions.
  
\begin{codeexample}[]
\Huge
\begin{tikzpicture}
  \node[name=s,shape=rectangle split, rectangle split parts=4, shape example] 
    {\nodepart{text}Text\nodepart{second}Second
		\nodepart{third}Third\nodepart{fourth}fourth};
  \foreach \anchor/\placement in
    {text/left,   text east/above,   text west/above, 
     second/left, second east/above, second west/above,    
     third/left,  third east/below,  third west/below,
     fourth/left, fourth east/below, fourth west/below,
     text split/left,   text split east/above,   text split west/above,
     second split/left, second split east/above, second split west/above,    
     third split/left,  third split east/below,  third split west/below, 
     north/above,  south/below,  east/below,  west/below,  
     center/above, 70/above,     mid/above,   base/below,
     north west/above, north east/above, south west/below, south east/below}
     \draw[shift=(s.\anchor)] plot[mark=x] coordinates{(0,0)}
       node[\placement] {\scriptsize\texttt{(s.\anchor)}};
\end{tikzpicture}
\end{codeexample}

\end{shape}





\subsection{Callout Shapes}

\begin{pgflibrary}{shapes.callout}
  Producing basic callouts can be done quite easily in \pgfname{} and 
	\tikzname{} by creating a node and then subsequently drawing a path
	from the border of the node to the required point. This library
	provides more fancy, `balloon'-style callouts.
    
\end{pgflibrary}

Callouts consist of a 
main shape, and a pointer (which is part of the shape) which points
to something in (or outside) the picture. The position on the border
of the main shape to which the pointer is connected is determined
automatically. However, the pointer is ignored when calculating the 
minimum size of the shape, and also when positioning anchors.

\begin{codeexample}[]
\begin{tikzpicture}[remember picture]
  \node[ellipse callout, draw] (hallo) {Hallo!};
\end{tikzpicture}
\end{codeexample}

There are two kinds of pointer:	the ``relative'' pointer and the 
``absolute'' pointer.	The relative pointer calculates the angle of a
specified coordinate relative to the center of the main shape, locates 
the point on the border to which this angle corresponds, and then adds 
the coordinate to this point. This seemingly over-complex approach 
means than you do not have to guess the size of the main shape: the 
relative pointer will always be outside. 
The absolute pointer, on the 
other hand, is much simpler: it points to the specified coordinate
absolutely (and can even point to named coordinates in different 
pictures).


\begin{codeexample}[]
\begin{tikzpicture}[remember picture, note/.style={rectangle callout, fill=#1}]
  \draw [help lines] grid(3,2);
  \node [note=red!50,  callout relative pointer={(0,1)}] at (3,1) {Relative};
  \node [note=blue!50, callout absolute pointer={(0,1)}] at (1,0) {Absolute};
  \node [note=green!50, opacity=.5, overlay, 
         callout absolute pointer={(hallo.south)}]       at (1,2) {Outside};
\end{tikzpicture}
\end{codeexample}


The following keys are common to all callouts. Please remember
that the |callout| |relative| |pointer|, and |callout| |absolute|
|pointer| keys take a different format for their value depending 
on whether they are being used in \pgfname{} or \tikzname{}.
  
  
\begin{key}{/pgf/callout relative pointer=\meta{coordinate} (initially {\ttfamily\char`\\pgfpointpolar\char`\{315\char`\}\char`\{.5cm\char`\}})}
  
  Set the vector of the callout pointer `relative' to the callout 
  shape. 
  
\end{key}

\begin{key}{/pgf/callout absolute pointer=\meta{coordinate}}
  
  Set the vector of the callout pointer absolutely within the picture.
  
\end{key}



\begin{key}{/tikz/callout relative pointer=\meta{coordinate} (initially {(315:.5cm)})}
	The \tikzname{} version of the |callout relative pointer| key. Here,
	\meta{coordinate} can be specified using the \tikzname{} format for
	coordinates.
\end{key}

\begin{key}{/tikz/callout absolute pointer=\meta{coordinate}}
	The \tikzname{} version of the |callout absolute pointer| key. Here,
	\meta{coordinate} can be specified using the \tikzname{} format for
	coordinates.
\end{key}

	It is also possible to shorten the pointer by some distance, using 
	the following key:
	
\begin{key}{/pgf/callout pointer shorten=\meta{distance} (initially 0cm)}
	Move the callout pointer towards the center of the callouts main 
	shape by \meta{distance}. 
	
\begin{codeexample}[]
\begin{tikzpicture}
	\tikzset{callout/.style={ellipse callout, callout pointer arc=30,
	  callout absolute pointer={#1}}}
  \draw (0,0) grid (3,2);
  \node[callout={(3,1.5)}, fill=red!50] at (0,1.5) {A};
  \node[callout={(3,.5)},  fill=green!50, callout pointer shorten=1cm]          
    at (0,.5)  {B};
\end{tikzpicture} 
\end{codeexample}
\end{key}
	
  
\begin{shape}{rectangle callout}%
	
  This shape is a callout whose main shape is a rectangle, which 
  tightly fits the node contents (including any |inner sep|).
  It  supports the keys described above and also the following 
  key:
  
  
\begin{key}{/pgf/callout pointer width=\meta{length} (initially .25cm)} 
  Set the width of the pointer at the border of the rectangle.
\end{key}
		
	The anchors for this shape are shown below (anchor |60| is an
	example of a border anchor). The pointer direction is ignored when 
	placing anchors. 
	Additionally, when using an absolute pointer, the |pointer| 
	anchor should not be used to used to position the shape as it is 
	calculated whilst the shape is being drawn.
	
\begin{codeexample}[]
\Huge
\begin{tikzpicture}
  \node[name=s,shape=rectangle callout, callout relative pointer={(1.25cm,-1cm)}, 
     callout pointer width=2cm, shape example, inner xsep=2cm, inner ysep=1cm] 
  	{Rectangle Callout\vrule width 1pt height 2cm};
  \foreach \anchor/\placement in
    {center/above, text/below,      60/above,
     mid/right,    mid west/left,  mid east/right, 
     base/below,   base west/below, base east/below, 
     north/above,  south/below, east/above, west/above,
     north west/above, north east/above,     
     south west/below, south east/below,
     pointer/below}
     \draw[shift=(s.\anchor)] plot[mark=x] coordinates{(0,0)}
       node[\placement] {\scriptsize\texttt{(s.\anchor)}};
\end{tikzpicture}
\end{codeexample}

\end{shape}%


\begin{shape}{ellipse callout}%
	
  This shape is a callout whose main shape is a ellipse, which 
  tightly fits the node contents (including any |inner sep|).
  It uses the |absolute callout pointer|, 
	|relative callout pointer| and |callout pointer shorten| keys, and
	also the following key:
  
  
\begin{key}{/pgf/callout pointer arc=\meta{angle} (initially 15)} 
  Set the width of pointer at the border of the ellipse according
  to an arc of length \meta{angle}.
\end{key}
	
	
	The anchors for this shape are shown below (anchor |60| is an
	example of a border anchor). The pointer direction is ignored 
	when placing anchors and the |pointer| anchor can only be
	used to position the shape when the relative anchor is 
	specified.
	
\begin{codeexample}[]
\Huge
\begin{tikzpicture}
  \node[name=s,shape=ellipse callout, callout relative pointer={(1.25cm,-1cm)}, 
    callout pointer width=2cm, shape example, inner xsep=1cm, inner ysep=.5cm] 
  	{Ellipse Callout\vrule width 1pt height 2cm};
  \foreach \anchor/\placement in
    {center/above, text/below,      60/above,
     mid/above,    mid west/right,  mid east/left, 
     base/below,   base west/below, base east/below, 
     north/above,  south/below, east/above, west/above,
     north west/above left,    north east/above right,     
     south west/below left,    south east/below right,
     pointer/below}
     \draw[shift=(s.\anchor)] plot[mark=x] coordinates{(0,0)}
       node[\placement] {\scriptsize\texttt{(s.\anchor)}};
\end{tikzpicture}
\end{codeexample}

\end{shape}


\begin{shape}{cloud callout}
This shape is a callout whose main shape is a cloud which fits the 
node contents. The pointer is segmented, consisting of a series of 
shrinking ellipses. This callout requires the symbol shape library 
(for the cloud shape). If this library is not loaded an error will 
result.

\begin{codeexample}[]
\begin{tikzpicture}
  \node[cloud callout, cloud puffs=15, aspect=2.5, cloud puff arc=120, 
    shading=ball,text=white] {\bf Imagine...};
\end{tikzpicture}
\end{codeexample}

The |cloud callout| supports the |absolute callout pointer|,
|relative callout pointer| and |callout pointer shorten| keys, as 
described above.
The main shape can be modified using the same keys as the |cloud| 
shape. The following keys are also supported:

\begin{key}{/pgf/callout pointer start size=\meta{value} (initially .2 of callout)}
	Set the size of the first segment in the pointer (i.e., the segment
	nearest the main cloud shape). There are three possible forms for
	\meta{value}:
	\begin{itemize}
		\item
			A single dimension (e.g., |5pt|), in which case the first ellipse 
			will have equal diameters of 5pt.
		\item
			Two dimensions (e.g., |10pt and 2.5pt|), which sets the $x$ and 
			$y$ diameters of the first ellipse.
		\item
			A decimal fraction (e.g., |.2 of callout|), in which case
			the $x$ and $y$ diameters of the first ellipse will be set as 
			fractions of the width and height of the main shape. The keyword
			|of callout| cannot be omitted.	
	\end{itemize}
\end{key}

\begin{key}{/pgf/callout pointer end size=\meta{value} (initially .1 of callout)}
	Set the size of the last ellipse in the pointer.
\end{key}

\begin{key}{/pgf/callout pointer segments=\meta{number} (initially 2)}
	Set the number of segments in the pointer. Note that \pgfname{} will 
	happily	overlap segments if too many are specified. 
\end{key}

The anchors for this shape are shown below (anchor |70| is an example 
of a border anchor). The pointer direction is ignored when placing 
anchors and the pointer anchor can only be used to position the
shape when the relative anchor is specified. Note that the center
of the last segment is drawn at the |pointer| anchor.

\begin{codeexample}[]
\Huge
\begin{tikzpicture}
  \node[name=s, shape=cloud callout, style=shape example, cloud puffs=11, aspect=1.5,
    cloud puff arc=120,inner xsep=.5cm, callout pointer start size=.25 of callout,
    callout pointer end size=.15 of callout, callout relative pointer={(315:4cm)},
    callout pointer segments=2] {Cloud Callout\vrule width 1pt height 2cm};
  \foreach \anchor/\placement in
    {puff 1/above, puff 2/above, puff 3/above, puff 4/below,
     puff 5/left, puff 6/below, puff 7/below, puff 8/right,
     puff 9/below, puff 10/above, puff 11/above, 70/right,
     center/above, base/below, mid/right, text/left,
     north/below, south/below, east/above, west/above,
     north west/left, north east/right,
     south west/below, south east/below,pointer/above}
  \draw[shift=(s.\anchor)] plot[mark=x] coordinates{(0,0)}
    node[\placement] {\scriptsize\texttt{(s.\anchor)}};
\end{tikzpicture}
\end{codeexample}%
\end{shape}




\subsection{Logic Gate Shapes}

\subsubsection{Overview}
	\pgfname{} provides two libraries of logic gates, one providing
	``American'' style gates and the other, providing ``rectangular'' 
	logic gates.	
	Each library suffixes the gate names with an identifer:
	|US| for the American style gates, and |IEC| for the rectangular
	gates (additionally, two shapes in the |US| library use the
	suffix |CDH|). Keys which are specific
	to a particular library	also contain this identifier (e.g., 
	|/pgf/and gate IEC symbol|).
	However, as described below, a \tikzname{} key is provided which
	sets up several styles allowing the identifier to be omitted,
	for example, |and gate| can become a synonym for |shape=and gate US|.
	
	Multiple inputs can be specified for a logic gate (provided they
  support multiple inputs: a not gate --- also known as an inverter ---
  does not). However, there is an upper limit for the number of inputs 
  which has been set at 1024, which should be \emph{way} 
  more than would ever be needed.
    
  There are some \pgfname{} keys which are common to both 
  libraries, which have no library identifier contained in them:
  
  
  \begin{key}{/pgf/logic gate inputs=\meta{input list} (initially \char`\{normal,normal\char`\})}
  Specify the inputs for for the logic gate. The keyword |inverted|
  indicates an inverted input which will mean \pgfname{} will draw a
  circle attached to the main shape of the logic gate. Any keyword
  that is not |inverted| will be treated as a ``normal'' or 
  ``non-inverted'' input (however, for readability, you may wish to 
  use |normal| or |non-inverted|), and \pgfname{} will not draw the 
  circle.  
  In both cases the anchors for the inputs will be set 
  up appropriately, numbered from top to bottom |input 1|, |input 2|,
  \ldots and so on. If the gate only supports one input the anchor
  is simply called |input| with no numerical index.
  
\begin{codeexample}[]
\begin{tikzpicture}[minimum height=0.75cm]
  \node[and gate IEC, draw, logic gate inputs={inverted, normal, inverted}] 
  (A) {};
  \foreach \a in {1,...,3}
    \draw (A.input \a -| -1,0) -- (A.input \a);
  \draw (A.output) -- ([xshift=0.5cm]A.output);
\end{tikzpicture}
\end{codeexample} 
  
  For multiple inputs it may be somewhat unweildy to specify a long
  list, thus, the following ``shorthand'' is permitted (this is an  
  extension of ideas due to Juergen Werber and Christoph Bartoschek):
  Using |i| for inverted and |n| for normal inputs, \meta{input list}
  can be specfied \emph{without the commas}. So, for example,
  |ini| is equivalent to |inverted, normal, inverted|.
  
\begin{codeexample}[]
\begin{tikzpicture}[minimum height=0.75cm]
  \node[or gate US, draw,logic gate inputs=inini] (A) {};
  \foreach \a in {1,...,5}
    \draw (A.input \a -| -1,0) -- (A.input \a);
  \draw (A.output) -- ([xshift=0.5cm]A.output);
\end{tikzpicture}
\end{codeexample} 
 
\end{key}


The height of the gate may be increased to accommodate the number 
of inputs. In fact, it depends on three variables:
$n$, the number of inputs, $r$, the radius of the circle used
to indicate an inverted input and $s$, the distance between
the centers of the inputs.
The default height is then calculated according to the expression 
$(n+1)\times\max(2r,s)$. This then may
be increased to accommodate the node contents or any
minimum size specifications.

The radius of the inverted input circle and the distance between the 
centers of the inputs can be customised using the following keys:

\begin{key}{/pgf/logic gate inverted radius=\meta{length} (initially 2pt)}
  Set the radius of the circle that is used to indicate inverted
  inputs. This is also the radius of the circle used for the inverted
  output of the |nand|, |nor|, |xnor| and |not| gates. 
    
\begin{codeexample}[]
\begin{tikzpicture}[minimum height=0.75cm]
  \tikzset{every node/.style={shape=nand gate CDH, draw, logic gate inputs=ii}}
  \node[logic gate inverted radius=2pt] {A};
  \node[logic gate inverted radius=4pt] at (0,-1) {B};
\end{tikzpicture}
\end{codeexample} 
\end{key}

\begin{key}{/pgf/logic gate input sep=\meta{length} (initially .125cm)}
  Set the distance between the \emph{centers} of the inputs to the
  logic gate. 
  
\begin{codeexample}[]
\begin{tikzpicture}[minimum size=0.75cm]
  \draw [help lines] grid (3,2);
  \tikzset{every node/.style={shape=and gate IEC, draw, logic gate inputs=ini}}
  \node[logic gate input sep=0.33333cm] at (1,1)(A) {A};
  \node[logic gate input sep=0.5cm]     at (3,1) (B) {B};
  \foreach \a in {1,...,3}
    \draw (A.input \a -| 0,0) -- (A.input \a)
          (B.input \a -| 2,0) -- (B.input \a);
\end{tikzpicture}
\end{codeexample} 
\end{key}


\subsubsection{US Logic Gates}

\begin{pgflibrary}{shapes.gates.logic.US}
 This library provides ``American'' logic gate shapes whose names are 
  suffixed with the identifier |US|. Additionally,
  alternative |and| and |nand| gates are provided which are based on the 
  logic symbols used in A. Croft, R. Davidson, and M. Hargreaves (1992), 
  \emph{Engineering Mathematics}, Addison-Wesley, 82--95. These two 
  shapes are suffixed with |CDH|. 
\end{pgflibrary}

  To use the shapes in \tikzname{} without their suffixes, the 
  following keys are provided:
  
\begin{key}{/tikz/use US style logic gates}
	This allows the the shapes suffixed with |US| to be used without
	the suffix. So, for example, |and gate| becomes a synonym for
	|shape=and gate US|.
\begin{codeexample}[]
\tikz\node[draw, and gate US, red]{and};
\space
\tikz[use US style logic gates,blue]\node[draw, and gate]{and};
\end{codeexample}
\end{key}

\begin{key}{/tikz/use CDH style logic gates}
	This key again allows the the shapes suffixed with |US| to be used 
	without	the |US| suffix. However, |and gate| becomes a synonym for
	|shape=and gate CDH| and |nand gate| becomes a synonym for
	|shape=nand gate CDH|, providing alternative symbols for these
	gates.
	
\begin{codeexample}[]
\begin{tikzpicture}[minimum height=1cm]
  \node[draw, and gate US, red]  at  (0,1.5) {and};
  \node[draw, nand gate US, red] at (2,1.5) {nand};
  \tikzset{use CDH style logic gates}
  \node[draw, and gate, blue]  at (0,0) {and};
  \node[draw, nand gate, blue] at (2,0) {nand};
\end{tikzpicture}
\end{codeexample}
\end{key}



As described above, \pgfname{} will increase the size of the 
logic gate to accommodate the number of inputs, and the size
of the inverted radius and the separation between the inputs.
However with all shapes in this library, any increase in size 
(including any minimum size requirements) will be applied so that 
the default aspect ratio is unaltered. This means that changing
the height will change the width and vice versa. 

The ``compass point'' anchors apply to the main part of the shape
and do not include any inverted inputs or outputs. This library
provides an additonal feature to facilitate the relative positioning
of logic gates:

\begin{key}{/pgf/logic gate anchors use bounding box=\meta{boolean} (initially false)}
When set to |true| this key will ensure that the 
compass point anchors use the bounding rectangle of the
main shape, which, ignore any inverted inputs or outputs, but
includes any |outer sep|. 
This \emph{only} affects the compass point anchors
and is not set on a shape by shape basis: whether the bounding
box is used is determined by value of this key when the anchor
is accessed.

\begin{codeexample}[]
\begin{tikzpicture}[minimum height=1.5cm]
  \node[xnor gate US, draw, gray!50,line width=2pt] (A) {};
  \foreach \x/\y/\z in {false/blue/1pt, true/red/2pt}
    \foreach \a in {north, south, east, west, north east, 
      south east, north west, south west}
      \draw[logic gate anchors use bounding box=\x, color=\y]	
        (A.\a) circle(\z);
\end{tikzpicture}
\end{codeexample} 



\end{key}




\begin{shape}{and gate US}
  This shape is an and gate which supports two or more inputs. If
	less than two inputs are specified an error will result. 
	The anchors for this gate with two
  non-inverted inputs (using the normal compass point anchors) are
  shown below. Anchor |30| is an example of a border anchor.
  
\begin{codeexample}[]
\Huge
\begin{tikzpicture}
  \node[name=s,shape=and gate US,shape example, inner sep=0cm,
    logic gate inverted radius=.5cm] {And Gate\vrule width1pt height2cm};
  \foreach \anchor/\placement in
    {center/above, text/above, 30/above right,
     mid/right, mid east/left, mid west/above,
     base/below, base east/right, base west/left,
     north/above, south/below, east/above, west/above,
     north east/above, south east/below, south west/below, north west/above,
     output/right, input 1/above, input 2/below}
     \draw[shift=(s.\anchor)] plot[mark=x] coordinates{(0,0)}
       node[\placement] {\scriptsize\texttt{(s.\anchor)}};
\end{tikzpicture}
\end{codeexample}
\end{shape}

\begin{shape}{nand gate US}
  This shape is a nand gate, which supports two or more inputs. If
	less than two inputs are specified an error will result. 
	The anchors for this gate with two
  non-inverted inputs (using the normal compass point anchors) are
  shown below. Anchor |30| is an example of a border anchor.
 
\begin{codeexample}[]
\Huge
\begin{tikzpicture}
  \node[name=s,shape=nand gate US,shape example, inner sep=0cm,
  logic gate inverted radius=.5cm] {Nand Gate\vrule width1pt height2cm};
  \foreach \anchor/\placement in
    {center/above, text/above, 30/above right,
     mid/right, mid east/left, mid west/above,
     base/below, base east/below, base west/left,
     north/above, south/below, east/above, west/above,
     north east/above, south east/below, south west/below, north west/above,
     output/right, input 1/above, input 2/below}
     \draw[shift=(s.\anchor)] plot[mark=x] coordinates{(0,0)}
       node[\placement] {\scriptsize\texttt{(s.\anchor)}};
\end{tikzpicture}
\end{codeexample}

\end{shape}

\begin{shape}{or gate US}

	This shape is an or gate, which supports two or more inputs. If
	less than two inputs are specified an error will result. 
	The anchors for this gate with two
  non-inverted inputs (using the normal compass point anchors) are
  shown below. Anchor |30| is an example of a border anchor.
  
\begin{codeexample}[]
\Huge
\begin{tikzpicture}
  \node[name=s,shape=or gate US,shape example, inner sep=0cm,
  logic gate inverted radius=.5cm] {Or Gate\vrule width1pt height2cm};
  \foreach \anchor/\placement in
    {center/above, text/above, 30/above right,
     mid/right, mid east/left, mid west/above,
     base/below, base east/below, base west/left,
     north/above, south/below, east/above, west/above,
     north east/above, south east/below, south west/below, north west/above,
     output/right, input 1/left, input 2/below}
     \draw[shift=(s.\anchor)] plot[mark=x] coordinates{(0,0)}
       node[\placement] {\scriptsize\texttt{(s.\anchor)}};
\end{tikzpicture}
\end{codeexample}
\end{shape}

\begin{shape}{nor gate US}
	This shape is a nor gate, which supports two or more inputs. If
	less than two inputs are specified an error will result. 
	The anchors for this gate with two
  non-inverted inputs (using the normal compass point anchors) are
  shown below. Anchor |30| is an example of a border anchor.

\begin{codeexample}[]
\Huge
\begin{tikzpicture}
  \node[name=s,shape=nor gate US,shape example, inner sep=0cm,
  logic gate inverted radius=.5cm] {Nor Gate\vrule width1pt height2cm};
  \foreach \anchor/\placement in
    {center/above, text/above, 30/above right,
     mid/right, mid east/left, mid west/above,
     base/below, base east/below, base west/left,
     north/above, south/below, east/above, west/above,
     north east/above, south east/below, south west/below, north west/above,
     output/right, input 1/left, input 2/below}
     \draw[shift=(s.\anchor)] plot[mark=x] coordinates{(0,0)}
       node[\placement] {\scriptsize\texttt{(s.\anchor)}};
\end{tikzpicture}
\end{codeexample}

\end{shape}

\begin{shape}{xor gate US}
  This shape is an xor gate, which supports only two inputs. If
  less than two inputs are specified an error will result. If more
  than two inputs are specified, the extra inputs are ignored.
  The anchors for this gate with two
  non-inverted inputs (using the normal compass point anchors) are
  shown below. Anchor |30| is an example of a border anchor.
  
\begin{codeexample}[]
\Huge
\begin{tikzpicture}
  \node[name=s,shape=xor gate US,shape example, inner sep=0cm,
    logic gate inverted radius=.5cm] {Xor Gate\vrule width1pt height2cm};
  \foreach \anchor/\placement in
    {center/above, text/above, 30/above right,
     mid/right, mid east/left, mid west/above,
     base/below, base east/below, base west/left,
     north/above, south/below, east/above, west/above,
     north east/above, south east/below, south west/below, north west/above,
     output/right, input 1/above, input 2/below}
     \draw[shift=(s.\anchor)] plot[mark=x] coordinates{(0,0)}
       node[\placement] {\scriptsize\texttt{(s.\anchor)}};
\end{tikzpicture}
\end{codeexample}

\end{shape}

\begin{shape}{xnor gate US}

	This shape is an xnor gate, which supports only two inputs. If
	less than two inputs are specified an error will result. If more
	than two inputs are specified, the extra inputs are ignored.
	The anchors for this gate with two
  non-inverted inputs (using the normal compass point anchors) are
  shown below. Anchor |30| is an example of a border anchor.
  
\begin{codeexample}[]
\Huge
\begin{tikzpicture}
  \node[name=s,shape=xnor gate US,shape example, inner sep=0cm,
  logic gate inverted radius=.5cm] {Xnor Gate\vrule width1pt height2cm};
  \foreach \anchor/\placement in
    {center/above, text/above, 30/above right,
     mid/right, mid east/left, mid west/above,
     base/below, base east/below, base west/left,
     north/above, south/below, east/above, west/above,
     north east/above, south east/below, south west/below, north west/above,
     output/above, input 1/above, input 2/below}
     \draw[shift=(s.\anchor)] plot[mark=x] coordinates{(0,0)}
       node[\placement] {\scriptsize\texttt{(s.\anchor)}};
\end{tikzpicture}
\end{codeexample}


\end{shape}

\begin{shape}{not gate US}
	This shape is a not gate, which supports only one input. If
	no inputs are specified an error will result. If more
	than one input is specified, the extra inputs are ignored.
	The anchors for this gate with two
  non-inverted inputs (using the normal compass point anchors) are
  shown below. Anchor |30| is an example of a border anchor.
  
\begin{codeexample}[]
\Huge
\begin{tikzpicture}
  \node[name=s,shape=not gate US,shape example, inner sep=1.5cm,
  logic gate inverted radius=.5cm] 
  {Not Gate\vrule width1pt height2cm};
  \foreach \anchor/\placement in
    {center/above, text/above, 30/above right,
     mid/right, mid east/left, mid west/above,
     base/below, base east/below, base west/below,
     north/above, south/below, east/above, west/above,
     north east/above, south east/below, south west/below, north west/above,
     output/above}
     \draw[shift=(s.\anchor)] plot[mark=x] coordinates{(0,0)}
       node[\placement] {\scriptsize\texttt{(s.\anchor)}};
\end{tikzpicture}
\end{codeexample}

\end{shape}

\begin{shape}{buffer gate US}
	This shape is a not gate, which supports only one input. If
	no inputs are specified an error will result. If more
	than one input is specified, the extra inputs are ignored.
	The anchors for this gate with two
  non-inverted inputs (using the normal compass point anchors) are
  shown below. Anchor |30| is an example of a border anchor.
  
\begin{codeexample}[]
\Huge
\begin{tikzpicture}
  \node[name=s,shape=buffer gate US,shape example, inner sep=1.5cm,
  logic gate inverted radius=.5cm] 
  {Buffer Gate\vrule width1pt height2cm};
  \foreach \anchor/\placement in
    {center/above, text/above, 30/above right,
     mid/right, mid east/left, mid west/above,
     base/below, base east/below, base west/below,
     north/above, south/below, east/above, west/above,
     north east/above, south east/below, south west/below, north west/above,
     output/below}
     \draw[shift=(s.\anchor)] plot[mark=x] coordinates{(0,0)}
       node[\placement] {\scriptsize\texttt{(s.\anchor)}};
\end{tikzpicture}
\end{codeexample}

\end{shape}





\begin{shape}{and gate CDH}
  This shape is the alternative and gate. It supports two or more inputs.
  If less than two inputs are specified an error will result. 
	The anchors for this gate with two
  non-inverted inputs (using the normal compass point anchors) are
  shown below. Anchor |30| is an example of a border anchor.
  
\begin{codeexample}[]
\Huge
\begin{tikzpicture}
  \node[name=s,shape=and gate CDH,shape example, inner sep=0cm,
    logic gate inverted radius=.5cm] {And Gate\vrule width1pt height2cm};
  \foreach \anchor/\placement in
    {center/above, text/above, 30/above right,
     mid/right, mid east/left, mid west/above,
     base/below, base east/below, base west/left,
     north/above, south/below, east/above, west/above,
     north east/above, south east/below, south west/below, north west/above,
     output/right, input 1/above, input 2/below}
     \draw[shift=(s.\anchor)] plot[mark=x] coordinates{(0,0)}
       node[\placement] {\scriptsize\texttt{(s.\anchor)}};
\end{tikzpicture}
\end{codeexample}
\end{shape}

\begin{shape}{nand gate CDH}
  This shape is the alternative nand gate. It supports two or more inputs.
  If less than two inputs are specified an error will result. 
	The anchors for this gate with two
  non-inverted inputs (using the normal compass point anchors) are
  shown below. Anchor |30| is an example of a border anchor.
  
\begin{codeexample}[]
\Huge
\begin{tikzpicture}
  \node[name=s,shape=nand gate CDH,shape example, inner xsep=0cm,
    logic gate inverted radius=.5cm] {Nand Gate\vrule width1pt height2cm};
  \foreach \anchor/\placement in
    {center/above, text/above, 30/above right,
     mid/right, mid east/left, mid west/above,
     base/below, base east/below, base west/left,
     north/above, south/below, east/above, west/above,
     north east/above, south east/below, south west/below, north west/above,
     output/right, input 1/above, input 2/below}
     \draw[shift=(s.\anchor)] plot[mark=x] coordinates{(0,0)}
       node[\placement] {\scriptsize\texttt{(s.\anchor)}};
\end{tikzpicture}
\end{codeexample}

\end{shape}





\subsubsection{IEC Logic Gates}

\begin{pgflibrary}{shapes.gates.logic.IEC}
  This library provides rectangular logic gate shapes. These shapes
  are suffixed with |IEC| as they are based on gates recommended by
  the International Electrotechincal Commission.
\end{pgflibrary}

  In order to use these shapes in \tikzname{} without the |IEC|
  suffix, the following key is provided:
  
\begin{key}{/tikz/use IEC style logic gates}
	This allows the the shapes suffixed with |IEC| to be used without
	the suffix. So, for example, |and gate| becomes a synonym for
	|shape=and gate IEC|. In addtion the |IEC| specific keys can be
	used without |IEC|, so |and gate symbol| can be
	used for |and gate IEC symbol|.
\end{key}

  By default each gate is drawn with a symbol, $\char`\&$ for |and| and 
  |nand| gates, $\geq1$ for |or| and |nor| gates, $1$ for |not| and 
  |buffer| gates, and $=1$ for |xor| and |xnor| gates. These symbols 
  are drawn automatically (internally they are drawn using the 
  ``foreground'' path), and are not strictly speaking part of the node
  contents. However, the gate is enlarged to make sure the symbols are 
  within the border of the node.
  It is possible to change
  the symbols and their position within the node using the following
  keys:
  
\begin{key}{/pgf/and gate IEC symbol=\meta{text} (initially \char`\\char\char`\`\char`\\\char`\&)}
  Set the symbol for the |and gate|. Note that if the node is filled,
  this color will be used for the symbol, making it invisible, so
  it will be necessary set \meta{text} to something like
  |\color{black}\char`\&|. Alternatively, the 
  |logic gate IEC symbol color| key can be used to set the color
  of all symbols simultaneously.
  
  In \tikzname, when the |use IEC style logic gates| key has been 
  used, this key can be replaced by |and gate symbol|.
\end{key}

\begin{key}{/pgf/nand gate IEC symbol=\meta{text} (initially \char`\\char\char`\`\char`\\\char`\&)}
  Set the symbol for the |nand gate|.  
  In \tikzname, when the |use IEC style logic gates| key has been 
  used, this key can be replaced by |nand gate symbol|.
\end{key}

\begin{key}{/pgf/or gate IEC symbol=\meta{text} (initially \char`\$\char`\\geq1\char`\$)}
  Set the symbol for the |or gate|.  
  In \tikzname, when the |use IEC style logic gates| key has been 
  used, this key can be replaced by |or gate symbol|.
\end{key}

\begin{key}{/pgf/nor gate IEC symbol=\meta{text} (initially \char`\$\char`\\geq1\char`\$)}
  Set the symbol for the |nor gate|.  
  In \tikzname, when the |use IEC style logic gates| key has been 
  used, this key can be replaced by |nor gate symbol|.
\end{key}

\begin{key}{/pgf/xor gate IEC symbol=\meta{text} (initially \char`\{\char`\$=1\char`\$\char`\})}
  Set the symbol for the |xor gate|. Note the necessity for braces,
  as the symbol contains |=|.
  In \tikzname, when the |use IEC style logic gates| key has been 
  used, this key can be replaced by |or gate symbol|.
\end{key}

\begin{key}{/pgf/xnor gate IEC symbol=\meta{text} (initially  \char`\{\char`\$=1\char`\$\char`\})}
  Set the symbol for the |xnor gate|.  
  In \tikzname, when the |use IEC style logic gates| key has been 
  used, this key can be replaced by |xnor gate symbol|.
\end{key}

\begin{key}{/pgf/not gate IEC symbol=\meta{text} (initially 1)}
  Set the symbol for the |not gate|.  
  In \tikzname, when the |use IEC style logic gates| key has been 
  used, this key can be replaced by |not gate symbol|.
\end{key}

\begin{key}{/pgf/buffer gate IEC symbol=\meta{text} (initially 1)}
  Set the symbol for the |buffer gate|.  
  In \tikzname, when the |use IEC style logic gates| key has been 
  used, this key can be replaced by |buffer gate symbol|.
\end{key}

\begin{key}{/pgf/logic gate IEC symbol align=\meta{align} (initially top)}
  Set the alignment of the logic gate symbol (in \tikzname, when the 
  |use IEC style logic gates| key has been used, |IEC| can be omitted.
  The specification in \meta{align} is a comma separated list from
  |top|, |bottom|, |left| or |right|. The distance between the border
  of the node and the outer edge of the symbol is determined by the values 
  of the |inner xsep| and |inner ysep|.
  
\begin{codeexample}[]
\begin{tikzpicture}[minimum size=1cm, use IEC style logic gates]
	\tikzset{every node/.style={nor gate, draw}}
  \node (A) at (0,1.5) {};
  \node [logic gate symbol align={bottom, right}] (B) at (0,0) {}; 
  \foreach \g in {A, B}{
    \foreach \i in {1,2}
      \draw ([xshift=-0.5cm]\g.input \i) -- (\g.input \i);
    \draw (\g.output) -- ([xshift=0.5cm]\g.output);
  }
\end{tikzpicture}
\end{codeexample} 

\end{key}


\begin{key}{/pgf/logic gate IEC symbol color=\meta{color}}
  This key sets the color for all symbols simultaneously. This color
  can be overridden on a case by case basis by specifying a color
  when seting the symbol text.
\end{key}


\begin{shape}{and gate IEC}
  This shape is an and gate. It supports two or more inputs.
  If less than two inputs are specified an error will result. 
	The anchors for this gate with two
  non-inverted inputs are
  shown below. Anchor |30| is an example of a border anchor.
  
\begin{codeexample}[]
\Huge
\begin{tikzpicture}
  \node[name=s,shape=and gate IEC ,shape example, inner xsep=1cm, inner ysep=1cm,
    minimum height=6cm, and gate IEC symbol=\color{black!30}\char`\&] 
  {And Gate\vrule width1pt height2cm};
  \foreach \anchor/\placement in
    {center/above, text/above, 30/above right,
     mid/right, mid east/left, mid west/above,
     base/below, base east/below, base west/left,
     north/above, south/below, east/above, west/above,
     north east/above, south east/below, south west/below, north west/above,
     output/right, input 1/above, input 2/below}
     \draw[shift=(s.\anchor)] plot[mark=x] coordinates{(0,0)}
       node[\placement] {\scriptsize\texttt{(s.\anchor)}};
\end{tikzpicture}
\end{codeexample}
\end{shape}


\begin{shape}{nand gate IEC}
  This shape is a nand gate. It supports two or more inputs.
  If less than two inputs are specified an error will result. 
	The anchors for this gate with two
  non-inverted inputs are
  shown below. Anchor |30| is an example of a border anchor.
  
\begin{codeexample}[]
\Huge
\begin{tikzpicture}
  \node[name=s,shape=nand gate IEC ,shape example, inner xsep=1cm, inner ysep=1cm,
    minimum height=6cm, nand gate IEC symbol=\color{black!30}\char`\&,
    logic gate inverted radius=0.65cm] 
  {Nand Gate\vrule width1pt height2cm};
  \foreach \anchor/\placement in
    {center/above, text/above, 30/above right,
     mid/right, mid east/left, mid west/above,
     base/below, base east/below, base west/left,
     north/above, south/below, east/above, west/above,
     north east/above, south east/below, south west/below, north west/above,
     output/right, input 1/above, input 2/below}
     \draw[shift=(s.\anchor)] plot[mark=x] coordinates{(0,0)}
       node[\placement] {\scriptsize\texttt{(s.\anchor)}};
\end{tikzpicture}
\end{codeexample}
\end{shape}

\begin{shape}{or gate IEC}
  This shape is an or gate. It supports two or more inputs.
  If less than two inputs are specified an error will result. 
	See the |and gate IEC| shape for the anchors.
	
\begin{codeexample}[]
\begin{tikzpicture}[minimum width=.875cm, minimum height=1cm]
  \node[or gate IEC, draw, logic gate inputs=in] (A) {};
  \draw (A.input 1 -| -1,0) -- (A.input 1) (A.input 2 -| -1,0) -- (A.input 2)
        (A.output) -- ([xshift=0.5cm]A.output);
\end{tikzpicture}
\end{codeexample} 

\end{shape}


\begin{shape}{nor gate IEC}
  This shape is an nor gate. It supports two or more inputs.
  If less than two inputs are specified an error will result. 
	See the |nand gate IEC| shape for the anchors.
	
\begin{codeexample}[]
\begin{tikzpicture}[minimum width=.875cm, minimum height=1cm]
  \node[nor gate IEC, draw, logic gate inputs=in] (A) {};
  \draw (A.input 1 -| -1,0) -- (A.input 1) (A.input 2 -| -1,0) -- (A.input 2)
        (A.output) -- ([xshift=0.5cm]A.output);
\end{tikzpicture}
\end{codeexample}

\end{shape}

\begin{shape}{xor gate IEC}
  This shape is an xor gate. It supports only two inputs.
   If less than two inputs are specified an error will result.
  Any extra inputs are ignored.  
		See the |and gate IEC| shape for the anchors.
	
\begin{codeexample}[]
\begin{tikzpicture}[minimum width=.875cm, minimum height=1cm]
  \node[xor gate IEC, draw, logic gate inputs=in] (A) {};
  \draw (A.input 1 -| -1,0) -- (A.input 1) (A.input 2 -| -1,0) -- (A.input 2)
        (A.output) -- ([xshift=0.5cm]A.output);
\end{tikzpicture}
\end{codeexample}

\end{shape}


\begin{shape}{xnor gate IEC}
  This shape is an xnor gate. It supports only two inputs.
  If less than two inputs are specified an error will result.
  Any extra inputs are ignored.  
	See the |nand gate IEC| shape for the anchors.
	
\begin{codeexample}[]
\begin{tikzpicture}[minimum width=.875cm, minimum height=1cm]
  \node[xnor gate IEC, draw, logic gate inputs=in] (A) {};
  \draw (A.input 1 -| -1,0) -- (A.input 1) (A.input 2 -| -1,0) -- (A.input 2)
        (A.output) -- ([xshift=0.5cm]A.output);
\end{tikzpicture}
\end{codeexample}

\end{shape}

\begin{shape}{buffer gate IEC}
   This shape is a buffer gate. It supports only one input.
  If less than one input is specified an error will result.
  Any extra inputs are ignored. 
	See the |and gate IEC| shape for the anchors.

\begin{codeexample}[]
\begin{tikzpicture}[minimum width=.875cm, minimum height=1cm]
  \node[buffer gate IEC, draw] (A) {};
  \draw (A.input -| -1,0) -- (A.input) (A.output) -- ([xshift=0.5cm]A.output);
\end{tikzpicture}
\end{codeexample}

\end{shape}


\begin{shape}{not gate IEC}
  This shape is a not gate. It supports only one input.
  If less than one input is specified an error will result.
  Any extra inputs are ignored. 
  See the |nand gate IEC| shape for the anchors.

\begin{codeexample}[]
\begin{tikzpicture}[minimum width=.875cm, minimum height=1cm]
  \node[not gate IEC, draw] (A) {};
  \draw (A.input -| -1,0) -- (A.input) (A.output) -- ([xshift=0.5cm]A.output);
\end{tikzpicture}
\end{codeexample}


\end{shape}






\subsection{Miscellaneous Shapes}

\begin{pgflibrary}{shapes.misc}
  This library defines general-purpose shapes that do not fit in the
  previous categories.
\end{pgflibrary}

\begin{shape}{cross out}
  This shape ``crosses out'' the node. Its foreground path are simply
  two diagonal lines that between the corners of the node's bounding
  box. Here is an example:  
\begin{codeexample}[]
\begin{tikzpicture} 
  \draw [help lines] (0,0) grid (3,2); 
  \node [cross out,draw=red] at (1.5,1) {cross out}; 
\end{tikzpicture}
\end{codeexample}

  A useful application is inside text as in the following example:
\begin{codeexample}[]
Cross \tikz[baseline] \node [cross out,draw,anchor=text] {me}; out!  
\end{codeexample}

  This shape inherits all anchors from the |rectangle| shape, see also
  the following figure:
\begin{codeexample}[]
\Huge
\begin{tikzpicture}
  \node[name=s,shape=cross out,shape example] {cross out\vrule width 1pt height 2cm};
  \foreach \anchor/\placement in
    {north west/above left, north/above, north east/above right, 
     west/left, center/above, east/right, 
     mid west/right, mid/above, mid east/left, 
     base west/left, base/below, base east/right, 
     south west/below left, south/below, south east/below right, 
     text/left, 10/right, 130/above}
     \draw[shift=(s.\anchor)] plot[mark=x] coordinates{(0,0)}
       node[\placement] {\scriptsize\texttt{(s.\anchor)}};
\end{tikzpicture}
\end{codeexample}
\end{shape}


\begin{shape}{cross out}
  This shape ``crosses out'' the node. Its foreground path are simply
  two diagonal lines that between the corners of the node's bounding
  box. Here is an example:

\begin{codeexample}[]
\begin{tikzpicture}
  \draw[help lines] (0,0) grid (3,2);
  \node [cross out,draw=red] at (1.5,1) {cross out};
\end{tikzpicture}
\end{codeexample}

  A useful application is inside text as in the following example:
\begin{codeexample}[]
Cross \tikz[baseline] \node [cross out,draw,anchor=text] {me}; out!  
\end{codeexample}

  This shape inherits all anchors from the |rectangle| shape, see also
  the following figure:
\begin{codeexample}[]
\Huge
\begin{tikzpicture}
  \node[name=s,shape=cross out,shape example] {cross out\vrule width 1pt height 2cm};
  \foreach \anchor/\placement in
    {north west/above left, north/above, north east/above right, 
     west/left, center/above, east/right, 
     mid west/right, mid/above, mid east/left, 
     base west/left, base/below, base east/right, 
     south west/below left, south/below, south east/below right, 
     text/left, 10/right, 130/above}
     \draw[shift=(s.\anchor)] plot[mark=x] coordinates{(0,0)}
       node[\placement] {\scriptsize\texttt{(s.\anchor)}};
\end{tikzpicture}
\end{codeexample}
\end{shape}

\begin{shape}{strike out}
  This shape is idential to the |cross out| shape, only its foreground
  path consists of a single line from the lower left to the upper
  right.
  
\begin{codeexample}[]
Strike \tikz[baseline] \node [strike out,draw,anchor=text] {me}; out!  
\end{codeexample}

  See the |cross out| shape for the anchors.
\end{shape}




\begin{shape}{rounded rectangle}
	This shape is a rectangle which can be optionally round sides.

\begin{codeexample}[]
\begin{tikzpicture}
  \node[rounded rectangle, draw, fill=red!20]{Hallo};
\end{tikzpicture}
\end{codeexample}

	There are keys to specify how the sides are rounded (to use
	these keys in \tikzname, simply remove the \declare{|/pgf/|} path).


\begin{key}{/pgf/rounded rectangle arc length=\meta{angle} (initially 180)}

	Set the length of the arcs for the rounded ends. Recommended values 
	for	\meta{angle} are between |90| and |180|. 
	
\begin{codeexample}[]
\begin{tikzpicture}
  \matrix[row sep=5pt, every node/.style={draw, rounded rectangle}]{
    \node[rounded rectangle arc length=180] {180}; \\
    \node[rounded rectangle arc length=120] {120}; \\
    \node[rounded rectangle arc length=90]  {90};  \\};
\end{tikzpicture} 
\end{codeexample}

\end{key}

\begin{key}{/pgf/rounded rectangle west arc=\meta{arc type} (initially convex)}
	Set the style of the rounding for the left side. The permitted values
	for \meta{arc type} are |concave|, |convex|, or |none|.

\begin{codeexample}[]
\begin{tikzpicture}
  \matrix[row sep=5pt, every node/.style={draw, rounded rectangle}]{
  	\node[rounded rectangle west arc=concave] {Concave}; \\
  	\node[rounded rectangle west arc=convex]  {Convex};  \\
  	\node[rounded rectangle left arc=none]    {None};    \\};
\end{tikzpicture} 
\end{codeexample}
\end{key}

\begin{stylekey}{/pgf/rounded rectangle left arc=\meta{arc type}}
	Alternative key for specifying the west arc.
\end{stylekey}

\begin{key}{/pgf/rounded rectangle east arc=\meta{arc type} (initially convex)}
	Set the style of the rounding for the east side.
\end{key}

\begin{stylekey}{/pgf/rounded rectangle right arc=\meta{arc type}}
	Alternative key for specifying the east arc.
\end{stylekey}

	The anchors for this shape are shown below (anchor |10| is an example
	of a border angle). Note that if only one side is rounded, the 
	|center| anchor will not be the precise center of the shape.
	
\begin{codeexample}[]
\Huge
\begin{tikzpicture}
  \node[name=s,shape=rounded rectangle, shape example, inner xsep=1.5cm, inner ysep=1cm] 
  	{Rounded Rectangle\vrule width 1pt height 2cm};
  \foreach \anchor/\placement in
    {center/above, text/below,      10/above,
     mid/above,    mid west/right,  mid east/left, 
     base/below,   base west/below, base east/below, 
     north/above,  south/below, east/above, west/above, 
     north west/above left,    north east/above right,     
     south west/below left,    south east/below right}
     \draw[shift=(s.\anchor)] plot[mark=x] coordinates{(0,0)}
       node[\placement] {\scriptsize\texttt{(s.\anchor)}};
\end{tikzpicture}
\end{codeexample}

\end{shape}


\begin{shape}{chamfered rectangle}

	This shape is a rectangle with optionally chamfered corners.
	
\begin{codeexample}[]
\begin{tikzpicture}
  \node[chamfered rectangle, white, fill=red, double=red, draw, very thick]
    {\bf STOP!};
\end{tikzpicture}
\end{codeexample}

	There are \pgfname{} keys to specify how this shape is drawn (to use
	these keys in \tikzname{} simply remove the \declare{|/pgf/|} path).

\begin{key}{/pgf/chamfered rectangle angle=\meta{angle} (initially 45)}
	Set the angle \emph{from the vertical} for the chamfer.
	
\begin{codeexample}[]
\begin{tikzpicture}
  \tikzset{every node/.style={chamfered rectangle, draw}}
  \node[chamfered rectangle angle=30] {abc};
  \node[chamfered rectangle angle=60] at (1.5,0) {123};
\end{tikzpicture}
\end{codeexample}
\end{key}

\begin{key}{/pgf/chamfered rectangle xsep=\meta{length} (initially .666ex)}
	Set the distance that the chamfer extends horizontally beyond the node 
	contents (which includes the |inner sep|). 
	If \meta{length} is large, such
	that the top and bottom chamfered edges would cross, then 
	\meta{length} is ignored and the chamfered edges are drawn so that
	they meet in the middle.

\begin{codeexample}[]
\begin{tikzpicture}
  \tikzset{every node/.style={chamfered rectangle, draw}}
  \node[chamfered rectangle xsep=2pt] {def};
  \node[chamfered rectangle xsep=2cm] at (1.5,0) {456};
\end{tikzpicture}
\end{codeexample}
	
\end{key}

\begin{key}{/pgf/chamfered rectangle ysep=\meta{length} (initially .666ex)}
	Set the distance that the chamfer extends vertically beyond the node 
	contents. 
	If \meta{length} is large, such that the left and right chamfered 
	edges would cross, then \meta{length} is ignored and the chamfered 
	edges are drawn so that	they meet in the middle.
\end{key}

\begin{key}{/pgf/chamfered rectangle sep=\meta{length} (initially .666ex)}
	Set both the |xsep| and |ysep| simultaneously.
\end{key}

\begin{key}{/pgf/chamfered rectangle corners=\meta{list} (initially chamfer all)}
	Specify which corners are chamfered. The corners are identified by 
	their ``compass point'' directions (i.e. |north east|, |north west|,
	|south west|, and |south east|), and must be separated by commas (so
	if there is more than one corner in the list, it must be surrounded
	by braces). Any corners not mentioned in 
	\meta{list} are automatically not chamfered. Two additional values
	|chamfer all| and |chamfer none|, are also permitted.

\begin{codeexample}[]
\begin{tikzpicture}
  \tikzset{every node/.style={chamfered rectangle, draw}}
  \node[chamfered rectangle corners=north west] {ghi};
  \node[chamfered rectangle corners={north east, south east}] at (1.5,0) {789};
\end{tikzpicture}
\end{codeexample}
\end{key}


	The anchors for this shape are shown below (anchor |60| is an example
	of a border angle.
	
\begin{codeexample}[]
\Huge
\begin{tikzpicture}
  \node[name=s,shape=chamfered rectangle, chamfered rectangle sep=1cm,
        shape example, inner ysep=1cm, inner xsep=.75cm] 
    {Chamfered Rectangle\vrule width1pt height2cm};
  \foreach \anchor/\placement in
    {text/right, center/above,    70/above, 
     base/below, base east/left, base west/right,
     mid/right,  mid east/above,   mid west/above,  
     north/above, south/below, east/above, west/above,
     before north east/above, north east/above, after north east/above,
     before north west/above, north west/above, after north west/above,
     before south west/below, south west/below, after south west/below,
     before south east/below, south east/below, after south east/below}     
     \draw[shift=(s.\anchor)] plot[mark=x] coordinates{(0,0)}
       node[\placement] {\scriptsize\texttt{(s.\anchor)}};
\end{tikzpicture}
\end{codeexample}

\end{shape}



%%% Local Variables: 
%%% mode: latex
%%% TeX-master: "pgfmanual-pdftex-version"
%%% End: 

%
���}	� �
����DEV��INO��SYN��SV~�pgfmanual-en-library-3d.tex�J��J���J�j��P�_.j�
% Copyright 2006 by Till Tantau
%
% This file may be distributed and/or modified
%
% 1. under the LaTeX Project Public License and/or
% 2. under the GNU Free Documentation License.
%
% See the file doc/generic/pgf/licenses/LICENSE for more details.

\section{To Path Library}

\label{library-to-paths}

\begin{tikzlibrary}{topaths}
  This library provides predefined to paths for use with the |to|
  path operation. After loading this package, you can say for instance
  |to [loop]| to add a loop to a node.

  This library is loaded automatically by \tikzname, so you do not
  need to load it yourself.
\end{tikzlibrary}


\subsection{Straight Lines}

The following style installs a to path that is simply a straight line
from the start coordinate to the target coordinate.

\begin{key}{/tikz/line to}
  Causes a straight line to be added to the path upon a |to| or an
  |edge| operation.
\begin{codeexample}[]
\tikz {\draw (0,0) to[line to] (1,0);}
\end{codeexample}
\end{key}


\subsection{Curves}

The |curve to| style causes the to path to be set to a curve. The
exact way this curve looks can be influenced via a number of options.

\begin{key}{/tikz/curve to}
  Specifies that the |to path| should be a curve. This curve will
  leave the start coordinate at a certain angle, which can be
  specified using the |out| option. It reaches the target coordinate
  also at a certain angle, which is specified using the |in|
  option. The control points of the curve are at a certain distance
  that is computed in different ways, depending on which options are
  set.

  All of the following options implictly cause the |curve to| style to
  be installed.

  \begin{key}{/tikz/out=\meta{angle}}
    The angle at which the curve leaves the start coordinate. If the
    start coordinate is a node, the start coordinate is the point on the
    border of the node at the given \meta{angle}. The control point
    will, thus, lie at a certain distance in the direction \meta{angle}
    from the start coordinate.
\begin{codeexample}[]
\begin{tikzpicture}[out=45,in=135]
  \draw (0,0) to (1,0)
        (0,0) to (2,0)
        (0,0) to (3,0);
\end{tikzpicture}
\end{codeexample}
  \end{key}
  \begin{key}{/tikz/in=\meta{angle}}
    The angle at which the curve reaches the target coordinate.
  \end{key}

  \begin{key}{/tikz/relative=\meta{true or false} (default true)}
    This option tells \tikzname\ whether the |in| and |out| angles
    should be considered absolute or relative. Absolute means that an
    |out| angle of 30$^\circ$ means that the curve leaves the start
    coordinate at an angle of 30$^\circ$ relative to the paper (unless,
    of course, further transformations have been installed). A
    \emph{relative} angle is, by comparison, measured relative to a
    straight line from the start coordinate to the target
    coordinate. Thus, a relative angle of 30$^\circ$ means that the
    curve will bend to the left from the line going straight from the
    start to the target. For the target, the relative coordinate is
    measured in the same manner, namely relative to the line going from
    the start to the target. Thus, an angle of 150$^\circ$ means that
    the curve will reach target coming slightly from the left.

\begin{codeexample}[]
\begin{tikzpicture}[out=45,in=135,relative]
  \draw (0,0) to (1,0)
              to (2,1)
              to (2,2);
\end{tikzpicture}
\end{codeexample}

\begin{codeexample}[]
\begin{tikzpicture}[out=90,in=90,relative]
  \node [circle,draw] (a) at (0,0) {a};
  \node [circle,draw] (b) at (1,1) {b};
  \node [circle,draw] (c) at (2,2) {c};

  \path (a) edge (b)
            edge (c);
\end{tikzpicture}
\end{codeexample}
  \end{key}

  \begin{key}{/tikz/bend left=\meta{angle} (default \normalfont last value)}
    This option sets |out=|\meta{angle}|,in=|$180-\meta{angle}$|,relative|. If no
    \meta{angle} is given, the last given |bend left| or |bend right|
    angle is used.  
  
\begin{codeexample}[]
\begin{tikzpicture}[shorten >=1pt,node distance=2cm,on grid]
  \node[state,initial]  (q_0)                {$q_0$};
  \node[state]          (q_1) [right=of q_0] {$q_1$};
  \node[state,accepting](q_2) [right=of q_1] {$q_2$};

  \path[->] (q_0) edge              node [above]  {0} (q_1)
                  edge [loop above] node          {1} ()
                  edge [bend left]  node [above]  {1} (q_2)
                  edge [bend right] node [below]  {0} (q_2)
            (q_1) edge              node [above]  {1} (q_2);
\end{tikzpicture}
\end{codeexample}

\begin{codeexample}[]
\begin{tikzpicture}
  \foreach \angle in {0,45,...,315}
    \node[rectangle,draw=black!50] (\angle) at (\angle:2) {\angle};

  \foreach \from/\to in {0/45,45/90,90/135,135/180,
                         180/225,225/270,270/315,315/0}
    \path (\from) edge [->,bend right=22,looseness=0.8] (\to)
                  edge [<-,bend left=22,looseness=0.8] (\to);
\end{tikzpicture}
\end{codeexample}
  \end{key}

  \begin{key}{/tikz/bend right=\meta{angle} (default \normalfont last  value)}
    Works like the |bend left| option, only the bend is to the other side.
  \end{key}

  \begin{key}{/tikz/bend angle=\meta{angle}}
    Sets the angle to be used by the |bend left| or |bend right|, but
    without actually selecting the |curve to| or the |relative|
    option. This is useful for globally specifying a |bend angle| for a
    whole picture.
  \end{key}

  \begin{key}{/tikz/looseness=\meta{number} (initially 1)}
    This number specifies how ``loose'' the curve will be. In detail,
    the following happens: \tikzname\ computes the distance between the
    start and the target coordinate (if the start and/or target
    coordinate are nodes, the distance is computed between the points on
    their border). This distance is then multiplied by a fixed factor
    and also by the factor \meta{number}. The resulting distance, let us
    call it $d$, is then used as the distance of the control points from
    the start and target coordinates.

    The fixed factor has been chosen in such a way that if \meta{number}
    is |1|, if the |in| and |out| angles differ by
    90$\circ$, then a quarter circle results:
  \begin{codeexample}[]
\tikz \draw (0,0) to [out=0,in=-90]               (1,1);
\tikz \draw (0,0) to [out=0,in=-90,looseness=0.5] (1,1);
  \end{codeexample}
  \end{key}

  \begin{key}{/tikz/out looseness=\meta{number}}
    specifies the looseness factor for the out distance only.
  \end{key}

  \begin{key}{/tikz/in looseness=\meta{number}}
    specifies the looseness factor for the in distance only.
  \end{key}
  \begin{key}{/tikz/min distance=\meta{distance}}
    If the computed distance for the start and target coordinates are
    below \meta{distance}, then \meta{distance} is used instead.
  \end{key}
  \begin{key}{/tikz/max distance=\meta{distance}}
    If the computed distance for the start and target coordinates are
    above \meta{distance}, then \meta{distance} is used instead.
  \end{key}
  \begin{key}{/tikz/out min distance=\meta{distance}}
    The mininimum distance set only for the start coordinate.
  \end{key}
  \begin{key}{/tikz/out max distance=\meta{distance}}
    The maximum distance set only for the start coordinate.
  \end{key}
  \begin{key}{/tikz/in min distance=\meta{distance}}
    The min distance set only for the target coordinate.
  \end{key}
  \begin{key}{/tikz/in max distance=\meta{distance}}
    The max distance set only for the target coordinate.
  \end{key}
  \begin{key}{/tikz/distance=\meta{distance}}
    Set the min and max distance to the same value \meta{distance}. Note
    that this causes any computed distance $d$ to be ignored and
    \meta{distance} to be used instead.
\begin{codeexample}[]
\begin{tikzpicture}[out=45,in=135,distance=1cm]
  \draw (0,0) to (1,0)
        (0,0) to (2,0)
        (0,0) to (3,0);
\end{tikzpicture}
\end{codeexample}
  \end{key}
  \begin{key}{/tikz/out distance=\meta{distance}}
    Sets the min and max out distance.
  \end{key}
  \begin{key}{/tikz/in distance=\meta{distance}}
    Sets the min and max in distance.
  \end{key}
  \begin{key}{/tikz/out control=\meta{coordinate}}
    This option causes the \meta{coordinate} to be used as the start
    control point. All computations of $d$ are ignored. You can use a
    coordinate like |+(1,0)| to specify a point relative to the start
    coordinate.
  \end{key}
  \begin{key}{/tikz/in control=\meta{coordinate}}
    This option causes the \meta{coordinate} to be used as the target
    control point.
  \end{key}
  \begin{key}{/tikz/controls=\meta{coordinate}| and |\meta{coordinate}}
    This option causes the \meta{coordinate}s to be used as control
    points. 
\begin{codeexample}[]
\tikz \draw (0,0) to [controls=+(90:1) and +(90:1)] (3,0);
\end{codeexample}
  \end{key}
\end{key}


\subsection{Loops}

\begin{key}{/tikz/loop}
  This key is similar to the |curve to| key, but differs in the
  following ways: First, the actual target coordinate is ignored and the
  start coordiante is used as the target coordinate. Thus, it is
  allowed not to provide any target coordinate, which can be useful
  with unnamed nodes. Second, the |looseness| is set to |8| and the
  |min distance| to |5mm|. These settings result in rather nice loops
  when the opening angle (difference between |in| and |out|) is
  30$^\circ$.
\begin{codeexample}[]
\begin{tikzpicture}
  \node [circle,draw] {a} edge [in=30,out=60,loop] ();    
\end{tikzpicture}
\end{codeexample}
\end{key}

\begin{stylekey}{/tikz/loop above}
  Sets the |loop| style and sets in and out angles such that
  loop is above the node. Furthermore, the |above| option is set,
  which causes a node label to be placed at the correct position. 
\begin{codeexample}[]
\begin{tikzpicture}
  \node [circle,draw] {a} edge [loop above] node {x} ();    
\end{tikzpicture}
\end{codeexample}
\end{stylekey}
\begin{stylekey}{/tikz/loop below} Works like the previous option. \end{stylekey}
\begin{stylekey}{/tikz/loop left} Works like the previous option. \end{stylekey}
\begin{stylekey}{/tikz/loop right} Works like the previous option. \end{stylekey}
\begin{stylekey}{/tikz/every loop (initially {->,shorten >=1pt})}
  This style is installed at the beginning of
  every loop.
\begin{codeexample}[]
\begin{tikzpicture}[every loop/.style={}]
  \draw (0,0) to [loop above] () to [loop right] ()
              to [loop below] () to [loop left]  ();
\end{tikzpicture}
\end{codeexample}
\end{stylekey}



%%% Local Variables: 
%%% mode: latex
%%% TeX-master: "pgfmanual-pdftex-version"
%%% End: 


�� �}	� �
����DEV��INO��SYN��SV~�pgfmanual-en-library-through.tex�J��Jŀ�J�j��6P�_.j�

���}	� �
����DEV��INO��SYN��SV~�pgfmanual-en-library-trees.tex�J��Jƀ�J�j��8P�_.j�




\part{Utilities}
\label{part-utilities}

{\Large \emph{by Till Tantau}}


\bigskip
\noindent
The utility packages are not directly involved in creating graphics,
but you may find them useful nonetheless. All of them either directly
depend on \pgfname\ or they are designed to work well together with
\pgfname\ even though they can be used in a stand-alone way.

\vskip2cm
\medskip
\noindent
\begin{codeexample}[graphic=white]
\begin{tikzpicture}[scale=2]
  \shade[top color=blue,bottom color=gray!50] (0,0) parabola (1.5,2.25) |- (0,0);
  \draw (1.05cm,2pt) node[above] {$\displaystyle\int_0^{3/2} \!\!x^2\mathrm{d}x$};
  
  \draw[help lines] (0,0) grid (3.9,3.9)
       [step=0.25cm]      (1,2) grid +(1,1);

  \draw[->] (-0.2,0) -- (4,0) node[right] {$x$};
  \draw[->] (0,-0.2) -- (0,4) node[above] {$f(x)$};

  \foreach \x/\xtext in {1/1, 1.5/1\frac{1}{2}, 2/2, 3/3}
    \draw[shift={(\x,0)}] (0pt,2pt) -- (0pt,-2pt) node[below] {$\xtext$};

  \foreach \y/\ytext in {1/1, 2/2, 2.25/2\frac{1}{4}, 3/3}
    \draw[shift={(0,\y)}] (2pt,0pt) -- (-2pt,0pt) node[left] {$\ytext$};
    
  \draw (-.5,.25) parabola bend (0,0) (2,4) node[below right] {$x^2$};
\end{tikzpicture}
\end{codeexample}


���}	� �
����DEV��INO��SYN��SV~�pgfmanual-en-pgfkeys.tex�J��J̀�J�j��LP�_.j�

���}	� �
����DEV��INO��SYN��SV~�pgfmanual-en-pgffor.tex�J��Jˀ�J�j��JP�_.j�

���}	� �
����DEV��INO��SYN��SV~�pgfmanual-en-pgfcalendar.tex�J��Jˀ�J�j��HP�_.j�

���}	� �
����DEV��INO��SYN��SV~�pgfmanual-en-pages.tex�J��Jʀ�J�j��FP�_.j�

���}	� �
����DEV��INO��SYN��SV~�pgfmanual-en-xxcolor.tex�J��J؀�J�j��vP�_.j�


\part{Mathematical Engine}

{\Large \emph{by Mark Wibrow and Till Tantau}}


\bigskip
\noindent
\pgfname\ comes with its own mathematical engine. The job of this
engine is to support mathematical operations like addition,
subtraction, multiplication and division, using both integers and 
non-integers, but also functions such as square-roots, sine, cosine,
and generate pseudo-random numbers.

Mostly, you will use the mathematical facilities of \pgfname\
indirectly, namely when you write a coordinate like |(5cm*3,6cm/4)|,
but the mathematical engine can also be used independently of
\pgfname\ and \tikzname. 

\vskip1cm
\begin{codeexample}[graphic=white]
\pgfmathsetseed{1}
\foreach \col in {black,red,green,blue}
{
  \begin{tikzpicture}[x=10pt,y=10pt,ultra thick,baseline,line cap=round]
    \coordinate (current point) at (0,0);
    \coordinate (old velocity) at (0,0);
    \coordinate (new velocity) at (rand,rand);
    
    \foreach \i in {0,1,...,100}
    {
      \draw[\col!\i] (current point)
      .. controls ++([scale=-1]old velocity) and
                  ++(new velocity) .. ++(rand,rand)
         coordinate (current point);
      \coordinate (old velocity) at (new velocity);
      \coordinate (new velocity) at (rand,rand);
    }
  \end{tikzpicture}
}
\end{codeexample}


���}	� �
����DEV��INO��SYN��SV~�pgfmanual-en-math-design.tex�J��Jɀ�J�j��BP�_.j�

���}	� �
����DEV��INO��SYN��SV~�pgfmanual-en-math-parsing.tex�J��Jɀ�J�j��DP�_.j�
% Copyright 2007 by Mark Wibrow
%
% This file may be distributed and/or modified
%
% 1. under the LaTeX Project Public License and/or
% 2. under the GNU Free Documentation License.
%
% See the file doc/generic/pgf/licenses/LICENSE for more details.

\section{Evaluating Mathematical Operations}

\label{pgfmath-commands}

Instead of parsing and evaluating complex expressions, you can also
use the mathematical engine to evaluate a single mathematical
operation. The macros used for these computations are described in the
following. 


\subsection{Basic Operations and Functions}

\label{pgfmath-operations}

\begin{command}{\pgfmathadd\marg{x}\marg{y}}  
	Defines |\pgfmathresult| as $\meta{x}+\meta{y}$.
\end{command}

\begin{command}{\pgfmathsubtract\marg{x}\marg{y}}      
	Defines |\pgfmathresult| as $\meta{x}-\meta{y}$.                                       
\end{command}

\begin{command}{\pgfmathmultiply\marg{x}\marg{y}}      
	Defines |\pgfmathresult| as $\meta{x}\times\meta{y}$.                                
\end{command}

\begin{command}{\pgfmathdivide\marg{x}\marg{y}}        
	Defines |\pgfmathresult| as $\meta{x}\div\meta{y}$. An error will
	result if \meta{y} is	|0|, or if the result of the division is
	too big for the mathematical engine.
	Please remember	when using this command that accurate (and reasonably 
	quick) division of non-integers is particularly tricky in \TeX{}. 	
	There are three different forms of division used in this command:
	\begin{itemize}
		\item 
		If \meta{y} is an integer then the native |\divide| operation of 
		\TeX{} is used.
		\item
		If \vrule\meta{y}\vrule$<1$, then |\pgfmathreciprocal| is employed.
		\item
		For all other values of \meta{y} an optimised long division 
		algorithm is used. In theory this should be accurate
		to any finite precision, but in practice it is constrained by the
		limits of \TeX{}'s native mathematics.
	\end{itemize}
	                             
\end{command}

\begin{command}{\pgfmathreciprocal\marg{x}}         
	Defines |\pgfmathresult| as $1\div\meta{x}$.                            
\end{command}

\begin{command}{\pgfmathgreaterthan\marg{x}\marg{y}}   
	Defines |\pgfmathresult| as 1.0 if \meta{x} $>$ \meta{y}, but 0.0 otherwise.                 
\end{command}

\begin{command}{\pgfmathlessthan\marg{x}\marg{y}} 
	Defines |\pgfmathresult| as 1.0 if \meta{x} $<$ \meta{y}, but 0.0 otherwise.             
\end{command}
	
\begin{command}{\pgfmathequalto\marg{x}\marg{y}}       
	Defines |\pgfmathresult| 1.0 if \meta{x} $=$ \meta{y}, but 0.0 otherwise.                    
\end{command}

\begin{command}{\pgfmathround\marg{x}}              
	Defines |\pgfmathresult| as $\left\lfloor\textrm{\meta{x}}\right\rceil$.	
	This uses asymmetric	half-up rounding.                          
\end{command}

\begin{command}{\pgfmathfloor\marg{x}}              
	Defines |\pgfmathresult| as $\left\lfloor\textrm{\meta{x}}\right\rfloor$.
\end{command}

\begin{command}{\pgfmathceil\marg{x}}               
	Defines |\pgfmathresult| as $\left\lceil\textrm{\meta{x}}\right\rceil$.                           
\end{command}
	
\begin{command}{\pgfmathpow\marg{x}\marg{y}}         
	Defines |\pgfmathresult| as $\meta{x}^{\meta{y}}$.  For greatest 
	accuracy \mvar{y} should be an integer. If \mvar{y} is not an integer 
	the actual calculation will be an approximation of $e^{y\ln(x)}$.
\end{command}

\begin{command}{\pgfmathmod\marg{x}\marg{y}}           
	Defines |\pgfmathresult| as \meta{x} modulo \meta{y}.                       
\end{command}

\begin{command}{\pgfmathmax\marg{x}\marg{y}}           
	Defines |\pgfmathresult| as the maximum of \meta{x} or \meta{y}.                       
\end{command}

\begin{command}{\pgfmathmin\marg{x}\marg{y}}           
	Defines |\pgfmathresult| as the minimum \meta{x} or \meta{y}.                       
\end{command}
	
\begin{command}{\pgfmathabs\marg{x}}                
	Defines |\pgfmathresult| as  absolute value of \meta{x}.                                 
\end{command}
	
\begin{command}{\pgfmathexp\marg{x}}                
	Defines |\pgfmathresult| as $e^{\meta{x}}$. Here, \meta{x} can be a 
	non-integer. The algorithm	uses a Maclaurin series.               
\end{command}

\begin{command}{\pgfmathln\marg{x}}                
	Defines |\pgfmathresult| as $\ln{\meta{x}}$. This uses an algorithm
	due to Rouben Rostamian, and coefficients suggested by
	Alain Matthes.             
\end{command}
	
\begin{command}{\pgfmathsqrt\marg{x}} 
	Defines |\pgfmathresult| as $\sqrt{\meta{x}}$. 
\end{command}
	
\begin{command}{\pgfmathveclen\marg{x}\marg{y}}        
	Defines |\pgfmathresult| as $\sqrt{\meta{x}^2+\meta{y}^2}$. This uses
	a polynomial approximation, based on ideas due to Rouben Rostamian.                                    
\end{command}

\subsection{Trignometric Functions}

\label{pgfmath-trigonmetry}

\begin{command}{\pgfmathpi}
  	Defines |\pgfmathresult| as $3.14159$.
\end{command}
   
\begin{command}{\pgfmathdeg{\marg{x}}} 
	Defines |\pgfmathresult| as \meta{x} (given in radians) converted to 
	degrees. 
\end{command}

\begin{command}{\pgfmathrad{\marg{x}}} 
	Defines |\pgfmathresult| as \meta{x} (given in degrees) converted to 
	radians. 
\end{command}

\begin{command}{\pgfmathsin{\marg{x}}}  
	Defines |\pgfmathresult| as the sine of \meta{x}.  
\end{command}

\begin{command}{\pgfmathcos{\marg{x}}}
	Defines |\pgfmathresult| as the cosine of \meta{x}.
\end{command}

\begin{command}{\pgfmathtan{\marg{x}}}  
	Defines |\pgfmathresult| as the tangant of \meta{x}.  
\end{command}

\begin{command}{\pgfmathsec{\marg{x}}}
	Defines |\pgfmathresult| as the secant of \meta{x}.
\end{command}

\begin{command}{\pgfmathcosec{\marg{x}}}  
	Defines |\pgfmathresult| as the cosecant of \meta{x}.  
\end{command}

\begin{command}{\pgfmathcot{\marg{x}}}  
	Defines |\pgfmathresult| as the cotangant of \meta{x}.  
\end{command}

\begin{command}{\pgfmathasin{\marg{x}}}
	Defines |\pgfmathresult| as the arcsine of \meta{x}. 
	The result will be in the range $\pm90^\circ$.
\end{command}

\begin{command}{\pgfmathacos{\marg{x}}}
	Defines |\pgfmathresult| as the arccosine of \meta{x}.
	The result will be in the range $\pm90^\circ$.
\end{command}

\begin{command}{\pgfmathatan{\marg{x}}}
 	Defines |\pgfmathresult| as the arctangent of \meta{x}.
\end{command}



\subsection{Pseudo-Random Numbers}

\label{pgfmath-random}


\begin{command}{\pgfmathgeneratepseudorandomnumber}
	Defines |\pgfmathresult| as a pseudo-random integer between 1 and 
	$2^{31}-1$. This uses a linear congruency generator, based on ideas
	due to Erich Janka.
\end{command}

\begin{command}{\pgfmathrnd}
	Defines |\pgfmathresult| as a pseudo-random number between |0| and |1|.
\end{command}

\begin{command}{\pgfmathrand}
	Defines |\pgfmathresult| as a pseudo-random number between |-1| and |1|.
\end{command}

\begin{command}{\pgfmathrandominteger\marg{macro}\marg{maximum}\marg{minimum}}
	This defines \meta{macro} as a pseudo-randomly generated integer from 
	the range \meta{maximum} to \meta{minimum} (inclusive).
	
\begin{codeexample}[]
\begin{pgfpicture}
   \foreach \x in {1,...,50}{
      \pgfmathrandominteger{\a}{1}{50}
      \pgfmathrandominteger{\b}{1}{50}
      \pgfpathcircle{\pgfpoint{+\a pt}{+\b pt}}{+2pt}
      \color{blue!40!white}
      \pgfsetstrokecolor{blue!80!black}
      \pgfusepath{stroke, fill}
   }	  
\end{pgfpicture}
\end{codeexample}
\end{command}

\begin{command}{\pgfmathdeclarerandomlist\marg{list name}\{\marg{item-1}\marg{item 2}...\}}
	This creates a list of items with the name \meta{list name}.
\end{command}

\begin{command}{\pgfmathrandomitem\marg{macro}\marg{list name}}
	Select an item from a random list \meta{list name}. The
	selected item is placed in \meta{macro}.
\end{command}

\begin{codeexample}[]
\begin{pgfpicture}
   \pgfmathdeclarerandomlist{color}{{red}{blue}{green}{yellow}{white}}
   \foreach \a in {1,...,50}{
      \pgfmathrandominteger{\x}{1}{85}
      \pgfmathrandominteger{\y}{1}{85}
      \pgfmathrandominteger{\r}{5}{10}
      \pgfmathrandomitem{\c}{color}
      \pgfpathcircle{\pgfpoint{+\x pt}{+\y pt}}{+\r pt}
      \color{\c!40!white}
      \pgfsetstrokecolor{\c!80!black}
      \pgfusepath{stroke, fill}
   }	  
\end{pgfpicture}
\end{codeexample}

\begin{command}{\pgfmathsetseed\marg{integer}}
  Explicitly set seed for the pseudo-random number generator. By
  default it is set to the value of |\time|$\times$|\year|.
\end{command}


      
\subsection{Conversion Between Bases}
	
\label{pgfmath-bases}

\pgfname{} provides limited support for conversion between 
\emph{representations} of numbers. Currently the numbers must be
positive integers in the range $0$ to $2^{31}-1$, and the bases in the
range $2$ to $36$. All digits representing numbers greater than 9 (in
base ten), are alphabetic, but may be upper or lower case. 

\begin{command}{\pgfmathbasetodec\marg{macro}\marg{number}\marg{base}}
	Defines \meta{macro} as the result of converting \meta{number} from
	base \meta{base} to base 10. Alphabetic digits can be upper or lower
	case.

\medskip{\def\medskip{}

\begin{codeexample}[]
\pgfmathbasetodec\mynumber{107f}{16} \mynumber
\end{codeexample}


\begin{codeexample}[]
\pgfmathbasetodec\mynumber{33FC}{20} \mynumber
\end{codeexample}

}\medskip

\end{command}

\begin{command}{\pgfmathdectobase\marg{macro}\marg{number}\marg{base}}
	Defines \meta{macro} as the result of converting \meta{number} from
	base 10 to base \meta{base}. Any resulting alphabetic digits are in
	\emph{lower case}.
	
\begin{codeexample}[]
\pgfmathdectobase\mynumber{65535}{16} \mynumber
\end{codeexample}

\end{command}

\begin{command}{\pgfmathdectoBase\marg{macro}\marg{number}\marg{base}}
	Defines \meta{macro} as the result of converting \meta{number} from
	base 10 to base \meta{base}. Any resulting alphabetic digits are in
	\emph{upper case}.
	
\begin{codeexample}[]
\pgfmathdectoBase\mynumber{65535}{16} \mynumber
\end{codeexample}

\end{command}

\begin{command}{\pgfmathbasetobase\marg{macro}\marg{number}\marg{base-1}\marg{base-2}}
	Defines \meta{macro} as the result of converting \meta{number} from
	base \meta{base-1} to base \meta{base-2}. Alphabetic digits in 
	\meta{number} can be upper or lower case, but any resulting 
	alphabetic digits are in \emph{lower case}.
	
\begin{codeexample}[]
\pgfmathbasetobase\mynumber{11011011}{2}{16} \mynumber
\end{codeexample}

\end{command}

\begin{command}{\pgfmathbasetoBase\marg{macro}\marg{number}\marg{base-1}\marg{base-2}}
	Defines \meta{macro} as the result of converting \meta{number} from
	base \meta{base-1} to base \meta{base-2}. Alphabetic digits in 
	\meta{number} can be upper or lower case, but any resulting 
	alphabetic digits are in \emph{upper case}.
	
\begin{codeexample}[]
\pgfmathbasetoBase\mynumber{121212}{3}{12} \mynumber
\end{codeexample}

\end{command}


\begin{command}{\pgfmathsetbasenumberlength\marg{integer}}
	Set the number of digits in the result of a base conversion to 
	\meta{integer}. If the result of a conversion has less digits
	than this number it is prefixed with zeros.

\begin{codeexample}[]
\pgfmathsetbasenumberlength{8}
\pgfmathdectobase\mynumber{15}{2} \mynumber
\end{codeexample}

\end{command}
% Copyright 2007 by Mark Wibrow
%
% This file may be distributed and/or modified
%
% 1. under the LaTeX Project Public License and/or
% 2. under the GNU Free Documentation License.
%
% See the file doc/generic/pgf/licenses/LICENSE for more details.


\section[Reimplementing the Computations of the Mathematical Engine]
  {Reimplementing the Computations of the\\ Mathematical Engine}

\label{pgfmath-reimplement}

Perhaps you are not satisfied with the Maclaurin series for
$e^x$. Perhaps you have a fantastically more accurate
and efficient way of calculating the sine or cosine of angles. Perhaps
 you would like the library to interface with a package such as |fp| 
 for fixed-point arithmetic (but you may find that exclusively
 using |fp| can cause a significant increase in compile time for
 documents involving many hundreds of calculations).
In these cases you will want to replace the current implementations of
the computations done by the mathematical engine by your own code. 

The mathematical engine was designed with such a replacement in
mind. For this reason, the operations and functions like |\pgfmathadd|
are implemented in the following manner: 

\begin{itemize}
\item |\pgfmath|\meta{function name} 

  This macro is the ``public'' interface for the function
  \meta{function name}. All arguments passed to this macro are 
  evaluated using |\pgfmathparse| and then passed on to the following
  function:
  
\item |\pgfmath|\meta{function name}|@|
  
  This macro is the ``non-public'' implementation of the functions 
  algorithm (but note that, for speed, the parser calls this macro 
  rather than the ``public'' one). Arguments passed to this macro 
  are expected to be numbers \emph{without units}. This is the macro 
  which should be rewritten with your prize-winning new algorithm.

  Note, furthermore, that if the function takes more than one
  argument, the second argument should not involve the dimensions
  |\pgfmath@x| nor |\pgfmath@xa| nor |\pgf@x| nor |\pgf@xa| since
  these may be set to the value of the first argument when the
  second argument is parsed.
\end{itemize}

The effect of |\pgfmath|\meta{function name}|@| should be to set the
macro |\pgfmathresult| to the correct value (namely to the result of
the computation without units). Furthermore, the function should have
no other side effects, that is, it should not change any global
values. One way to achieve this is to use the following code:

\begin{codeexample}[code only]
\def\pgfmath...@#1#2...{%
   \begingroup%
      ... code for algorithm ...
      \pgfmath@returnone\pgfmath@x%
   \endgroup%
}
\end{codeexample}


The macro |\pgfmath@returnone|\meta{code} must be directly followed by an
|\endgroup| and will save result of the algorithm, by defining
|\pgfmathresult| as the  expansion of \meta{code} \emph{without units}
outside the group. The \meta{code} should expand to a dimension
register or to a dimension. By performing the algorithm within a
\TeX{} group, \pgfname{} registers such as |\pgf@x|, |\pgf@y| and 
|\c@pgf@counta|, |\c@pgfcountb|, and so forth, can be used at will.

\pgfname{} uses the last known definition of a function within the
prevailing scope, so it is possible for a function to be redefined 
or |\let| to an alternative definition locally.
You should also remember that any |.sty| or |.tex| file contatining any
re-implementions should be loaded \emph{after} \pgfname-Math.


\part{The Basic Layer}

{\Large \emph{by Till Tantau}}


\bigskip
\noindent
\vskip1cm
\begin{codeexample}[graphic=white]
\begin{tikzpicture}
  \draw[gray,very thin] (-1.9,-1.9) grid (2.9,3.9)
          [step=0.25cm] (-1,-1) grid (1,1);
  \draw[blue] (1,-2.1) -- (1,4.1); % asymptote
                
  \draw[->] (-2,0) -- (3,0) node[right] {$x(t)$};
  \draw[->] (0,-2) -- (0,4) node[above] {$y(t)$};

  \foreach \pos in {-1,2}
    \draw[shift={(\pos,0)}] (0pt,2pt) -- (0pt,-2pt) node[below] {$\pos$};

  \foreach \pos in {-1,1,2,3}
    \draw[shift={(0,\pos)}] (2pt,0pt) -- (-2pt,0pt) node[left] {$\pos$};

  \fill (0,0) circle (0.064cm);
  \draw[thick,parametric,domain=0.4:1.5,samples=200]
    % The plot is reparameterised such that there are more samples
    % near the center.
    plot[id=asymptotic-example] function{(t*t*t)*sin(1/(t*t*t)),(t*t*t)*cos(1/(t*t*t))}
    node[right] {$\bigl(x(t),y(t)\bigr) = (t\sin \frac{1}{t}, t\cos \frac{1}{t})$};

  \fill[red] (0.63662,0) circle (2pt)
    node [below right,fill=white,yshift=-4pt] {$(\frac{2}{\pi},0)$};
\end{tikzpicture}
\end{codeexample}


% Copyright 2006 by Till Tantau
%
% This file may be distributed and/or modified
%
% 1. under the LaTeX Project Public License and/or
% 2. under the GNU Free Documentation License.
%
% See the file doc/generic/pgf/licenses/LICENSE for more details.


\section{Design Principles}

This section describes the basic layer of \pgfname. This layer is
build on top of the system layer. Whereas the system layer just
provides the absolute minimum for drawing graphics, the basic
layer provides numerous commands that make it possible to create
sophisticated graphics easily and also quickly.

The basic layer does not provide a convenient syntax for describing
graphics, which is left to frontends like \tikzname. For this reason, the
basic layer is typically used only by ``other programs.'' For example,
the \textsc{beamer} package uses the basic layer extensively, but does
not need a convenient input syntax. Rather, speed and flexibility are
needed when \textsc{beamer} creates graphics.

The following basic design principles underlie the basic layer:
\begin{enumerate}
\item Structuring into a core and modules.
\item Consistently named \TeX\ macros for all graphics commands.
\item Path-centered description of graphics.
\item Coordinate transformation system.
\end{enumerate}



\subsection{Core and Modules}

The basic layer consists of a \emph{core package}, called |pgfcore|,
which provides the most basic commands, and several
\emph{modules} like commands for plotting (in the |plot| module).
Modules are loaded using the |\usepgfmodule| command. 

If you say |\usepackage{pgf}| or |\input pgf.tex| or
|\usemodule[pgf]|, the |plot| and |shapes| modules are preloaded (as
well as the core and the system layer).



\subsection{Communicating with the Basic Layer via Macros}

In order to ``communicate'' with the basic layer you use long
sequences of commands that start with |\pgf|. You are only allowed to
give these commands inside a |{pgfpicture}| environment. (Note that
|{tikzpicture}| opens a |{pgfpicture}| internally, so you can freely
mix \pgfname\ commands and \tikzname\ commands inside a
|{tikzpicture}|.) It is possible to ``do other things'' between the
commands. For example, you might use one command to move to a certain
point, then have a complicated computation of the next point, and then
move there. 

\begin{codeexample}[]
\newdimen\myypos
\begin{pgfpicture}
  \pgfpathmoveto{\pgfpoint{0cm}{\myypos}}
  \pgfpathlineto{\pgfpoint{1cm}{\myypos}}
  \advance \myypos by 1cm
  \pgfpathlineto{\pgfpoint{1cm}{\myypos}}
  \pgfpathclose
  \pgfusepath{stroke}
\end{pgfpicture}
\end{codeexample}

The following naming conventions are used in the basic layer:

\begin{enumerate}
\item
  All commands and environments start with |pgf|.
\item
  All commands that specify a point (a coordinate) start with |\pgfpoint|.
\item
  All commands that extend the current path start with |\pgfpath|.
\item
  All commands that set/change a graphics parameter start with |\pgfset|.
\item
  All commands that use a previously declared object (like a path,
  image or shading) start with |\pgfuse|.
\item
  All commands having to do with coordinate transformations start with
  |\pgftransform|. 
\item
  All commands having to do with arrow tips start with |\pgfarrows|.
\item
  All commands for ``quickly'' extending or drawing a path start with
  |\pgfpathq| or |\pgfusepathq|.
\item
  All commands having to do with matrices start with |\pgfmatrix|.
\end{enumerate}


\subsection{Path-Centered Approach}

In \pgfname\ the most important entity is the \emph{path}. All
graphics are composed of numerous paths that can be stroked,
filled, shaded, or clipped against. Paths can be closed or open, they
can self-intersect and consist of unconnected parts.

Paths are first \emph{constructed} and then \emph{used}. In order to
construct a path, you can use commands starting with |\pgfpath|. Each
time such a command is called, the current path is extended in some
way.

Once a path has been completely constructed, you can use it using the
command |\pgfusepath|. Depending on the parameters given to this
command, the path will be stroked (drawn) or filled or subsequent
drawings will be clipped against this path.




\subsection{Coordinate Versus Canvas Transformations}

\label{section-design-transformations}

\pgfname\ provides two transformation systems: \pgfname's own
\emph{coordinate} transformation matrix and \pdf's or PostScript's
\emph{canvas} transformation matrix. These two systems are quite
different. Whereas a scaling by a factor of, say, $2$ of the canvas
causes \emph{everything} to be scaled by this factor (including
the thickness of lines and text), a scaling of two in the coordinate 
system causes only the \emph{coordinates} to be scaled, but not the
line width nor text.

By default, all transformations only apply to the coordinate
transformation system. However, using the command |\pgflowlevel|
it is possible to apply a transformation to the canvas.

Coordinate transformations are often preferable over canvas
transformations. Text and lines that are transformed using canvas 
transformations suffer from differing sizes and lines whose thickness 
differs depending on whether the line is horizontal or vertical. To
appreciate the difference, consider the following two ``circles'' both
of which have been scaled in the $x$-direction by a factor of $3$ and
by a factor of $0.5$ in the $y$-direction. The left circle uses a
canvas transformation, the right uses \pgfname's coordinate
transformation (some viewers will render the left graphic incorrectly
since they do no apply the low-level transformation the way they
should):  

\begin{tikzpicture}[line width=5pt]
  \useasboundingbox (-1.75,-1) rectangle (14,1);
  
  \begin{scope}
    \pgflowlevel{\pgftransformxscale{3}}
    \pgflowlevel{\pgftransformyscale{.5}}

    \draw (0,0) circle (0.5cm);
    \draw (.55cm,0pt) node[right] {canvas};
  \end{scope}
  \begin{scope}[xshift=9cm,xscale=3,yscale=.5]
    \draw (0,0) circle (0.5cm);
    \draw (.55cm,0pt) node[right] {coordinate};
  \end{scope}
\end{tikzpicture}



���}	� �
����DEV��INO��SYN��SV~�pgfmanual-en-base-scopes.tex�J��J���J�j���P�_.j�

���}	� �
����DEV��INO��SYN��SV~�pgfmanual-en-base-points.tex�J��J���J�j���P�_.j�
% Copyright 2006 by Till Tantau
%
% This file may be distributed and/or modified
%
% 1. under the LaTeX Project Public License and/or
% 2. under the GNU Free Documentation License.
%
% See the file doc/generic/pgf/licenses/LICENSE for more details.


\section{Constructing Paths}

\subsection{Overview}

The ``basic entity of drawing'' in \pgfname\ is the \emph{path}. A
path consists of several parts, each of which is either a closed or
open curve. An open curve has a starting point and an end point and,
in between, consists of several \emph{segments}, each of which is
either a straight line or a B�zier curve. Here is an example of a
path (in red) consisting of two parts, one open, one closed:

\begin{codeexample}[]
\begin{tikzpicture}[scale=2]
  \draw[thick,red]
       (0,0) coordinate (a)
    -- coordinate (ab) (1,.5) coordinate (b)
    .. coordinate (bc) controls +(up:1cm) and +(left:1cm) .. (3,1)  coordinate (c)
       (0,1) -- (2,1) -- coordinate (x) (1,2) -- cycle;

  \draw (a)  node[below] {start part 1}
        (ab) node[below right] {straight segment}
        (b)  node[right] {end first segment}
        (c)  node[right] {end part 1}
        (x)  node[above right]  {part 2 (closed)};        
\end{tikzpicture}
\end{codeexample}

A path, by itself, has no ``effect,'' that is, it does not leave any
marks on the page. It is just a set of points on the plane. However,
you can \emph{use} a path in different ways. The most natural actions
are \emph{stroking} (also known as \emph{drawing}) and
\emph{filling}. Stroking can be imagined as picking up a pen of a
certain diameter and ``moving it along the path.'' Filling means that
everything ``inside'' the path is filled with a uniform
color. Naturally, the open parts of a path must first be closed before
a path can be filled.

In \pgfname, there are numerous commands for constructing paths, all
of which start with |\pgfpath|. There are also commands for
\emph{using} paths, though most operations can be performed by calling
|\pgfusepath| with an appropriate parameter.

As a side-effect, the path construction commands keep track of two
bounding boxes. One is the bounding box for the current path, the
other is a bounding box for all paths in the current picture. See
Section~\ref{section-bb} for more details.

Each path construction command extends the current path in some
way. The ``current path'' is a global entity that persists across
\TeX\ groups. Thus, between calls to the path construction commands
you can perform arbitrary computations and even open and closed \TeX\
groups. The current path only gets ``flushed'' when the |\pgfusepath|
command is called (or when the soft-path subsystem is used directly,
see Section~\ref{section-soft-paths}).

\subsection{The Move-To Path Operation}

The most basic operation is the move-to operation. It must be given at
the beginning of paths, though some path construction command (like
|\pgfpathrectangle|) generate move-tos implicitly. A move-to operation
can also be used to start a new part of a path. 

\begin{command}{\pgfpathmoveto\marg{coordinate}}
  This command expects a \pgfname-coordinate like |\pgfpointorigin| as
  its parameter. When the current path is empty, this operation will
  start the path at the given \meta{coordinate}. If a path has already
  been partly constructed, this command will end the current part of
  the path and start a new one.
\begin{codeexample}[]
\begin{pgfpicture}
  \pgfpathmoveto{\pgfpointorigin}
  \pgfpathlineto{\pgfpoint{1cm}{1cm}}
  \pgfpathlineto{\pgfpoint{2cm}{1cm}}
  \pgfpathlineto{\pgfpoint{3cm}{0.5cm}}
  \pgfpathlineto{\pgfpoint{3cm}{0cm}}
  \pgfsetfillcolor{examplefill}
  \pgfusepath{fill,stroke}
\end{pgfpicture}
\end{codeexample}
\begin{codeexample}[]
\begin{pgfpicture}
  \pgfpathmoveto{\pgfpointorigin}
  \pgfpathlineto{\pgfpoint{1cm}{1cm}}
  \pgfpathlineto{\pgfpoint{2cm}{1cm}}
  \pgfpathmoveto{\pgfpoint{2cm}{1cm}} % New part
  \pgfpathlineto{\pgfpoint{3cm}{0.5cm}}
  \pgfpathlineto{\pgfpoint{3cm}{0cm}}
  \pgfsetfillcolor{examplefill}
  \pgfusepath{fill,stroke}
\end{pgfpicture}
\end{codeexample}
  The command will apply the current coordinate transformation matrix
  to \meta{coordinate} before using it.

  The command will update the bounding box of the current path and
  picture, if necessary. 
\end{command}


\subsection{The Line-To Path Operation}

\begin{command}{\pgfpathlineto\marg{coordinate}}
  This command extends the current path in a straight line to the
  given \meta{coordinate}. If this command is given at the beginning
  of path without any other path construction command given before (in
  particular without a move-to operation), the \TeX\ file may compile
  without an error message, but a viewer application may display an
  error message when trying to render the picture. 
\begin{codeexample}[]
\begin{pgfpicture}
  \pgfpathmoveto{\pgfpointorigin}
  \pgfpathlineto{\pgfpoint{1cm}{1cm}}
  \pgfpathlineto{\pgfpoint{2cm}{1cm}}
  \pgfsetfillcolor{examplefill}
  \pgfusepath{fill,stroke}
\end{pgfpicture}
\end{codeexample}
  The command will apply the current coordinate transformation matrix
  to \meta{coordinate} before using it.

  The command will update the bounding box of the current path and
  picture, if necessary. 
\end{command}


\subsection{The Curve-To Path Operation}

\begin{command}{\pgfpathcurveto\marg{support 1}\marg{support 2}\marg{coordinate}}
  This command extends the current path with a B�zier curve from the
  last point of the path to  \meta{coordinate}. The \meta{support 1}
  and \meta{support 2} are the first and second support point of the
  B�zier curve. For more information on B�zier curve, please consult a
  standard textbook on computer graphics.

  Like the line-to command, this command may not be the first path
  construction command in a path.
\begin{codeexample}[]
\begin{pgfpicture}
  \pgfpathmoveto{\pgfpointorigin}
  \pgfpathcurveto
    {\pgfpoint{1cm}{1cm}}{\pgfpoint{2cm}{1cm}}{\pgfpoint{3cm}{0cm}}
  \pgfsetfillcolor{examplefill}
  \pgfusepath{fill,stroke}
\end{pgfpicture}
\end{codeexample}
  The command will apply the current coordinate transformation matrix
  to \meta{coordinate} before using it.

  The command will update the bounding box of the current path and
  picture, if necessary. However, the bounding box is simply made
  large enough such that it encompasses all of the support points and
  the \meta{coordinate}. This will guarantee that the curve is
  completely inside the bounding box, but the bounding box will
  typically be quite a bit too large. It is not clear (to me) how this 
  can be avoided without resorting to ``some serious math'' in order
  to calculate a precise bounding box. 
\end{command}


\subsection{The Close Path Operation}

\begin{command}{\pgfpathclose}
  This command closes the current part of the path by appending a
  straight line to the start point of the current part. Note that there
  \emph{is} a difference between closing a path and using the line-to
  operation to add a straight line to the start of the current
  path. The difference is demonstrated by the upper corners of the triangles
  in the following example: 
\begin{codeexample}[]
\begin{tikzpicture}
  \draw[help lines] (0,0) grid (3,2);
  \pgfsetlinewidth{5pt}
  \pgfpathmoveto{\pgfpoint{1cm}{1cm}}
  \pgfpathlineto{\pgfpoint{0cm}{-1cm}}
  \pgfpathlineto{\pgfpoint{1cm}{-1cm}}
  \pgfpathclose
  \pgfpathmoveto{\pgfpoint{2.5cm}{1cm}}
  \pgfpathlineto{\pgfpoint{1.5cm}{-1cm}}
  \pgfpathlineto{\pgfpoint{2.5cm}{-1cm}}
  \pgfpathlineto{\pgfpoint{2.5cm}{1cm}}
  \pgfusepath{stroke}
\end{tikzpicture}
\end{codeexample}
\end{command}


\subsection{Arc, Ellipse and Circle Path Operations}

The path construction commands that we have discussed up to now are
sufficient to create all paths that can be created ``at all.''
However, it is useful to have special commands to create certain
shapes, like circles, that arise often in practice.

In the following, the commands for adding (parts of) (transformed)
circles to a path are described.

\begin{command}{\pgfpatharc\marg{start angle}\marg{end
      angle}{\ttfamily\char`\{}\meta{radius}\opt{| and |\meta{y-radius}}{\ttfamily\char`\}}}
  This command appends a part of a circle (or an ellipse) to the current
  path. Imaging the curve between \meta{start angle} and \meta{end
    angle} on a circle of radius \meta{radius} (if $\meta{start angle}
  < \meta{end angle}$, the curve goes around the circle
  counterclockwise, otherwise clockwise). This curve is now moved such
  that the point where the curve starts is the previous last point of the
  path. Note that this command will \emph{not} start a new part of the
  path, which is important for example for filling purposes. 

\begin{codeexample}[]
\begin{tikzpicture}
  \draw[help lines] (0,0) grid (3,2);
  \pgfpathmoveto{\pgfpointorigin}
  \pgfpathlineto{\pgfpoint{0cm}{1cm}}
  \pgfpatharc{180}{90}{.5cm}
  \pgfpathlineto{\pgfpoint{3cm}{1.5cm}}
  \pgfpatharc{90}{-45}{.5cm}
  \pgfusepath{fill}
\end{tikzpicture}
\end{codeexample}

  Saying |\pgfpatharc{0}{360}{1cm}| ``nearly'' gives you a full
  circle. The ``nearly'' refers to the fact that the circle will not
  be closed. You can close it using |\pgfpathclose|.

  If the optional \meta{y-radius} is given, the \meta{radius} is the
  $x$-radius and the \meta{y-radius} the $y$-radius of the ellipse
  from which the curve is taken:

\begin{codeexample}[]
\begin{tikzpicture}
  \draw[help lines] (0,0) grid (3,2);
  \pgfpathmoveto{\pgfpointorigin}
  \pgfpatharc{180}{45}{2cm and 1cm}
  \pgfusepath{draw}
\end{tikzpicture}
\end{codeexample}

  The axes of the circle or ellipse from which the arc is ``taken''
  always point up and right. However, the current coordinate
  transformation matrix will have an effect on the arc. This can be
  used to, say, rotate an arc:

\begin{codeexample}[]
\begin{tikzpicture}
  \draw[help lines] (0,0) grid (3,2);
  \pgftransformrotate{30}
  \pgfpathmoveto{\pgfpointorigin}
  \pgfpatharc{180}{45}{2cm and 1cm}
  \pgfusepath{draw}
\end{tikzpicture}
\end{codeexample}

  The command will update the bounding box of the current path and
  picture, if necessary. Unless rotation or shearing transformations
  are applied, the bounding box will be tight.
\end{command}

\begin{command}{\pgfpatharcaxes\marg{start angle}\marg{end
      angle}\marg{first axis}\marg{second axis}}
  This command is similar to |\pgfpatharc|. The main difference is how
  the ellipse or circle is specified from which the arc is taken. The
  two parameters \meta{first axis} and \meta{second axis} are the
  $0^\circ$-axis and the $90^\circ$-axis of the ellipse from which the
  path is taken. Thus, |\pgfpatharc{0}{90}{1cm and 2cm}| has the same effect
  as
\begin{verbatim}
\pgfpatharcaxes{0}{90}{\pgfpoint{1cm}{0cm}}{\pgfpoint{0cm}{2cm}}
\end{verbatim}
\begin{codeexample}[]
\begin{tikzpicture}
  \draw[help lines] (0,0) grid (3,2);
  \draw (0,0) -- (2cm,5mm) (0,0) -- (0cm,1cm);
  
  \pgfpathmoveto{\pgfpoint{2cm}{5mm}}
  \pgfpatharcaxes{0}{90}{\pgfpoint{2cm}{5mm}}{\pgfpoint{0cm}{1cm}}
  \pgfusepath{draw}
\end{tikzpicture}
\end{codeexample}
\end{command}  


\begin{command}{\pgfpathellipse\marg{center}\marg{first
      axis}\marg{second axis}}
  The effect of this command is to append an ellipse to the current
  path (if the path is not empty, a new part is started). The
  ellipse's center will be \meta{center} and \meta{first axis} and
  \meta{second axis} are the axis \emph{vectors}. The same effect as
  this command can also be achieved using an appropriate sequence of
  move-to, arc, and close operations, but this command is easier and
  faster. 

\begin{codeexample}[]
\begin{tikzpicture}
  \draw[help lines] (0,0) grid (3,2);
  \pgfpathellipse{\pgfpoint{1cm}{0cm}}
                 {\pgfpoint{1.5cm}{0cm}}
                 {\pgfpoint{0cm}{1cm}}
  \pgfusepath{draw}
  \color{red}               
  \pgfpathellipse{\pgfpoint{1cm}{0cm}}
                 {\pgfpoint{1cm}{1cm}}
                 {\pgfpoint{-0.5cm}{0.5cm}}
  \pgfusepath{draw}
\end{tikzpicture}
\end{codeexample}

  The command will apply coordinate transformations to all coordinates
  of the ellipse. However, the coordinate transformations are applied
  only after the ellipse is ``finished conceptually.'' Thus, a
  transformation of 1cm to the right will simply shift the ellipse one
  centimeter to the right; it will not add 1cm to the $x$-coordinates
  of the two axis vectors.

  The command will update the bounding box of the current path and
  picture, if necessary. 
\end{command}

\begin{command}{\pgfpathcirlce\marg{center}\marg{radius}}
  A shorthand for |\pgfpathellipse| applied to \meta{center} and the
  two axis vectors $(\meta{radius},0)$ and $(0,\meta{radius})$. 
\end{command}


\subsection{Rectangle Path Operations}

Another shape that arises frequently is the rectangle. Two commands
can be used to add a rectangle to the current path. Both commands will
start a new part of the path.


\begin{command}{\pgfpathrectangle\marg{corner}\marg{diagonal vector}}
  Adds a rectangle to the path whose one corner is \meta{corner} and
  whose opposite corner is given by $\meta{corner} + \meta{diagonal
    vector}$.

\begin{codeexample}[]
\begin{tikzpicture}
  \draw[help lines] (0,0) grid (3,2);
  \pgfpathrectangle{\pgfpoint{1cm}{0cm}}{\pgfpoint{1.5cm}{1cm}}
  \pgfpathrectangle{\pgfpoint{1.5cm}{0.25cm}}{\pgfpoint{1.5cm}{1cm}}
  \pgfpathrectangle{\pgfpoint{2cm}{0.5cm}}{\pgfpoint{1.5cm}{1cm}}
  \pgfusepath{draw}
\end{tikzpicture}
\end{codeexample}
  The command will apply coordinate transformations and update the
  bounding boxes tightly.
\end{command}


\begin{command}{\pgfpathrectanglecorners\marg{corner}\marg{opposite corner}}
  Adds a rectangle to the path whose two opposing corners are
  \meta{corner} and \meta{opposite corner}.
\begin{codeexample}[]
\begin{tikzpicture}
  \draw[help lines] (0,0) grid (3,2);
  \pgfpathrectanglecorners{\pgfpoint{1cm}{0cm}}{\pgfpoint{1.5cm}{1cm}}
  \pgfusepath{draw}
\end{tikzpicture}
\end{codeexample}
  The command will apply coordinate transformations and update the
  bounding boxes tightly.
\end{command}



\subsection{The Grid Path Operation}

\begin{command}{\pgfpathgrid\oarg{options}\marg{lower left}\marg{upper right}}
  Appends a grid to the current path. That is, a (possibly large)
  number of parts are added to the path, each part consisting of a
  single horizontal or vertical straight line segment.

  Conceptually, the origin is part of the grid and the grid is clipped 
  to the rectangle specified by the \meta{lower left} and
  the \meta{upper right} corner. However, no clipping occurs (this
  command just adds parts to the current path). Rather, the points
  where the lines enter and leave the ``clipping area'' are computed
  and used to add simple lines to the current path.

  The following keys influence the grid:
  \begin{key}{/pgf/stepx=\meta{dimension} (initially 1cm)}
    The horizontal stepping.
  \end{key}
  \begin{key}{/pgf/stepy=\meta{dimension} (initially 1cm)}
    The vertical stepping.
  \end{key}
  \begin{key}{/pgf/step=\meta{vector}}
    Sets the horizontal stepping to the $x$-coordinate of
    \meta{vector} and the vertical stepping to its $y$-coordinate.
  \end{key}
\begin{codeexample}[]
\begin{pgfpicture}
  \pgfsetlinewidth{0.8pt}
  \pgfpathgrid[step={\pgfpoint{1cm}{1cm}}]
    {\pgfpoint{-3mm}{-3mm}}{\pgfpoint{33mm}{23mm}}
  \pgfusepath{stroke}
  \pgfsetlinewidth{0.4pt}
  \pgfpathgrid[stepx=1mm,stepy=1mm]
    {\pgfpoint{-1.5mm}{-1.5mm}}{\pgfpoint{31.5mm}{21.5mm}}
  \pgfusepath{stroke}
\end{pgfpicture}
\end{codeexample}
  The command will apply coordinate transformations and update the
  bounding boxes tightly. As for ellipses, the transformations are
  applied to the ``conceptually finished'' grid. 
\begin{codeexample}[]
\begin{pgfpicture}
  \pgftransformrotate{10}
  \pgfpathgrid[stepx=1mm,stepy=2mm]{\pgfpoint{0mm}{0mm}}{\pgfpoint{30mm}{30mm}}
  \pgfusepath{stroke}
\end{pgfpicture}
\end{codeexample}
\end{command}


\subsection{The Parabola Path Operation}

\begin{command}{\pgfpathparabola\marg{bend vector}\marg{end vector}}
  This command appends two half-parabolas to the  current path. The
  first starts at the current point and ends at the current point plus
  \meta{bend vector}. At his point, it has its bend. The second half
  parabola starts at that bend point and end at point that is given by
  the bend plus \meta{end vector}.

  If you set \meta{end vector} to the null vector, you append only a
  half parabola that goes from the current point to the bend; by
  setting \meta{bend vector} to the null vector, you append only a
  half parabola that goes to current point plus \meta{end vector} and
  has its bend at the current point.

  It is not possible to use this command to draw a part of a parabola
  that does not contain the bend.

\begin{codeexample}[]
\begin{pgfpicture}
  % Half-parabola going ``up and right''
  \pgfpathmoveto{\pgfpointorigin}
  \pgfpathparabola{\pgfpointorigin}{\pgfpoint{2cm}{4cm}}
  \color{red}
  \pgfusepath{stroke}

  % Half-parabola going ``down and right''
  \pgfpathmoveto{\pgfpointorigin}
  \pgfpathparabola{\pgfpoint{-2cm}{4cm}}{\pgfpointorigin}
  \color{blue}
  \pgfusepath{stroke}

  % Full parabola
  \pgfpathmoveto{\pgfpoint{-2cm}{2cm}}
  \pgfpathparabola{\pgfpoint{1cm}{-1cm}}{\pgfpoint{2cm}{4cm}}
  \color{orange}
  \pgfusepath{stroke}
\end{pgfpicture}
\end{codeexample}
  The command will apply coordinate transformations and update the
  bounding boxes.
\end{command}


\subsection{Sine and Cosine Path Operations}

Sine and cosine curves often need to be drawn and the following commands
may help with this. However, they only allow you to append sine and
cosine curves in intervals that are multiples of $\pi/2$.

\begin{command}{\pgfpathsine\marg{vector}}
  This command appends a sine curve in the interval $[0,\pi/2]$ to the
  current path. The sine curve is squeezed or stretched such that the
  curve starts at the current point and ends at the current point plus
  \meta{vector}.
\begin{codeexample}[]
\begin{tikzpicture}
  \draw[help lines] (0,0) grid (3,1);
  \pgfpathmoveto{\pgfpoint{1cm}{0cm}}
  \pgfpathsine{\pgfpoint{1cm}{1cm}}
  \pgfusepath{stroke}

  \color{red}
  \pgfpathmoveto{\pgfpoint{1cm}{0cm}}
  \pgfpathsine{\pgfpoint{-2cm}{-2cm}}
  \pgfusepath{stroke}
\end{tikzpicture}
\end{codeexample}
  The command will apply coordinate transformations and update the
  bounding boxes.  
\end{command}

\begin{command}{\pgfpathcosine\marg{vector}}
  This command appends a cosine curve in the interval $[0,\pi/2]$ to the
  current path. The curve is squeezed or stretched such that the
  curve starts at the current point and ends at the current point plus
  \meta{vector}. Using several sine and cosine operations in sequence
  allows you to produce a complete sine or cosine curve
\begin{codeexample}[]
\begin{pgfpicture}
  \pgfpathmoveto{\pgfpoint{0cm}{0cm}}
  \pgfpathsine{\pgfpoint{1cm}{1cm}}
  \pgfpathcosine{\pgfpoint{1cm}{-1cm}}
  \pgfpathsine{\pgfpoint{1cm}{-1cm}}
  \pgfpathcosine{\pgfpoint{1cm}{1cm}}
  \pgfsetfillcolor{examplefill}
  \pgfusepath{fill,stroke}
\end{pgfpicture}
\end{codeexample}
  The command will apply coordinate transformations and update the
  bounding boxes.  
\end{command}


\subsection{Plot Path Operations}

There exist several commands for appending
plots to a path. These
commands are available through the module |plot|. They are
documented in Section~\ref{section-plots}.


\subsection{Rounded Corners}

Normally, when you connect two straight line segments or when you
connect two curves that end and start ``at different angles'' you get
``sharp corners'' between the lines or curves. In some cases it is
desirable to produce ``rounded corners'' instead. Thus, the lines
or curves should be shortened a bit and then connected by arcs.

\pgfname\ offers an easy way to achieve this effect, by calling the
following two commands.

\begin{command}{\pgfsetcornersarced\marg{point}}
  This command causes all subsequent corners to be replaced by little
  arcs. The effect of this command lasts till the end of the current
  \TeX\ scope.

  The \meta{point} dictates how large the corner arc will be. Consider
  a corner made by two lines $l$ and~$r$ and assume that the line $l$
  comes first on the path. The $x$-dimension of the \meta{point}
  decides by how much the line~$l$ will be shortened, the
  $y$-dimension of \meta{point} decides by how much the line $r$ will
  be shortened. Then, the shortened lines are connected by an arc.

\begin{codeexample}[]
\begin{tikzpicture}
  \draw[help lines] (0,0) grid (3,2);

  \pgfsetcornersarced{\pgfpoint{5mm}{5mm}}
  \pgfpathrectanglecorners{\pgfpointorigin}{\pgfpoint{3cm}{2cm}}
  \pgfusepath{stroke}
\end{tikzpicture}
\end{codeexample}

\begin{codeexample}[]
\begin{tikzpicture}
  \draw[help lines] (0,0) grid (3,2);

  \pgfsetcornersarced{\pgfpoint{10mm}{5mm}}
  % 10mm entering,
  % 5mm leaving.
  \pgfpathmoveto{\pgfpointorigin}
  \pgfpathlineto{\pgfpoint{0cm}{2cm}}
  \pgfpathlineto{\pgfpoint{3cm}{2cm}}
  \pgfpathcurveto
    {\pgfpoint{3cm}{0cm}}
    {\pgfpoint{2cm}{0cm}}
    {\pgfpoint{1cm}{0cm}}
  \pgfusepath{stroke}
\end{tikzpicture}
\end{codeexample}

  If the $x$- and $y$-coordinates of \meta{point} are the same and the
  corner is a right angle, you will get a perfect quarter circle
  (well, not quite perfect, but perfect up to six decimals). When the
  angle is not $90^\circ$, you only get a fair approximation.

  More or less ``all'' corners will be rounded, even the corner
  generated by a |\pgfpathclose| command. (The author is a bit proud
  of this feature.)
  
\begin{codeexample}[]
\begin{pgfpicture}
  \pgfsetcornersarced{\pgfpoint{4pt}{4pt}}
  \pgfpathmoveto{\pgfpointpolar{0}{1cm}}
  \pgfpathlineto{\pgfpointpolar{72}{1cm}}
  \pgfpathlineto{\pgfpointpolar{144}{1cm}}
  \pgfpathlineto{\pgfpointpolar{216}{1cm}}
  \pgfpathlineto{\pgfpointpolar{288}{1cm}}
  \pgfpathclose
  \pgfusepath{stroke}
\end{pgfpicture}
\end{codeexample}

  To return to normal (unrounded) corners, use
  |\pgfsetcornersarced{\pgfpointorigin}|.

  Note that the rounding will produce strange and undesirable effects
  if the lines at the corners are too short. In this case the
  shortening may cause the lines to ``suddenly extend over the other
  end'' which is rarely desirable. 
\end{command}




\subsection{Internal Tracking of Bounding Boxes for Paths and Pictures}

\label{section-bb}

\makeatletter

The path construction commands keep track of two bounding boxes: One
for the current path, which is reset whenever the path is used and
thereby flushed, and a bounding box for the current |{pgfpicture}|. 

The bounding boxes are not accessible by ``normal'' macros. Rather,
two sets of four dimension variables are used for this, all of which
contain the letter~|@|.

\begin{textoken}{\pgf@pathminx}
  The minimum $x$-coordinate ``mentioned'' in the current
  path. Initially, this is set to $16000$pt.
\end{textoken}

\begin{textoken}{\pgf@pathmaxx}
  The maximum $x$-coordinate ``mentioned'' in the current
  path. Initially, this is set to $-16000$pt.
\end{textoken}

\begin{textoken}{\pgf@pathminy}
  The minimum $y$-coordinate ``mentioned'' in the current
  path. Initially, this is set to $16000$pt.
\end{textoken}

\begin{textoken}{\pgf@pathmaxy}
  The maximum $y$-coordinate ``mentioned'' in the current
  path. Initially, this is set to $-16000$pt.
\end{textoken}

\begin{textoken}{\pgf@picminx}
  The minimum $x$-coordinate ``mentioned'' in the current
  picture. Initially, this is set to $16000$pt.
\end{textoken}

\begin{textoken}{\pgf@picmaxx}
  The maximum $x$-coordinate ``mentioned'' in the current
  picture. Initially, this is set to $-16000$pt.
\end{textoken}

\begin{textoken}{\pgf@picminy}
  The minimum $y$-coordinate ``mentioned'' in the current
  picture. Initially, this is set to $16000$pt.
\end{textoken}

\begin{textoken}{\pgf@picmaxy}
  The maximum $y$-coordinate ``mentioned'' in the current
  picture. Initially, this is set to $-16000$pt.
\end{textoken}


Each time a path construction command is called, the above variables
are (globally) updated. To facilitate this, you can use the following
command:

\begin{command}{\pgf@protocolsizes\marg{x-dimension}\marg{y-dimension}}
  Updates all of the above dimension in such a way that the point
  specified by the two arguments is inside both bounding boxes. For
  the picture's bounding box this updating occurs only if
  |\ifpgf@relevantforpicturesize| is true, see below.
\end{command}

For the bounding box of the picture it is not always desirable that
every path construction command affects this bounding box. For
example, if you have just used a clip command, you do not want anything
outside the clipping area to affect the bounding box. For this reason,
there exists a special ``\TeX\ if'' that (locally) decides whether
updating should be applied to the picture's bounding box. Clipping
will set this if to false, as will certain other commands.

\begin{command}{\pgf@relevantforpicturesizefalse}
  Suppresses updating of the picture's bounding box.
\end{command}

\begin{command}{\pgf@relevantforpicturesizetrue}
  Causes updating of the picture's bounding box.
\end{command}



��!�}	� �
����DEV��INO��SYN��SV~�pgfmanual-en-base-decorations.tex�J��J���J�j���P�_.j�
% Copyright 2006 by Till Tantau
%
% This file may be distributed and/or modified
%
% 1. under the LaTeX Project Public License and/or
% 2. under the GNU Free Documentation License.
%
% See the file doc/generic/pgf/licenses/LICENSE for more details.


\section{Using Paths}

\subsection{Overview}

Once a path has been constructed, it can be \emph{used} in different
ways. For example, you can draw the path or fill it or use it for
clipping.

Numerous graph parameters influence how a path will be rendered. For
example, when you draw a path, the line width is important as well as
the dashing pattern. The options that govern how paths are rendered
can all be set with commands starting with |\pgfset|. \emph{All
  options that influence how a path is rendered always influence the
  complete path.} Thus, it is not possible to draw part of a path
using, say, a red color and drawing another part using a green
color. To achieve such an effect, you must use two paths.

In detail, paths can be used in the following ways:

\begin{enumerate}
\item
  You can \emph{stroke} (also known as \emph{draw}) a path.
\item
  You can \emph{fill} a path with a uniform color.
\item
  You can \emph{clip} subsequent renderings against the path.
\item
  You can \emph{shade} a path.
\item
  You can \emph{use the path as bounding box} for the whole picture.
\end{enumerate}
You can also perform any combination of the above, though it makes no
sense to fill and shade a path at the same time.

To perform (a combination of) the first three actions, you can use the
following command:
\begin{command}{\pgfusepath\marg{actions}}
  Applies the given \meta{actions} to the current path. Afterwards,
  the current path is (globally) empty. The following actions are
  possible:
  \begin{itemize}
  \item \declare{|fill|}
    fills the path. See Section~\ref{section-fill} for further details.
\begin{codeexample}[]
\begin{pgfpicture}
  \pgfpathmoveto{\pgfpointorigin}
  \pgfpathlineto{\pgfpoint{1cm}{1cm}}
  \pgfpathlineto{\pgfpoint{1cm}{0cm}}
  \pgfusepath{fill}
\end{pgfpicture}
\end{codeexample}
  \item \declare{|stroke|}
    strokes the path. See Section~\ref{section-stroke} for further details.
\begin{codeexample}[]
\begin{pgfpicture}
  \pgfpathmoveto{\pgfpointorigin}
  \pgfpathlineto{\pgfpoint{1cm}{1cm}}
  \pgfpathlineto{\pgfpoint{1cm}{0cm}}
  \pgfusepath{stroke}
\end{pgfpicture}
\end{codeexample}
  \item \declare{|clip|}
    clips all subsequent drawings against the path. See
    Section~\ref{section-clip} for further details.
\begin{codeexample}[]
\begin{pgfpicture}
  \pgfpathmoveto{\pgfpointorigin}
  \pgfpathlineto{\pgfpoint{1cm}{1cm}}
  \pgfpathlineto{\pgfpoint{1cm}{0cm}}
  \pgfusepath{stroke,clip}
  \pgfpathcircle{\pgfpoint{1cm}{1cm}}{0.5cm}
  \pgfusepath{fill}
\end{pgfpicture}
\end{codeexample}
  \item \declare{|discard|}
    discards the path, that is, it is not used at all. Giving this
    option (alone) has the same effect as giving an empty options
    list.
  \end{itemize}
  When more than one of the first three actions are given, they are
  applied in the above ordering, regardless of their ordering in
  \meta{actions}. Thus, |{stroke,fill}| and |{fill,stroke}| have the
  same effect. 
\end{command}

To shade a path, use the |\pgfshadepath| command, which is explained
in Section~\ref{section-shadings}.



\subsection{Stroking a Path}
\label{section-stroke}

When you use |\pgfusepath{stroke}| to stroke a path, several graphic
parameters influence how the path is drawn. The commands for setting
these parameters are explained in the following.

Note that all graphic parameters apply to the path as a whole, never
only to a part of it.

All graphic parameters are local to the current |{pgfscope}|, but they
persists past \TeX\ groups, \emph{except} for the interior rule
(even-odd or nonzero) and the arrow tip kinds. The latter graphic
parameters only persist till the end of the current \TeX\ group, but 
this may change in the future, so do not count on this.

\subsubsection{Graphic Parameter: Line Width}

\begin{command}{\pgfsetlinewidth\marg{line width}}
  This command sets the line width for subsequent strokes (in the
  current |pgfscope|). The line width is given as a normal \TeX\
  dimension like |0.4pt| or |1mm|.

\begin{codeexample}[]
\begin{pgfpicture}
  \pgfsetlinewidth{1mm}
  \pgfpathmoveto{\pgfpoint{0mm}{0mm}}
  \pgfpathlineto{\pgfpoint{2cm}{0mm}}
  \pgfusepath{stroke}
  \pgfsetlinewidth{2\pgflinewidth} % double in size
  \pgfpathmoveto{\pgfpoint{0mm}{5mm}}
  \pgfpathlineto{\pgfpoint{2cm}{5mm}}
  \pgfusepath{stroke}
\end{pgfpicture}
\end{codeexample}
\end{command}

\begin{textoken}{\pgflinewidth}
  You can access the current line width via the \TeX\ dimension
  |\pgflinewidth|. It will be set to the correct line width, that is,
  even when a \TeX\ group closed, the value will be correct since it
  is set globally, but when a |{pgfscope}| closes, the value is set to
  the correct value it had before the scope.
\end{textoken}


\subsubsection{Graphic Parameter: Caps and Joins}

\begin{command}{\pgfsetbuttcap}
  Sets the line cap to a butt cap. See Section~\ref{section-cap-joins}
  for an explanation of what this is.
\end{command}
\begin{command}{\pgfsetroundcap}
  Sets the line cap to a round cap. See again
  Section~\ref{section-cap-joins}.
\end{command}
\begin{command}{\pgfsetrectcap}
  Sets the line cap to a square cap. See again
  Section~\ref{section-cap-joins}. 
\end{command}
\begin{command}{\pgfsetroundjoin}
  Sets the line join to a round join. See again
  Section~\ref{section-cap-joins}. 
\end{command}
\begin{command}{\pgfsetbeveljoin}
  Sets the line join to a bevel join. See again
  Section~\ref{section-cap-joins}. 
\end{command}
\begin{command}{\pgfsetmiterjoin}
  Sets the line join to a miter join. See again
  Section~\ref{section-cap-joins}. 
\end{command}
\begin{command}{\pgfsetmiterlimit\marg{miter limit factor}}
  Sets the miter limit to  \meta{miter limit factor}. See again 
  Section~\ref{section-cap-joins}. 
\end{command}

\subsubsection{Graphic Parameter: Dashing}

\begin{command}{\pgfsetdash\marg{list of even length of dimensions}\marg{phase}}
  Sets the dashing of a line. The first entry in the list specifies
  the length of the first solid part of the list. The second entry
  specifies the length of the following gap. Then comes the length of
  the second solid part, following by the length of the second gap,
  and so on. The \meta{phase} specifies where the first solid part
  starts relative to the beginning of the line.

\begin{codeexample}[]
\begin{pgfpicture}
  \pgfsetdash{{0.5cm}{0.5cm}{0.1cm}{0.2cm}}{0cm}
  \pgfpathmoveto{\pgfpoint{0mm}{0mm}}
  \pgfpathlineto{\pgfpoint{2cm}{0mm}}
  \pgfusepath{stroke}
  \pgfsetdash{{0.5cm}{0.5cm}{0.1cm}{0.2cm}}{0.1cm}
  \pgfpathmoveto{\pgfpoint{0mm}{1mm}}
  \pgfpathlineto{\pgfpoint{2cm}{1mm}}
  \pgfusepath{stroke}
  \pgfsetdash{{0.5cm}{0.5cm}{0.1cm}{0.2cm}}{0.2cm}
  \pgfpathmoveto{\pgfpoint{0mm}{2mm}}
  \pgfpathlineto{\pgfpoint{2cm}{2mm}}
  \pgfusepath{stroke}
\end{pgfpicture}
\end{codeexample}

  Use |\pgfsetdash{}{0pt}| to get a solid dashing.
\end{command}

\subsubsection{Graphic Parameter: Stroke Color}

\begin{command}{\pgfsetstrokecolor\marg{color}}
  Sets the color used for stroking lines to \meta{color}, where
  \meta{color} is a \LaTeX\ color like |red| or |black!20!red|. Unlike
  the |\color| command, the effect of this command lasts till the end
  of the current |{pgfscope}| and not till the end of the current
  \TeX\ group.

  The color used for stroking may be different from the color used for
  filling. However, a |\color| command will always ``immediately
  override'' any special settings for the stroke and fill colors.

  In plain \TeX, this command will also work, but the problem of
  \emph{defining} a color arises. After all, plain \TeX\ does not
  provide \LaTeX\ colors. For this reason, \pgfname\ implements a
  minimalistic ``emulation'' of the |\definecolor|, |\colorlet|, and
  |\color| commands. Only gray-scale and rgb colors are supported. For
  most cases this turns out to be enough.

\begin{codeexample}[]
\begin{pgfpicture}
  \pgfsetlinewidth{1pt}
  \color{red}
  \pgfpathcircle{\pgfpoint{0cm}{0cm}}{3mm} \pgfusepath{fill,stroke}
  \pgfsetstrokecolor{black}
  \pgfpathcircle{\pgfpoint{1cm}{0cm}}{3mm} \pgfusepath{fill,stroke}
  \color{red}
  \pgfpathcircle{\pgfpoint{2cm}{0cm}}{3mm} \pgfusepath{fill,stroke}
\end{pgfpicture}
\end{codeexample}
\end{command}

\begin{command}{\pgfsetcolor\marg{color}}
  Sets both the stroke and fill color. The difference to the normal
  |\color| command is that the effect lasts till the end of the
  current |{pgfscope}|, not only till the end of the current \TeX\
  group. 
\end{command}


\subsubsection{Graphic Parameter: Stroke Opacity}

You can set the stroke opacity using |\pgfsetstrokeopacity|. This
command is described in Section~\ref{section-transparency}.

\subsubsection{Graphic Parameter: Arrows}

After a path has been drawn, \pgfname\ can add arrow tips at the
ends. Currently, it will only add arrows correctly at the end of paths
that consist of a single open part. For other paths, like closed paths
or path consisting of multiple parts, the result is not defined.

\begin{command}{\pgfsetarrowsstart\marg{arrow kind}}
  Sets the arrow tip kind used at the start of a (possibly curved)
  path. When this option is used, the line will often be slightly
  shortened to ensure that the tip of the arrow will exactly ``touch''
  the ``real'' start of the line.

  To ``clear'' the start arrow, say |\pgfsetarrowsstart{}|.
\begin{codeexample}[]
\begin{pgfpicture}
  \pgfsetarrowsstart{latex}
  \pgfpathmoveto{\pgfpointorigin}
  \pgfpathlineto{\pgfpoint{1cm}{0cm}}
  \pgfusepath{stroke}
  \pgfsetarrowsstart{to}
  \pgfpathmoveto{\pgfpoint{0cm}{2mm}}
  \pgfpathlineto{\pgfpoint{1cm}{2mm}}
  \pgfusepath{stroke}
\end{pgfpicture}
\end{codeexample}

  The effect of this command persists only till the end of the current
  \TeX\ scope.

  The different possible arrow kinds are explained in
  Section~\ref{section-arrows}.  
\end{command}

\begin{command}{\pgfsetarrowsend\marg{arrow kind}}
  Sets the arrow tip kind used at the end of a path.
\begin{codeexample}[]
\begin{pgfpicture}
  \pgfsetarrowsstart{latex}
  \pgfsetarrowsend{to}
  \pgfpathmoveto{\pgfpointorigin}
  \pgfpathlineto{\pgfpoint{1cm}{0cm}}
  \pgfusepath{stroke}
\end{pgfpicture}
\end{codeexample}
\end{command}

\begin{command}{\pgfsetarrows{\texttt{\char`\{}}\meta{start kind}|-|\meta{end kind}{\texttt{\char`\}}}}
  Sets the start arrow kind to \meta{start kind} and the end kind to
  \meta{end kind}.
\begin{codeexample}[]
\begin{pgfpicture}
  \pgfsetarrows{latex-to}
  \pgfpathmoveto{\pgfpointorigin}
  \pgfpathlineto{\pgfpoint{1cm}{0cm}}
  \pgfusepath{stroke}
\end{pgfpicture}
\end{codeexample}
\end{command}

\begin{command}{\pgfsetshortenstart\marg{dimension}}
  This command will shortened the start of every stroked path by the
  given dimension. This shortening is done in addition to automatic
  shortening done by a start arrow, but it can be used even if no
  start arrow is given.

  This command is useful if you wish arrows or lines to ``stop shortly
  before'' a given point.
\begin{codeexample}[]
\begin{pgfpicture}
  \pgfpathcircle{\pgfpointorigin}{5mm}
  \pgfusepath{stroke}
  \pgfsetarrows{latex-}
  \pgfsetshortenstart{4pt}
  \pgfpathmoveto{\pgfpoint{5mm}{0cm}} % would be on the circle
  \pgfpathlineto{\pgfpoint{2cm}{0cm}}
  \pgfusepath{stroke}
\end{pgfpicture}
\end{codeexample}
\end{command}
  
\begin{command}{\pgfsetshortenend\marg{dimension}}
  Works like |\pgfsetshortenstart|.
\end{command}



\subsection{Filling a Path}
\label{section-fill}

Filling a path means coloring every interior point of the path with
the current fill color. It is not always obvious whether a point is
``inside'' a  path when the path is self-intersecting and/or consists
or multiple parts. In this case either the nonzero winding number rule
or the even-odd crossing number rule is used to decide, which points
lie ``inside.'' These rules are explained in
Section~\ref{section-rules}. 

\subsubsection{Graphic Parameter: Interior Rule}

You can set which rule is used using the following commands:

\begin{command}{\pgfseteorule}
  Dictates that the even-odd rule is used in subsequent fillings in
  the current \emph{\TeX\ scope}. Thus, for once, the effect of this
  command does not persist past the current \TeX\ scope.

\begin{codeexample}[]
\begin{pgfpicture}
  \pgfseteorule
  \pgfpathcircle{\pgfpoint{0mm}{0cm}}{7mm}
  \pgfpathcircle{\pgfpoint{5mm}{0cm}}{7mm}
  \pgfusepath{fill}
\end{pgfpicture}
\end{codeexample}
\end{command}

\begin{command}{\pgfsetnonzerorule}
  Dictates that the nonzero winding number rule is used in subsequent
  fillings in the current \TeX\ scope. This is the default.

\begin{codeexample}[]
\begin{pgfpicture}
  \pgfsetnonzerorule
  \pgfpathcircle{\pgfpoint{0mm}{0cm}}{7mm}
  \pgfpathcircle{\pgfpoint{5mm}{0cm}}{7mm}
  \pgfusepath{fill}
\end{pgfpicture}
\end{codeexample}
\end{command}

\subsubsection{Graphic Parameter: Filling Color}

\begin{command}{\pgfsetfillcolor\marg{color}}
  Sets the color used for filling paths to \meta{color}. Like the
  stroke color, the effect lasts only till the next use of |\color|. 
\end{command}


\subsubsection{Graphic Parameter: Fill Opacity}

You can set the fill opacity using |\pgfsetfillopacity|. This
command is described in Section~\ref{section-transparency}.

\subsection{Clipping a Path}
\label{section-clip}

When you add the |clip| option, the current path is used for
clipping subsequent drawings. The same rule as for filling is used to
decide whether a point is inside or outside the path, that is, either
the even-odd rule or the nonzero rule.

Clipping never enlarges the clipping area. Thus, when you clip against
a certain path and then clip again against another path, you clip
against the intersection of both.

The only way to enlarge the clipping path is to end the |{pgfscope}|
in which the clipping was done. At the end of a |{pgfscope}| the
clipping path that was in force at the beginning of the scope is
reinstalled. 

\subsection{Using a Path as a Bounding Box}
\label{section-using-bb}

When you add the |use as bounding box| option, the bounding box of the
picture will be enlarged such that the path in encompassed, but any
\emph{subsequent} paths of the current \TeX\ scope will not have any
effect on the size of the bounding box. Typically, you use this
command at the very beginning of a |{pgfpicture}| environment.

\begin{codeexample}[]
Left
\begin{pgfpicture}
  \pgfpathrectangle{\pgfpointorigin}{\pgfpoint{2ex}{1ex}}
  \pgfusepath{use as bounding box} % draws nothing

  \pgfpathcircle{\pgfpointorigin}{2ex}
  \pgfusepath{stroke}
\end{pgfpicture}
right.
\end{codeexample}


% Copyright 2006 by Till Tantau
%
% This file may be distributed and/or modified
%
% 1. under the LaTeX Project Public License and/or
% 2. under the GNU Free Documentation License.
%
% See the file doc/generic/pgf/licenses/LICENSE for more details.


\section{Arrow Tips}
\label{section-arrows}


\subsection{Overview}

\subsubsection{When Does PGF Draw Arrow Tips?}

\pgfname\ offers an interface for placing \emph{arrow tips} at the end
of lines. The interface works as follows:

\begin{enumerate}
\item
  You (or someone else) assigns a name to a certain kind of arrow
  tips. For example, the 
  arrow tip |latex| is the arrow tip used by the standard \LaTeX\
  picture environment; the arrow tip |to| looks like the tip of the
  arrow in \TeX's |\to| command; and so on.

  This is done once at the beginning of the document.
\item
  Inside some picture, at some point you specify that in the current
  scope from now on you would like tips of, say, kind |to| to be added
  at the end and/or beginning of all paths.

  When an arrow kind has been installed and when \pgfname\ is about to
  stroke a path, the following things happen:
  \begin{enumerate}
  \item
    The beginning and/or end of the path is shortened appropriately.
  \item
    The path is stroked.
  \item
    The arrow tip is drawn at the beginning and/or end of the path,
    appropriately rotated and appropriately resized.
  \end{enumerate}
\end{enumerate}

In the above description, there are a number of ``appropriately.''
The exact details are not quite trivial and described later on.

\subsubsection{Meta-Arrow Tips}

In \pgfname, arrows are ``meta-arrows'' in the same way that fonts in
\TeX\ are ``meta-fonts.'' When a meta-arrow is resized, it is not
simply scaled, but a possibly complicated transformation is applied to
the size.

A meta-font is not one particular font at a specific size with a
specific stroke width (and with a large number of other parameters
being fixed). Rather, it is a ``blueprint'' (actually, more like a
program) for generating such a font at a particular size and
width. This allows the designer of a meta-font to make sure that, say,
the font is somewhat thicker and wider at very small sizes. To
appreciate the difference: Compare the following texts: ``Berlin'' and
``\tikz{\node [scale=2,inner sep=0pt,outer sep=0pt]{\tiny
    Berlin};}''. The first is a ``normal'' text, the second is the tiny
version scaled by a factor of two. Obviously, the first look
better. Now, compare  ``\tikz{\node [scale=.5,inner sep=0pt,outer
  sep=0pt]{Berlin};}'' and ``{\tiny Berlin}''. This time, the normal
text was scaled down, while the second text is a ``normal'' tiny
text. The second text is easier to read. 

\pgfname's meta-arrows work in a similar fashion: The shape of an
arrow tip can vary according to the line width of the arrow tip is
used. Thus, an arrow tip drawn at a line width of 5pt will typically
\emph{not} be five times as large as an arrow tip of line width
1pt. Instead, the size of the arrow will get bigger only slowly as the
line width increases.

To appreciate the difference, here are the |latex| and |to| arrows, as
drawn by \pgfname\ at four different sizes:

\medskip
\begin{tikzpicture}
  \draw[-latex,line width=0.1pt] (0pt,0ex) -- +(3,0)  node[thin,right] {line width is 0.1pt};
  \draw[-latex,line width=0.4pt] (0pt,-2em) -- +(3,0) node[thin,right] {line width is 0.4pt};
  \draw[-latex,line width=1.2pt] (0pt,-4em) -- +(3,0) node[thin,right] {line width is 1.2pt};
  \draw[-latex,line width=5pt]   (0pt,-6em) -- +(3,0) node[thin,right] {line width is 5pt};

  \draw[-to,line width=0.1pt] (6cm,0ex) -- +(3,0)  node[thin,right] {line width is 0.1pt};
  \draw[-to,line width=0.4pt] (6cm,-2em) -- +(3,0) node[thin,right] {line width is 0.4pt};
  \draw[-to,line width=1.2pt] (6cm,-4em) -- +(3,0) node[thin,right] {line width is 1.2pt};
  \draw[-to,line width=5pt]   (6cm,-6em) -- +(3,0) node[thin,right] {line width is 5pt};
\end{tikzpicture}

\medskip
Here, by comparison, is the same arrow when it is simply ``resized''
(as done by most programs):

\pgfarrowsdeclare{bad latex}{bad latex}
{
  \pgfarrowsleftextend{-1\pgflinewidth}
  \pgfarrowsrightextend{9\pgflinewidth}
}
{
  \pgfpathmoveto{\pgfpoint{9\pgflinewidth}{0pt}}
  \pgfpathcurveto
  {\pgfpoint{6.3333\pgflinewidth}{.5\pgflinewidth}}
  {\pgfpoint{2\pgflinewidth}{2\pgflinewidth}}
  {\pgfpoint{-1\pgflinewidth}{3.75\pgflinewidth}}
  \pgfpathlineto{\pgfpoint{-1\pgflinewidth}{-3.75\pgflinewidth}}
  \pgfpathcurveto
  {\pgfpoint{2\pgflinewidth}{-2\pgflinewidth}}
  {\pgfpoint{6.3333\pgflinewidth}{-.5\pgflinewidth}}
  {\pgfpoint{9\pgflinewidth}{0pt}}
  \pgfusepathqfill
}

\pgfarrowsdeclare{bad to}{bad to}
{
  \pgfarrowsleftextend{-2\pgflinewidth}
  \pgfarrowsrightextend{\pgflinewidth}
}
{
  \pgfsetlinewidth{0.8\pgflinewidth}
  \pgfsetdash{}{0pt}
  \pgfsetroundcap
  \pgfsetroundjoin
  \pgfpathmoveto{\pgfpoint{-3\pgflinewidth}{4\pgflinewidth}}
  \pgfpathcurveto
  {\pgfpoint{-2.75\pgflinewidth}{2.5\pgflinewidth}}
  {\pgfpoint{0pt}{0.25\pgflinewidth}}
  {\pgfpoint{0.75\pgflinewidth}{0pt}}
  \pgfpathcurveto
  {\pgfpoint{0pt}{-0.25\pgflinewidth}}
  {\pgfpoint{-2.75\pgflinewidth}{-2.5\pgflinewidth}}
  {\pgfpoint{-3\pgflinewidth}{-4\pgflinewidth}}
  \pgfusepathqstroke
}

\medskip
\begin{tikzpicture}
  \draw[-bad latex,line width=0.1pt] (0pt,0ex) -- +(3,0)  node[thin,right] {line width is 0.1pt};
  \draw[-bad latex,line width=0.4pt] (0pt,-2em) -- +(3,0) node[thin,right] {line width is 0.4pt};
  \draw[-bad latex,line width=1.2pt] (0pt,-4em) -- +(3,0) node[thin,right] {line width is 1.2pt};
  \draw[-bad latex,line width=5pt]   (0pt,-6em) -- +(3,0) node[thin,right] {line width is 5pt};

  \draw[-bad to,line width=0.1pt] (6cm,0ex) -- +(3,0)  node[thin,right] {line width is 0.1pt};
  \draw[-bad to,line width=0.4pt] (6cm,-2em) -- +(3,0) node[thin,right] {line width is 0.4pt};
  \draw[-bad to,line width=1.2pt] (6cm,-4em) -- +(3,0) node[thin,right] {line width is 1.2pt};
  \draw[-bad to,line width=5pt]   (6cm,-6em) -- +(3,0) node[thin,right] {line width is 5pt};
\end{tikzpicture}

\bigskip
As can be seen, simple scaling produces arrow tips that are way too
large at larger sizes and way too small at smaller sizes.



\subsection{Declaring an Arrow Tip Kind}

To declare an arrow kind ``from scratch,'' the following command is
used:

\begin{command}{\pgfarrowsdeclare\marg{start name}\marg{end
      name}\marg{extend code}\marg{arrow tip code}}
  This command declares a new arrow kind. An arrow kind has two names,
  which will typically be the same. When the arrow tip needs to be
  drawn, the \meta{arrow tip code} will be invoked, but the canvas
  transformation is setup beforehand to a rotation such that when an
  arrow tip pointing right is specified, the arrow tip that is
  actually drawn points in the direction of the line.

  \medskip
  \textbf{Naming the arrow kind.}
  The \meta{start name} is the name
  used for the arrow tip when it is at the start of a path, the \meta{end
    name} is the name used at the end of a path. For example, the
  arrow kind that looks like a parenthesis has the \meta{start
    name} |(| and the \meta{end name} |)| so that you can say
  |\pgfsetarrows{(-)}| to specify that you want parenthesis arrows and
  both ends.

  The \meta{end name} and \meta{start name} can be quite arbitrary and
  may contain spaces.

  \medskip
  \textbf{Basics of the arrow tip code.}
  Let us next have a look at the \meta{arrow tip code}. This code will
  be used to draw the arrow tip when \pgfname\ thinks this is
  necessary. The code should draw an arrow that ``points right,''
  which means that is should draw an arrow at the end of a line coming
  from the left and ending at the origin.

  As an example, suppose we wanted to declare an arrow tip consisting
  of two arcs, that is, we want the arrow tip to look more or less
  like the red part of the following picture:
\begin{codeexample}[]
\begin{tikzpicture}[line width=3pt]
  \draw (-2,0) -- (0,0);
  \draw[red,line join=round,line cap=round]
        (-10pt,10pt) arc (180:270:10pt) arc (90:180:10pt);
\end{tikzpicture}
\end{codeexample}

  We could use the following as \meta{arrow tip code} for this:
\begin{codeexample}[code only]
\pgfarrowsdeclare{arcs}{arcs}{...}
{
  \pgfsetdash{}{0pt} % do not dash
  \pgfsetroundjoin   % fix join
  \pgfsetroundcap    % fix cap
  \pgfpathmoveto{\pgfpoint{-10pt}{10pt}}
  \pgfpatharc{180}{270}{10pt}
  \pgfpatharc{90}{180}{10pt}
  \pgfusepathqstroke
}
\end{codeexample}

  Indeed, when the |...| is set appropriately (in a moment), we can
  write the following:
\pgfarrowsdeclare{arcs}{arcs}{\pgfarrowsleftextend{0pt}\pgfarrowsrightextend{0pt}}
{
  \pgfsetdash{}{0pt} % do not dash
  \pgfsetroundjoin   % fix join
  \pgfsetroundcap    % fix cap
  \pgfpathmoveto{\pgfpoint{-10pt}{10pt}}
  \pgfpatharc{180}{270}{10pt}
  \pgfpatharc{90}{180}{10pt}
  \pgfusepathqstroke
}
\begin{codeexample}[]
\begin{tikzpicture}
  \draw[-arcs,line width=3pt] (-2,0)  -- (0,0);
  \draw[arcs-arcs,line width=1pt] (-2,-1.5) -- (0,-1);
  \useasboundingbox (-2,-2) rectangle (0,0.75);
\end{tikzpicture}
\end{codeexample}

  As can be seen in the second example, the arrow tip is automatically
  rotated as needed when the arrow is drawn. This is achieved by a
  canvas rotation.

  \medskip
  \textbf{Special considerations about the arrow tip code.}
  There are several things you need to be aware of when designing
  arrow tip code:
  \begin{itemize}
  \item
    Inside the code, you may not use the |\pgfusepath|
    command. The reason is that this command internally calls arrow
    construction commands, which is something you obviously do not want
    to happen.

    Instead of |\pgfusepath|, use the quick versions. Typically, you
    will use |\pgfusepathqstroke|, |\pgfusepathqfill|, or
    |\pgfusepathqfillstroke|.
  \item
    The code will be executed only once, namely the first time the
    arrow tip needs to be drawn. The resulting low-level driver
    commands are protocoled and stored away. In all subsequent 
    uses of the arrow tip, the protocoled code is directly inserted.
  \item
    However, the code will be executed anew for each line width. Thus,
    an arrow of line width 2pt may result in a different protocol than
    the same arrow for a line width of 0.4pt.
  \item
    If you stroke the path that you construct, you should first set
    the dashing to solid and setup fixed joins and caps, as
    needed. This will ensure that the arrow tip will always look the
    same.
  \item
    When the arrow tip code is executed, it is automatically put
    inside a low-level scope, so nothing will ``leak out'' from the
    scope.
  \item
    The high-level coordinate transformation matrix will be set to the
    identity matrix when the code is executed for the first time.
  \end{itemize}

  \medskip
  \textbf{Designing meta-arrows.}
  The \meta{arrow tip code} should adjust the size of the arrow in
  accordance with the line width. For a small line width, the arrow
  tip should be small, for a large line width, it should be
  larger. However, the size of the arrow typically \emph{should not}
  grow in direct proportion to the line width. On the other hand, the
  size of the arrow head typically \emph{should} grow ``a bit'' with
  the line width. 

  For these reasons, \pgfname\ will not simply executed your arrow
  code within a scaled scope, where the scaling depends on the line
  width. Instead, your \meta{arrow tip code} is reexecuted again for
  each different line width.

  In our example, we could use the following code for the new arrow
  tip kind |arc'| (note the prime):
\begin{codeexample}[code only]
\newdimen\arrowsize    
\pgfarrowsdeclare{arcs'}{arcs'}{...}
{
  \arrowsize=0.2pt
  \advance\arrowsize by .5\pgflinewidth
  \pgfsetdash{}{0pt} % do not dash
  \pgfsetroundjoin   % fix join
  \pgfsetroundcap    % fix cap
  \pgfpathmoveto{\pgfpoint{-4\arrowsize}{4\arrowsize}}
  \pgfpatharc{180}{270}{4\arrowsize}
  \pgfpatharc{90}{180}{4\arrowsize}
  \pgfusepathqstroke
}
\end{codeexample}
\newdimen\arrowsize    
\pgfarrowsdeclare{arcs'}{arcs'}{\pgfarrowsleftextend{0pt}\pgfarrowsrightextend{0pt}}
{
  \arrowsize=0.2pt
  \advance\arrowsize by .5\pgflinewidth
  \pgfsetdash{}{0pt} % do not dash
  \pgfsetroundjoin   % fix join
  \pgfsetroundcap    % fix cap
  \pgfpathmoveto{\pgfpoint{-4\arrowsize}{4\arrowsize}}
  \pgfpatharc{180}{270}{4\arrowsize}
  \pgfusepathqstroke
  \pgfpathmoveto{\pgfpointorigin}
  \pgfpatharc{90}{180}{4\arrowsize}
  \pgfusepathqstroke
}
\begin{codeexample}[]
\begin{tikzpicture}
  \draw[-arcs',line width=3pt] (-2,0)  -- (0,0);
  \draw[arcs'-arcs',line width=1pt] (-2,-1.5) -- (0,-1);
  \useasboundingbox (-2,-1.75) rectangle (0,0.5);
\end{tikzpicture}
\end{codeexample}
  
  However, sometimes, it can also be useful to have arrows that do not
  resize at all when the line width changes. This can be achieved by
  giving absolute size coordinates in the code, as done for |arc|. On
  the other hand, you can also have the arrow resize linearly with the
  line width by specifying all coordinates as multiples of
  |\pgflinewidth|.

  \textbf{The left and right extend.}
  Let us have another look at the exact left and right ``ends'' of our
  arrow tip. Let us draw the arrow tip |arc'| at a very large size:

\begin{codeexample}[]
\begin{tikzpicture}
  \draw[help lines] (-2,-1) grid (1,1);
  \draw[line width=10pt,-arcs'] (-2,0) -- (0,0);
  \draw[line width=2pt,white] (-2,0) -- (0,0);
\end{tikzpicture}
\end{codeexample}

  As one can see, the arrow tip does not ``touch'' the origin as it
  should, but protrudes a little over the origin. One remedy to this
  undesirable effect is to change the code of the arrow tip such that
  everything is shifted half an |\arrowsize| to the left. While this
  will cause the arrow tip to touch the origin, the line itself will
  then interfere with the arrow: The arrow tip will be partly
  ``hidden'' by the line itself.

  \pgfname\ uses a different approach to solving the problem: The
  \meta{extend code} argument can be used to ``tell'' \pgfname\ how
  much the arrow protrudes over the origin. The argument is also used
  to tell \pgfname\ where the ``left'' end of the arrow is. However,
  this number is important only when the arrow is being reversed or
  composed with other arrow tips.

  Once \pgfname\ knows the right extend of an arrow kind, it can
  \emph{shorten} lines by this amount when drawing arrows.

  Here is a picture that shows what the visualizes the extends. The
  arrow tip itself is shown in red once more:

  \medskip
  \begin{tikzpicture}
    \draw[line width=1cm,-arcs',red] (-6,0) -- (0,0);
    \draw[line width=1cm,black]      (-6,0) -- (0,0);
    \draw[help lines] (-6,0) -- (2,0)     (0,-3) -- (0,3) coordinate (a);
    \draw[help lines,xshift=0.5cm]        (0,-3) -- (0,3) coordinate (b);
    \draw[help lines,xshift=-2.5cm-0.8pt] (0,-3) -- (0,3) coordinate (c);

    \coordinate (xline 1) at (0,1.5);
    \coordinate (xline 2) at (0,2.8);
    
    \draw[|->|] (xline 1 -| a) -- node[above=2pt] {right extend} (xline 1 -| b);    
    \draw[|<-|] (xline 2 -| c) -- node[above=2pt] {left extend}  (xline 2 -| a);

    \draw (0,0) -- (1,-1) node[below right] {origin};
   \end{tikzpicture}
  

  The \meta{extend code} is normal \TeX\ code that is executed
  whenever \pgfname\ wants to know how far the arrow tip will protrude
  to the right and left. The code should call the following two
  commands: \declare{|\pgfarrowsrightextend|} and
  \declare{|\pgfarrowsleftextend|}. Both arguments take one argument
  that specifies the size. Here is the final code for the |arc''| arrow
  tip: 
\begin{codeexample}[]
\pgfarrowsdeclare{arcs''}{arcs''}
{
  \arrowsize=0.2pt
  \advance\arrowsize by .5\pgflinewidth
  \pgfarrowsleftextend{-4\arrowsize-.5\pgflinewidth}
  \pgfarrowsrightextend{.5\pgflinewidth}
}
{
  \arrowsize=0.2pt
  \advance\arrowsize by .5\pgflinewidth
  \pgfsetdash{}{0pt} % do not dash
  \pgfsetroundjoin   % fix join
  \pgfsetroundcap    % fix cap
  \pgfpathmoveto{\pgfpoint{-4\arrowsize}{4\arrowsize}}
  \pgfpatharc{180}{270}{4\arrowsize}
  \pgfusepathqstroke
  \pgfpathmoveto{\pgfpointorigin}
  \pgfpatharc{90}{180}{4\arrowsize}
  \pgfusepathqstroke
}
\begin{tikzpicture}
  \draw[help lines] (-2,-1) grid (1,1);
  \draw[line width=10pt,-arcs''] (-2,0) -- (0,0);
  \draw[line width=2pt,white] (-2,0) -- (0,0);
\end{tikzpicture}
\end{codeexample}
\end{command}

\pgfarrowsdeclare{arcs''}{arcs''}
{
  \arrowsize=0.2pt
  \advance\arrowsize by .5\pgflinewidth
  \pgfarrowsleftextend{-4\arrowsize-.5\pgflinewidth}
  \pgfarrowsrightextend{.5\pgflinewidth}
}
{
  \arrowsize=0.2pt
  \advance\arrowsize by .5\pgflinewidth
  \pgfsetdash{}{0pt} % do not dash
  \pgfsetroundjoin   % fix join
  \pgfsetroundcap    % fix cap
  \pgfpathmoveto{\pgfpoint{-4\arrowsize}{4\arrowsize}}
  \pgfpatharc{180}{270}{4\arrowsize}
  \pgfusepathqstroke
  \pgfpathmoveto{\pgfpointorigin}
  \pgfpatharc{90}{180}{4\arrowsize}
  \pgfusepathqstroke
}


\subsection{Declaring a Derived Arrow Tip Kind}

It is possible to declare arrow kinds in terms of existing ones. For
these command to work correctly, the left and right extends must be
set correctly.

\begin{command}{\pgfarrowsdeclarealias\marg{start name}\marg{end
      name}\marg{old start name}\marg{old end name}}
  This command can be used to create an alias (another name) for an
  existing arrow kind.

\begin{codeexample}[]
\pgfarrowsdeclarealias{<}{>}{arcs''}{arcs''}%
\begin{tikzpicture}
  \pgfsetarrows{<->}
  \pgfsetlinewidth{1ex}
  \pgfpathmoveto{\pgfpointorigin}
  \pgfpathlineto{\pgfpoint{3.5cm}{2cm}}
  \pgfusepath{stroke}
  \useasboundingbox (-0.25,-0.25) rectangle (3.75,2.25);
\end{tikzpicture}
\end{codeexample}
\end{command}


\begin{command}{\pgfarrowsdeclarereversed\marg{start name}\marg{end
      name}\marg{old start name}\marg{old end name}}
  This command creates a new arrow kind that is the ``reverse'' of an
  existing arrow kind. The (automatically cerated) code of the new
  arrow kind will contain a flip of the canvas and the meanings of the
  left and right extend will be reversed. 

\begin{codeexample}[]
\pgfarrowsdeclarereversed{arcs reversed}{arcs reversed}{arcs''}{arcs''}%
\begin{tikzpicture}
  \pgfsetarrows{arcs reversed-arcs reversed}
  \pgfsetlinewidth{1ex}
  \pgfpathmoveto{\pgfpointorigin}
  \pgfpathlineto{\pgfpoint{3.5cm}{2cm}}
  \pgfusepath{stroke}
  \useasboundingbox (-0.25,-0.25) rectangle (3.75,2.25);
\end{tikzpicture}
\end{codeexample}
\end{command}



\begin{command}{\pgfarrowsdeclarecombine\opt{|*|}\opt{\oarg{offset}}\marg{start
      name}\marg{end name}\marg{first start name}\marg{first end
      name}\penalty0\marg{second start name}\marg{second end name}}
  This command creates a new arrow kind that combines two existing
  arrow kinds. The first arrow kind is the ``innermost'' arrow kind,
  the second arrow kind is the ``outermost.''

  The code for the combined arrow kind will install a canvas
  translation before the innermost arrow kind in drawn. This
  translation is calculated such that the right tip of the innermost
  arrow touches the right  end of the outermost arrow. The optional
  \meta{offset} can be used to increase (or decrease) the distance
  between the inner and outermost arrow.

\begin{codeexample}[]
\pgfarrowsdeclarecombine[\pgflinewidth]
  {combined}{combined}{arcs''}{arcs''}{latex}{latex}%
\begin{tikzpicture}
  \pgfsetarrows{combined-combined}
  \pgfsetlinewidth{1ex}
  \pgfpathmoveto{\pgfpointorigin}
  \pgfpathlineto{\pgfpoint{3.5cm}{2cm}}
  \pgfusepath{stroke}
  \useasboundingbox (-0.25,-0.25) rectangle (3.75,2.25);
\end{tikzpicture}
\end{codeexample}

  In the star variant, the end of the line is not in the outermost
  arrow, but inside the innermost arrow.

\begin{codeexample}[]
\pgfarrowsdeclarecombine*[\pgflinewidth]
  {combined'}{combined'}{arcs''}{arcs''}{latex}{latex}%
\begin{tikzpicture}
  \pgfsetarrows{combined'-combined'}
  \pgfsetlinewidth{1ex}
  \pgfpathmoveto{\pgfpointorigin}
  \pgfpathlineto{\pgfpoint{3.5cm}{2cm}}
  \pgfusepath{stroke}
  \useasboundingbox (-0.25,-0.25) rectangle (3.75,2.25);
\end{tikzpicture}
\end{codeexample}
\end{command}


\begin{command}{\pgfarrowsdeclaredouble\opt{\oarg{offset}}\marg{start
      name}\marg{end name}\marg{old start name}\marg{old end
      name}}
  This command is a shortcut for combining an arrow kind with itself.

\begin{codeexample}[]
\pgfarrowsdeclaredouble{<<}{>>}{arcs''}{arcs''}%
\begin{tikzpicture}
  \pgfsetarrows{<<->>}
  \pgfsetlinewidth{1ex}
  \pgfpathmoveto{\pgfpointorigin}
  \pgfpathlineto{\pgfpoint{3.5cm}{2cm}}
  \pgfusepath{stroke}
  \useasboundingbox (-0.25,-0.25) rectangle (3.75,2.25);
\end{tikzpicture}
\end{codeexample} 
\end{command}


\begin{command}{\pgfarrowsdeclaretriple\opt{\oarg{offset}}\marg{start
      name}\marg{end name}\marg{old start name}\marg{old end
      name}}
  This command is a shortcut for combining an arrow kind with itself
  and then again.

\begin{codeexample}[]
\pgfarrowsdeclaretriple{<<<}{>>>}{arcs''}{arcs''}%
\begin{tikzpicture}
  \pgfsetarrows{<<<->>>}
  \pgfsetlinewidth{1ex}
  \pgfpathmoveto{\pgfpointorigin}
  \pgfpathlineto{\pgfpoint{3.5cm}{2cm}}
  \pgfusepath{stroke}
  \useasboundingbox (-0.25,-0.25) rectangle (3.75,2.25);
\end{tikzpicture}
\end{codeexample} 
\end{command}





\subsection{Using an Arrow Tip Kind}

The following commands install the arrow kind that will be used when
stroking is done.

\begin{command}{\pgfsetarrowsstart\marg{start arrow kind}}
  Installs the given \meta{start arrow kind} for all subsequent
  strokes in the in the current \TeX-group. If \meta{start arrow kind}
  is empty, no arrow tips will be drawn at the start of the last
  segment of paths.
\begin{codeexample}[]
\begin{tikzpicture}
  \pgfsetarrowsstart{latex}
  \pgfsetlinewidth{1ex}
  \pgfpathmoveto{\pgfpointorigin}
  \pgfpathlineto{\pgfpoint{3.5cm}{2cm}}
  \pgfusepath{stroke}
  \useasboundingbox (-0.25,-0.25) rectangle (3.75,2.25);
\end{tikzpicture}
\end{codeexample} 
\end{command}

\begin{command}{\pgfsetarrowsend\marg{start arrow kind}}
  Like |\pgfsetarrowsstart|, only for the end of the arrow.
\begin{codeexample}[]
\begin{tikzpicture}
  \pgfsetarrowsend{latex}
  \pgfsetlinewidth{1ex}
  \pgfpathmoveto{\pgfpointorigin}
  \pgfpathlineto{\pgfpoint{3.5cm}{2cm}}
  \pgfusepath{stroke}
  \useasboundingbox (-0.25,-0.25) rectangle (3.75,2.25);
\end{tikzpicture}
\end{codeexample} 
\end{command}

\emph{Warning:} If the compatibility mode is active (which is the
default), there also exist old commands called |\pgfsetstartarrow| and 
|\pgfsetendarrow|, which are incompatible with the meta-arrow
management.


\begin{command}{\pgfsetarrows\texttt{\char`\{}\meta{start kind}|-|\meta{end kind}\texttt{\char`\}}}
  Calls |\pgfsetarrowsstart| for \meta{start kind} and
  |\pgfsetarrowsend| for \meta{end kind}.
\begin{codeexample}[]
\begin{tikzpicture}
  \pgfsetarrows{latex-to}
  \pgfsetlinewidth{1ex}
  \pgfpathmoveto{\pgfpointorigin}
  \pgfpathlineto{\pgfpoint{3.5cm}{2cm}}
  \pgfusepath{stroke}
  \useasboundingbox (-0.25,-0.25) rectangle (3.75,2.25);
\end{tikzpicture}
\end{codeexample} 
\end{command}


\subsection{Predefined Arrow Tip Kinds}

\label{standard-arrows}

The following arrow tip kinds are always defined:

{
\bigskip
\catcode`\|=12
\begin{tabular}{ll}
  \sarrow{stealth}{stealth} \\
  \sarrow{stealth reversed}{stealth reversed}  \\
  \sarrow{to}{to} \\
  \sarrow{to reversed}{to reversed}  \\
  \sarrow{latex}{latex} \\
  \sarrow{latex reversed}{latex reversed}  \\
  \index{*vbar@\protect\texttt{\protect\myvbar} arrow tip}%
  \index{Arrow tips!*vbar@\protect\texttt{\protect\myvbar}}
  \texttt{|-|}& yields thick  
  \begin{tikzpicture}[arrows={|-|},thick]
    \useasboundingbox (0pt,-0.5ex) rectangle (1cm,2ex);
    \draw (0,0) -- (1,0);
  \end{tikzpicture} and thin
  \begin{tikzpicture}[arrows={|-|},thin]
    \useasboundingbox (0pt,-0.5ex) rectangle (1cm,2ex);
    \draw (0,0) -- (1,0);
  \end{tikzpicture}
\end{tabular}
}

For further arrow tips, see page~\pageref{section-library-arrows}.

%%% Local Variables: 
%%% mode: latex
%%% TeX-master: "pgfmanual"
%%% End: 


���}	� �
����DEV��INO��SYN��SV~�pgfmanual-en-base-nodes.tex�J��J���J�j���P�_.j�

���}	� �
����DEV��INO��SYN��SV~�pgfmanual-en-base-matrices.tex�J��J���J�j���P�_.j�

��%�}	� �
����DEV��INO��SYN��SV~�pgfmanual-en-base-transformations.tex�J��J���J�j��P�_.j�
% Copyright 2006 by Till Tantau
%
% This file may be distributed and/or modified
%
% 1. under the LaTeX Project Public License and/or
% 2. under the GNU Free Documentation License.
%
% See the file doc/generic/pgf/licenses/LICENSE for more details.


\section{Patterns}

\label{section-patterns}

\subsection{Overview}

There are many ways of filling a path. First, you can fill it using a
solid color and this is also the fasted method. Second, you can also
fill it using a shading, which means that the color changes smoothly
between two (or more) different colors. Third, you can fill it using a
tiling pattern and it is explained in the following how this is done.

A tiling pattern can be imagined as a rectangular tile (hence the
name) on which a small picture is painted. There is not a single tile,
but (conceptually) an infinite number of tiles, all showing the same 
picture, and these tiles are arranged horizontally and vertically to
fill the plane. When you use a tiling pattern to fill a path, what
happens is that the path clips out a ``window'' through which we see
part of this infinite plane.

Patterns come in two versions: \emph{inherently colored patterns} and
\emph{form-only patterns}. (These are often called ``color patterns''
and ``uncolored patterns,'' but these names are misleading since
uncolored patterns do have a color and the color changes. As I said, 
the name is misleading\dots) An inherently colored pattern is just a
colored tile like, say, a red star with a black outline. A form-only
pattern can be imagined as a tile that is a kind of rubber stamp. When
this pattern is used, the stamp is used to print copies of the stamp
picture onto the plane, but we can use a different stamp color each
time we use a form-only pattern.

\pgfname\ provides a special support for patterns. You can declare a
pattern and then use it very much like a fill color. \pgfname\
directly maps patterns to the pattern facilities of the underlying
graphic languages (PostScript, \textsc{pdf}, and \textsc{svg}). This
means that filling a path using a pattern will be nearly as fast as if
you used a uniform color.

There are a number of pitfalls and restrictions when using
patterns. First, once a pattern has been declared, you cannot change
it anymore. In particular, it is not possible to enlarge it or change
the line width. Such flexibility would require that the repeating of
the pattern were not done by the graphic language, but on the
\pgfname\ level. This would make patterns orders of magnitude slower
to produce and to render.

Second, the phase of patterns is not well-defined, that is, it is not
clear where origin of the ``first'' tile is. To be more precise,
PostScript and \textsc{pdf} on the one hand and \textsc{svg} on the
other hand define the origin differently. PostScript and \textsc{pdf}
define a fixed origin that is independent of where the path lies. This
has the highly desirable effect that if you use the same pattern to
fill multiple paths, this has the same effect as if you used the
pattern to will a single path that is the union of all the paths. By
comparison, \textsc{svg} uses the upper-left (?) corner of the path to
be filled as the origin. However, the \textsc{svg} specification is a
bit vague on this question.


\subsection{Declaring a Pattern}

Before a pattern can be used, it must be declared. The following
command is used for this:

\begin{command}{\pgfdeclarepatternformonly%
    \marg{name}%
    \marg{lower left}%
    \marg{upper right}%
    \marg{tile size}%
    \marg{code}}
  This command declares a new form-only pattern. The \marg{name} is a
  name for later reference. The two parameters \marg{lower left} and
  \marg{upper right} must describe a bounding box that is large enough
  to encompass the complete tile.

  The size of a tile is given by \meta{tile size}, that is, a tile is
  a rectangle whose lower left   corner is the origin and whose upper
  right corner is given by \meta{tile size}. This might make you
  wonder why the second and third parameters are needed. First, the
  bounding box might be smaller than the tile size if the tile is
  larger than the picture on the tile. Second, the bounding box might
  be bigger, in which case the picture will ``bleed'' over the tile.

  The \meta{code} should be \pgfname\ code than can be protocolled. It
  should not contain any color code.

  
\begin{codeexample}[]
\pgfdeclarepatternformonly{stars}
{\pgfpointorigin}{\pgfpoint{1cm}{1cm}}
{\pgfpoint{1cm}{1cm}}
{
  \pgftransformshift{\pgfpoint{.5cm}{.5cm}}
  \pgfpathmoveto{\pgfpointpolar{0}{4mm}}
  \pgfpathlineto{\pgfpointpolar{144}{4mm}}
  \pgfpathlineto{\pgfpointpolar{288}{4mm}}
  \pgfpathlineto{\pgfpointpolar{72}{4mm}}
  \pgfpathlineto{\pgfpointpolar{216}{4mm}}
  \pgfpathclose%
  \pgfusepath{fill}
}
\begin{tikzpicture}
  \filldraw[pattern=stars] (0,0)   rectangle (1.5,2);  
  \filldraw[pattern=stars,pattern color=red]
                           (1.5,0) rectangle (3,2);  
\end{tikzpicture}
\end{codeexample}
\end{command}

\begin{command}{\pgfdeclarepatterninherentlycolored
    \marg{name}
    \marg{lower left}
    \marg{upper right}
    \marg{tile size}
    \marg{code}}
  This command works like |\pgfdeclarepatternuncolored|, only the
  pattern will have an inherent color. To set the color, you should
  use \pgfname's color commands, not the |\color| command, since this
  fill not be protocolled.
  
\begin{codeexample}[]
\pgfdeclarepatterninherentlycolored{green stars}
{\pgfpointorigin}{\pgfpoint{1cm}{1cm}}
{\pgfpoint{1cm}{1cm}}
{
  \pgfsetfillcolor{green!50!black}
  \pgftransformshift{\pgfpoint{.5cm}{.5cm}}
  \pgfpathmoveto{\pgfpointpolar{0}{4mm}}
  \pgfpathlineto{\pgfpointpolar{144}{4mm}}
  \pgfpathlineto{\pgfpointpolar{288}{4mm}}
  \pgfpathlineto{\pgfpointpolar{72}{4mm}}
  \pgfpathlineto{\pgfpointpolar{216}{4mm}}
  \pgfpathclose%
  \pgfusepath{stroke,fill}
}
\begin{tikzpicture}
  \filldraw[pattern=green stars] (0,0) rectangle (3,2);  
\end{tikzpicture}
\end{codeexample}
\end{command}


\subsection{Setting a Pattern}

Once a pattern has been declared, it can be used.

\begin{command}{\pgfsetfillpattern\marg{name}\marg{color}}
  This command specifies that paths that are filled should be filled
  with the ``color'' by the pattern \meta{name}. For an inherently
  colored pattern, the \meta{color} parameter is ignored. For
  form-only patterns, the \meta{color} parameter specified the color
  to be used for the pattern.
\begin{codeexample}[]
\begin{tikzpicture}
  \pgfsetfillpattern{stars}{red}
  \filldraw (0,0) rectangle (1.5,2);  

  \pgfsetfillpattern{green stars}{red}
  \filldraw (1.5,0) rectangle (3,2);  
\end{tikzpicture}
\end{codeexample} 
\end{command}



%%% Local Variables: 
%%% mode: latex
%%% TeX-master: "pgfmanual"
%%% End: 

%
���}	� �
����DEV��INO��SYN��SV~�pgfmanual-en-base-images.tex�J��J���J�j���P�_.j�

���}	� �
����DEV��INO��SYN��SV~�pgfmanual-en-base-external.tex�J��J���J�j���P�_.j�
% Copyright 2006 by Till Tantau
%
% This file may be distributed and/or modified
%
% 1. under the LaTeX Project Public License and/or
% 2. under the GNU Free Documentation License.
%
% See the file doc/generic/pgf/licenses/LICENSE for more details.


\section{Creating Plots}

\label{section-plots}

This section describes the |plot| module.

\begin{pgfmodule}{plot}
  This module provides a set of commands that are intended to make it
  reasonably easy to plot functions using \pgfname. It is loaded
  automatically by |pgf|, but you can load it manually if you have
  only included |pgfcore|.  
\end{pgfmodule}


\subsection{Overview}

There are different reasons for using \pgfname\ for creating plots
rather than some more powerful program such as \textsc{gnuplot} or
\textsc{mathematica}, as discussed in
Section~\ref{section-why-pgname-for-plots}. So, let us assume that --
for whatever reason -- you wish to use \pgfname\ for generating a plot.

\pgfname\ (conceptually) uses a two-stage process for generating
plots. First, a \emph{plot stream} must be produced. This stream
consists (more or less) of a large number of coordinates. Second a 
\emph{plot handler} is applied to the stream. A plot handler ``does
something'' with the stream. The standard handler will issue
line-to operations to the coordinates in the stream. However, a
handler might also try to issue appropriate curve-to operations in
order to smooth the curve. A handler may even do something else
entirely, like writing each coordinate to another stream, thereby
duplicating the original stream.

Both for the creation of streams and the handling of streams different
sets of commands exist. The commands for creating streams start with
|\pgfplotstream|, the commands for setting the handler start with
|\pgfplothandler|.



\subsection{Generating Plot Streams}

\subsubsection{Basic Building Blocks of Plot Streams}
A \emph{plot stream} is a (long) sequence of the following three
commands:
\begin{enumerate}
\item
  |\pgfplotstreamstart|,
\item
  |\pgfplotstreampoint|, and
\item
  |\pgfplotstreamend|.
\end{enumerate}
Between calls of these commands arbitrary other code may be
called. Obviously, the stream should start with the first command and
end with the last command. Here is an example of a plot stream:
\begin{codeexample}[code only]
\pgfplotstreamstart
\pgfplotstreampoint{\pgfpoint{1cm}{1cm}}
\newdimen\mydim
\mydim=2cm
\pgfplotstreampoint{\pgfpoint{\mydim}{2cm}}
\advance \mydim by 3cm
\pgfplotstreampoint{\pgfpoint{\mydim}{2cm}}
\pgfplotstreamend
\end{codeexample}

\begin{command}{\pgfplotstreamstart}
  This command signals that a plot stream starts. The effect of this
  command is to call the internal command |\pgf@plotstreamstart|,
  which is set by the current plot handler to do whatever needs to be
  done at the beginning of the plot.
\end{command}

\begin{command}{\pgfplotstreampoint\marg{point}}
  This command adds a \meta{point} to the current plot stream. The
  effect of this command is to call the internal command |\pgf@plotstreampoint|,
  which is also set by the current plot handler. This command should
  now ``handle'' the point in some sensible way. For example, a
  line-to command might be issued for the point.
\end{command}

\begin{command}{\pgfplotstreamend}
  This command signals that a plot stream ends. It calls
  |\pgf@plotstreamend|, which should now do any necessary ``cleanup.''
\end{command}

Note that plot streams are not buffered, that is, the different points
are handled immediately. However, using the recording handler, it is
possible to record a stream.

\subsubsection{Commands That Generate Plot Streams}

Plot streams can be created ``by hand'' as in the earlier
example. However, most of the time the coordinates will be produced
internally by some command. For example, the |\pgfplotxyfile| reads a
file and converts it into a plot stream.

\begin{command}{\pgfplotxyfile\marg{filename}}
  This command will try to open the file \meta{filename}. If this
  succeeds, it will convert the file contents into a plot stream as
  follows: A |\pgfplotstreamstart| is issued. Then, each nonempty line
  of the file should start with two numbers separated by a space, such
  as |0.1 1| or |100 -.3|. Anything following the numbers is ignored.

  Each pair \meta{x} and \meta{y} of numbers is converted into one
  plot stream point in the xy-coordinate system. Thus, a line like
\begin{codeexample}[code only]
2 -5 some text
\end{codeexample}
  is turned into 
\begin{codeexample}[code only]
\pgfplotstreampoint{\pgfpointxy{2}{-5}}
\end{codeexample}

  The two characters |%| and |#| are also allowed in a file and they
  are both treated as comment characters. Thus, a line starting with
  either of them is empty and, hence, ignored.

  When the file has been read completely, |\pgfplotstreamend| is
  called. 
\end{command}


\begin{command}{\pgfplotxyzfile\marg{filename}}
  This command works like |\pgfplotxyfile|, only \emph{three} numbers
  are expected on each non-empty line. They are converted into points
  in the xyz-coordinate system. Consider, the following file:
\begin{codeexample}[code only]
% Some comments
# more comments
2 -5  1 first entry
2 -.2 2 second entry
2 -5  2 third entry
\end{codeexample}
  It is turned into the following stream:
\begin{codeexample}[code only]
\pgfplotstreamstart
\pgfplotstreampoint{\pgfpointxyz{2}{-5}{1}}
\pgfplotstreampoint{\pgfpointxyz{2}{-.2}{2}}
\pgfplotstreampoint{\pgfpointxyz{2}{-5}{2}}
\pgfplotstreamend
\end{codeexample}
\end{command}


Currently, there is no command that can decide automatically whether
the xy-coordinate system should be used or whether the xyz-system
should be used. However, it would not be terribly difficult to write a
``smart file reader'' that parses coordinate files a bit more
intelligently. 


\begin{command}{\pgfplotfunction\marg{variable}\marg{sample list}\marg{point}} 
  This command will produce coordinates by iterating the
  \meta{variable} over all values in \meta{sample list}, which should
  be a list in the |\foreach| syntax. For each value of
  \meta{variable}, the \meta{point} is evaluated and the resulting
  coordinate is inserted into the plot stream.

\begin{codeexample}[]
\begin{tikzpicture}[x=3.8cm/360]
  \pgfplothandlerlineto
  \pgfplotfunction{\x}{0,5,...,360}{\pgfpointxy{\x}{sin(\x)+sin(3*\x)}}
  \pgfusepath{stroke}  
\end{tikzpicture}
\end{codeexample}

\begin{codeexample}[]
\begin{tikzpicture}[y=3cm/360]
  \pgfplothandlerlineto
  \pgfplotfunction{\y}{0,5,...,360}{\pgfpointxyz{sin(2*\y)}{\y}{cos(2*\y)}}
  \pgfusepath{stroke}  
\end{tikzpicture}
\end{codeexample}

  Be warnded that if the expressions that need to evaluated for each
  point are complex, then this command can be very slow.
\end{command}



\begin{command}{\pgfplotgnuplot\oarg{prefix}\marg{function}}
  This command will ``try'' to call the \textsc{gnuplot} program to
  generate the coordinates of the \meta{function}. In detail, the
  following happens:

  This command works with two files: \meta{prefix}|.gnuplot| and
  \meta{prefix}|.table|.  If the optional argument \meta{prefix} is
  not given, it is set to |\jobname|.

  Let us start with the situation where none of these files
  exists. Then \pgfname\ will first generate the file
  \meta{prefix}|.gnuplot|. In this file it writes
\begin{codeexample}[code only]
set terminal table; set output "#1.table"; set format "%.5f"
\end{codeexample}
  where |#1| is replaced by \meta{prefix}. Then, in a second line, it
  writes the text \meta{function}.

  Next, \pgfname\ will try to invoke the program |gnuplot| with the
  argument \meta{prefix}|.gnuplot|. This call may or may not succeed,
  depending on whether the |\write18| mechanism (also known as
  shell escape) is switched on and whether the |gnuplot| program is
  available.

  Assuming that the call succeeded, the next step is to invoke
  |\pgfplotxyfile| on the file \meta{prefix}|.table|; which is exactly
  the file that has just been created by |gnuplot|.
  
\begin{codeexample}[]
\begin{tikzpicture}
  \draw[help lines] (0,-1) grid (4,1);
  \pgfplothandlerlineto
  \pgfplotgnuplot[plots/pgfplotgnuplot-example]{plot [x=0:3.5] x*sin(x)}
  \pgfusepath{stroke}
\end{tikzpicture}
\end{codeexample}

  The more difficult situation arises when the |.gnuplot| file exists,
  which will be the case on the second run of \TeX\ on the \TeX\
  file. In this case \pgfname\ will read this file and check whether
  it contains exactly what \pgfname\ ``would have written'' into
  this file. If this is not the case, the file contents is overwritten
  with what ``should be there'' and, as above, |gnuplot| is invoked to
  generate a new |.table| file. However, if the file contents is ``as
  expected,'' the external |gnuplot| program is \emph{not}
  called. Instead, the \meta{prefix}|.table| file is immediately
  read.

  As explained in Section~\ref{section-tikz-gnuplot}, the net effect
  of the above mechanism is that |gnuplot| is called as little as
  possible and that when you pass along the |.gnuplot| and |.table|
  files with your |.tex| file to someone else, that person can
  \TeX\ the |.tex| file without having |gnuplot| installed and without
  having the |\write18| mechanism switched on.
\end{command}



\subsection{Plot Handlers}

\label{section-plot-handlers}

A \emph{plot handler}  prescribes what ``should be done'' with a
plot stream. You must set the plot handler before the stream starts.
The following commands install the most basic plot handlers; more plot
handlers are defined in the file |pgflibraryplothandlers|, which is
documented in Section~\ref{section-library-plothandlers}.

All plot handlers work by setting redefining the following three
macros: |\pgf@plotstreamstart|, |\pgf@plotstreampoint|, and
|\pgf@plotstreamend|.

\begin{command}{\pgfplothandlerlineto}
  This handler will issue a |\pgfpathlineto| command for each point of
  the plot, \emph{except} possibly for the first. What happens with
  the first point can be specified using the two commands described
  below.

\begin{codeexample}[]
\begin{pgfpicture}
  \pgfpathmoveto{\pgfpointorigin}
  \pgfplothandlerlineto
  \pgfplotstreamstart
  \pgfplotstreampoint{\pgfpoint{1cm}{0cm}}
  \pgfplotstreampoint{\pgfpoint{2cm}{1cm}}
  \pgfplotstreampoint{\pgfpoint{3cm}{2cm}}
  \pgfplotstreampoint{\pgfpoint{1cm}{2cm}}
  \pgfplotstreamend
  \pgfusepath{stroke}
\end{pgfpicture}
\end{codeexample}
\end{command}

\begin{command}{\pgfsetmovetofirstplotpoint}
  Specifies that the line-to plot handler (and also some other plot 
  handlers) should issue a move-to command for the
  first point of the plot instead of a line-to. This will start a new
  part of the current path, which is not always, but often,
  desirable. This is the default.
\end{command}

\begin{command}{\pgfsetlinetofirstplotpoint}
  Specifies that  plot handlers should issue a line-to command for the
  first point of the plot.

\begin{codeexample}[]
\begin{pgfpicture}
  \pgfpathmoveto{\pgfpointorigin}
  \pgfsetlinetofirstplotpoint
  \pgfplothandlerlineto
  \pgfplotstreamstart
  \pgfplotstreampoint{\pgfpoint{1cm}{0cm}}
  \pgfplotstreampoint{\pgfpoint{2cm}{1cm}}
  \pgfplotstreampoint{\pgfpoint{3cm}{2cm}}
  \pgfplotstreampoint{\pgfpoint{1cm}{2cm}}
  \pgfplotstreamend
  \pgfusepath{stroke}
\end{pgfpicture}
\end{codeexample}
\end{command}

\begin{command}{\pgfplothandlerdiscard}
  This handler will simply throw away the stream.
\end{command}

\begin{command}{\pgfplothandlerrecord\marg{macro}}
  When this handler is installed, each time a plot stream command is
  called, this command will be appended to \meta{macros}. Thus, at
  the end of the stream, \meta{macro} will contain all the
  commands that were issued on the stream. You can then install
  another handler and invoke \meta{macro} to ``replay'' the stream
  (possibly many times).
 
\begin{codeexample}[]
\begin{pgfpicture}
  \pgfplothandlerrecord{\mystream}
  \pgfplotstreamstart
  \pgfplotstreampoint{\pgfpoint{1cm}{0cm}}
  \pgfplotstreampoint{\pgfpoint{2cm}{1cm}}
  \pgfplotstreampoint{\pgfpoint{3cm}{1cm}}
  \pgfplotstreampoint{\pgfpoint{1cm}{2cm}}
  \pgfplotstreamend
  \pgfplothandlerlineto
  \mystream
  \pgfplothandlerclosedcurve
  \mystream
  \pgfusepath{stroke}
\end{pgfpicture}
\end{codeexample} 
\end{command}

%%% Local Variables: 
%%% mode: latex
%%% TeX-master: "pgfmanual"
%%% End: 

% Copyright 2006 by Till Tantau
%
% This file may be distributed and/or modified
%
% 1. under the LaTeX Project Public License and/or
% 2. under the GNU Free Documentation License.
%
% See the file doc/generic/pgf/licenses/LICENSE for more details.


\section{Layered Graphics}

\label{section-layers}

\subsection{Overview}

\pgfname\ provides a layering mechanism for composing graphics from
multiple layers. (This mechanism is not be confused with the
conceptual ``software layers'' the \pgfname\ system is composed of.)
Layers are often used in graphic programs. The idea is that you can
draw on the different layers in any order. So you might start drawing
something on the ``background'' layer, then something on the
``foreground'' layer, then something on the ``middle'' layer, and then
something on the background layer once more, and so on. At the end, no
matter in which ordering you drew on the different layers, the layers
are ``stacked on top of each other'' in a fixed ordering to produce
the final picture. Thus, anything drawn on the middle layer would come
on top of everything of the background layer.

Normally, you do not need to use different layers since you will have
little trouble ``ordering'' your graphic commands in such a way that
layers are superfluous. However, in certain situations you only
``know'' what you should draw behind something else after the
``something else'' has been drawn.

For example, suppose you wish to draw a yellow background behind your
picture. The background should be as large as the bounding box of the
picture, plus a little border. If you know the size of the bounding box
of the picture at its beginning, this is easy to accomplish. However,
in general this is not the case and you need to create a
``background'' layer in addition to the standard ``main'' layer. Then,
at the end of the picture, when the bounding box has been established,
you can add a rectangle of the appropriate size to the picture.



\subsection{Declaring Layers}

In \pgfname\ layers are referenced using names. The standard layer,
which is a bit special in certain ways, is called |main|. If nothing
else is specified, all graphic commands are added to the |main|
layer. You can declare a new layer using the following command:

\begin{command}{\pgfdeclarelayer\marg{name}}
  This command declares a layer named \meta{name} for later
  use. Mainly, this will setup some internal bookkeeping.
\end{command}

The next step toward using a layer is to tell \pgfname\ which layers
will be part of the actual picture and which will be their
ordering. Thus, it is possible to have more layers declared than are
actually used.

\begin{command}{\pgfsetlayers\marg{layer list}}
  This command, which should be used \emph{outside} a |{pgfpicture}|
  environment, tells \pgfname\ which layers will be used in
  pictures. They are stacked on top of each other in the order
  given. The layer |main| should always be part of the list. Here is
  an example:
\begin{codeexample}[code only]
\pgfdeclarelayer{background}
\pgfdeclarelayer{foreground}  
\pgfsetlayers{background,main,foreground}
\end{codeexample}
\end{command}


\subsection{Using Layers}

Once the layers of your picture have been declared, you can start to
``fill'' them. As said before, all graphics commands are normally
added to the |main| layer. Using the |{pgfonlayer}| environment, you
can tell \pgfname\ that certain commands should, instead, be added to
the given layer.

\begin{environment}{{pgfonlayer}\marg{layer name}}
  The whole \meta{environment contents} is added to the layer with the
  name \meta{layer name}. This environment can be used anywhere inside
  a picture. Thus, even if it is used inside a |{pgfscope}| or a \TeX\
  group, the contents will still be added to the ``whole'' picture.
  Using this environment multiple times inside the same picture will
  cause the \meta{environment contents} to accumulate.

  \emph{Note:} You can \emph{not} add anything to the |main| layer
  using this environment. The only way to add anything to the main
  layer is to give graphic commands outside all |{pgfonlayer}|
  environments. 

\begin{codeexample}[]
\pgfdeclarelayer{background layer}
\pgfdeclarelayer{foreground layer}
\pgfsetlayers{background layer,main,foreground layer}
\begin{tikzpicture}
  % On main layer:
  \fill[blue] (0,0) circle (1cm);
  
  \begin{pgfonlayer}{background layer}
    \fill[yellow] (-1,-1) rectangle (1,1);
  \end{pgfonlayer}
  
  \begin{pgfonlayer}{foreground layer}
    \node[white] {foreground};
  \end{pgfonlayer}
  
  \begin{pgfonlayer}{background layer}
    \fill[black] (-.8,-.8) rectangle (.8,.8);
  \end{pgfonlayer}

  % On main layer again:
  \fill[blue!50] (-.5,-1) rectangle (.5,1);
\end{tikzpicture}
\end{codeexample}
\end{environment}

\begin{plainenvironment}{{pgfonlayer}\marg{layer name}}
  This is the plain \TeX\ version of the environment.
\end{plainenvironment}

\begin{contextenvironment}{{pgfonlayer}\marg{layer name}}
  This is the Con\TeX t version of the environment.
\end{contextenvironment}






%%% Local Variables: 
%%% mode: latex
%%% TeX-master: "pgfmanual"
%%% End: 


���}	� �
����DEV��INO��SYN��SV~�pgfmanual-en-base-shadings.tex�J��J���J�j��P�_.j�
% Copyright 2006 by Till Tantau
%
% This file may be distributed and/or modified
%
% 1. under the LaTeX Project Public License and/or
% 2. under the GNU Free Documentation License.
%
% See the file doc/generic/pgf/licenses/LICENSE for more details.


\section{Transparency}

\label{section-transparency}


For an introduction to the notion of transparency, fadings, and
transparency groups, please consult Section~\ref{section-tikz-transparency}. 


\subsection{Specifying a Uniform Opacity}

Specifying a stroke and/or fill opacity is quite easy.

\begin{command}{\pgfsetstrokeopacity\marg{value}}
  Sets the opacity of stroking operations. The \meta{value} should be
  a number between |0| and |1|, where |1| means ``fully opaque'' and
  |0| means ``fully transparent.'' A value like |0.5| will cause paths
  to be stroked in a semitransparent way.
  
\begin{codeexample}[]
\begin{pgfpicture}
  \pgfsetlinewidth{5mm}
  \color{red}
  \pgfpathcircle{\pgfpoint{0cm}{0cm}}{10mm} \pgfusepath{stroke}
  \color{black}
  \pgfsetstrokeopacity{0.5}
  \pgfpathcircle{\pgfpoint{1cm}{0cm}}{10mm} \pgfusepath{stroke}
\end{pgfpicture}
\end{codeexample}
\end{command}


\begin{command}{\pgfsetfillopacity\marg{value}}
  Sets the opacity of filling operations. As for stroking, the
  \meta{value} should be a number between |0| and~|1|.

  The ``filling transparency'' will also be used for text and images.  
  
\begin{codeexample}[]
\begin{tikzpicture}
  \pgfsetfillopacity{0.5}
  \fill[red]   (90:1cm)  circle (11mm);
  \fill[green] (210:1cm) circle (11mm);
  \fill[blue]  (-30:1cm) circle (11mm);
\end{tikzpicture}
\end{codeexample}
\end{command}

Note the following effect: If you setup a certain opacity for stroking
or filling and you stroke or fill the same area twice, the effect
accumulates:

\begin{codeexample}[]
\begin{tikzpicture}
  \pgfsetfillopacity{0.5}
  \fill[red] (0,0) circle (1);
  \fill[red] (1,0) circle (1);
\end{tikzpicture}
\end{codeexample}

Often, this is exactly what you intend, but not always. You can use
transparency groups, see the end of this section, to change this.


\subsection{Specifying a Fading}

The method used by \pgfname\ for specifying fadings is quite
general: You ``paint'' the fading using any of the standard graphics
commands. In more detail: You create a normal picture, which may even
contain text, image, and shadings. Then, you create a fading based on
this picture. For this, the \emph{luminosity} of each pixel of the
picture is analysed (the brighter the pixel, the higher the luminosity
-- a black pixel has luminosity $0$, a white pixel has luminosity $1$,
a gray pixel has some intermediate value as does a red pixel). Then,
when the fading is used, the luminosity of the pixel determines the
opacity of the fading at that position. Positions in the fading where
the picture was black will be completely transparent, positions where
the picture was white will be completely opaque. Positions that have
not been painted at all in the picture are always completely
transparent.


\begin{command}{\pgfdeclarefading\marg{name}\marg{contents}}
  This command declare a fading named \meta{name} for later use. The
  ``picture'' on which the fading is based is given by the
  \meta{contents}. This \meta{contents} is normally typeset in a \TeX\
  box. The resulting box is then used as the ``picture.'' In
  particular, inside the \meta{contents} you must explicitly open a
  |{pgfpicture}| environment if you wish to use \pgfname\ commands.

  Let's start with an easy example. Our first fading picture is just
  some text:
\begin{codeexample}[]
\pgfdeclarefading{fading1}{\color{white}Ti\emph{k}Z}    
\begin{tikzpicture}
  \fill [black!20] (0,0) rectangle (2,2);
  \fill [black!30] (0,0) arc (180:0:1);
  \pgfsetfading{fading1}{\pgftransformshift{\pgfpoint{1cm}{1cm}}}
  \fill [red] (0,0) rectangle (2,2);
\end{tikzpicture}
\end{codeexample}
  What's happening here? The ``fading picture'' is mostly transparent,
  except for the pixels that are part of the word Ti\emph{k}Z. Now,
  these pixels are \emph{white} and, thus, have a high
  luminosity. This in turn means that these pixels of the fading will
  be highly opaque. For this reason, only those pixels of the big red
  rectangle ``shine through'' that are at the positions of these
  opaque pixels.

  It is somewhat counter-intuitive that the white pixels in a fading
  picture are opaque in a fading. For this reason, the color
  |pgftransparent| is defined to be the same as |black|. This allows
  one to write |pgftransparent| for completely transparent parts of a
  fading picture and |pgftransparent!0| for the opaque parts and
  things liek |pgftransparent!20| for parts that are 20\%
  transparent.

  Furthermore, the color |pgftransparent!0| (which is the same as
  white and which corresponds to completely opaque) is installed at
  the beginning of a fading picture. Thus, in the above example the
  |\color{white}| was not really necessary.

  Next, let us create a fading that gets more and more transparent as
  we go from left to right. For this, we put a shading inside the
  fading picture that has the color |pgftransparent!0| at the
  left-hand side and the color |pgftransparent!100| at the right-hand
  side. 
\begin{codeexample}[]
\pgfdeclarefading{fading2}
{\tikz \shade[left color=pgftransparent!0,
              right color=pgftransparent!100] (0,0) rectangle (2,2);}    
\begin{tikzpicture}
  \fill [black!20] (0,0) rectangle (2,2);
  \fill [black!30] (0,0) arc (180:0:1);
  \pgfsetfading{fading2}{\pgftransformshift{\pgfpoint{1cm}{1cm}}}
  \fill [red] (0,0) rectangle (2,2);
\end{tikzpicture}
\end{codeexample}

  In our final example, we create a fading that is based on a radial
  shading.
\begin{codeexample}[]
\pgfdeclareradialshading{myshading}{\pgfpointorigin}
{
  color(0mm)=(pgftransparent!0);
  color(5mm)=(pgftransparent!0);
  color(8mm)=(pgftransparent!100);
  color(15mm)=(pgftransparent!100)
}
\pgfdeclarefading{fading3}{\pgfuseshading{myshading}}
\begin{tikzpicture}
  \fill [black!20] (0,0) rectangle (2,2);
  \fill [black!30] (0,0) arc (180:0:1);
  \pgfsetfading{fading3}{\pgftransformshift{\pgfpoint{1cm}{1cm}}}
  \fill [red] (0,0) rectangle (2,2);
\end{tikzpicture}
\end{codeexample}
\end{command}

After having declared a fading, we can use it. As for shadings, there
are two different commands for using fadings:

\begin{command}{\pgfsetfading\marg{name}\marg{transformations}}
  This command sets the graphic state parameter ``fading'' to a
  previously defined fading \meta{name}. This graphic state works like
  other graphic states, that is, is persists till the end of the
  current scope or until a different transparency setting is chosen.

  When the fading is installed, it will be centered on the origin with
  its natural size. Anything outside the fading pictures's original
  bounding box will be transparent and, thus, the fading effectively
  clips against this bounding box.

  The \meta{transformations} are applied to the fading before it is
  used. They contain normal \pgfname\ transformation commands like
  |\pgftransformshift|. You can also scale the fading using this
  command. Note, however, that the transformation needs to be inverted
  internally, which may result in inaccuracies and the following
  graphics may be slightly distorted if you use a strong
  \meta{transformation}.
\begin{codeexample}[]
\pgfdeclarefading{fading2}
{\tikz \shade[left color=pgftransparent!0,
              right color=pgftransparent!100] (0,0) rectangle (2,2);}    
\begin{tikzpicture}
  \fill [black!20] (0,0) rectangle (2,2);
  \fill [black!30] (0,0) arc (180:0:1);
  \pgfsetfading{fading2}{}
  \fill [red] (0,0) rectangle (2,2);
\end{tikzpicture}
\end{codeexample}
\begin{codeexample}[]
\begin{tikzpicture}
  \fill [black!20] (0,0) rectangle (2,2);
  \fill [black!30] (0,0) arc (180:0:1);
  \pgfsetfading{fading2}{\pgftransformshift{\pgfpoint{1cm}{1cm}}
                         \pgftransformrotate{20}}
  \fill [red] (0,0) rectangle (2,2);
\end{tikzpicture}
\end{codeexample}
\end{command}

\begin{command}{\pgfsetfadingforcurrentpath\marg{name}\marg{transformations}}
  This command works like |\pgfsetfading|, but the fading is scaled
  are transformed according to the following rules:
  \begin{enumerate}
  \item
    If the current path is empty, the command has the same effect as
    |\pgfsetfading|. 
  \item
    Otherwise it is assumed that the fading has a size of 100bp times
    100bp. 
  \item
    The fading is resized and shiften (using appropriate
    transformations) such that the position
    $(25\mathrm{bp},25\mathrm{bp})$ lies at the lower-left corner of
    the current path and the position $(75\mathrm{bp},75\mathrm{bp})$
    lies at the upper-right corner of the current path.
  \end{enumerate}
  Note that these rules are the same as the ones used in
  |\pgfshadepath| for shadings. After these transformations, the
  \meta{transformations} are executed (typically a rotation).
\begin{codeexample}[]
\pgfdeclarehorizontalshading{shading}{100bp}
{ color(0pt)=(transparent!0);    color(25bp)=(transparent!0);
  color(75bp)=(transparent!100); color(100bp)=(transparent!100)}

\pgfdeclarefading{fading}{\pgfuseshading{shading}}

\begin{tikzpicture}
  \fill [black!20] (0,0) rectangle (2,2);
  \fill [black!30] (0,0) arc (180:0:1);

  \pgfpathrectangle{\pgfpointorigin}{\pgfpoint{2cm}{1cm}}
  \pgfsetfadingforcurrentpath{fading}{}
  \pgfusepath{discard}
  
  \fill [red] (0,0) rectangle (2,1);

  \pgfpathrectangle{\pgfpoint{0cm}{1cm}}{\pgfpoint{2cm}{1cm}}
  \pgfsetfadingforcurrentpath{fading}{\pgftransformrotate{90}}
  \pgfusepath{discard}

  \fill [red] (0,1) rectangle (2,2);
\end{tikzpicture}
\end{codeexample}

\end{command}

\subsection{Transparency Groups}

Transparency groups are declared using the following commands.

\begin{environment}{{pgftransparencygroup}}
  This environment should only be used inside a |{pgfpicture}|. It has
  the following effect:
  \begin{enumerate}
  \item The \meta{environment contents} is stroked/filled
    ``ignoring any outside transparency.'' This means, all previous
    transparency settings are ignored (you can still set transparency
    inside the group, but never mind). This means that if in the
    \meta{environment contents} you stroke a pixel three times in
    black, it is just black. Stroking it white afterwards yields a
    white pixel, and so on.
  \item When the group is finished, it is painted as a whole. The 
    \emph{fill} transparency settings are now applied to the resulting
    picutre. For instance, the pixel that has been painted three times
    in black and once in white is just white at the end, so this white
    color will be blended with whatever is ``behind'' the group on the
    page.
  \end{enumerate}

  Note that, depending on the driver, \pgfname\ may have to guess the
  size of the contents of the transparency group (because such a group
  is put in an XForm in \textsc{pdf} and a bounding box must be
  supplied). \pgfname\ will use normally use the size of the picture's
  bounding box at the end of the transparency group plus a safety
  margin of 1cm. Under normal circumstances, this will work nicely
  since the picture's bounding box contains everything
  anyway. However, if you have switched off the picture size tracking
  or if you are using canvas transformations, you may have to make
  sure that the bounding box is big enough. The trick is to locallly
  create a picture that is ``large enough'' and then insert this
  picture into the main picture while ignoring the size. The following
  example shows how this is done:

  
\begin{codeexample}[]
\begin{tikzpicture}
  \draw [help lines] (0,0) grid (2,2);

  % Stuff outside the picture, but still in a transparency group.
  \node [left,overlay] at (0,1) {
    \begin{tikzpicture}
      \pgfsetfillopacity{0.5}
      \pgftransparencygroup
      \node at (2,0) [forbidden sign,line width=2ex,draw=red,fill=white]
        {Smoking};
      \endpgftransparencygroup
    \end{tikzpicture}  
  };
\end{tikzpicture}
\end{codeexample}


\begin{plainenvironment}{{pgftransparencygroup}}
  Plain \TeX\ version of the |{pgftransparencygroup}| environment.
\end{plainenvironment}

\begin{contextenvironment}{{pgftransparencygroup}}
  This is the Con\TeX t version of the environment.
\end{contextenvironment}

\end{environment}



%%% Local Variables: 
%%% mode: latex
%%% TeX-master: "pgfmanual"
%%% End: 


���}	� �
����DEV��INO��SYN��SV~�pgfmanual-en-base-quick.tex�J��J���J�j���P�_.j�




\part{The System Layer}
\label{part-system}

{\Large \emph{by Till Tantau}}


\bigskip
\noindent
This part describes the low-level interface of \pgfname, called the
\emph{system layer}. This interface provides a complete abstraction of
the internals of the underlying drivers. 

Unless you intend to port \pgfname\ to another driver or unless you intend
to write your own optimized frontend, you need not read this part.

In the following it is assumed that you are familiar with the basic
workings of the |graphics| package and that you know what
\TeX-drivers are and how they work.

\vskip1cm
\begin{codeexample}[graphic=white]
\begin{tikzpicture}
  [shorten >=1pt,->,
   vertex/.style={circle,fill=black!25,minimum size=17pt,inner sep=0pt}]
  
  \foreach \name/\x in {s/1, 2/2, 3/3, 4/4, 15/11, 16/12, 17/13, 18/14, 19/15, t/16}
    \node[vertex] (G-\name) at (\x,0) {$\name$};

  \foreach \name/\angle/\text in {P-1/234/5, P-2/162/6, P-3/90/7, P-4/18/8, P-5/-54/9}
    \node[vertex,xshift=6cm,yshift=.5cm] (\name) at (\angle:1cm) {$\text$};
  
  \foreach \name/\angle/\text in {Q-1/234/10, Q-2/162/11, Q-3/90/12, Q-4/18/13, Q-5/-54/14}
    \node[vertex,xshift=9cm,yshift=.5cm] (\name) at (\angle:1cm) {$\text$};

  \foreach \from/\to in {s/2,2/3,3/4,3/4,15/16,16/17,17/18,18/19,19/t}
    \draw (G-\from) -- (G-\to);  

  \foreach \from/\to in {1/2,2/3,3/4,4/5,5/1,1/3,2/4,3/5,4/1,5/2}
    { \draw (P-\from) -- (P-\to); \draw (Q-\from) -- (Q-\to); }

  \draw (G-3) .. controls +(-30:2cm) and +(-150:1cm) .. (Q-1);
  \draw (Q-5) -- (G-15);
\end{tikzpicture}
\end{codeexample}


�� �}	� �
����DEV��INO��SYN��SV~�pgfmanual-en-pgfsys-overview.tex�J��J̀�J�j��PP�_.j�

�� �}	� �
����DEV��INO��SYN��SV~�pgfmanual-en-pgfsys-commands.tex�J��J̀�J�j��NP�_.j�

���}	� �
����DEV��INO��SYN��SV~�pgfmanual-en-pgfsys-paths.tex�J��J̀�J�j��RP�_.j�

�� �}	� �
����DEV��INO��SYN��SV~�pgfmanual-en-pgfsys-protocol.tex�J��J΀�J�j��TP�_.j�



\part{References and Index}

\vskip1cm
\begin{codeexample}[graphic=white]
\begin{tikzpicture}
  \draw[line width=0.3cm,color=red!30,line cap=round,line join=round] (0,0)--(2,0)--(2,5);
  \draw[help lines] (-2.5,-2.5) grid (5.5,7.5);
  \draw[very thick] (1,-1)--(-1,-1)--(-1,1)--(0,1)--(0,0)--
    (1,0)--(1,-1)--(3,-1)--(3,2)--(2,2)--(2,3)--(3,3)--
    (3,5)--(1,5)--(1,4)--(0,4)--(0,6)--(1,6)--(1,5)
    (3,3)--(4,3)--(4,5)--(3,5)--(3,6)
    (3,-1)--(4,-1);
  \draw[below left] (0,0) node(s){$s$};
  \draw[below left] (2,5) node(t){$t$};
  \fill (0,0) circle (0.06cm) (2,5) circle (0.06cm);
  \draw[->,rounded corners=0.2cm,shorten >=2pt]
    (1.5,0.5)-- ++(0,-1)-- ++(1,0)-- ++(0,2)-- ++(-1,0)-- ++(0,2)-- ++(1,0)--
    ++(0,1)-- ++(-1,0)-- ++(0,-1)-- ++(-2,0)-- ++(0,3)-- ++(2,0)-- ++(0,-1)--
    ++(1,0)-- ++(0,1)-- ++(1,0)-- ++(0,-1)-- ++(1,0)-- ++(0,-3)-- ++(-2,0)--
    ++(1,0)-- ++(0,-3)-- ++(1,0)-- ++(0,-1)-- ++(-6,0)-- ++(0,3)-- ++(2,0)--
    ++(0,-1)-- ++(1,0);
\end{tikzpicture}
\end{codeexample}

\printindex

\end{document}



%%% Local Variables: 
%%% mode: latex
%%% TeX-master: "~/texmf/tex/generic/pgf/doc/pgf/version-for-pdftex/en/pgfmanual"
%%% End: 
