\section{The Semi-Markov Process Model}
\label{sec:SMPModel}

This paper follows the general formulation of the continuous-time discrete-state semi-Markov process model as developed in \cite{HowardA} and \cite{HowardB}. 

Let the model have $N$ states. Let $f_{ij}(t)$ and $F_{ij}(t)$ represent the   $pdf$ and $cdf$ respectively of the event corresponding to the transition from state $i$ to state $j$ at time $t$. 

Assume that the process is in state $i$. From this state, there could be $k$ different states to which the process could transit to in a single step. These states could be completion of a repair, further degradation of the system or a failure mode with an underlying failure mechanism. Also assumed in this model is that all these $k$ possibilities are independent of the occurrence of each other. At a time instant $t$, the process chooses only one state from these choices such that the time to be spent in the current state $i$ is the minimum before instantaneously jumping to the chosen state. The probability that the next state is $j$ and not any other state $k$ reachable from $i$ is given by:

\begin{align}
\label{eq:CompRisk}
c_{ij}(t) =  f_{ij} (t)\prod\limits_{k \ne j} {(1 - F_{ik} (t))} 
\end{align}

For $N$=2, $c_{ij}(t) =  f_{ij} (t)$. The matrix $C(t)=[ c_{ij}(t) ]$ is called the kernel or core of the semi-Markov process model and
\begin{align}
\label{eq:waiting}
w_i(t) = \sum\limits_{j=1}^{N}{c_{ij}(t)}
\end{align}
is called the waiting time distribution for the state $i$. It represents the probability that the system waits in state $i$ for $t$ time units before making a transition. Hence it is an unconditional probability distribution. 
It is assumed that any row $i$ of the kernel $C=[c_{ij}]$ satisfies the condition:
\begin{eqnarray}
\int\limits_0^\infty  { \sum\limits_{j}^{}{c_{ij}(t)} dt \approx 1} 
\label{eq:MREAssumption}
\end{eqnarray}
This assumption assures that there is unit probability that the system
will be in one of the $N$ states of the system at time $t$, given the initial 
state as $i$. The probability that the system does not leave state $i$ by time $t$ is given by:
\begin{align}
\label{eq:staying}
W_i (t) = 1 - \int\limits_0^t {w_i(t)dt} 
\end{align}
The objective of the model is to determine the probability $\phi_{ij}(t)$  of being in each state  $j$ given that the system initially is in a particular state $i$. $\phi_{ij}(t)$  can be determined by solving a system of integral equations:
\begin{align}
{\phi _{ij} (t)} = \delta _{ij} W_i (t) + \sum\limits_k {\int\limits_0^t {c_{ik} (\tau ){\phi _{kj} (t - \tau )}d\tau } } 
\label{CTMRE}
\end{align}
Where $i=j=k=0,1,2,...N-1$.	 


The right hand side of Equation \ref{CTMRE} describes the following probabilities:
\begin{enumerate}
\item	$i=j$ and second term=0: $W_i (t)$ is the probability that the process does not leave state $i$ by time $t$.
\item	$i=j$ and second term not 0: process leaves state $i$ and returns to $i$ by time $t$.
\item	$i \neq j$  and second term $\neq j$ : process leaves state $i$ and reaches state $j$ by time $t$.
\end{enumerate}

The system of equations can alternatively be written in a compact form as a matrix:
\begin{align}
{\phi(t)} = diag(W(t)) + {\int\limits_0^t {C(\tau ){\phi (t - \tau )}d\tau } } 
\label{MatMRE}
\end{align}

Given that the system started its operation in state $i$ and that state $j$ is the only absorbing state, the failure probability of the system is given by ${\phi_{ij}(t)}$.


